
\chapter{Rekurziók és generátorfüggvények}\label{chap:generatorf}

\section{Mi a generátorfüggvény?}\label{sec:mi_a_generatorf}
\begin{description}
{\large \item [{Szerző:}] Csapó Hajnalka (Didaktikai mesteri -- Matematika, I. év)}
\end{description}

\subsection*{Bevezető feladatok}
\begin{problem}
Legyen 
\[
A(x)=1+x+x^{2}+x^{3}+\dots+x^{n}+\dots
\]

Hogyan számolnánk ki ezt az összeget 6. osztályban?

Nézzük meg sajátos esetekre. Mennyi lesz az összeg $x=\dfrac{1}{2}$
és $x=\dfrac{1}{3}$ esetén?
\end{problem}

\begin{problem}\textbf{Csokoládé feladat:} Kapok két tábla csokoládét. Rögtön megeszem az egyiket, majd fokozatosan
spórolok vele. Másnap a másik csoki felét eszem meg, következő nap
a maradék felét, és így tovább. Mennyi csokit ettem meg $n$ nap alatt?
\end{problem}
\begin{solution} Tekintsük a következő ábrát:
\begin{center}
	\includegraphics[width=5cm]{\string"content/Csapo_Hajnalka-Lukacs_Andor/csoki1\string".JPG}
\end{center}
\begin{itemize}
	\item Első nap megettem 1 csokit, maradt 1.
	\item Második nap megettem fél csokit, maradt fél.
	\item Harmadik nap megettem negyed csokit, maradt negyed.
	\item \ldots
	\item $n$-edik nap megettem $\dfrac{1}{2^{n-1}}$ csokit, maradt $\dfrac{1}{2^{n-1}}$.
\end{itemize}
Tehát $n$ nap alatt megettem $1+\dfrac{1}{2}+\dfrac{1}{2^{2}}+\dfrac{1}{2^{3}}+\dots+\dfrac{1}{2^{n-1}}$
csokit és maradt $\dfrac{1}{2^{n-1}}$, ahonnan következik, hogy 
\[
1+\dfrac{1}{2}+\dfrac{1}{2^{2}}+\dfrac{1}{2^{3}}+\dots+\dfrac{1}{2^{n-1}}=2-\dfrac{1}{2^{n-1}}
\]
Innen következik, hogy ha a ,,végtelenségig'' ennénk, akkor 
\[
1+\dfrac{1}{2}+\dfrac{1}{2^{2}}+\dfrac{1}{2^{3}}+\dots=2
\]
\end{solution}


\begin{problem}\textbf{Torta feladat:} Ketten eszünk egy tortát. Fokozatosan spórolunk vele. Első nap megesszük
mindketten egy-egy harmadát. Másnap a maradék torta egy-egy harmadát
esszük meg, következő nap a maradék egy-egy harmadát, és így tovább.
Mennyi tortát eszünk meg $n$ nap alatt?
\end{problem}
\begin{solution}
	Tekintsük a következő ábrát:
\begin{center}
	\includegraphics[width=5cm]{\string"content/Csapo_Hajnalka-Lukacs_Andor/torta\string".JPG}
\end{center}

\begin{itemize}
	\item Első nap mindketten megettünk $\dfrac{1}{3}$ tortát, maradt $\dfrac{1}{3}$.
	\item Második nap $\dfrac{1}{9}$, maradt $\dfrac{1}{9}$.
	\item Harmadik nap $2\cdot\dfrac{1}{27}$, maradt $\dfrac{1}{27}$.
	\item \ldots
	\item $n$-edik nap $2\cdot\dfrac{1}{3^{n}}$, maradt $\dfrac{1}{3^{n}}$.
\end{itemize}
\[
2\cdot\left(\dfrac{1}{3}+\dfrac{1}{3^{2}}+\dots+\dfrac{1}{3^{n}}\right)=1-\dfrac{1}{3^{n}}
\]
Tehát végtelen sok nap után: 
\[
2\cdot\left(\dfrac{1}{3}+\dfrac{1}{3^{2}}+\dots\right)=1\Rightarrow\dfrac{1}{3}+\dfrac{1}{3^{2}}+\dots=\dfrac{1}{2}
\]

Hasonlóan el lehet játszadozni és igazolni más ilyen összegeket is,
például, ha hárman esszük a tortát,, akkor kiderül, hogy 
\[
3\cdot\left(\dfrac{1}{4}+\dfrac{1}{4^{2}}+\dots\right)=1\Rightarrow\dfrac{1}{4}+\dfrac{1}{4^{2}}+\dots=\dfrac{1}{3}
\]
\end{solution}

\subsection*{Számolás másképp}

Ugyanezeket az összegeket megkaphatjuk másképp is.

Legyen 
\[
S_{1}=1+\dfrac{1}{2}+\dfrac{1}{2^{2}}+\dfrac{1}{2^{3}}+\dots,
\]
ekkor 
\[
2S_{1}=2+1+\dfrac{1}{2}+\dfrac{1}{2^{2}}+\dots.
\]

Tehát $2S_{1}=2+S_{1}$, ahonnnan $S_{1}=2$.

Ha 
\[
S_{2}=\dfrac{1}{3}+\dfrac{1}{3^{2}}+\dfrac{1}{3^{3}}+\dots,
\]
akkor 
\[
3S_{2}=1+\dfrac{1}{3}+\dfrac{1}{3^{2}}+\dots,
\]
tehát $3S_{2}=1+S_{2}$, ahonnan $S_{2}=\frac{1}{2}.$

\subsection*{Visszatérés $A(x)$-hez}

Térjünk vissza az $A(x)=1+x+x^{2}+x^{3}+\dots+x^{n}+\dots$ összeghez.

$x=\dfrac{1}{k}$ esetén már megvan elemi szinten, de mi történik,
ha $x$ tetszőleges valós szám?

Az 
\[
A(x)=1+x+x^{2}+x^{3}+\dots+x^{n}+\dots
\]
egyenlőség mindkét oldalát szorozzuk be $x$-szel.

\[
xA(x)=x+x^{2}+x^{3}+\dots+x^{n}+\dots
\]

Tehát:

\[
A(x)-xA(x)=1
\]
\[
A(x)=\dfrac{1}{1-x}
\]
\[
1+x+x^{2}+x^{3}+\dots+x^{n}+\dots=\dfrac{1}{1-x}
\]

\textbf{Megjegyzés} Ebben a fejezetben feltételezzük, hogy ezek az
általunk számolt összegek léteznek és végesek, ami a fenti esetben
is csak $|x|<1$ esetén biztosított. 

\subsection*{Újabb összeg}

\begin{problem}
Számoljuk ki a $B(x)=1+2x+3x^{2}+4x^{3}+\dots$ összeget!
\end{problem}
\begin{solution}
A fentiekhez hasonlóan, ha a 
\[
B(x)=1+2x+3x^{2}+4x^{3}+\dots
\]
egyenlőség mindkét oldalát beszorozzuk $x$-szel, kapjuk:

\[
xB(x)=x+2x^{2}+3x^{3}+\dots+(n-1)x^{n-1}+nx^{n}+\dots
\]

\[
B(x)-xB(x)=1+x+x^{2}+\dots+x^{n}+\dots=\dfrac{1}{1-x},
\]
tehát 
\[
B(x)=\dfrac{1}{(1-x)^{2}}
\]
\[
1+2x+3x^{2}+4x^{3}+...+n\cdot x^{n-1}+\dots=\dfrac{1}{(1-x)^{2}}
\]
\end{solution}
\begin{definition}{def:genfugg}
Egy $a_{0},a_{1},a_{2},\dots$ sorozat generátorfüggvénye: 
\[
A(x)=a_{0}+a_{1}x+a_{2}x^{2}+a_{3}x^{3}+\dots=\sum_{n=0}^{\infty}a_{n}x^{n}
\]
Ez egy formális hatványsor, amely ``kódolja'' a sorozatot.
\end{definition}
\textbf{Megjegyzés:} Ez abban az esetben létezik, ha ${\displaystyle |x|<\lim_{n\to\infty}\left|\frac{a_{n}}{a_{n+1}}\right|}$

\begin{example}
Az előbbi feladatok alapján 
\begin{itemize}
\item Az $\left(a_{n}\right)_{n\geq0}$, $a_{n}=1$, bármely $n\geq0$ sorozat
generátorfüggvénye: 
\[
A(x)=\dfrac{1}{1-x}
\]
\item Az $\left(a_{n}\right)_{n\geq0}$, $a_{n}=n+1$, bármely $n\geq0$
sorozat generátorfüggvénye: 
\[
B(x)=\dfrac{1}{(1-x)^{2}}.
\]
\end{itemize}
\end{example}

\subsection*{Feladatok}
\begin{problem} Válaszoljunk a következő kérdésekre:
\begin{enumerate}
\item Az $\dfrac{1}{(1-x)^{3}}$ milyen sorozat generátorfüggvénye? 
\item Az $\dfrac{1}{(1-x)^{n}}$ milyen sorozat generátorfüggvénye? 
\item Mi az $1,0,1,0,\dots$ sorozat generátorfüggvénye?: 
\item Mi az $1,0,0,1,0,0,\dots$ sorozat generátorfüggvénye?: 
\item Hányféleképpen fedhető le $2\times n$ téglalap $1\times2$ dominókkal? 
\end{enumerate}
\end{problem}
\begin{solution}
\begin{enumerate}
\item Az előbbiek alapján $B(x)=1+2x+3x^{2}+4x^{3}+\dots=\dfrac{1}{(1-x)^{2}},$
ekkor 
\[
C(x)=\dfrac{1}{(1-x)^{3}}=\dfrac{B(x)}{1-x},
\]
tehát 
\[
C(x)-xC(x)=B(x)
\]
Legyen a keresett sorozat $\left(c_{n}\right)_{n\geq0}$, ekkor 
\[
C(x)=c_{0}+c_{1}x+c_{2}x^{2}+\dots
\]
és 
\[
xC(x)=c_{0}x+c_{1}x^{2}+c_{2}x^{3}+\dots,
\]
ahonnan 
\[
C(x)-xC(x)=c_{0}+(c_{1}-c_{0})x+(c_{2}-c_{1})x^{2}+(c_{3}-c_{2})x^{3}+\dots
\]
és

\[
B(x)=1+2x+3x^{2}+4x^{3}+\dots
\]

Tehát $c_{0}=1,c_{1}-c_{0}=2,c_{2}-c_{1}=3,c_{3}-c_{2}=4,\dots c_{n}-c_{n-1}=n+1,$
ezeket összeadva kapjuk, hogy $c_{n}=1+2+3+\dots+n+1=\dfrac{(n+1)(n+2)}{2}$
\item Vezessük be az $S_{k}(x)=\dfrac{1}{(1-x)^{k}}$ jelölést és legyen
a sorozat, amelynek a generátorfüggvénye $\left(a_{n}^{k}\right)_{n\geq0}.$

Az eddigiek alapján

\[
S_{1}(x)=\dfrac{1}{1-x}=1+x+x^{2}+x^{3}+\dots=\sum_{n=0}^{\infty}x^{n}\Rightarrow a_{n}^{1}=1,n\geq0
\]

\[
S_{2}(x)=\dfrac{1}{(1-x)^{2}}=1+2x+3x^{2}+4x^{3}+\dots=\sum_{n=0}^{\infty}(n+1)x^{n}\Rightarrow a_{n}^{2}=n+1,n\geq0
\]

\[
S_{3}(x)=\dfrac{1}{(1-x)^{3}}=\sum_{n=0}^{\infty}\dfrac{(n+1)(n+2)}{2}x^{k}\Rightarrow a_{n}^{3}=\dfrac{(n+1)(n+2)}{2},k\geq0
\]

Tehát $a_{n}^{1}=C_{n}^{0},a_{n}^{2}=C_{n+1}^{1},a_{n}^{3}=C_{n+2}^{2}$

Az a sejtésünk, hogy $S_{k}(x)=\dfrac{1}{(1-x)^{k}}$ az $\left(a_{n}^{k}\right)_{n\geq0},$
$a_{n}^{k}=C_{n+k-1}^{k-1}$ sorozat generátorfüggvénye. Ezt a matematikai
indukció módszerével bizonyítjuk.

Feltételezzük, hogy $a_{n}^{k}=C_{n+k-1}^{k-1}$ bármely $n\geq0$,
$k\geq1$ és bizonyítjuk, hogy $a_{n}^{k}=C_{n+k}^{k}$ bármely $n\geq0$.

\[
S_{k+1}(x)=\dfrac{1}{(1-x)^{k+1}}=\dfrac{S_{k}(x)}{1-x},
\]

tehát 
\[
S_{k+1}(x)-xS_{k+1}(x)=S_{k}(x).
\]
\[
S_{k+1}(x)=a_{0}^{k+1}+a_{1}^{k+1}x+a_{2}^{k+1}x^{2}+\dots
\]
és 
\[
xS_{k+1}(x)=a_{0}^{k+1}x+a_{1}^{k+1}x^{2}+a_{2}^{k+1}x^{3}+\dots,
\]
ahonnan 
\[
S_{k+1}(x)-xS_{k+1}(x)=a_{0}^{k+1}+(a_{1}^{k+1}-a_{0}^{k+1})x+(a_{2}^{k+1}-a_{1}^{k+1})x^{2}+(a_{3}^{k+1}-a_{2}^{k+1})x^{3}+\dots
\]
és

\[
S_{k}(x)=a_{0}^{k}+a_{1}^{k}x+a_{2}^{k}x^{2}+a_{3}^{k}x^{3}+\dots
\]

Tehát 
\[
a_{0}^{k+1}=a_{0}^{k},
\]
\[
a_{1}^{k+1}-a_{0}^{k+1}=a_{1}^{k},
\]
\[
a_{2}^{k+1}-a_{1}^{k+1}=a_{2}^{k},
\]
\[
a_{3}^{k+1}-a_{2}^{k+1}=a_{3}^{k}\dots
\]
\[
a_{n}^{k+1}-a_{n-1}^{k+1}=a_{n}^{k},
\]
ezeket összeadva kapjuk, hogy 
\[
a_{n}^{k+1}=a_{0}^{k}+a_{1}^{k}+a_{2}^{k}+\dots+a_{n}^{k}=C_{k-1}^{k-1}+C_{k}^{k-1}+C_{k+1}^{k-1}+\dots+C_{n+k-1}^{k-1}
\]

\[
C_{k-1}^{k-1}=C_{k}^{k}\Rightarrow C_{k-1}^{k-1}+C_{k}^{k-1}=C_{k}^{k}+C_{k}^{k-1}=C_{k+1}^{k}
\]

\[
C_{k+1}^{k}+C_{k+1}^{k-1}=C_{k+2}^{k}\Rightarrow C_{k-1}^{k-1}+C_{k}^{k-1}+C_{k+1}^{k-1}=C_{k+2}^{k}
\]

Ezt folytatva, kapjuk, hogy

\[
C_{k-1}^{k-1}+C_{k}^{k-1}+C_{k+1}^{k-1}+\dots+C_{n+k-1}^{k-1}=C_{n+k}^{k},
\]
amit bizonyítani akartunk.

Tehát a matematikai indukció alapján 
\[
\dfrac{1}{(1-x)^{k}}=\sum_{n=0}^{\infty}C_{n+k-1}^{k-1}x^{n}
\]

\item $1+x^{2}+x^{4}+\dots=S_{1}(x^{2})=\dfrac{1}{1-x^{2}}$ 
\item $1+x^{3}+x^{6}+\dots=S_{1}(x^{3})=\dfrac{1}{1-x^{3}}$ %\item 
 %Ha $a$ darab 1-es, $b$ darab $2$-es és $c$ darab 5-ös pénzegységet használunk, akkor az$(a,b,c)$ számhármasok számát kell meghatározzuk, ha $a+2b+5c=10$ és $a,b,c$ nemnegatív egészek.
 % Az $a+2b+5c$ az $x^{10}$ egy\"utthat\'oja az $(1+x+x^2+x^3+\dots)(1+x^2+x^4+x^6+\dots)(1+x^5+x^10+x^15+\dots)$ szorzatban, amit az $\dfrac{1}{(1-x)(1-x^2)(1-x^5)}$ sorbafejt\'es\'evel is kisz\'amolhatunk, de ebben az esetben nem \'erdemes. Nagyobb számoknál igen.
\item Dominós probléma: Nézzünk meg néhány sajátos esetet.

$n=1$ esetén a lefedéSek száma $T_{1}=1$

\includegraphics[width=0.5cm]{\string"content/Csapo_Hajnalka-Lukacs_Andor/d1\string".JPG}

$n=2$ esetén a lefedéSek száma $T_{2}=2$

\includegraphics[width=3cm]{\string"content/Csapo_Hajnalka-Lukacs_Andor/d2\string".JPG}

$n=3$ esetén a lefedések száma $T_{3}=3$

A $2\times3$-as téglalap lefödéseit csoportos\'{t}hatjuk aszerint,
hogy az első (vagy az utolsó) oszlop egy teljes dominó vagy két fél
dominó. Előbbiket a $2\times2$-es téglalapookból kapjuk egy függőleges
dominó balra (vagy jobbra) illesztésével, utóbbiakat a $2\times1$-esből
2 vízsszintes dominó balra (vagy jobbra) illesztésével.

\includegraphics[width=6cm]{\string"content/Csapo_Hajnalka-Lukacs_Andor/d32\string".JPG}

$n=4$ esetén a lefedések száma $T_{4}=5$

Az előző esethez hasonlóan képezzük.

\includegraphics[width=12cm]{\string"content/Csapo_Hajnalka-Lukacs_Andor/d42\string".JPG}

Formálisan adjuk össze az összes $2\times n$-es lefödést. Vezessük
be a lefödésk szorzását, legyen ez az összetapasztás. Fogadjuk el,
hogy a $2\times0$-s téglalap egyféleképpen födhető le, ez egy 2 egyységnyi
függőleges vonal, ami a szorzásra nézve semleges elem.

\begin{center}
	\includegraphics[width=0.5cm]{\string"content/Csapo_Hajnalka-Lukacs_Andor/0\string".JPG}
\end{center}


Az előbb említett összeg:

\includegraphics[width=\textwidth]{\string"content/Csapo_Hajnalka-Lukacs_Andor/T\string".JPG}

\includegraphics[width=\textwidth]{\string"content/Csapo_Hajnalka-Lukacs_Andor/t2\string".JPG}

Kiemeljük a pirossal jelölt függőleges dominókat és a kékkel jelölt
dupla vízszintes dominókat és kapjuk, hogy

\includegraphics[width=\textwidth]{\string"content/Csapo_Hajnalka-Lukacs_Andor/T3\string".JPG}

A fentiekből
\begin{center}
	\includegraphics[width=10cm]{\string"content/Csapo_Hajnalka-Lukacs_Andor/Tforma\string".JPG}
\end{center}
Tulajdonképpen az utóbbi tört a generátorfüggvénye a lefödések számának
sorozatának.

A továbbiakban helyettesítsük a függőleges dominókat $x$-szel és
a vízszintes dominókat $y$-nal, ekkor $T=1+xT+yT$.

Vegyük észre, hogy egy lefödést helyettesítő monom fokszáma éppen
az illető téglalap hosszúsága. Azaz, ha $y=x$-et helyettes`t-nk a
tov'bbiakban, akkor megkapjuk a sorozat generátorfüggvényét, ugyanis
éppen annyiszor szerepel az összegben a $2\times n$-es téglalap,
ahány ilyen lefödés van, tehát az $x^{n}$ együtthatója éppen a lefödések
száma lesz.

Tehát 
\[
T=T_{0}+T_{1}x+T_{2}x^{2}+T_{3}x^{3}+\dots=\dfrac{1}{1-x-x^{2}}
\]

A sorozat meghatározása érdekében bontsuk elemi törtekre az utóbbi
törtet.

A nevező gyökei $x_{1,2}=\dfrac{-1\pm\sqrt{5}}{2}$

\[
\dfrac{1}{1-x-x^{2}}=\dfrac{1}{\sqrt{5}}\cdot\dfrac{1}{x-\dfrac{-1-\sqrt{5}}{2}}-\dfrac{1}{\sqrt{5}}\cdot\dfrac{1}{x-\dfrac{-1+\sqrt{5}}{2}}=
\]

\[
=\dfrac{1}{\sqrt{5}}\cdot\dfrac{1}{{\dfrac{1+\sqrt{5}}{2}}\left(1-\dfrac{1-\sqrt{5}}{2}x\right)}-\dfrac{1}{\sqrt{5}}\cdot\dfrac{1}{{\dfrac{1-\sqrt{5}}{2}}\left(1-\dfrac{1+\sqrt{5}}{2}x\right)}=
\]

\[
=\dfrac{\sqrt{5}-1}{2\sqrt{5}}\cdot\dfrac{1}{1-\dfrac{1-\sqrt{5}}{2}x}+\dfrac{\sqrt{5}+1}{2\sqrt{5}}\cdot\dfrac{1}{1-\dfrac{1+\sqrt{5}}{2}x}
\]

Tudjuk, hogy $\dfrac{1}{1-ax}=1+ax+a^{2}x^{2}+a^{3}x^{3}+\dots$,
tehát

\[
T=\dfrac{\sqrt{5}-1}{2\sqrt{5}}\cdot\left(1+\dfrac{1-\sqrt{5}}{2}x+\left(\dfrac{1-\sqrt{5}}{2}x\right)^{2}+\dots\right)+
\]

\[
\dfrac{\sqrt{5}+1}{2\sqrt{5}}\cdot\left(1+\dfrac{1+\sqrt{5}}{2}x+\left(\dfrac{1+\sqrt{5}}{2}x\right)^{2}+\dots\right)
\]

Ez alapján: 
\[
T_{n}=\dfrac{\sqrt{5}-1}{2\sqrt{5}}\cdot\left(\dfrac{1-\sqrt{5}}{2}\right)^{n}+\dfrac{\sqrt{5}+1}{2\sqrt{5}}\cdot\left(\dfrac{1+\sqrt{5}}{2}\right)^{n}=\dfrac{1}{\sqrt{5}}\left(\left(\dfrac{1+\sqrt{5}}{2}\right)^{n+1}-\left(\dfrac{1-\sqrt{5}}{2}\right)^{n+1}\right)
\]

\end{enumerate}
\end{solution}

\subsection*{Fibonacci-sorozat}
\begin{problem}
Határozzuk meg a Fibonacci-sorozat generátorfüggvényét és számoljuk
ki annak segítségével az általános tag képletét. 
\end{problem}
\begin{solution}
\[
F_{0}=0,\ F_{1}=1,\ F_{n+2}=F_{n}+F_{n+1}
\]
\[
F(x)=x+x^{2}+2x^{3}+3x^{4}+5x^{5}+8x^{6}+\dots
\]

\[
F(x)=x+xF(x)+x^{2}F(x)\Rightarrow F(x)=\dfrac{x}{1-x-x^{2}}
\]

Ezt elemi törtekre bontjuk:

\[
\dfrac{x}{1-x-x^{2}}=\dfrac{1-\sqrt{5}}{2\sqrt{5}}\cdot\dfrac{1}{x-\dfrac{-1+\sqrt{5}}{2}}+\dfrac{-1-\sqrt{5}}{2\sqrt{5}}\cdot\dfrac{1}{x-\dfrac{-1-\sqrt{5}}{2}}=
\]
\[
=\dfrac{1}{\sqrt{5}}\cdot\dfrac{1}{1-\dfrac{1+\sqrt{5}}{2}x}-\dfrac{1}{\sqrt{5}}\cdot\dfrac{1}{1-\dfrac{1-\sqrt{5}}{2}x}
\]

Tudjuk, hogy $\dfrac{1}{1-ax}=1+ax+a^{2}x^{2}+a^{3}x^{3}+\dots$,
tehát

\[
F(x)=\dfrac{1}{\sqrt{5}}\left(1+\dfrac{1+\sqrt{5}}{2}x+\left(\dfrac{1+\sqrt{5}}{2}x\right)^{2}+\dots\right)-
\]
\[
-\dfrac{1}{\sqrt{5}}\left(1+\dfrac{1-\sqrt{5}}{2}x+\left(\dfrac{1-\sqrt{5}}{2}x\right)^{2}+\dots\right)
\]

Ez alapján: 
\[
F_{n}=\dfrac{1}{\sqrt{5}}\left(\left(\dfrac{1+\sqrt{5}}{2}\right)^{n}-\left(\dfrac{1-\sqrt{5}}{2}\right)^{n}\right)
\]
\end{solution}

Az eddigiekben láthattuk, hogy a generátorfüggvények segítségével
megadhatjuk sorozatok általános tagjának képletét algebrai úton.

\section{Catalan-számok}\label{sec:Catalan}
\begin{description}
{\large \item [{Szerző}] Lukács Andor (Didaktikai mesteri -- Matematika, I. év)}
\end{description}
\begin{problem}[Catalan-számok]
Minden $n\ge1$ esetén legyen $C_{n+2}$ egy konvex $(n+2)$-szög
háromszögekre való felbontásainak a száma olyan átlók (és a sokszög
oldalai) segítségével, amelyek belső pontban nem metszik egymást.
Az alábbi ábrán szemléltetjük az $n=3$ esetet ($C_{3}=5$):
 
\begin{center}
\begin{tikzpicture}[line cap=round,line join=round,>=triangle 45,x=1.0cm,y=1.0cm]
\clip(-1.5,-10.5) rectangle (15.5,-2.1);
\draw[line width=.4pt] (10.,-10.) -- (12.,-10.) -- (12.618033988749895,-8.097886967409693) -- (11.,-6.922316462824747) -- (9.381966011250105,-8.097886967409693) -- cycle;
\draw[line width=0.4pt] (0.,-6.) -- (2.,-6.) -- (2.618033988749895,-4.097886967409694) -- (1.,-2.922316462824747) -- (-0.6180339887498947,-4.097886967409693) -- cycle;
\draw[line width=0.4pt] (6.,-6.) -- (8.,-6.) -- (8.618033988749895,-4.097886967409694) -- (7.,-2.922316462824747) -- (5.381966011250105,-4.097886967409693) -- cycle;
\draw[line width=0.4pt] (12.,-6.) -- (14.,-6.) -- (14.618033988749895,-4.097886967409694) -- (13.,-2.922316462824747) -- (11.381966011250105,-4.097886967409693) -- cycle;
\draw[line width=0.4pt] (2.,-10.) -- (4.,-10.) -- (4.618033988749895,-8.097886967409693) -- (3.,-6.922316462824747) -- (1.3819660112501053,-8.097886967409693) -- cycle;
\draw [line width=0.4pt] (10.,-10.)-- (12.,-10.);
\draw [line width=0.4pt] (12.,-10.)-- (12.618033988749895,-8.097886967409693);
\draw [line width=0.4pt] (12.618033988749895,-8.097886967409693)-- (11.,-6.922316462824747);
\draw [line width=0.4pt] (11.,-6.922316462824747)-- (9.381966011250105,-8.097886967409693);
\draw [line width=0.4pt] (9.381966011250105,-8.097886967409693)-- (10.,-10.);
\draw [line width=0.4pt] (0.,-6.)-- (2.,-6.);
\draw [line width=0.4pt] (2.,-6.)-- (2.618033988749895,-4.097886967409694);
\draw [line width=0.4pt] (2.618033988749895,-4.097886967409694)-- (1.,-2.922316462824747);
\draw [line width=0.4pt] (1.,-2.922316462824747)-- (-0.6180339887498947,-4.097886967409693);
\draw [line width=0.4pt] (-0.6180339887498947,-4.097886967409693)-- (0.,-6.);
\draw [line width=0.4pt] (6.,-6.)-- (8.,-6.);
\draw [line width=0.4pt] (8.,-6.)-- (8.618033988749895,-4.097886967409694);
\draw [line width=0.4pt] (8.618033988749895,-4.097886967409694)-- (7.,-2.922316462824747);
\draw [line width=0.4pt] (7.,-2.922316462824747)-- (5.381966011250105,-4.097886967409693);
\draw [line width=0.4pt] (5.381966011250105,-4.097886967409693)-- (6.,-6.);
\draw [line width=0.4pt] (12.,-6.)-- (14.,-6.);
\draw [line width=0.4pt] (14.,-6.)-- (14.618033988749895,-4.097886967409694);
\draw [line width=0.4pt] (14.618033988749895,-4.097886967409694)-- (13.,-2.922316462824747);
\draw [line width=0.4pt] (13.,-2.922316462824747)-- (11.381966011250105,-4.097886967409693);
\draw [line width=0.4pt] (11.381966011250105,-4.097886967409693)-- (12.,-6.);
\draw [line width=0.4pt] (2.,-10.)-- (4.,-10.);
\draw [line width=0.4pt] (4.,-10.)-- (4.618033988749895,-8.097886967409693);
\draw [line width=0.4pt] (4.618033988749895,-8.097886967409693)-- (3.,-6.922316462824747);
\draw [line width=0.4pt] (3.,-6.922316462824747)-- (1.3819660112501053,-8.097886967409693);
\draw [line width=0.4pt] (1.3819660112501053,-8.097886967409693)-- (2.,-10.);
\draw [line width=0.8pt] (0.,-6.)-- (2.618033988749895,-4.097886967409694);
\draw [line width=0.8pt] (0.,-6.)-- (1.,-2.922316462824747);
\draw [line width=0.8pt] (6.,-6.)-- (8.618033988749895,-4.097886967409694);
\draw [line width=0.8pt] (8.618033988749895,-4.097886967409694)-- (5.381966011250105,-4.097886967409693);
\draw [line width=0.8pt] (12.,-6.)-- (13.,-2.922316462824747);
\draw [line width=0.8pt] (13.,-2.922316462824747)-- (14.,-6.);
\draw [line width=0.8pt] (1.3819660112501053,-8.097886967409693)-- (4.,-10.);
\draw [line width=0.8pt] (4.,-10.)-- (3.,-6.922316462824747);
\draw [line width=0.8pt] (9.381966011250105,-8.097886967409693)-- (12.,-10.);
\draw [line width=0.8pt] (9.381966011250105,-8.097886967409693)-- (12.618033988749895,-8.097886967409693);
\begin{scriptsize}
\draw [fill=black] (10.,-10.) circle (1.0pt);
\draw[color=black] (9.70656367900068,-9.911223268383722) node {$A_1$};
\draw [fill=black] (12.,-10.) circle (1.0pt);
\draw[color=black] (12.337461099221592,-9.952331040574673) node {$A_2$};
\draw [fill=black] (12.618033988749895,-8.097886967409693) circle (1.0pt);
\draw[color=black] (12.830754365513013,-7.83528077274066) node {$A_3$};
\draw [fill=black] (11.,-6.922316462824747) circle (1.0pt);
\draw[color=black] (10.980904616920183,-6.760939834821157) node {$A_4$};
\draw [fill=black] (9.381966011250105,-8.097886967409693) circle (1.0pt);
\draw[color=black] (9.11050098223188,-7.9174963171225645) node {$A_5$};
\draw [fill=black] (0.,-6.) circle (1.0pt);
\draw[color=black] (-0.3031788494960696,-5.923769365861405) node {$A_1$};
\draw [fill=black] (2.,-6.) circle (1.0pt);
\draw[color=black] (2.2455030263429387,-5.882661593670454) node {$A_2$};
\draw [fill=black] (2.618033988749895,-4.097886967409694) circle (1.0pt);
\draw[color=black] (2.841565723111739,-3.8272729841228674) node {$A_3$};
\draw [fill=black] (1.,-2.922316462824747) circle (1.0pt);
\draw[color=black] (0.9917159745189104,-2.752932046203364) node {$A_4$};
\draw [fill=black] (-0.6180339887498947,-4.097886967409693) circle (1.0pt);
\draw[color=black] (-0.89924154626487,-3.909488528504771) node {$A_5$};
\draw [fill=black] (6.,-6.) circle (1.0pt);
\draw[color=black] (5.6985558903828855,-5.923769365861405) node {$A_1$};
\draw [fill=black] (8.,-6.) circle (1.0pt);
\draw[color=black] (8.226683880126417,-5.882661593670454) node {$A_2$};
\draw [fill=black] (8.618033988749895,-4.097886967409694) circle (1.0pt);
\draw[color=black] (8.843300462990694,-3.8272729841228674) node {$A_3$};
\draw [fill=black] (7.,-2.922316462824747) circle (1.0pt);
\draw[color=black] (6.9934507143978655,-2.752932046203364) node {$A_4$};
\draw [fill=black] (5.381966011250105,-4.097886967409693) circle (1.0pt);
\draw[color=black] (5.102493193614086,-3.909488528504771) node {$A_5$};
\draw [fill=black] (12.,-6.) circle (1.0pt);
\draw[color=black] (11.70029063026184,-5.923769365861405) node {$A_1$};
\draw [fill=black] (14.,-6.) circle (1.0pt);
\draw[color=black] (14.228418620005373,-5.882661593670454) node {$A_2$};
\draw [fill=black] (14.618033988749895,-4.097886967409694) circle (1.0pt);
\draw[color=black] (14.824481316774172,-3.8272729841228674) node {$A_3$};
\draw [fill=black] (13.,-2.922316462824747) circle (1.0pt);
\draw[color=black] (12.99518545427682,-2.752932046203364) node {$A_4$};
\draw [fill=black] (11.381966011250105,-4.097886967409693) circle (1.0pt);
\draw[color=black] (11.10422793349304,-3.909488528504771) node {$A_5$};
\draw [fill=black] (2.,-10.) circle (1.0pt);
\draw[color=black] (1.711101987860566,-9.911223268383722) node {$A_1$};
\draw [fill=black] (4.,-10.) circle (1.0pt);
\draw[color=black] (4.239229977604099,-9.87011549619277) node {$A_2$};
\draw [fill=black] (4.618033988749895,-8.097886967409693) circle (1.0pt);
\draw[color=black] (4.835292674372899,-7.83528077274066) node {$A_3$};
\draw [fill=black] (3.,-6.922316462824747) circle (1.0pt);
\draw[color=black] (2.9854429257800703,-6.760939834821157) node {$A_4$};
\draw [fill=black] (1.3819660112501053,-8.097886967409693) circle (1.0pt);
\draw[color=black] (1.0944854049962898,-7.9174963171225645) node {$A_5$};
\end{scriptsize}
\end{tikzpicture}
\end{center}

\noindent Írj fel egy rekurziót $C_{n+1}$-re!
\end{problem}
\begin{solution}
Egyszerű számolással belátható, hogy $C_{1}=1,C_{2}=2,C_{3}=5$. Legyenek
$A_{1},A_{2},\dots A_{n+3}$ egy $n+3$ oldalú sokszög csúcsai. Vegyük
észre, hogy tetszőleges felbontás esetén, a sokszög $A_{1}A_{2}$
oldala mindig pontosan egy, az aktuális felbontásban szereplő háromszögnek
oldala: az $A_{1}A_{2}A_{k}$-nak, valamilyen $k\in\{3,4,\dots n+3\}$
esetén, és emiatt $C_{n}=\sum_{k=3}^{n+3}T_{k},$
ahol $T_{k}$ azoknak a felbontásoknak a száma, amelyekben megjelenik
az $A_{1}A_{2}A_{k}$ háromszög. A $T_{k}$-t vissza tudjuk vezetni
a korábbi $C_{j}$ értékekre, a következő módon: 
\begin{center}
\definecolor{zzttqq}{rgb}{0.6,0.2,0.} \definecolor{qqqqff}{rgb}{0.,0.,1.}
\begin{tikzpicture}[line cap=round,line join=round,>=triangle 45,x=0.5cm,y=0.5cm]
\clip(1.,-12.7) rectangle (14.5,0.1);
\draw[line width=0.4pt] (6.,-12.) -- (10.,-12.) -- (13.064177772475912,-9.428849561253845) -- (13.758770483143632,-5.489618549205014) -- (11.758770483143634,-2.0255169340672587) -- (8.,-0.6574363607645841) -- (4.24122951685637,-2.025516934067256) -- (2.241229516856368,-5.489618549205009) -- (2.9358222275240875,-9.428849561253841) -- cycle;
\fill[line width=0.4pt,color=zzttqq,fill=zzttqq,fill opacity=0.10000000149011612] (10.,-12.) -- (13.064177772475912,-9.428849561253845) -- (13.758770483143632,-5.489618549205014) -- (11.758770483143634,-2.0255169340672587) -- cycle;
\fill[line width=0.4pt,color=zzttqq,fill=zzttqq,fill opacity=0.10000000149011612] (6.,-12.) -- (11.758770483143634,-2.0255169340672587) -- (8.,-0.6574363607645841) -- (4.24122951685637,-2.025516934067256) -- (2.241229516856368,-5.489618549205009) -- (2.9358222275240875,-9.428849561253841) -- cycle;
\draw [line width=0.4pt] (6.,-12.)-- (10.,-12.);
\draw [line width=0.4pt] (10.,-12.)-- (13.064177772475912,-9.428849561253845);
\draw [line width=0.4pt] (13.064177772475912,-9.428849561253845)-- (13.758770483143632,-5.489618549205014);
\draw [line width=0.4pt] (13.758770483143632,-5.489618549205014)-- (11.758770483143634,-2.0255169340672587);
\draw [line width=0.4pt] (11.758770483143634,-2.0255169340672587)-- (8.,-0.6574363607645841);
\draw [line width=0.4pt] (8.,-0.6574363607645841)-- (4.24122951685637,-2.025516934067256);
\draw [line width=0.4pt] (4.24122951685637,-2.025516934067256)-- (2.241229516856368,-5.489618549205009);
\draw [line width=0.4pt] (2.241229516856368,-5.489618549205009)-- (2.9358222275240875,-9.428849561253841);
\draw [line width=0.4pt] (2.9358222275240875,-9.428849561253841)-- (6.,-12.);
\draw [line width=1.2pt,color=qqqqff] (6.,-12.)-- (10.,-12.);
\draw [line width=1.2pt,color=qqqqff] (10.,-12.)-- (11.758770483143634,-2.0255169340672587);
\draw [line width=1.2pt,color=qqqqff] (11.758770483143634,-2.0255169340672587)-- (6.,-12.);
\draw [line width=0.4pt,color=zzttqq] (10.,-12.)-- (13.064177772475912,-9.428849561253845);
\draw [line width=0.4pt,color=zzttqq] (13.064177772475912,-9.428849561253845)-- (13.758770483143632,-5.489618549205014);
\draw [line width=0.4pt,color=zzttqq] (13.758770483143632,-5.489618549205014)-- (11.758770483143634,-2.0255169340672587);
\draw [line width=0.4pt,color=zzttqq] (11.758770483143634,-2.0255169340672587)-- (10.,-12.);
\draw [line width=0.4pt,color=zzttqq] (6.,-12.)-- (11.758770483143634,-2.0255169340672587);
\draw [line width=0.4pt,color=zzttqq] (11.758770483143634,-2.0255169340672587)-- (8.,-0.6574363607645841);
\draw [line width=0.4pt,color=zzttqq] (8.,-0.6574363607645841)-- (4.24122951685637,-2.025516934067256);
\draw [line width=0.4pt,color=zzttqq] (4.24122951685637,-2.025516934067256)-- (2.241229516856368,-5.489618549205009);
\draw [line width=0.4pt,color=zzttqq] (2.241229516856368,-5.489618549205009)-- (2.9358222275240875,-9.428849561253841);
\draw [line width=0.4pt,color=zzttqq] (2.9358222275240875,-9.428849561253841)-- (6.,-12.);
\begin{scriptsize}
  \draw (3.2426838677557385,-6.259909113061673) node[anchor=north west] {$(n+5-k) \text{-oldalú}$};
\draw (10.066554842756481,-6.259910224941581) node[anchor=north west] {$(k-1) \text{-oldalú}$};
\draw [fill=black] (6.,-12.) circle (1.0pt);
\draw[color=black] (5.719001565177595,-12.225966560190215) node {$A_1$};
\draw [fill=black] (10.,-12.) circle (1.0pt);
\draw[color=black] (10.280246411000518,-12.228766189563579) node {$A_2$};
\draw [fill=black] (13.064177772475912,-9.428849561253845) circle (1.0pt);
\draw[color=black] (13.576477358324603,-9.27313913381917) node {$A_3$};
\draw [fill=black] (13.758770483143632,-5.489618549205014) circle (1.0pt);
\draw [fill=black] (11.758770483143634,-2.0255169340672587) circle (1.0pt);
\draw[color=black] (11.931463459825762,-1.7400640819254292) node {$A_k$};
\draw [fill=black] (8.,-0.6574363607645841) circle (1.0pt);
\draw[color=black] (8.248626767788826,-0.2892496274866347) node {$A_{k+1}$};
\draw [fill=black] (4.24122951685637,-2.025516934067256) circle (1.0pt);
\draw [fill=black] (2.241229516856368,-5.489618549205009) circle (1.0pt);
\draw [fill=black] (2.9358222275240875,-9.428849561253841) circle (1.0pt);
\draw[color=black] (2.1535665409605176,-9.421940616325712) node {$A_{n+3}$};
\end{scriptsize}
\end{tikzpicture}
\end{center}

\par Vágjuk ki az $A_{1}A_{2}A_{k}$ háromszöget a sokszögből, így
\begin{itemize} 

\item $k=3$, vagy $k=n+3$ esetén egy-egy $n+2$ oldalú sokszög
keletkezik, ezeket összesen $C_{n}+C_{n}$ módon tudjuk felbontani; 

\item $k\in\{4,5,\dots n+2\}$ esetén keletkezik az $A_{2}A_{3}\dots A_{k}$
és az $A_{k}\dots A_{n+3}A_{1}$ sokszög, ezek $k-1$, valamint $n+5-k$
oldalúak, tehát a kettőt külön-külön $C_{k-3}$, illetve $C_{n+3-k}$
módon tudjuk felbontani. \end{itemize} Az előbbiek alapján 
\[
C_{n+1}=C_{n}+C_{n}+\sum_{k=4}^{n+2}C_{k-3}C_{n+3-k}=C_{n}+C_{n}+\sum_{j=1}^{n-1}C_{j}C_{n-j},
\]
és ha egyezményesen bevezetjük a $C_{0}=1$ értéket, azt kapjuk, hogy
\begin{equation}
C_{n+1}=\sum_{j=0}^{n}C_{j}C_{n-j}=C_{0}C_{n}+C_{1}C_{n-1}+C_{2}C_{n-2}+\ldots+C_{n-2}C_{2}+C_{n-1}C_{1}+C_{n}C_{0},\quad\forall n\ge0.\label{eq:segner}
\end{equation}

A $(C_{n})$ sorozat a \emph{Catalan}\cprotect\footnote{Eugène Charles Catalan, 1814-1894, francia-belga matematikus, \url{https://en.wikipedia.org/wiki/Eugène_Charles_Catalan}}\emph{-számok}
sorozata, \told\aref({eq:segner})+as{} rekurzió pedig a \emph{Segner}\cprotect\footnote{Segner János András, 1704-1777, magyar természettudós, matematikus,
\url{https://hu.wikipedia.org/wiki/Segner_János_András}}\emph{-rekurzió} néven ismert.
\end{solution}
\begin{problem}
Határozd meg a $(C_{n})$ sorozat általános tagját!
\end{problem}
\begin{solution}
A $(C_{n})$ sorozat általános tagját a \told\aref({eq:segner})+as{}
rekurzió alapján, a sorozat generátorfüggvényének segítségével határozzuk
meg. Legyen 
\[
C(x)=\sum_{n=0}^{\infty}C_{n}x^{n}=C_{0}+C_{1}x+C_{2}x^{2}+\ldots
\]
Írhatjuk, hogy 
\begin{align*}
C^{2}(x) & =\left(\sum_{n=0}^{\infty}C_{n}x^{n}\right)\left(\sum_{n=0}^{\infty}C_{n}x^{n}\right)\\
 & =\sum_{n=0}^{\infty}\left(\sum_{\substack{i+j=n\\
i,j\geq0
}
}C_{i}C_{j}\right)x^{n}\\
 & =\sum_{n=0}^{\infty}\left(\sum_{j=0}^{n}C_{j}C_{n-j}\right)x^{n}\\
 & =\sum_{n=0}^{\infty}C_{n+1}x^{n}\\
 & =\frac{1}{x}\cdot\sum_{n=0}^{\infty}C_{n+1}x^{n+1}\\
 & =\frac{1}{x}\left(C(x)-C_{0}\right)\\
 & =\frac{1}{x}\left(C(x)-1\right),
\end{align*}
tehát 
\[
xC^{2}(x)-C(x)+1=0.
\]
Ha az előbbi összefüggést $C(x)$-ben másodfokú egyenletnek tekintjük,
akkor a megoldóképlet alapján 
\[
C(x)=\frac{1\pm\sqrt{1-4x}}{2x}.
\]
Mivel valójában a hatványsor konvergencia-intervallumán belül minden
rögzített $x\ne0$ esetén külön megoldottuk a másodfokú egyenletet,
ezért első ránézésre az előző képletben a ``$\pm$'' előjelek függnek
$x$-től (különböző $x$ értékekhez különböző előjelek tartozhatnak).
Viszont tudjuk, hogy a konvergencia-intervallumon belül $C$ folytonos
függvény, ezért vagy minden $x\ne0$-ra ``$+$'' előjelű, vagy minden
$x\ne0$-ra ``$-$'' előjelű a képletben megjelenő ``$\pm$''.
Ugyanakkor, a $C$ függvény az $x=0$-ban is folytonos, tehát 
\[
\lim\limits_{x\to0}C(x)=\lim\limits_{x\to0^{+}}C(x)=C(0)=C_{0}=1
\]
kell teljesüljön. Mivel a l'Hospital-szabály alapján 
\begin{align*}
\lim\limits_{x\to0^{+}}\frac{1-\sqrt{1-4x}}{2x} & =\lim\limits_{x\to0^{+}}\frac{2}{2\sqrt{1-4x}}=1,\textnormal{ valamint}\\
\lim\limits_{x\to0^{+}}\frac{1+\sqrt{1-4x}}{2x} & =\frac{2}{0^{+}}=+\infty,
\end{align*}
ezért a ``$+$'' előjelű megoldás nem lehet helyes, tehát a hatványsor
konvergencia-intervallumán belül minden $x\ne0$ esetén 
\[
C(x)=\frac{1-\sqrt{1-4x}}{2x}.
\]
Ha felírjuk ennek a függvénynek a Maclaurin-sorát, akkor a megfelelő
együtthatók megadják $(C_{n})$ sorozat általános tagjának a zárt
alakját. Ehhez előbb tekintsük a $g(x)=\sqrt{1-4x}$ függvény Maclaurin-sorát.
Írhatjuk, hogy 
\begin{align*}
g'(x) & =(-4)\cdot\left(\frac{1}{2}\right)\cdot(1-4x)^{-1/2},\\
g''(x) & =(-4)^{2}\cdot\left(\frac{1}{2}\right)\cdot\left(-\frac{1}{2}\right)\cdot(1-4x)^{-3/2},\\
g'''(x) & =(-4)^{3}\cdot\left(\frac{1}{2}\right)\cdot\left(-\frac{1}{2}\right)\cdot\left(-\frac{3}{2}\right)\cdot(1-4x)^{-5/2},\\
 & \ \vdots\\
g^{(n)}(x) & =(-4)^{n}\cdot\left(\frac{1}{2}\right)\cdot\left(-\frac{1}{2}\right)\cdot\left(-\frac{3}{2}\right)\cdots\left(-\frac{2n-3}{2}\right)\cdot(1-4x)^{-(2n-1)/2},
\end{align*}
tehát $g(0)=1$, $g'(0)=-2$, $g''(0)=-4$, $g'''(0)=(-8)\cdot1\cdot3$,
és általában 
\[
g^{(n)}(0)=-2^{n}\cdot1\cdot3\cdot5\cdots(2n-3)=-2^{n}\cdot(2n-3)!!.
\]
Innen következik, hogy 
\begin{align*}
g(x) & =1-2x-\frac{2^{2}}{2!}\cdot(2n-3)!!\cdot x^{2}-\frac{2^{3}}{3!}\cdot(2n-3)!!\cdot x^{3}-\ldots\\
 & =1-2x-\sum_{n=2}^{\infty}\frac{2^{n}}{n!}\cdot(2n-3)!!\cdot x^{n},
\end{align*}
tehát 
\begin{align*}
C(x) & =\frac{1-g(x)}{2x}=1+\sum_{n=2}^{\infty}\frac{2^{n-1}}{n!}\cdot(2n-3)!!\cdot x^{n-1}\\
 & =1+\sum_{n=1}^{\infty}\frac{2^{n}}{(n+1)!}\cdot(2n-1)!!\cdot x^{n}.
\end{align*}
Tehát minden $n\ge1$ esetén 
\[
C_{n}=2^{n}\cdot\frac{1\cdot3\cdot5\cdots(2n-1)}{(n+1)!}=\frac{1}{n+1}\cdot2^{n}\cdot\frac{(2n)!}{n!\cdot(2^{n}\cdot n!)}=\frac{C_{2n}^{n}}{n+1}.
\]
\end{solution}

\section*{Házi feladatok}
\begin{problem}
Mi a generátorfüggvénye az $a_{0}=0,a_{n+1}=2a_{n}+1$ sorozatnak? 
\end{problem}

\begin{solution}
\[
A(x)=a_{0}+a_{1}x+a_{2}x^{2}+a_{3}x^{3}+\dots
\]

\[
2A(x)+1+x+x^{+}x^{3}+\dots=(2a_{0}+1)+(2a_{1}+1)x+(2a_{2}+1)x^{2}+\dots=
\]
\[
=a_{1}+a_{2}x+a_{3}x^{2}+\dots
\]

Tehát 
\[
x(2A(x)+1+x+x^{2}+x^{3}+\dots)=A(x)-a_{0},
\]
azaz 
\[
2xA(x)+\dfrac{x}{1-x}=A(x).
\]

Tehát 
\[
A(x)=\dfrac{x}{(x-1)(2x-1)}
\]
\end{solution}
\begin{problem}
Mi a generátorfüggvénye az $a_{n}=n,n\geq0$ sorozatnak? 
\end{problem}

\begin{solution}
\[
A(x)=0+x+2x^{2}+3x^{3}+\dots=x(1+2x+3x^{2}+\dots)=\dfrac{x}{(1-x)^{2}}
\]
\end{solution}
\begin{problem}
Mi a generátorfüggvénye az $a_{n}=n\cdot2^{n},n\geq0$ sorozatnak? 
\end{problem}

\begin{solution}
\[
A(x)=0+2x+2\cdot2^{2}x^{2}+3\cdot2^{3}x^{3}+\dots=\dfrac{2x}{(1-2x)^{2}}
\]
\end{solution}
\begin{problem}
Minek a generátorfüggvénye az $F(x)=\dfrac{x}{x^{2}-1}$? 
\end{problem}

\begin{solution}
Elemi törtekre bontva

\[
\frac{x}{x^{2}-1}=\frac{1}{2(x-1)}+\frac{1}{2(x+1)}
\]

\[
\frac{1}{x-1}=-\sum_{n=0}^{\infty}x^{n}\Rightarrow\frac{1}{2(x-1)}=-\frac{1}{2}\sum_{n=0}^{\infty}x^{n}
\]

\[
\frac{1}{x+1}=\sum_{n=0}^{\infty}(-1)^{n}x^{n}\Rightarrow\frac{1}{2(x+1)}=\frac{1}{2}\sum_{n=0}^{\infty}(-1)^{n}x^{n}
\]

\[
F(x)=\sum_{n=0}^{\infty}\left(\frac{-1+(-1)^{n}}{2}\right)x^{n}
\]

Ez alapján a sorozattagok:

\[
a_{n}=\frac{-1+(-1)^{n}}{2}=\begin{cases}
0, & \text{ha }n\text{ páros}\\
1, & \text{ha }n\text{ páratlan}
\end{cases}
\]
\end{solution}
\begin{problem}
Határozzuk meg az $a_{0}=a_{1}=1,a_{n+2}=4a_{n+1}-4a_{n},n\geq0$
sorozat generátorfüggvényét és ennek segítségével számoljuk ki az
általános tag képletét! 
\end{problem}

\begin{solution}
\[
A(x)=a_{0}+a_{1}x+a_{2}x^{2}+a_{3}x^{3}+\dots
\]

De mivel 
\[
a_{n+2}-4a_{n+1}+4a_{n}=0,
\]
szorozzuk mindkét oldalt $x^{n+2}$-nel és összegezzük $n\geq0$-tól:

\[
\sum_{n=0}^{\infty}(a_{n+2}-4a_{n+1}+4a_{n})x^{n+2}=0
\]

Vagyis: 
\[
\sum_{n=2}^{\infty}a_{n}x^{n}-4\sum_{n=1}^{\infty}a_{n}x^{n+1}+4\sum_{n=0}^{\infty}a_{n}x^{n+2}=0
\]

Alakítsuk ki az egyenletet $A(x)$-ben: 
\[
A(x)-a_{0}-a_{1}x-4xA(x)+4a_{0}x+4x^{2}A(x)=0
\]

Helyettesítsük be $a_{0}=a_{1}=1$:

\[
A(x)-1-x-4xA(x)+4x+4x^{2}A(x)=0\Rightarrow A(x)(4x^{2}-4x+1)=-3x+1\Rightarrow
\]

\[
A(x)=\frac{3x-1}{(2x-1)^{2}}=\frac{\frac{3}{2}}{1-2x}-\frac{\frac{1}{2}}{(2x-1)^{2}}
\]

\[
\frac{\frac{3}{2}}{1-2x}=\frac{3}{2}\left(1+2x+2^{2}x^{2}+2^{3}x^{3}+\dots\right)
\]

\[
-\frac{\frac{1}{2}}{(2x-1)^{2}}=-\frac{1}{2}\left(1+2\cdot2x+3\cdot2^{2}x^{2}+4\cdot2^{3}x^{3}+\dots\right)
\]

Tehát 
\[
a_{n}=\frac{3}{2}\cdot2^{n}-\frac{1}{2}(n+1)\cdot2^{n}=2^{n-1}(2-n)
\]
\end{solution}

\section*{Nehezebb feladatok}
\begin{extraproblem}[Czofa Vivien]
\textit{\emph{Számítsuk ki a következő összeget:}}\emph{ }
\[
\binom{n}{1}+2\binom{n}{2}+\cdots+n\binom{n}{n}!
\]
\end{extraproblem}

\bigskip{}

\begin{solution}
Vizsgáljuk meg az alábbi súlyozott binomiális összeg viselkedését
a binomiális tétel segítségével. A klasszikus Newton-féle binomiális
formula szerint:

\[
(1+x)^{n}=\sum_{k=0}^{n}\binom{n}{k}x^{k}.
\]

Deriváljuk mindkét oldalt az $x$ változó szerint:

\[
\frac{d}{dx}(1+x)^{n}=n(1+x)^{n-1},
\]

\[
\frac{d}{dx}\left(\sum_{k=0}^{n}\binom{n}{k}x^{k}\right)=\sum_{k=1}^{n}k\binom{n}{k}x^{k-1}.
\]

Az így kapott összefüggés tehát:

\[
n(1+x)^{n-1}=\sum_{k=1}^{n}k\binom{n}{k}x^{k-1}.
\]

Most helyettesítsük be $x=1$-et:

\[
n(1+1)^{n-1}=\sum_{k=1}^{n}k\binom{n}{k}\cdot1^{k-1},
\]

\[
n\cdot2^{n-1}=\sum_{k=1}^{n}k\binom{n}{k}.
\]

Tehát:

\[
\binom{n}{1}+2\binom{n}{2}+\cdots+n\binom{n}{n}=n\cdot2^{n-1}.
\]
\end{solution}
\begin{extraproblem}[Czofa Vivien]
\textit{\emph{Számítsuk ki a következő összeget:}} 
\[
\binom{n}{1}-2\binom{n}{2}+\cdots+(-1)^{n-1}n\binom{n}{n}!
\]
\end{extraproblem}

\bigskip{}

\begin{solution}
Az összeg meghatározásához használjuk a generátorfüggvények elvét.
Tekintsük a binomiális tétel következő módosított alakját:

\[
(1-x)^{n}=\sum_{k=0}^{n}(-1)^{k}\binom{n}{k}x^{k}.
\]

Deriváljuk mindkét oldalt az $x$ szerint:

\[
\frac{d}{dx}(1-x)^{n}=-n(1-x)^{n-1},
\]

\[
\frac{d}{dx}\left(\sum_{k=0}^{n}(-1)^{k}\binom{n}{k}x^{k}\right)=\sum_{k=1}^{n}(-1)^{k}k\binom{n}{k}x^{k-1}.
\]

Az így kapott derivált alak:

\[
-n(1-x)^{n-1}=\sum_{k=1}^{n}(-1)^{k}k\binom{n}{k}x^{k-1}.
\]

Most helyettesítsünk be $x=1$-et:

\[
-n(1-1)^{n-1}=\sum_{k=1}^{n}(-1)^{k}k\binom{n}{k}\cdot1^{k-1},
\]

\[
0=\sum_{k=1}^{n}(-1)^{k}k\binom{n}{k}.
\]

Vagyis:

\[
\binom{n}{1}-2\binom{n}{2}+\cdots+(-1)^{n-1}n\binom{n}{n}=0.
\]
\end{solution}
\begin{extraproblem}[Fábián Nóra]
Határozzuk meg az 
\[
x_{n+3}=2x_{n+2}-x_{n+1}+2x_{n}\quad(n\geq0)
\]
rekurzióval definiált sorozat \textbf{generátorfüggvényét}, majd ebből
kiindulva adjuk meg az általános tag képletét is egy választott kezdőérték-hármas
alapján. 
\end{extraproblem}

\begin{solution}
Megoldás 1 Legyen a generátorfüggvény: 
\[
X(x)=\sum_{n=0}^{\infty}x_{n}x^{n}
\]

A rekurzió: 
\[
x_{n+3}-2x_{n+2}+x_{n+1}-2x_{n}=0
\]

Alkalmazzuk a generátorfüggvényre: 
\[
\sum_{n=0}^{\infty}(x_{n+3}-2x_{n+2}+x_{n+1}-2x_{n})x^{n}=0
\]

Szétbontva: 
\[
\sum_{n=0}^{\infty}x_{n+3}x^{n}-2\sum_{n=0}^{\infty}x_{n+2}x^{n}+\sum_{n=0}^{\infty}x_{n+1}x^{n}-2\sum_{n=0}^{\infty}x_{n}x^{n}=0
\]

Minden tagot a $X(x)$ segítségével fejezünk ki: 
\begin{align*}
\sum_{n=0}^{\infty}x_{n+3}x^{n} & =\frac{X(x)-x_{0}-x_{1}x-x_{2}x^{2}}{x^{3}}\\
\sum_{n=0}^{\infty}x_{n+2}x^{n} & =\frac{X(x)-x_{0}-x_{1}x}{x^{2}}\\
\sum_{n=0}^{\infty}x_{n+1}x^{n} & =\frac{X(x)-x_{0}}{x}\\
\sum_{n=0}^{\infty}x_{n}x^{n} & =X(x)
\end{align*}

Behelyettesítve: 
\[
\frac{X(x)-x_{0}-x_{1}x-x_{2}x^{2}}{x^{3}}-2\cdot\frac{X(x)-x_{0}-x_{1}x}{x^{2}}+\frac{X(x)-x_{0}}{x}-2X(x)=0
\]

Szorozzuk meg $x^{3}$-mal: 
\[
(X(x)-x_{0}-x_{1}x-x_{2}x^{2})-2x(X(x)-x_{0}-x_{1}x)+x^{2}(X(x)-x_{0})-2x^{3}X(x)=0
\]

Rendezzük: 
\begin{align*}
X(x)(1-2x+x^{2}-2x^{3}) & =x_{0}(2x-1-x^{2})+x_{1}x(-1+2x)-x_{2}x^{2}
\end{align*}

\textbf{Példa kezdőértékekkel:}\\

Legyen például
\[
x_{0}=1,\quad x_{1}=0,\quad x_{2}=2
\]

Számláló: 
\[
x_{0}(2x-1-x^{2})+x_{1}x(-1+2x)-x_{2}x^{2}=2x-1-3x^{2}
\]

Nevező: 
\[
1-2x+x^{2}-2x^{3}
\]

\textbf{Végső alak:}\\

\[
\boxed{X(x)=\frac{2x-1-3x^{2}}{1-2x+x^{2}-2x^{3}}}
\]
\end{solution}
\begin{extraproblem}[Gál Tamara]
Számítsuk ki a következő összeget: 
\[
C_{n}^{1}+2C_{n}^{2}+\cdots+nC_{n}^{n}!
\]
\end{extraproblem}

\begin{solution}
Ha keressük a $(C_{n}^{1},2C_{n}^{2},\cdots,nC_{n}^{n})$ generátorfüggvényét,
akkor szintén a Newton-féle binomiális összegre kell támaszkodnunk,
amely alapján írhatjuk a következőket: 
\[
(1+x)^{n}=C_{n}^{0}+C_{n}^{1}x+C_{n}^{2}x^{2}+\cdots+C_{n}^{n}x^{n}.
\]
Az egyenlőség mindkét oldalát deriválva kapjuk, hogy 
\[
n(1+x)^{n-1}=C_{n}^{1}+2C_{n}^{2}x+\cdots+nC_{n}^{n}x^{n-1},
\]
amelybe az $x=1$ értéket helyettesítve a keresett összefüggéshez
jutunk. 
\[
n(1+1)^{n-1}=n2^{n-1}=C_{n}^{1}+2C_{n}^{2}+\cdots+nC_{n}^{n},
\]
tehát 
\[
C_{n}^{1}+2C_{n}^{2}+\cdots+nC_{n}^{n}=n2^{n-1}.
\]
\end{solution}
\begin{extraproblem}[Gál Tamara]
Számítsuk ki a következő összeget: 
\[
C_{n}^{2}+2C_{n}^{3}+3C_{n}^{4}+\cdots+(n-1)C_{n}^{n}!
\]
\end{extraproblem}

\begin{solution}
Keressük meg a saját generátorfüggvényét. Az első feladathoz képest
azt vesszük észre, hogy a kombinációs együtthatókat szorzó számok
eggyel el vannak tolva balra (shift-left), tehát arra következtethetünk,
hogy ezt a deriválásból származó $x$ hatványa adja, amelyet csökkentenünk
kell. Ezek alapján a generátor függvény a következő alakban írható:
\[
\frac{(1+x)^{n}-1}{x}=C_{n}^{1}+C_{n}^{2}x+\cdots+C_{n}^{n}x^{n-1},
\]
amelyet deriválva $x$ szerint: 
\[
\frac{n(1+x)^{n-1}x-[(1+x)^{n}-1]}{x^{2}}=C_{n}^{2}+\cdots+(n-1)C_{n}^{n}x^{n-2},
\]
ha $x$ helyébe $1$-et teszünk, kapjuk a kért összefüggést: 
\[
\frac{n(1+1)^{n-1}-[(1+1)^{n}-1]}{1^{2}}=C_{n}^{2}+\cdots+(n-1)C_{n}^{n},
\]
vagyis 
\[
C_{n}^{2}+\cdots+(n-1)C_{n}^{n}=n2^{n-1}-2^{n}+1=(n-2)2^{n-1}+1.
\]
\end{solution}
\begin{extraproblem}[Gergely Verona]
Feladat 1 Az $f(x)$ függvényre tetszőleges $x$ valós szám esetén
teljesül, hogy $2\cdot f(x)+f(1-x)=x^{2}$.

Milyen $n$ pozitív egész számra igaz, hogy $f(1)+f(2)+\dots+f(n)=19\cdot n$?

(XII. Nemzetközi Magyar Matematika Verseny, 2003) 
\end{extraproblem}

\begin{solution}
A feltétel szerint minden valós $x$-re teljesül, hogy 
\begin{eqnarray*}
2\cdot f(x)+f(1-x)=x^{2}.\quad(1)
\end{eqnarray*}

Helyettesítsünk (1)-ben $x$ helyébe $1-x$-et! 
\begin{eqnarray*}
2\cdot f(1-x)+f(x)={(1-x)}^{2}.\quad(2)
\end{eqnarray*}

Szorozzuk be (1) mindkét oldalát 2-vel! 
\begin{eqnarray*}
4\cdot f(x)+2\cdot f(1-x)=2\cdot x^{2}.\quad(3)
\end{eqnarray*}

A (3) és (2) egyenletek megfelelő oldalainak különbségét véve azt
kapjuk, hogy: $3\cdot f(x)=2\cdot x^{2}-{(1-x)}^{2}$, innen pedig
\begin{eqnarray*}
f(x)=\frac{x^{2}+2x-1}{3}.\quad(4)
\end{eqnarray*}

Mivel (4) bármely valós $x$-re igaz, ezért:

\[
\begin{array}{c}
f(1)=\frac{1^{2}+2\cdot1-1}{3},\\
f(2)=\frac{2^{2}+2\cdot2-1}{3},\\
.\\
.\\
.\\
f(n)=\frac{n^{2}+2\cdot n-1}{3}.
\end{array}
\]

Adjuk össze a kapott egyenletek megfelelő oldalait! Ekkor: 
\begin{eqnarray*}
f(1)+f(2)+\dots+f(n)=\frac{1^{2}+2^{2}+\dots+n^{2}+2\cdot(1+2+\dots+n)-n}{3}.\quad(5)
\end{eqnarray*}

Ismeretes, hogy $1^{2}+2^{2}+\dots+n^{2}=\frac{n\cdot(n+1)\cdot(2n+1)}{6}$
és $1+2+\dots+n=\frac{n\cdot(n+1)}{2}$ ezért 
\[
f(1)+f(2)+\dots+f(n)=\frac{\frac{n\cdot(n+1)\cdot(2n+1)}{6}+2\cdot\frac{n\cdot(n+1)}{2}-n}{3},
\]
ahonnan rendezés után az alábbi összefüggés adódik: 
\begin{eqnarray*}
f(1)+f(2)+\dots+f(n)=\frac{2n^{2}+9n+1}{18}\cdot n.\quad(6)
\end{eqnarray*}

A feltétel miatt $\frac{2n^{2}+9n+1}{18}\cdot n=19\cdot n$, ahonnan
az $n\ne0$-val való osztás után a $2n^{2}+9n-341=0$ egyenletre jutunk,
amelynek megoldásai: $n_{1}=11$ és $n_{2}=-15,5$.

Az $n_{2}$ nyilván nem felel meg a feltételnek, a feladat kérdésére
tehát $n_{1}=11$ a válasz.

Ellenőrizhető, hogy valóban $f(1)+f(2)+\dots+f(11)=19\cdot11=209$. 
\end{solution}
\begin{extraproblem}[Kiss Andrea-Tímea]
Tekintsük az ${\displaystyle {A=\begin{pmatrix}a & b\\
c & d
\end{pmatrix}}}$ mátrixot és jelöljék $\lambda_{1},\lambda_{2}$ az 
\[
\lambda^{2}-(a+d)\lambda+\det A=0\phantom{xxx}(a,b,c,d\in\mathbb{R})
\]
egyenlet gyökeit. Lássuk be, hogy tetszőleges $n\in\mathbb{N}$, $n\geq2$,
illetve az ${\displaystyle {I_{2}=\begin{pmatrix}1 & 0\\
0 & 1
\end{pmatrix}}}$ egységmátrix esetén 
\begin{enumerate}
\item ha $\lambda_{1}\neq\lambda_{2}$, akkor 
\[
{\displaystyle {A^{n}=\frac{\lambda_{1}^{n}-\lambda_{2}^{n}}{\lambda_{1}-\lambda_{2}}\cdot A-\det A\cdot\frac{\lambda_{1}^{n-1}-\lambda_{2}^{n-2}}{\lambda_{1}-\lambda_{2}}\cdot I_{2};}}
\]
\item ha $\lambda_{1}=\lambda_{2}=\lambda$, akkor 
\[
{\displaystyle {A^{n}=n\lambda^{n-1}\cdot A-(n-1)\det A\cdot\lambda^{n-2}\cdot I_{2}.}}
\]
\end{enumerate}
\end{extraproblem}

\begin{solution}
Elsősorban belátjuk, hogy minden $n\geq1$ természetes szám esetén
léteznek az $x_{n}$ és $y_{n}$ valós számok úgy, hogy $A^{n}=x_{n}\cdot A+y_{n}\cdot I_{2}$.

$n=1$ esetén mivel $A=1\cdot A+0\cdot I_{2}$, ezért $x_{1}=1$ és
$y_{1}=0$.

$n=2$ esetén, a Cayley-Hamilton-tétel alapján $A^{2}-(a+d)\cdot A+\det A\cdot I_{2}=0$,
ahonnan $A^{2}=(a+d)\cdot A-\det A\cdot I_{2}$, így $x_{2}=a+d$
és $y_{2}=-\det A$.

Feltételezzük, hogy tetszőleges $n\geq1$ esetén léteznek az $x_{n},y_{n}\in R$
úgy, hogy $A^{n}=x_{n}\cdot A+y_{n}\cdot I_{2}$, és bizonyítsuk ez
a kijelentést $(n+1)$ - re. 
\[
\begin{array}{lcl}
A^{n+1} & = & A^{n}\cdot A=(x_{n}\cdot A+y_{n}\cdot I_{2})\cdot A=x_{n}\cdot A^{2}+y_{n}\cdot A\\
 & = & x_{n}(x_{2}\cdot A+y_{2}\cdot I_{2})+y_{n}\cdot A=(x_{2}x_{n}+y_{n})\cdot A+y_{2}x_{n}\cdot I_{2},
\end{array}
\]
ahonnan 
\[
\left\{ \begin{array}{l}
x_{n+1}=x_{2}\cdot x_{n}+y_{n}\in\mathbb{R}\\
y_{n+1}=y_{2}\cdot x_{n}\in\mathbb{R}.
\end{array}\right.
\]

Tehát léteznek $x_{n+1},y_{n+1}\in\mathbb{R}$ úgy, hogy $A^{n+1}=x_{n+1}\cdot A+y_{n+1}\cdot I_{2}$,
így a matematikai indukció alapján bármely $n\geq1$ természetes szám
esetén léteznek $x_{n},y_{n}$ valós számok úgy, hogy $A^{n}=x_{n}\cdot A+y_{n}\cdot I_{2}$.

A fent használt $x_{n},y_{n}$ valós számok sorozatát felírhatjuk,
mint a következő $(x_{n})_{n\geq0}$ és $(y_{n})_{n\geq0}$ rekurzív
sorozatokat, ahol $x_{0}=0$, $y_{0}=1$ (mert $I_{2}=A^{0}=0\cdot A+1\cdot I_{2}$),
$x_{1}=1$, $y_{1}=0$, ahol a rekurrens összefüggés: 
\[
\begin{array}{l}
x_{n+1}=(a+d)\cdot x_{n}+y_{n}\\
y_{n+1}=-\det A\cdot x_{n},
\end{array}
\]
minden $n\geq0$ esetén fennáll.

Ezek alapján 
\[
x_{n+1}=(a+d)\cdot x_{n}-\det A\cdot x_{n-1},\forall n\geq1,
\]
így az $(x_{n})_{n\geq0}$ sorozatot egy másodrendű lineáris homogén
rekurzió adja meg, amelyhez rendelt karakterisztikus egyenlet $\lambda^{2}-(a+d)\lambda+\det A=0$,
amelynek gyökei, a feladat feltételei alapján, $\lambda_{1}$ és $\lambda_{2}$.
A Viéte-féle összefüggések alapján $\lambda_{1}+\lambda_{2}=a+d$
és $\lambda_{1}\lambda_{2}=\det A$, ahonnan akkor 
\[
x_{n+1}=(\lambda_{1}+\lambda_{2})\cdot x_{n}-\lambda_{1}\lambda_{2}\cdot x_{n-1},\forall n\geq1
\]
\[
\Leftrightarrow x_{n+1}-\lambda_{1}\cdot x_{n}=\lambda_{2}\cdot(x_{n}-\lambda_{1}\cdot x_{n-1}),\forall n\geq1.
\]
Jelöljük az $(a_{n})_{n\geq1}$ sorozat tagjait a következőképpen
$a_{n}=x_{n}-\lambda_{1}\cdot x_{n-1}$. Így $a_{n+1}=\lambda_{2}\cdot a_{n},\forall n\geq1$,
vagyis az $(a_{n})$ sorozat egy mértani haladvány, ezért $a_{n}=\lambda_{2}^{n-1}\cdot a_{1}$,
ahonnan $x_{n}-\lambda_{1}\cdot x_{n-1}=\lambda_{2}^{n-1}\cdot(x_{1}-\lambda_{2}\cdot x_{0})=\lambda_{2}^{n-1}$,
$\forall n\geq1$. Felírva ezeket az összefüggéseket $n=1,2,3,\ldots,n$
esetén, beszorozva az egyenleteket megfelelőképpen és összegezve a
kapott egyenleteket, a következőkhöz jutunk: 
\[
\left\{ \begin{array}{ll}
x_{n}-\lambda_{1}\cdot x_{n-1}=\lambda_{2}^{n-1}\\
x_{n-1}-\lambda_{1}\cdot x_{n-2}=\lambda_{2}^{n-2} & \phantom{x}/\cdot\lambda_{1}\\
x_{n-2}-\lambda_{1}\cdot x_{n-3}=\lambda_{2}^{n-3} & \phantom{x}/\cdot\lambda_{1}^{2}\\
\vdots\\
x_{3}-\lambda_{1}\cdot x_{2}=\lambda_{2}^{2} & \phantom{x}/\cdot\lambda_{1}^{n-3}\\
x_{2}-\lambda_{1}\cdot x_{1}=\lambda_{2}^{1} & \phantom{x}/\cdot\lambda_{1}^{n-2}\\
x_{1}-\lambda_{1}\cdot x_{0}=1 & \phantom{x}/\cdot\lambda_{1}^{n-1}
\end{array}\right.\Rightarrow\left\{ \begin{array}{l}
x_{n}-\cancel{\lambda_{1}\cdot x_{n-1}}=\lambda_{2}^{n-1}\\
\cancel{\lambda_{1}\cdot x_{n-1}}-\cancel{\lambda_{1}^{2}\cdot x_{n-2}}=\lambda_{1}\cdot\lambda_{2}^{n-2}\\
\cancel{\lambda_{1}^{2}\cdot x_{n-2}}-\cancel{\lambda_{1}^{3}\cdot x_{n-3}}=\lambda_{1}^{2}\cdot\lambda_{2}^{n-3}\\
\vdots\\
\cancel{\lambda_{1}^{n-3}\cdot x_{3}}-\cancel{\lambda_{1}^{n-2}\cdot x_{2}}=\lambda_{1}^{n-3}\cdot\lambda_{2}^{2}\\
\cancel{\lambda_{1}^{n-2}\cdot x_{2}}-\cancel{\lambda_{1}^{n-1}\cdot x_{1}}=\lambda_{1}^{n-2}\cdot\lambda_{2}^{1}\\
\cancel{\lambda_{1}^{n-1}\cdot x_{1}}-\lambda_{1}^{n}\cdot x_{0}=\lambda_{1}^{n-1}
\end{array}\right.
\]
\[
\Rightarrow x_{n}=\lambda_{1}^{n-1}+\lambda_{1}^{n-2}\cdot\lambda_{2}+\lambda_{1}^{n-3}\cdot\lambda_{2}^{2}+\cdots+\lambda_{1}^{2}\cdot\lambda_{2}^{n-3}+\lambda_{1}\cdot\lambda_{2}^{n-2}+\lambda_{2}^{n-1}
\]

Ha $\lambda_{1}\neq\lambda_{2}$, akkor ${\displaystyle {x_{n}=\frac{\lambda_{1}^{n}-\lambda_{2}^{n}}{\lambda_{1}-\lambda_{2}}}}$.
Ekkor ${\displaystyle {y_{n}=-\det A\cdot x_{n-1}=-\det A\cdot\frac{\lambda_{1}^{n-1}-\lambda_{2}^{n-1}}{\lambda_{1}-\lambda_{2}}}}$.
Mivel $A^{n}=x_{n}\cdot A+y_{n}\cdot I_{2}$, ezért a $\lambda_{1}\neq\lambda_{2}$
esetben 
\[
{\displaystyle {A^{n}=\frac{\lambda_{1}^{n}-\lambda_{2}^{n}}{\lambda_{1}-\lambda_{2}}\cdot A-\det A\cdot\frac{\lambda_{1}^{n-1}-\lambda_{2}^{n-2}}{\lambda_{1}-\lambda_{2}}\cdot I_{2}.}}
\]

Ha $\lambda_{1}=\lambda_{2}=\lambda$, akkor $x_{n}=n\cdot\lambda^{n-1}$.
Ekkor $y_{n}=-\det A\cdot x_{n-1}=-\det A\cdot(n-1)\cdot\lambda^{n-2}$.
Mivel $A^{n}=x_{n}\cdot A+y_{n}\cdot I_{2}$, ezért a $\lambda_{1}=\lambda_{2}$
esetben 
\[
{\displaystyle {A^{n}=n\lambda^{n-1}\cdot A-(n-1)\det A\cdot\lambda^{n-2}\cdot I_{2}.}}
\]
\end{solution}
\begin{extraproblem}[Kiss Andrea-Tímea]
Adott a $p(x)=x^{2}-3x+2\in\mathbb{Z}[X]$ polinom. Mutassuk ki,
hogy bármely $n\geq2$ természetes szám esetén létezik egy $(a_{n},b_{n})$
egész számpár úgy, hogy a $q_{n}(x)=x^{n}-a_{n}x-b_{n}$ polinom osztható
legyen a $p(x)$ polinommal, majd mutassuk ki, hogy az $a_{n}$ és
$b_{n}$ kielégítik a $a_{n+1}=3a_{n}+b_{n}$ és $b_{n+1}=-2a_{n}$
rekurrens összefüggéseket minden $n\geq2$ esetén. 
\end{extraproblem}

\begin{solution}
A $p(x)=x^{2}-3x+2=(x-1)\cdot(x-2)$ polinom gyökei $x_{1}=1$ és
$x_{2}=2$. Mivel a $p$ polinom osztja a $q_{n}$ polinomot, ezért
a $p$ gyökei a $q_{n}$ gyökei is, azaz 
\[
\left\{ \begin{array}{l}
q_{n}(1)=0\\
q_{n}(2)=0
\end{array}\right.\Rightarrow\left\{ \begin{array}{l}
1-a_{n}-b_{n}=0\\
2^{n}-2a_{n}-b_{n}=0
\end{array}\right.\Rightarrow\left\{ \begin{array}{l}
a_{n}+b_{n}=1\\
2a_{n}+b_{n}=2^{n}
\end{array}\right.
\]

A második egyenletből kivonva az első egyenletet megkapjuk a $a_{n}$
általános tagját: $a_{n}=2^{n}-1,\forall n\geq2$. Ezt behelyettesítve
az első egyenletbe, és kifejezve a $b_{n}$-et, megkapjuk a $(b_{n})$
sorozat általános tagjának a képletét: $b_{n}=2-2^{n},\forall n\geq2$.

Ezek alapján 
\[
3a_{n}+b_{n}=3\cdot(2^{n}-1)+2-2^{n}=(3-1)\cdot2^{n}-3+2=2^{n+1}-1=a_{n+1},\forall n\geq2,
\]
és 
\[
-2a_{n}=-2\cdot(2^{n}-1)=-2^{n+1}+2=2-2^{n+1}=b_{n+1},\forall n\geq2.
\]
\end{solution}
\begin{extraproblem}[Kovács Levente]
Határozzuk meg a következő sorozat explicit képletét, és igazoljuk
generátorfüggvénnyel a következőt! 
\[
\begin{cases}
a_{0}=2,\\
a_{1}=5,\\
a_{n}=4\,a_{n-1}-4\,a_{n-2}+3^{n},\quad n\ge2.
\end{cases}
\]
\end{extraproblem}

\vspace{1em}

\begin{solution}
Először oldjuk meg a homogén részt, majd keressünk partikuláris megoldást,
végül ellenőrizzünk generátorfüggvénnyel.
\begin{enumerate}
\item \textbf{Homogén rész:} 
\[
h_{n}=4h_{n-1}-4h_{n-2},\quad r^{2}-4r+4=0\;\Rightarrow\;(r-2)^{2}=0\;\Rightarrow\;h_{n}=(C_{1}+C_{2}n)\,2^{n}.
\]
\item \textbf{Partikuláris megoldás:} próbáljuk $p_{n}=A\cdot3^{n}$: 
\[
A\cdot3^{n}=4A\cdot3^{n-1}-4A\cdot3^{n-2}+3^{n}\;\Rightarrow\;9A=(12A-4A)+9\;\Rightarrow\;A=9.
\]
Tehát $p_{n}=9\cdot3^{n}$. 
\item \textbf{Általános megoldás:} 
\[
a_{n}=(C_{1}+C_{2}n)\,2^{n}+9\cdot3^{n}.
\]
Indulóértékek: 
\[
n=0:\;C_{1}+9=2\;\Rightarrow\;C_{1}=-7,\qquad n=1:\;2(C_{1}+C_{2})+27=5\;\Rightarrow\;C_{2}=-4.
\]
Így 
\[
\boxed{a_{n}=(-7-4n)\,2^{n}+9\,3^{n}.}
\]
\item \textbf{Generátorfüggvényes ellenőrzés:} Legyen 
\[
A(x)=\sum_{n=0}^{\infty}a_{n}x^{n}.
\]
A rekurziót átalakítva és parciális törtekre bontva kapjuk: 
\[
A(x)=\frac{2+(5-8)x}{1-4x+4x^{2}}\;+\;\frac{9x^{2}}{1-3x},
\]
mely visszafejtve megegyezik az explicit formulával. 
\end{enumerate}
\end{solution}
\begin{extraproblem}[Kovács Levente]
Számítsuk ki az $b_{n}$ sorozat zárt alakját, és igazoljuk generátorfüggvénnyel!
\[
\begin{cases}
b_{0}=1,\\
b_{n}=2\sum_{k=0}^{n-1}b_{k}+5^{n},\quad n\ge1.
\end{cases}
\]
\end{extraproblem}

\vspace{1em}

\begin{solution}
Használjuk a parciális összegek jelölését, majd oldjuk meg a kialakult
rekurziót.
\begin{enumerate}
\item Legyen $S_{n}=\sum_{k=0}^{n}b_{k}$. Ekkor 
\[
b_{n}=2S_{n-1}+5^{n},\quad S_{n}=S_{n-1}+b_{n}=3S_{n-1}+5^{n},\;S_{0}=1.
\]
\item A lineáris nem homogén rekurzió: 
\[
S_{n}-3S_{n-1}=5^{n}.
\]
Homogén megoldás: $C\cdot3^{n}$. Partikuláris tipp: $D\cdot5^{n}$:
\[
D\cdot5^{n}-3D\cdot5^{n-1}=5^{n}\;\Rightarrow\;(5-3)D=1\;\Rightarrow\;D=\tfrac{1}{2}.
\]
Indulófeltétel $S_{0}=1$ adja $C+\tfrac{1}{2}=1$, tehát $C=\tfrac{1}{2}$.
Így 
\[
S_{n}=\tfrac{1}{2}(3^{n}+5^{n}).
\]
\item Vissza $b_{n}$-hez: 
\[
b_{n}=2S_{n-1}+5^{n}=2\cdot\tfrac{1}{2}(3^{n-1}+5^{n-1})+5^{n}=3^{n-1}+6\cdot5^{n-1}.
\]
\[
\boxed{b_{n}=3^{n-1}+6\cdot5^{n-1}.}
\]
\item \textbf{Generátorfüggvényes ellenőrzés:} Legyen 
\[
B(x)=\sum_{n=0}^{\infty}b_{n}x^{n}.
\]
A rekurzió kezelésével és parciális törtekre bontással kapjuk ugyanazt
a zárt alakot. 
\end{enumerate}
\end{solution}
\begin{extraproblem}[Lukács Andor]
Legyen $n$ egy természetes szám. Határozd meg azon $w$ szavak (betűk
véges sorozatai) számát, amelyek teljesítik a következő három tulajdonságot: 
\begin{itemize}
\item[(1)] $w$ pontosan $n$ betűből áll, melyek mindegyike az $\{\texttt{a},\texttt{b},\texttt{c},\texttt{d}\}$
ábécéből származik; 
\item[(2)] $w$ páros sok \texttt{a} betűt tartalmaz; 
\item[(3)] $w$ páros sok \texttt{b} betűt tartalmaz. 
\end{itemize}
\noindent (Például, $n=2$ esetén $6$ ilyen szó van: \texttt{aa},
\texttt{bb}, \texttt{cc}, \texttt{dd}, \texttt{cd} és \texttt{dc}.) 
\begin{flushright}
(IMC, 2020) 
\par\end{flushright}
\end{extraproblem}

\begin{solution}
Legyen $N=\{1,2,\ldots,n\}$. Tekintsünk egy $w$ szót, amely kielégíti
a feltételeket, és legyenek $A,B,C,D\subset N$ azon pozíciók halmazai,
ahol az \texttt{a}, \texttt{b}, \texttt{c} és \texttt{d} betűk találhatók
a $w$-ben. A szavak értelmezése szerint $A\sqcup B\sqcup C\sqcup D=N$.
Az $A$ és $B$ halmazok méretének párosnak kell lennie.

Ahhoz, hogy megalkossuk az összes megfelelő $w$ szót, először válasszuk
ki az $S=A\cup B$ halmazt; a feltételek szerint $|S|=|A|+|B|$ párosnak
kell lennie. Jól ismert, hogy egy $n$-elemű halmaznak (ahol $n\ge1$)
$2^{n-1}$ páros részhalmaza van, így $2^{n-1}$ lehetőség van $S$-re.

Ha $S=\emptyset$, akkor tetszőlegesen választhatjuk $C\subset N$-t,
és ekkor a $D=S\setminus C$ halmaz egyértelműen meghatározott. Mivel
$N$-nek $2^{n}$ részhalmaza van, $2^{n}$ lehetőségünk van a $C$
halmazra, és ezért $2^{n}$ megfelelő $w$ szó létezik $S=\emptyset$
esetén.

Egyébként, ha $k=|S|>0$, akkor ki kell választanunk egy tetszőleges
$C$ részhalmazt az $N\setminus S$-ből és egy páros $A$ részhalmazt
$S$-ből; ekkor $D=(N\setminus S)\setminus C$ és $B=S\setminus A$
meghatározottak, és $|B|=|S|-|A|$ automatikusan páros lesz. $2^{n-k}$
választási lehetőségünk van $C$-re és $2^{k-1}$ független választási
lehetőségünk $A$-ra; tehát minden nem üres páros $S$ esetén $2^{n-k}\cdot2^{k-1}=2^{n-1}$
megfelelő szó létezik.

A nem üres páros $S$ halmazok száma $2^{n-1}-1$, így összesen a
feltételeket kielégítő szavak száma 
\[
1\cdot2^{n}+(2^{n-1}-1)\cdot2^{n-1}=4^{n-1}+2^{n-1}.
\]

Legyen 
\[
A(z)=\sum_{k\ge0}\frac{z^{2k}}{(2k)!}=\frac{e^{z}+e^{-z}}{2}
\]
a páros sok \texttt{a} betűnek megfelelő elemi exponenciális generátorfüggvény,
\[
B(z)=\sum_{k\ge0}\frac{z^{2k}}{(2k)!}=\frac{e^{z}+e^{-z}}{2}
\]
pedig a \texttt{b} betűké. A \texttt{c} és \texttt{d} betűk esetén
nincs megkötés, ezért 
\[
C(z)=D(z)=\sum_{k\ge0}\frac{z^{k}}{k!}=e^{z}.
\]
Az exponenciális generátorfüggvények tulajdonságai alapján 
\[
G(z)=A(z)B(z)C(z)D(z)=\frac{(e^{z}+e^{-z})^{2}}{4}\,e^{2z}=\frac{1}{4}\bigl(e^{4z}+2e^{2z}+1\bigr)
\]
a megadott tulajdonságokat teljesítő szavak számának megfelelő exponenciális
generátorfüggvény, tehát az $n$-edik együttható 
\[
x_{n}=\frac{4^{n}+2\cdot2^{n}}{4}=4^{n-1}+2^{n-1},\qquad\forall n\ge1.
\]
\end{solution}
\begin{extraproblem}[Miklós Dóra]
Az $(a_{n})_{n\geq1}$ sorozatra $a_{1}=\frac{1}{2}$, $a_{2}=\frac{1}{3}$
és $a_{n+2}=\frac{a_{n}a_{n+1}}{3a_{n}-2a_{n+1}}$, ha $n\geq1$.
Határozzuk meg az általános tag képletét $n$ függvényében. \emph{(NMMV
1992) }
\end{extraproblem}

\begin{solution}
Vegyük a megadott képlet inverzét: 
\[
\frac{1}{a_{n+2}}=\frac{3a_{n}-2a_{n+1}}{a_{n}a_{n+1}}=\frac{3}{a_{n+1}}-\frac{2}{a_{n}}.
\]
A fenti átalakítást felhasználva bevezetjük a $b_{n}=\frac{1}{a_{n}}$,
$n\geq1$ sorozatot. Ez értelmezhető, mivel $a_{n+2}$ csak akkor
0, ha $a_{n}$ vagy $a_{n+1}$ egyenlő 0-val. Ez visszavezethető arra,
hogy a sorozat bármelyik tagja csak akkor lehet 0, ha $a_{1}$ vagy
$a_{2}$ egyenlő 0-val. Mivel ez nem teljesül a $b_{n}$ sorozat minden
tagja értelmezhető. Behelyettsítjük $b_{n}$-et a fenti egyenletbe
és egy állandó együtthatós, másodrendű lineáris homogén rekuziót kapunk:
\[
b_{n+2}=3b_{n+1}-2b_{n},\phantom{a}n\geq1.
\]
Ehhez hozzárendelhető az $r^{2}-3r+2=0$ karakterisztikus egyenlet.
Innen $r_{1}=1$ és $r_{2}=2$. Melyek alapján léteznek a $c_{1},c_{2}\in\mathbb{R}$
értékek úgy, hogy 
\[
b_{n}=c_{1}\cdot1^{n}+c_{2}\cdot2^{n}.
\]
Felhasználva, hogy $a_{1}=\frac{1}{2}$ és $a_{2}=\frac{1}{3}$ megkapjuk
a $b_{1}=2$ és $b_{2}=3$ értékeket. Ezeket behelyettesítve megkapjuk,
hogy $c_{1}=1$ és $c_{2}=\frac{1}{2}$. Tehát 
\[
b_{n}=2^{n-1}+1,\phantom{a}n\geq3.
\]
Visszatérve a sorozat értelmezéséhez megkapjuk az $a_{n}$ sorozat
általános tagjának képletét is: 
\[
a_{n}=\frac{1}{2^{n-1}+1},\phantom{a}n\geq3.
\]
\end{solution}
\begin{extraproblem}[Miklós Dóra]
 Legyen a $h$ hosszúságú pályán a balra lépés valószínűsége $p$,
a jobbra lépés valószínűsége $q=1-p$. Melyik mezőre kell kezdetben
helyezni a bolyongó pontot, hogy a játék megközelítőleg igazságos
legyen? \emph{(Orosz Gyula: Rekurzív sorozatok)}
\end{extraproblem}

\begin{solution}
Legyen $p_{i}$ annak a valószínűsége, hogy az $i.$ mezőn lévő bolyongó
pont a bal oldalon hagyja el a pályát. Célunk, hogy az $i$-t úgy
határozzuk meg, hogy $p_{i}$ megközelítőleg $\frac{1}{2}$ legyen.

Ha az $i.$ mezőn állunk, akkor $p$ valószínűséggel lépünk az $(i-1)$-re
és $q$ valószínűséggel az $(i+1)$-re. Ha az $(i-1).$ mezőre lépünk,
akkor onnan $p_{i-1}$ valószínűséggel hagyjuk el a bal oldalon a
pályát, valamint ha az $(i+1).$ mezőre lépünk, akkor $p_{i+1}$ ennek
a valószínűsége. Erre a folyamatra az alábbi rekurziót írhatjuk fel:
\[
p_{i}=p\cdot p_{i-1}+q\cdot p_{i+1}
\]
Ez egy másodrendű lineáris homogén rekurzió melynek megoldásához kezdeti
feltételekre van szükségünk. Ennek érdekére vesszük fel a 0. és a
$(h+1).$ mezőket melyek értékéről elmondható, hogy $p_{0}=1$ és
$p_{h+1}=0$. A rekurzió karakterisztikus egyenlete 
\[
q\cdot r^{2}-r+p=0,
\]
melynek diszkriminánsa $\Delta=1-4qp$. Két esetet fogunk tárgyalni
az szerint, hogy a diszkrimináns nulla vagy nullánál nagyobb, mert
a rekurzió megoldóképlete más a két esetben.

Ha $\Delta=0$, akkor $p=q=\frac{1}{2}$ és $r_{1,2}=1$. Így a sorozat
általános tagjának képlete: 
\[
p_{i}=a+b\cdot i.
\]
A kezdeti feltételeket felhasználva határozzuk meg az $a$, $b$ értéket:
\[
\left\{ \begin{array}{l}
p_{o}=a=1\\
p_{h+1}=a+b\cdot(h+1)=0
\end{array}\right.
\]
Innen kapjuk, hogy $a=1$, $b=-\frac{1}{h+1}$ és $p_{i}=1-\frac{i}{h+1}$.
Ebben az esetben a $p_{i}=\frac{1}{2}$ egyenlet megoldása $i=\frac{h+1}{2}$,
ami $h=2k+1$ páratlan érték esetén a középső mező.

Ha $\Delta>0$, akkor az egyenlet két megoldása $r_{1}=1$ és $r_{2}=\frac{p}{q}$.
Vezessük be a $c=\frac{p}{q}$ jelölést, így a sorozat általános tagjának
képlete: 
\[
p_{i}=a+b\cdot c^{i}.
\]
Ebben az esetben is a kezdeti feltételekből határozzuk meg az $a$,
$b$ értékeket: 
\[
\left\{ \begin{array}{l}
p_{o}=a+b=1\\
p_{h+1}=a+b\cdot c^{h+1}=0
\end{array}\right.
\]
Következik, hogy $a=-\frac{c^{h+1}}{1-c^{h+1}}$, $b=\frac{1}{1-c^{h+1}}$
és 
\[
p_{i}=\frac{c^{i}-c^{h+1}}{1-c^{h+1}}.
\]
A keresett mező meghatározásához a $p_{i}=\frac{1}{2}$ egyenletet
kell megoldanunk, melynek eredménye 
\[
i=\frac{\ln\left(\frac{1+c^{h+1}}{2}\right)}{\ln c}.
\]
\end{solution}
\begin{extraproblem}[Péter Róbert]
Hány $n$ csúcsú $k$ levelű bináris fa létezik? 
\end{extraproblem}

\begin{solution}
Jelöljük ezek számát $b_{n}^{(k)}$-val. Könnyű belátni, hogy $b_{n}^{(k)}=0$,
ha $k>\left\lfloor \frac{n+1}{2}\right\rfloor $. Egyszerű okoskodással
ki lehet számítani a $k=1$ esetet, vagyis $b_{n}^{(1)}=2^{n-1}$
tetszőleges $n\geq1$ természetes számra. Legyen $b_{0}^{(0)}=1$
konvenció alapján.

Akárcsak az előző feladatoknál, itt is a bal és jobb oldali részfákat
vizsgáljuk meg. Ha a bal oldali részfában $i$ csúcs és $j$ levél
van, akkor a jobb oldaliban $n-i-1$ csúcs és $k-j$ levél van. A
$b_{i}^{(j)}b_{n-i-1}^{(k-j)}$ szorzat éppen ezeknek a fáknak a száma.
Összegezve $k$ és $j$ szerint, a következő rekurzív képletet kapjuk:
\begin{equation}
b_{n}^{(k)}=2b_{n-1}^{(k)}+\sum_{i=1}^{n-2}\sum_{j=1}^{k-1}b_{i}^{(j)}b_{n-i-1}^{(k-j)}.\tag{1}
\end{equation}

Ennek a rekurzív egyenletnek a megoldására használjuk a következő
generátorfüggvényt: 
\[
B^{(k)}(z)=\sum_{n=0}^{\infty}b_{n}^{(k)}z^{n},\quad\text{ahol }k\geq1.
\]

A (1) egyenlet mindkét oldalát $z^{n}$-nel megszorozva, majd összeadva
$n=0,1,2,\dots$ értékekre, a következőt kapjuk: 
\[
\sum_{n=1}^{\infty}b_{n}^{(k)}z^{n}=2\sum_{n=1}^{\infty}b_{n-1}^{(k)}z^{n}+\sum_{n=1}^{\infty}\left(\sum_{i=1}^{n-2}\sum_{j=1}^{k-1}b_{i}^{(j)}b_{n-i-1}^{(k-j)}\right)z^{n}.
\]

\[
\sum_{n=1}^{\infty}b_{n}^{(k)}z^{n}=2zB^{(k)}(z)+z\sum_{j=1}^{k-1}B^{(j)}(z)B^{(k-j)}(z).
\]

Innen: 
\begin{equation}
B^{(k)}(z)=\frac{z}{1-2z}\sum_{j=1}^{k-1}B^{(j)}(z)B^{(k-j)}(z).\tag{2}
\end{equation}

Lépésről lépésre haladva, felírhatjuk a következőket: 
\begin{align*}
B^{(2)}(z) & =\frac{z}{1-2z}\left(B^{(1)}(z)\right)^{2},\\
B^{(3)}(z) & =\frac{2z^{2}}{(1-2z)^{2}}\left(B^{(1)}(z)\right)^{3},\\
B^{(4)}(z) & =\frac{5z^{3}}{(1-2z)^{3}}\left(B^{(1)}(z)\right)^{4}.
\end{align*}

Az általános megoldást megpróbáljuk a következő alakban keresni: 
\[
B^{(k)}(z)=c_{k}\frac{z^{k-1}}{(1-2z)^{k-1}}\left(B^{(1)}(z)\right)^{k},
\]

ahol, amint láttuk: $c_{2}=1$, $c_{3}=2$, $c_{4}=5$. A (2) képletbe
behelyettesítve a $c_{k}$ számokra a következő rekurzív összefüggést
kapjuk: 
\[
c_{k}=\sum_{i=1}^{k-1}c_{i}c_{k-i}.
\]

Ezt szintén generátorfüggvények segítségével oldjuk meg. Ha $k=2$,
akkor $c_{2}=c_{1}c_{1}$, és innen $c_{1}=1$. Legyen $c_{0}=1$.
Ha $C(z)=\sum_{n=0}^{\infty}c_{n}z^{n}$ a $c_{n}$ számok generátorfüggvénye,
akkor: 
\[
C(z)-1-z=(C(z)-1)^{2}\quad\Rightarrow\quad C^{2}(z)-3C(z)+z+2=0.
\]

Egyenletet megoldva: 
\[
C(z)=\frac{3-\sqrt{1-4z}}{2}.
\]

Mivel $C(0)=1$, csak a negatív előjelű gyök jó. Sorba fejtés után:
\begin{align*}
C(z) & =\frac{3}{2}-\frac{1}{2}(1-4z)^{1/2}\\
 & =\frac{3}{2}-\frac{1}{2}\sum_{n=0}^{\infty}\binom{2n}{n}\frac{(-1)^{n-1}}{2n-1}z^{n}\\
 & =1+\sum_{n=1}^{\infty}\frac{1}{2(2n-1)}\binom{2n}{n}z^{n}.
\end{align*}

Innen: 
\[
c_{n}=\frac{1}{2(2n-1)}\binom{2n}{n},\quad n\geq1.
\]

Mivel $b_{n}^{(1)}=2^{n-1}$, ha $n\geq1$, könnyen ellenőrizhető,
hogy $B^{(1)}(z)=\frac{z}{1-2z}$. Tehát: 
\[
B^{(k)}(z)=\frac{1}{2(2k-1)}\binom{2k}{k}\sum_{n=0}^{\infty}\binom{2k+n-2}{n}2^{n}z^{2k+n-1}.
\]

Innen pedig: 
\[
b_{n}^{(k)}=\frac{1}{2k-1}\binom{2k}{k}\binom{n-1}{2k-2}2^{n-2k},
\]

vagy: 
\[
b_{n}^{(k)}=\frac{1}{n}\binom{2k}{k}\binom{n}{2k-1}2^{n-2k}.
\]
\end{solution}
\begin{extraproblem}[Seres Brigitta-Alexandra]
Mutassuk ki, hogy a $2x_{n}=\sqrt{3}x_{n-1}+y_{n-1}$ és $2y_{n}=-x_{n-1}+\sqrt{3}y_{n-1}$,
$n\geq1$ összefüggéseket kielégítő valós számsorozatok periodikusak
és periódusuk ugyanaz. 
\begin{flushright}
\textit{(MATLAP 2005/10, L:1205)} 
\par\end{flushright}
\end{extraproblem}

\begin{solution}
Mindkét összefüggést elosztva 2-vel, kapjuk, hogy 
\[
x_{n}=\frac{\sqrt{3}}{2}x_{n-1}+\frac{1}{2}y_{n-1}\quad\text{és}\quad y_{n}=-\frac{1}{2}x_{n-1}+\frac{\sqrt{3}}{2}y_{n-1}.
\]
Látható, hogy a megadott összefüggések jobb oldalai csak konstansokban
térnek el, így felírható a következő 
\[
\begin{pmatrix}x_{n}\\
y_{n}
\end{pmatrix}=\begin{pmatrix}\frac{\sqrt{3}}{2} & \frac{1}{2}\\
-\frac{1}{2} & \frac{\sqrt{3}}{2}
\end{pmatrix}\cdot\begin{pmatrix}x_{n-1}\\
y_{n-1}
\end{pmatrix}.
\]
Tekintettel arra, hogy rekurzívan értelmezett az $(x_{n})_{n\geq0}$
és $(y_{n})_{n\geq0}$ sorozat, felírható 
\[
\begin{pmatrix}x_{n}\\
y_{n}
\end{pmatrix}=\begin{pmatrix}\frac{\sqrt{3}}{2} & \frac{1}{2}\\
-\frac{1}{2} & \frac{\sqrt{3}}{2}
\end{pmatrix}\cdot\begin{pmatrix}x_{n-1}\\
y_{n-1}
\end{pmatrix}=\begin{pmatrix}\frac{\sqrt{3}}{2} & \frac{1}{2}\\
-\frac{1}{2} & \frac{\sqrt{3}}{2}
\end{pmatrix}^{2}\cdot\begin{pmatrix}x_{n-2}\\
y_{n-2}
\end{pmatrix}=\dots=\begin{pmatrix}\frac{\sqrt{3}}{2} & \frac{1}{2}\\
-\frac{1}{2} & \frac{\sqrt{3}}{2}
\end{pmatrix}^{n}\cdot\begin{pmatrix}x_{0}\\
y_{0}
\end{pmatrix}.
\]
Mivel a feladat a periodicitásra kérdez rá, jó ötlet átírni az együtthatómátrixot
egy vele ekvivalens formába, felhasználva valamilyen periódikus függvényeket.
Látható, hogy $\sin x$ és $\cos x$ szögfüggvények megfelel\H{e}k
lesznek, hiszen periódikusak és bizonyos $x$-ekre visszakapjuk az
eredeti együtthatóinkat. Célunk az, hogy úgy válasszuk ki a megfelelő
ekvivalens alakot, hogy az együtthatómátrix minél nagyobb hatványraemelése
során valamilyen könnyen kezelhető alakot véljünk felfedezni.

Ha egyetlen szögfüggvényt akarunk csak felhasználni (pl. $\sin x$)
a kapott együtthatómátrix (jel. $A$) a következő lenne 
\[
A=\begin{pmatrix}\sin\frac{\pi}{3} & \sin\frac{\pi}{6}\\
-\sin\frac{\pi}{6} & \sin\frac{\pi}{3}
\end{pmatrix},
\]
négyzete pedig 
\[
A^{2}=\begin{pmatrix}\sin^{2}\frac{\pi}{3}-\sin^{2}\frac{\pi}{6} & 2\sin\frac{\pi}{3}\cdot\sin\frac{\pi}{6}\\
-2\sin\frac{\pi}{3}\cdot\sin\frac{\pi}{6} & \sin^{2}\frac{\pi}{3}-\sin^{2}\frac{\pi}{6}
\end{pmatrix}.
\]
Mivel nem azonosak a megjelenő argumentumok, nehezebb lenne általánosítani
még nagyobb hatványozás esetén. Próbáljuk meg két különboző szögfüggvény
felhasználásával (így azonosak lesznek az argumentumok) mit kapnánk.
\[
A=\begin{pmatrix}\cos\frac{\pi}{6} & \sin\frac{\pi}{6}\\
-\sin\frac{\pi}{6} & \cos\frac{\pi}{6}
\end{pmatrix},
\]
négyzete pedig 
\[
A^{2}=\begin{pmatrix}\cos^{2}\frac{\pi}{6}-\sin^{2}\frac{\pi}{6} & 2\cos\frac{\pi}{6}\cdot\sin\frac{\pi}{6}\\
-2\cos\frac{\pi}{6}\cdot\sin\frac{\pi}{6} & \cos^{2}\frac{\pi}{6}-\sin^{2}\frac{\pi}{6}
\end{pmatrix}.
\]
Mivel $\cos2\alpha=\cos^{2}\alpha-\sin^{2}\alpha$ és $\sin2\alpha=2\sin\alpha\cdot\cos\alpha$,
kapjuk hogy 
\[
A^{2}=\begin{pmatrix}\cos\frac{2\pi}{6} & \sin\frac{2\pi}{6}\\
-\sin\frac{2\pi}{6} & \cos\frac{2\pi}{6}
\end{pmatrix}.
\]
A következő sejtést 
\[
P(n):A^{n}=\begin{pmatrix}\cos\frac{n\pi}{6} & \sin\frac{n\pi}{6}\\
-\sin\frac{n\pi}{6} & \cos\frac{n\pi}{6}
\end{pmatrix},\forall n\geq1
\]
matematikai indukcióval bizonyítjuk. Látható volt fennebb, hogy $P(1),P(2)$
teljesül. Feltételezzük, hogy $P(k)$ is teljesül: 
\[
P(k):A^{k}=\begin{pmatrix}\cos\frac{k\pi}{6} & \sin\frac{k\pi}{6}\\
-\sin\frac{k\pi}{6} & \cos\frac{k\pi}{6}
\end{pmatrix},\forall k\geq1.
\]
Be kell látni, hogy $P(k+1)$-re is teljesül az állítás. Ekkor definíció
szerint 
\[
A^{k+1}=A^{k}\cdot A=\begin{pmatrix}\cos\frac{k\pi}{6} & \sin\frac{k\pi}{6}\\
-\sin\frac{k\pi}{6} & \cos\frac{k\pi}{6}
\end{pmatrix}\cdot\begin{pmatrix}\cos\frac{\pi}{6} & \sin\frac{\pi}{6}\\
-\sin\frac{\pi}{6} & \cos\frac{\pi}{6}
\end{pmatrix}.
\]
Így 
\[
A^{k+1}=\begin{pmatrix}\cos\frac{k\pi}{6}\cdot\cos\frac{\pi}{6}-\sin\frac{k\pi}{6}\cdot\sin\frac{\pi}{6} & \cos\frac{k\pi}{6}\cdot\sin\frac{\pi}{6}+\sin\frac{k\pi}{6}\cdot\cos\frac{\pi}{6}\\
-\cos\frac{k\pi}{6}\cdot\sin\frac{\pi}{6}-\sin\frac{k\pi}{6}\cdot\cos\frac{\pi}{6} & \cos\frac{k\pi}{6}\cdot\cos\frac{\pi}{6}-\sin\frac{k\pi}{6}\cdot\sin\frac{\pi}{6}
\end{pmatrix}
\]
Mivel $\cos(\alpha+\beta)=\cos\alpha\cdot\cos\beta-\sin\alpha\cdot\sin\beta$
és $\sin(\alpha+\beta)=\sin\alpha\cdot\cos\beta+\sin\beta\cdot\cos\alpha$,
így 
\[
A^{k+1}=\begin{pmatrix}\cos\frac{(k+1)\pi}{6} & \sin\frac{(k+1)\pi}{6}\\
-\sin\frac{(k+1)\pi}{6} & \cos\frac{(k+1)\pi}{6}
\end{pmatrix}=P(k+1),k\geq1.
\]
Tehát a matematikai indukció elve alapján beláttuk, hogy teljesül
\[
A^{n}=\begin{pmatrix}\cos\frac{n\pi}{6} & \sin\frac{n\pi}{6}\\
-\sin\frac{n\pi}{6} & \cos\frac{n\pi}{6}
\end{pmatrix},\forall n\in\mathbb{N^{*}}.
\]
Ekkor igaz, hogy 
\[
\begin{pmatrix}x_{n}\\
y_{n}
\end{pmatrix}=\begin{pmatrix}\cos\frac{n\pi}{6} & \sin\frac{n\pi}{6}\\
-\sin\frac{n\pi}{6} & \cos\frac{n\pi}{6}
\end{pmatrix}\cdot\begin{pmatrix}x_{0}\\
y_{0}
\end{pmatrix},\forall n\geq1.
\]
Következik, hogy 
\[
x_{n}=\cos\frac{n\pi}{6}x_{0}+\sin\frac{n\pi}{6}y_{0}\quad\text{és}\quad y_{n}=-\sin\frac{n\pi}{6}x_{0}+\cos\frac{n\pi}{6}y_{0},\forall n\geq1.
\]
Mivel $\cos x$ és $\sin x$ függvények periodikusak és periódusuk
$2\pi$, így $x_{n+12}=x_{n}$ és $y_{n+12}=y_{n}$, $\forall n\geq1$.
Látható, hogy a megadott számsorozatok periodikusak és periódusuk
ugyanaz ($=12$).
\end{solution}
\begin{extraproblem}[Seres Brigitta-Alexandra]
\textbf{a.)} Határozd meg az $(a_{n})_{n\geq1}$ sorozat általános
tagját, ha $a_{1}=1$ és 
\[
(n+1)!\cdot a_{n}-n!\cdot a_{n+1}=\frac{2n}{n^{4}+n^{2}+1}\cdot a_{n}\cdot a_{n+1},\quad\forall n\geq1.
\]
\\
 \textbf{b.)} Számítsd ki a $\lim\limits_{n\to\infty}\frac{2^{n}}{a_{n}}$
határértéket. 
\begin{flushright}
(\textit{EMMV 2016, XI. osztály, I. forduló, 3. feladat}) 
\par\end{flushright}
\end{extraproblem}

\begin{solution}
\textbf{a.)} Ha $a_{n}=0$, akkor $-n!\cdot a_{n+1}=0$, vagyis $a_{n+1}=0$,
hiszen $n!\neq0,\forall n\in\mathbb{N}$. Ha $a_{n+1}=0$, akkor $(n+1)!\cdot a_{n}=0$,
vagyis $a_{n}=0$, hiszen $n!\neq0,\forall n\in\mathbb{N}$. Vagyis
$a_{n}=0\iff a_{n+1}=0,\forall n\geq1$. Ez visszavezethető arra,
hogy a sorozat bármelyik tagja csak akkor lehet 0 ha $a_{1}=0$. De
$a_{1}=1\neq0$, így a sorozat egyetlen tagja sem lehet 0. Mivel $a_{n}\neq0,\forall n\geq1$,
végigosztunk az eredeti összefüggésben $a_{n}\cdot a_{n+1}$-el, így
kapjuk hogy 
\[
\frac{(n+1)!}{a_{n+1}}-\frac{n!}{a_{n}}=\frac{2n}{n^{4}+n^{2}+1},\quad\forall n\geq1.
\]
Mivel $n^{4}+n^{2}+1=(n^{2}-n+1)(n^{2}+n+1)$, így az egyenlőség jobb
oldalán szereplő törtet két tört összegeként írjuk fel. 
\[
\frac{2n}{n^{4}+n^{2}+1}=\frac{A}{n^{2}-n+1}+\frac{B}{n^{2}+n+1}=\frac{n^{2}(A+B)+n(A-B)+(A+B)}{(n^{2}-n+1)(n^{2}+n+1)}
\]
Ekkor a következő rendszert kell megoldani: 
\[
\begin{cases}
A+B=0\\
A-B=2
\end{cases}\iff\begin{cases}
A=-B\\
-B-B=2
\end{cases}\iff\begin{cases}
A=1\\
B=-1
\end{cases},
\]
vagyis az eredeti törtünk a következő alakú lesz: 
\[
\frac{2n}{n^{4}+n^{2}+1}=\frac{1}{n^{2}-n+1}-\frac{1}{n^{2}+n+1}.
\]
Ekkor a rekurzív összefüggésünk a következő: 
\[
\frac{(n+1)!}{a_{n+1}}-\frac{n!}{a_{n}}=\frac{1}{n^{2}-n+1}-\frac{1}{n^{2}+n+1},\forall n\geq1.
\]
A kapott összefüggést leírjuk egymás alá $n:=1,2,3,...,n-2,n-1$ értékekre:
\begin{align*}
\frac{2!}{a_{2}}-\frac{1!}{a_{1}} & =\frac{1}{1^{2}-1+1}-\frac{1}{1^{2}+1+1}\\
\frac{3!}{a_{3}}-\frac{2!}{a_{2}} & =\frac{1}{2^{2}-2+1}-\frac{1}{2^{2}+2+1}\\
\vdots\\
\frac{(n-1)!}{a_{n-1}}-\frac{(n-2)!}{a_{n-2}} & =\frac{1}{(n-2)^{2}-(n-2)+1}-\frac{1}{(n-2)^{2}+(n-2)+1}\\
\frac{n!}{a_{n}}-\frac{(n-1)!}{a_{n-1}} & =\frac{1}{(n-1)^{2}-(n-1)+1}-\frac{1}{(n-1)^{2}+(n-1)+1}
\end{align*}
Mivel $(k-1)^{2}-(k-1)+1=k^{2}-2k+1-k+2=k^{2}-3k+3=k^{2}-4k+4+k-2+1=(k-2)^{2}+(k-2)+1,\forall k\geq1$,
így összegezve a fennebb felírt összefüggéseket, kapjuk hogy: 
\[
\frac{n!}{a_{n}}-\frac{1!}{a_{1}}=1-\frac{1}{(n-1)^{2}+(n-1)+1}\iff\frac{n!}{a_{n}}-1=1-\frac{1}{n^{2}-n+1},\forall n\geq1.
\]
Ekkor az általános tag képlete: 
\[
a_{n}=\frac{n!\cdot(n^{2}-n+1)}{2n^{2}-2n+1},\forall n\geq1.
\]
\textbf{b.)} Mivel $0\leq n!,\forall n\geq1$ és $0\leq2^{n},\forall n\geq1$,
így 
\[
0\leq\frac{2^{n}}{n!}=\frac{2}{1}\cdot\frac{2\cdot2\cdot...\cdot2}{2\cdot3\cdot...\cdot(n-1)}\cdot\frac{2}{n}\leq\frac{2}{1}\cdot1\cdot\frac{2}{n}\leq\frac{4}{n},\forall n\geq1,
\]
így a fogó-tétel alapján 
\[
0\leq\lim\limits_{n\to\infty}\frac{2^{n}}{n!}\leq\lim\limits_{n\to\infty}\frac{4}{n}=0\Rightarrow\lim\limits_{n\to\infty}\frac{2^{n}}{n!}=0.
\]
Ekkor 
\begin{align*}
\lim\limits_{n\to\infty}\frac{2^{n}}{a_{n}} & =\lim\limits_{n\to\infty}\frac{2^{n}}{\frac{n!\cdot(n^{2}-n+1)}{2n^{2}-2n+1}}=\lim\limits_{n\to\infty}\frac{2^{n}\cdot(2n^{2}-2n+1)}{n!\cdot(n^{2}-n+1)}=\\
 & =\lim\limits_{n\to\infty}\frac{2^{n}}{n!}\cdot\lim\limits_{n\to\infty}\frac{n^{2}(2-\frac{2}{n}+\frac{1}{n^{2}})}{n^{2}(1-\frac{1}{n}+\frac{1}{n^{2}})}=0\cdot2=0.
\end{align*}
\end{solution}
\begin{extraproblem}[Sógor Bence]
Adj meg egy explicit képletet a következő sorozatra:
$$f_{n+1}=3f_n-f_{n-1}-5f_{n-2}, \text{ ahol } f_1=1, f_2=9, f_3=1.$$
\end{extraproblem}

\begin{solution}
Mivel a megadott rekurzió lineáris, ezért a karakterisztikus polinomja:
$$x^3-3x^2+x+5=0.$$
Vegyük észre, hogy $x_1=-1$ a polinomnak gyöke, innen
$$x^3-3x^2+x+5=(x+1)(x^2-4x+5)=(x+1)(x-2+i)(x-2-i).$$
Tehát a karakterisztikus polinom gyökei: $x_1=-1$, $x_2=2-i$, $x_3=2+i$.\\
Az $f_n$ álatlános alakja:
$$f_n=c_1x_1^n+c_2x_2^n+c_3x_3^n=c_1(-1)^n+c_2(2-i)^n+c_3(2+i)^n.$$
A rekurzió első pár értékét felhasználva:\\
\[
\begin{cases}
-c_1+c_2(2-i)+c_3(2+i)=1\\
c_1+c_2(2-i)^2+c_3(2+i)^2=9\\
-c_1+c_2(2-i)^3+c_3(2+i)^3=1\\
\end{cases}
\iff
\begin{cases}
-c_1+c_2(2-i)+c_3(2+i)=1\\
c_1+c_2(3-4i)+c_3(3+4i)=9\\
-c_1+c_2(2-11i)+c_3(2+11i)=1\\
\end{cases}\\
\begin{cases}
c_1=3\\
c_2=1\\
c_3=1\\
\end{cases}
\]
Az $f_n$ álatlános alakja:
$$f_n=3(-1)^n+(2-i)^n+(2+i)^n.$$
\end{solution}
\begin{extraproblem}[Száfta Antal]
Három számot, $a_{1},a_{2},a_{3}$ véletlenszerűen és visszatevés
nélkül kiválasztunk az $\{1,2,3,\ldots,1000\}$ halmazból. Ezután
három másik számot, $b_{1},b_{2},b_{3}$ véletlenszerűen és visszatevés
nélkül kiválasztunk a megmaradt 997 szám közül.

Legyen $p$ annak a valószínűsége, hogy egy alkalmas forgatással az
$a_{1}\times a_{2}\times a_{3}$ méretű téglatest elfér a $b_{1}\times b_{2}\times b_{3}$
méretű dobozban úgy, hogy a téglatest oldalai párhuzamosak a doboz
oldalaival.

Ha $p$ legegyszerűbb alakban felírt tört, akkor mi a számláló és
a nevező összegének értéke?

\emph{(1993 AIME, 7-es számú probléma) }
\end{extraproblem}

\begin{solution}
Legyen $x_{1}>x_{2}>x_{3}>x_{4}>x_{5}>x_{6}$, ahol mindegyik $x_{i}$
a kiválasztott 6 különböző szám egyike. Feltehetjük, hogy ezek a számok
csökkenő sorrendben vannak rendezve.

Azt kell meghatároznunk, hogy hányféleképpen lehet a 6 szám közül
3-at a doboz dimenzióinak, 3-at pedig a téglatest dimenzióinak választani
úgy, hogy a téglatest elférjen a dobozban (megfelelő forgatás után).

Ez azt jelenti, hogy a téglatest mindhárom dimenziója kisebb vagy
egyenlő kell legyen a doboz megfelelő dimenziójánál. Azt is tudjuk,
hogy a számok mind különbözőek, így valójában szigorúan kisebbnek
kell lenniük minden irányban -- azaz egy-egy dimenzió sorrendjét
be lehet állítani.

A helyes kiosztás akkor történik, ha a doboz méretei nagyobbak, mint
a téglatestéi minden irányban. Ez egy olyan kombinatorikai struktúra,
amit a Catalan-számok írnak le.

A $n$-edik Catalan-szám: 
\[
C_{n}=\frac{\binom{2n}{n}}{n+1}
\]

Esetünkben $n=3$, tehát a megfelelő kiosztások száma: 
\[
\frac{\binom{6}{3}}{4}=\frac{20}{4}=5
\]

Az összes lehetséges kiosztás száma: 
\[
\binom{6}{3}=20
\]

Ezért a keresett valószínűség: 
\[
\frac{\binom{6}{3}/4}{\binom{6}{3}}=\frac{1}{4}
\]

A legegyszerűbb tört: $\frac{1}{4}$, így a számláló és nevező összege:
\[
\boxed{005}
\]
\end{solution}
\begin{extraproblem}[Szélyes Klaudia]
Határozd meg az alábbi sorozat generálófüggvényét:
\end{extraproblem}

\[
a_{n}=\binom{n}{2}=\frac{n(n-1)}{2},\quad n\geq0
\]

\begin{solution}
A generálófüggvényt a következőképpen definiáljuk:

\[
G(x)=\sum_{n=0}^{\infty}a_{n}x^{n}=\sum_{n=0}^{\infty}\binom{n}{2}x^{n}
\]

Mivel 
\[
\binom{n}{2}=\frac{1}{2}(n^{2}-n),
\]
ezért a sorozat generálófüggvénye átírható így:

\[
G(x)=\sum_{n=0}^{\infty}\frac{1}{2}(n^{2}-n)x^{n}=\frac{1}{2}\sum_{n=0}^{\infty}(n^{2}-n)x^{n}=\frac{1}{2}\left(\sum_{n=0}^{\infty}n^{2}x^{n}-\sum_{n=0}^{\infty}nx^{n}\right)
\]

Ismeretes generálófüggvények:

\[
\sum_{n=0}^{\infty}nx^{n}=\frac{x}{(1-x)^{2}},\qquad\sum_{n=0}^{\infty}n^{2}x^{n}=\frac{x(x+1)}{(1-x)^{3}}
\]

Ezért:

\[
G(x)=\frac{1}{2}\left(\frac{x(x+1)}{(1-x)^{3}}-\frac{x}{(1-x)^{2}}\right)
\]

Tegyük közös nevezőre és egyszerűsítsük:

\[
G(x)=\frac{1}{2}\cdot\frac{x(x+1)(1-x)-x(1-x)}{(1-x)^{3}}=\frac{1}{2}\cdot\frac{x^{2}(1-x)}{(1-x)^{3}}=\frac{x^{2}}{2(1-x)^{2}}
\]

A generálófüggvény tehát:

\[
\boxed{\sum_{n=0}^{\infty}\binom{n}{2}x^{n}=\frac{x^{2}}{2(1-x)^{3}}}
\]
\end{solution}
\begin{extraproblem}[Szélyes Klaudia]
Legyen a sorozat az alábbi rekurzióval megadva: 
\[
a_{0}=0,\quad a_{1}=1,\quad a_{n}=2a_{n-1}+1\quad\text{minden }n\geq2
\]

Határozd meg a sorozat generálófüggvényét: 
\[
G(x)=\sum_{n=0}^{\infty}a_{n}x^{n}
\]
\end{extraproblem}

\begin{solution}
Írjuk fel a generálófüggvényt: 
\[
G(x)=a_{0}+a_{1}x+a_{2}x^{2}+a_{3}x^{3}+\dots
\]

Használjuk ki a rekurziót: 
\[
a_{n}=2a_{n-1}+1\quad(n\geq2)
\]

Szorzunk meg mindkét oldalt $x^{n}$-nel és összegezzük $n\geq2$-től:
\[
\sum_{n=2}^{\infty}a_{n}x^{n}=\sum_{n=2}^{\infty}(2a_{n-1}+1)x^{n}=2\sum_{n=2}^{\infty}a_{n-1}x^{n}+\sum_{n=2}^{\infty}x^{n}
\]

Az első összeg indexét toljuk le egyel: 
\[
\sum_{n=2}^{\infty}a_{n-1}x^{n}=x\sum_{n=1}^{\infty}a_{n}x^{n}=x(G(x)-a_{0})=xG(x)
\]

A második összeg egy geometriai sor: 
\[
\sum_{n=2}^{\infty}x^{n}=\frac{x^{2}}{1-x}
\]

Tehát: 
\[
\sum_{n=2}^{\infty}a_{n}x^{n}=2xG(x)+\frac{x^{2}}{1-x}
\]

Most bontsuk fel $G(x)$ első két tagját: 
\[
G(x)=a_{0}+a_{1}x+\sum_{n=2}^{\infty}a_{n}x^{n}=0+x+\sum_{n=2}^{\infty}a_{n}x^{n}=x+2xG(x)+\frac{x^{2}}{1-x}
\]

Rendezzük: 
\[
G(x)=x+2xG(x)+\frac{x^{2}}{1-x}
\]

Vigyük át a $G(x)$-es tagokat bal oldalra: 
\[
G(x)-2xG(x)=x+\frac{x^{2}}{1-x}\Rightarrow G(x)(1-2x)=x+\frac{x^{2}}{1-x}
\]

Most hozzuk közös nevezőre a jobb oldalt: 
\[
x+\frac{x^{2}}{1-x}=\frac{x(1-x)+x^{2}}{1-x}=\frac{x-x^{2}+x^{2}}{1-x}=\frac{x}{1-x}
\]

Így: 
\[
G(x)(1-2x)=\frac{x}{1-x}\Rightarrow G(x)=\frac{x}{(1-x)(1-2x)}
\]

A generálófüggvény:

\[
\boxed{G(x)=\sum_{n=0}^{\infty}a_{n}x^{n}=\frac{x}{(1-x)(1-2x)}}
\]
\end{solution}
\begin{extraproblem}[Czofa Vivien]
\textit{\emph{Számítsuk ki a következő összeget:}}\emph{ }
\[
\binom{n}{0}-2\binom{n}{1}+\cdots+(-1)^{n}(n+1)\binom{n}{n}!
\]
\end{extraproblem}

\begin{solution}
A kérdéses összeg meghatározásához célszerű egy
jól megválasztott generátorfüggvényt alkalmazni. Induljunk ki a binomiális
tétel következő formájából:

\[
(1-x)^{n}=\sum_{k=0}^{n}(-1)^{k}\binom{n}{k}x^{k}.
\]

Most szorozzuk meg ezt az egyenlőséget $x$-szel:

\[
x(1-x)^{n}=\sum_{k=0}^{n}(-1)^{k}\binom{n}{k}x^{k+1}=\sum_{k=1}^{n+1}(-1)^{k-1}\binom{n}{k-1}x^{k}.
\]

Deriváljuk az így kapott kifejezést:

\[
\frac{d}{dx}\left[x(1-x)^{n}\right]=(1-x)^{n}-nx(1-x)^{n-1},
\]

másrészt:

\[
\frac{d}{dx}\left[\sum_{k=1}^{n+1}(-1)^{k-1}\binom{n}{k-1}x^{k}\right]=\sum_{k=1}^{n+1}(-1)^{k-1}k\binom{n}{k-1}x^{k-1}.
\]

Most képezzük az alábbi két kifejezés különbségét:

\[
(1-x)^{n}-x(1-x)^{n}=(1-x)^{n}(1-x)=(1-x)^{n+1},
\]

illetve a megfelelő binomiális sorozatkülönbséget:

\[
\sum_{k=0}^{n}(-1)^{k}\binom{n}{k}x^{k}-\sum_{k=0}^{n}(-1)^{k}\binom{n}{k}x^{k+1}=\sum_{k=0}^{n}(-1)^{k}\binom{n}{k}x^{k}-\sum_{k=1}^{n+1}(-1)^{k-1}\binom{n}{k-1}x^{k}.
\]

Végül helyettesítsünk $x=1$-et:

\[
(1-1)^{n}-1\cdot(1-1)^{n}=0-0=0,
\]

illetve:

\[
\binom{n}{0}-2\binom{n}{1}+\cdots+(-1)^{n}(n+1)\binom{n}{n}=\begin{cases}
0, & \text{ha }n\geq1,\\
-1, & \text{ha }n=1.
\end{cases}
\]

Azaz:

\[
\sum_{k=0}^{n}(-1)^{k}(k+1)\binom{n}{k}=\begin{cases}
0, & \text{ha }n\geq1,\\
-1, & \text{ha }n=1.
\end{cases}
\]
\end{solution}
\begin{extraproblem}[Czofa Vivien]
\textit{\emph{Számítsuk ki a következő összeget:}}\emph{ 
\[
\frac{\binom{n}{0}}{1}+\frac{\binom{n}{1}}{2}+\cdots+\frac{\binom{n}{n}}{n+1}!
\]
}
\end{extraproblem}

\begin{solution}
 Induljunk ki ismét a binomiális tételből:

\[
(1+x)^{n}=\sum_{k=0}^{n}\binom{n}{k}x^{k}.
\]

Integráljuk mindkét oldalt az $x\in[0,t]$ intervallumon:

\[
\int_{0}^{t}(1+x)^{n}dx=\sum_{k=0}^{n}\binom{n}{k}\int_{0}^{t}x^{k}dx.
\]

Bal oldal:

\[
\int_{0}^{t}(1+x)^{n}dx=\frac{(1+t)^{n+1}-1}{n+1}.
\]

Jobb oldal:

\[
\sum_{k=0}^{n}\binom{n}{k}\cdot\frac{t^{k+1}}{k+1}.
\]

Ez tehát:

\[
\frac{(1+t)^{n+1}-1}{n+1}=\sum_{k=0}^{n}\frac{\binom{n}{k}}{k+1}t^{k+1}.
\]

Most legyen $t=1$:

\[
\frac{2^{n+1}-1}{n+1}=\sum_{k=0}^{n}\frac{\binom{n}{k}}{k+1}.
\]

Ez éppen:

\[
\frac{\binom{n}{0}}{1}+\frac{\binom{n}{1}}{2}+\cdots+\frac{\binom{n}{n}}{n+1}=\frac{2^{n+1}-1}{n+1}.
\]
\end{solution}
\begin{extraproblem}[Gál Tamara]
Hányféleképpen helyezhetünk el 5 zárójelpárt a $2\cdot3\cdot4\cdot5\cdot6\cdot7$
szorzatban úgy, hogy a műveletek elvégzésekor mindig egy zárójelen
belüli két tényezőt kell összeszoroznunk? Például $((2\cdot(3\cdot(4\cdot5)\cdot(6\cdot7))$
zárójelezéssel írhatjuk le.) 
\end{extraproblem}

\begin{solution}
Vizsgáljuk meg a feladatot kisebb esetekre! Jelöljük $b_{n}$ -nel
azt az értéket, ahányféleképpen egy n- tényezős szorzatban1n $n-1$
zárójelpár elhelyezhető. Ekkor a feladatunk $b_{6}$ meghatározása.
Egy 1-tényezős szorzat 1-féleképp zárójelezhető (úgy, hogy nem teszünk
bele zárójelet), így $b_{1}=1.$ Egy 2-tényezős szorzat is 1-féleképp
zárójelezhető: $(2\cdot3)$, így $b_{2}=1.$ Egy 3-tényezős szorzat
2-féleképp zárójelezhető: $((2\cdot3)\cdot4)$ és $(2\cdot(3\cdot4)),$
így $b_{3}=2.$ Egy 4-tényezős szorzat 5-féleképp zárójelezhető. Megfigyelhetjük,
hogy a legkülső zárójelpár mindig a műveletsor legelején és legvégén
áll (hiszen az utolsó szorzás elvégzésekor már csak két tényező szerepel,
és ez a zárójelpár ezeket határolja). Próbáljuk meg a 4-tényezős szorzat
zárójelezéseit visszavezetni a korábbi esetekre! Csoportosítsuk az
eseteket aszerint, hogy melyik szorzást végezzük el legutoljára! A
következőkben jelöljük az utoljára elvégzett szorzás jelét, és ennek
helye szerint tagoljuk oszlopokra az eseteket: $(2\textcolor{red}{\cdot}((3\cdot4)\cdot5))$,
$(2\textcolor{red}{\cdot}(3\cdot(4\cdot5)))$, $((2\cdot3)\textcolor{red}{\cdot}(4\cdot5))$,
$(((2\cdot3)\cdot4)\textcolor{red}{\cdot}5)$, $((2\cdot(3\cdot4))\textcolor{red}{\cdot}5)$
Vegyük észre, hogy ha az utolsó szorzásjel a 2 és a 3 közötti, akkor
ennek elvégzése előtt egy 1- tényezős (2) és egy 3-tényezős $(3\cdot4\cdot5)$
szorzatot kell egymástól függetlenül zárójeleznünk. Ha az utolsó szorzásjel
a 3 és a 4 közötti, akkor ennek elvégzése előtt egy 2-tényezős $2\cdot3$
és egy másik 2-tényezős $(4\cdot5)$ szorzatot kell egymástól függetlenül
zárójeleznünk. Ha pedig az utolsó szorzás- jel a 4 és az 5 közötti,
akkor ennek elvégzése előtt egy 3-tényezős $(2\cdot3\cdot4)$ és egy
1-tényezős (5) szorzatot kell egymástól függetlenül zárójeleznünk.
Így a $b_{4}=b_{1}\cdot b_{3}+b_{2}\cdot b_{2}+b_{3}\cdot b_{1}$
összefüggést kapjuk. Ugyanez a gondolatmenet továbbvihető a $b$ sorozat
következő néhány tagjának meghatározására: $b_{5}=b_{1}\cdot b_{4}+b_{2}\cdot b_{3}+b_{3}\cdot b_{2}+b_{4}\cdot b_{1}=1\cdot5+1\cdot2+2\cdot1+5\cdot1=14$
és $b_{6}=b_{1}\cdot b_{5}+b_{2}\cdot b_{4}+b_{3}\cdot b_{3}+b_{4}\cdot b_{2}+b_{5}\cdot b_{1}=14+5+4+5+14=42.$
Tehát a feladatban szereplő 6-tényezős $2\cdot3\cdot4\cdot5\cdot6\cdot7$
szorzatban 42-féleképpen helyezhetünk el 5 zárójelpárt. 
\end{solution}
\begin{extraproblem}[Kiss Andrea-Tímea]
Hányféleképpen lehet 24 egyforma robotot kiosztani négy szerelősorra
úgy, hogy legalább három robot jusson minden sorra? 
\end{extraproblem}

\begin{solution}
Ha tekintjük az $(a_{n})_{n\geq0}$ sorozatot, ahol $a_{n}=$ hányféleképpen
lehet $n$ azonos robotot kiosztani négy szerelősorra úgy, hogy legalább
három robot jusson minden sorra. Ekkor az $(a_{n})_{n\geq0}$ sorozathoz
rendelt generátorfüggvény: 
\[
A(x)=a_{0}+a_{1}x+a_{2}x^{2}+a_{3}x^{3}+\ldots,
\]
ahol mivel négy szerelősor van, és minden szerelősorra legalább három
robot kerül, ezért a robotok száma legalább $3\cdot4=12$. Ezek alapján
$a_{n}=0,\ \forall n<12$. Tehát 
\[
A(x)=a_{12}x^{12}+a_{13}x^{13}+a_{14}x^{14}+\ldots.
\]
Hogyan tudnánk másképp felírni az $A(x)$-et? Ha minden szerelősorra
legalább három robot jut, akkor az első, második, harmadik és a negyedik
szerelősorra $3,4,5,\ldots$ robot is juthat. 
\[
\begin{array}{lcl}
A(x) & = & (x^{3}+x^{4}+x^{5}+\ldots)\cdot(x^{3}+x^{4}+\ldots)\cdot(x^{3}+x^{4}+\ldots)\cdot(x^{3}+x^{4}+\ldots)\\
 & = & (x^{3}+x^{4}+x^{5}+\ldots)^{4}=(x^{3})^{4}\cdot(1+x+x^{2}+\ldots)^{4}\\
 & = & x^{12}\cdot\left(\dfrac{1}{1-x}\right)^{4}=x^{12}\cdot(1-x)^{-4}.
\end{array}
\]
Így az $A(x)$ generátorfüggvény esetén az $a_{24}$ az $x^{24}$
együtthatója. Ahhoz, hogy ezt a számot megkapjuk szükségünk lesz az
$(1-x)^{-4}$ függvény Taylor-sorba fejtésére.

Legyen $f:[0,1)\rightarrow\mathbb{R}$, $f(x)=(1-x)^{-4}$. Ekkor
\[
\begin{array}{l}
f'(x)=(-1)\cdot(-4)\cdot(1-x)^{-5},\\
f''(x)=(-1)^{2}\cdot(-4)\cdot(-5)\cdot(1-x)^{-6},\\
f'''(x)=(-1)^{3}\cdot(-4)\cdot(-5)\cdot(-6)\cdot(1-x)^{-7},\\
\vdots\\
f^{(n)}(x)=(-1)^{n}\cdot(-1)^{n}\cdot\dfrac{(n+3)!}{3!}\cdot(1-x)^{-(n+4)}=\dfrac{(n+3)!}{3!}\cdot(1-x)^{-(n+4)}.
\end{array}
\]

Ekkor az $f$ függvény Taylor-sorba fejtése az $x_{0}=0$ körül: 
\[
\begin{array}{lcl}
f(x) & = & f(0)+\dfrac{f'(0)}{1!}x+\dfrac{f''(0)}{2!}x^{2}+\dfrac{f'''(0)}{3!}x^{3}+\ldots+\dfrac{f^{(n)}(0)}{n!}x^{n}\ldots\\
 & = & {\displaystyle {\sum_{n=0}^{\infty}}\dfrac{f^{(n)}(0)}{n!}x^{n}={\displaystyle {\sum_{n=0}^{\infty}}\dfrac{\dfrac{(n+3)!}{3!}\cdot(1-0)^{-(n+4)}}{n!}x^{n}={\displaystyle {\sum_{n=0}^{\infty}}C_{n+3}^{3}x^{n}.}}}
\end{array}
\]

Tehát 
\[
A(x)=x^{12}\cdot(1-x)^{-4}=x^{12}\cdot{\displaystyle {\sum_{n=0}^{\infty}}C_{n+3}^{3}x^{n}={\displaystyle {\sum_{n=0}^{\infty}}C_{n+3}^{3}x^{n+12}.}}
\]
Így az $x^{24}$ együtthatóját akkor kapjuk meg, amikor az összegzésbe
$n=12$, vagyis $a_{24}=C_{12+3}^{3}=C_{15}^{3}=\dfrac{13\cdot14\cdot15}{3!}=455$.
Tehát $24$ robotot $455$-féleképpen lehet elosztani négy szerelősorra
úgy, hogy minden szerelősorra jusson legalább három robot. 
\end{solution}
\begin{extraproblem}[Kovács Levente]
Legyen a $(c_{n})$ sorozat úgy definiálva, hogy 
\[
\begin{cases}
c_{0}=1,\quad c_{1}=2,\\
c_{n}=5\,c_{n-1}-6\,c_{n-2}+2^{n},\quad n\ge2.
\end{cases}
\]
\begin{enumerate}
\item Határozzuk meg $c_{n}$ explicit zárt alakját. 
\item Igazoljuk az eredményt generátorfüggvénnyel. 
\end{enumerate}
\end{extraproblem}

\vspace{1em}

\begin{solution}
\textbf{1. Homogén rész és partikuláris tipp.} 
\[
h_{n}=5h_{n-1}-6h_{n-2},\quad r^{2}-5r+6=0\;\Rightarrow\;(r-2)(r-3)=0,
\]
tehát $h_{n}=A\cdot2^{n}+B\cdot3^{n}$.

A nem‐homogén tag $2^{n}$, de mivel $2$ gyök a karakterisztikusból,
próbáljuk 
\[
p_{n}=C\cdot n\cdot2^{n}.
\]
Helyettesítve: 
\[
C\,n\,2^{n}=5C\,(n-1)2^{n-1}-6C\,(n-2)2^{n-2}+2^{n}.
\]
Osszuk $2^{n-2}$-vel és rendezzük: 
\[
4Cn=10C(n-1)-6C(n-2)+4.
\]
Bontva: 
\[
4Cn=10Cn-10C-6Cn+12C+4\;\Longrightarrow\;0=(10C-6C-4C)n+(2C+4)\;\Longrightarrow\;2C+4=0,
\]
így $C=-2$. Tehát 
\[
p_{n}=-2n\,2^{n}.
\]

\textbf{2. Általános megoldás és konstansok.} 
\[
c_{n}=A\,2^{n}+B\,3^{n}-2n\,2^{n}.
\]
Indulófeltételek: 
\[
n=0:\;A+B=1,\quad n=1:\;2A+3B-4=2\;\Longrightarrow\;2A+3B=6.
\]
Megoldva: 
\[
\begin{cases}
A+B=1,\\
2A+3B=6,
\end{cases}\quad\Longrightarrow\;A=-3,\;B=4.
\]
Így 
\[
\boxed{c_{n}=-3\cdot2^{n}+4\cdot3^{n}-2n\,2^{n}.}
\]

\textbf{3. Generátorfüggvényes ellenőrzés.}

Legyen 
\[
C(x)=\sum_{n=0}^{\infty}c_{n}\,x^{n}.
\]
A rekurzióból 
\[
\sum_{n\ge2}c_{n}x^{n}=5x\sum_{n\ge2}c_{n-1}x^{n-1}-6x^{2}\sum_{n\ge2}c_{n-2}x^{n-2}+\sum_{n\ge2}2^{n}x^{n}.
\]
Bevezetve a teljes $C(x)$ és az első két tagot külön: 
\[
C(x)-1-2x=5x\bigl(C(x)-1\bigr)-6x^{2}C(x)+\frac{4x^{2}}{1-2x}.
\]
Rendezve: 
\[
C(x)\bigl(1-5x+6x^{2}\bigr)=1-3x+\frac{4x^{2}}{1-2x},
\]
\[
C(x)=\frac{1-3x}{(1-2x)(1-3x)}+\frac{4x^{2}}{(1-2x)(1-3x)}=\frac{1-3x+4x^{2}}{(1-2x)(1-3x)}.
\]
Parciális törtekre bontva visszafejtve kapjuk a fenti $c_{n}$-et.
\end{solution}
\vspace{1em}

\begin{extraproblem}[Kovács Levente]
Legyen $(d_{n})$ a következő rekurzív sorozat: 
\[
\begin{cases}
d_{0}=0,\;d_{1}=1,\\
d_{n}=3\,d_{n-1}+2\,d_{n-2}+n,\quad n\ge2.
\end{cases}
\]
\begin{enumerate}
\item Határozzuk meg $d_{n}$ explicit alakját. 
\item Igazoljuk a végeredményt generátorfüggvénnyel. 
\end{enumerate}
\end{extraproblem}

\vspace{1em}

\begin{solution}
\textbf{1. Homogén és partikuláris rész.}

Homogén: 
\[
h_{n}=3h_{n-1}+2h_{n-2},\quad r^{2}-3r-2=0\;\Rightarrow\;r=\frac{3\pm\sqrt{17}}{2},
\]
tehát 
\[
h_{n}=A\!\left(\frac{3+\sqrt{17}}{2}\right)^{n}+B\!\left(\frac{3-\sqrt{17}}{2}\right)^{n}.
\]

A nem‐homogén tag $n$ polinom fokú, próbáljuk 
\[
p_{n}=Cn+D.
\]
Helyettesítve: 
\[
Cn+D=3[C(n-1)+D]+2[C(n-2)+D]+n.
\]
Rendezzük: 
\[
Cn+D=3Cn-3C+3D+2Cn-4C+2D+n
\]
\[
\Longrightarrow\;0=(5C-C+1)n+(5D-7C-D)\;\Longrightarrow\;4C+1=0,\;4D-7C=0
\]
\[
C=-\tfrac{1}{4},\quad D=\tfrac{7}{16}.
\]
Így 
\[
p_{n}=-\tfrac{1}{4}n+\tfrac{7}{16}.
\]

\textbf{2. Általános megoldás és konstansok.} 
\[
d_{n}=Ar_{1}^{n}+Br_{2}^{n}-\tfrac{1}{4}n+\tfrac{7}{16},
\]
ahol $r_{1,2}=(3\pm\sqrt{17})/2$. Indulófeltételekből $n=0,1$ meghatározhatók
$A,B$.

\textbf{3. Generátorfüggvényes ellenőrzés.}

Legyen 
\[
D(x)=\sum_{n=0}^{\infty}d_{n}x^{n}.
\]
A rekurzió generátorfüggvényes alakja: 
\[
D(x)-x=3x\bigl(D(x)\bigr)+2x^{2}D(x)+\sum_{n\ge2}nx^{n}.
\]
Az utolsó sorozat zárt alakja $\frac{x}{(1-x)^{2}}$. Összerendezve:
\[
D(x)\bigl(1-3x-2x^{2}\bigr)=x+\frac{x}{(1-x)^{2}},
\]
\[
D(x)=\frac{x(1-x)^{2}+x}{(1-x)^{2}(1-3x-2x^{2})}=\frac{x(2-2x+x^{2})}{(1-x)^{2}(1-3x-2x^{2})}.
\]
Parciális törtekre bontva és visszafejtve megkapjuk a fenti explicit
$d_{n}$-et. 
\end{solution}
\begin{extraproblem}[Miklós Dóra]
Legyen a $h$ hosszúságú pályán a balra lépés valószínűsége $p$,
a jobbra lépés valószínűsége $q=1-p$. Melyik mezőre kell kezdetben
helyezni a bolyongó pontot, hogy a játék a legtovább tartson? (A bolyongásnak
akkor van vége, ha a pont valamelyik oldalon elhagyja a pályát.) 
\end{extraproblem}

\begin{solution}
Jelölje $L_{i}$ az átlagos lépésszámot, ha a bolyongó pont az $i.$
mezőn van. A feladat $L_{i}$ maximumhelyének meghatározása. Az $i.$
mezőről két irányba léphetünk: vagy az $(i-1).$, vagy az $(i+1).$
mezőre. Ez alapján az alábbi rekurzív összefüggést írhatjuk fel 
\[
L_{i}=p\cdot(1+L_{i-1})+q\cdot(1+L_{i+1}).
\]
Az egyenlet első tagja úgy értelmezhető, hogy a pont $p$ valószínűséggel
az $(i-1).$ mezőre kerül, tehát lép egyet, plusz még annyit, amennyi
átlagosan az $(i-1).$ mezőn állva várható, vagyis $L_{i-1}$-et.
Hasonlóan a második tag a jobbra lépést írja le. Ha a pont a fiktív
mezőkre kerül, a játéknak vége lesz, tehát $L_{0}=L_{h+1}=0.$ Átalakítva
az 
\[
qL_{i+1}=L_{i}-pL_{i-1}-1,\phantom{a}L_{0}=L_{h+1}=0
\]
inhomogén másodrendű rekurziót kapjuk. Ennek megoldásához bevezetjük
a $d_{i}=L_{i}-L_{i-1}$ különbségsorozatot, mely felhasználásával
az inhomogén rekurziót homogén rekurzióvá alakíthatjuk át 
\[
qd_{i+1}=qL_{i+1}-qL_{i}=L_{i}-pL_{i-1}-1-(L_{i-1}-pL_{i-2}-1)=d_{i}-pd_{i-1.}
\]
Ha a különbségsorozat képletét meg tudjuk határozni, akkor $L_{i}$
is kifejezhető, hiszen: 
\begin{align*}
d_{1} & =L_{1}-L_{0}\\
d_{2} & =L_{2}-L_{1}\\
d_{3} & =L_{3}-L_{2}\\
 & \cdots\\
d_{i} & =L_{i}-L_{i-1},\\
\end{align*}
mely egyenleteket összeadva kapjuk, hogy $L_{i}-L_{0}=\sum\limits_{j=1}^{i}d_{j}$
és mivel $L_{0}=0$, ezért $L_{i}=\sum\limits_{j=1}^{i}d_{j}$. A
továbbiakban a $qr^{2}-r+p=0$ karakterisztikus egyenlet vizsgálatát
ismét két részletben végezzük el.

Ha $p=q=\frac{1}{2}$, akkor az egyenlet két gyöke $r_{1,2}=1$, így
a $d$ sorozat $d_{i}=a+b\cdot i$ alakú. Mivel a $d$ sorozat peremfeltételeit
nem ismerjük, ezért az $a$ és $b$ állandókat az $L$ sorozat kezdeti
feltételeinek felhasználásával határozzuk meg. Először is $L_{h+1}=0$,
ezért $L_{h+1}=\sum\limits_{i=1}^{h+1}d_{i}=a\cdot(h+1)+b\cdot\frac{(h+1)(h+2)}{2}=0$.
A másik egyenlet meghatározásához kiszámoljuk a sorozat első néhány
tagját és felhasználjuk $L$ rekurzív képletét: 
\[
L_{0}=0,\phantom{a}L_{1}=d_{1}=a+b,\phantom{a}L_{2}=L_{1}+d_{2}=a+b+a+2b=2a+3b
\]
\[
L_{2}=\frac{L_{1}-pL_{0}-1}{q}=\frac{L_{1}-1}{\frac{1}{2}}=2L_{1}-2,\phantom{i}\textrm{ahonnan}\phantom{i}2a+3b=2a+2b-2.
\]
Így az alábbi egyenletrendszert kapjuk 
\[
\left\{ \begin{array}{l}
a+b\cdot\frac{h+2}{2}=0\\
2a+3b=2a+2b-2,
\end{array}\right.
\]
melynek megoldása $b=-2$ és $a=h+2$. Így $d_{i}=h+2-2i$ és 
\[
L_{i}=\sum\limits_{j=1}^{i}d_{j}=(h+2)i-2\cdot\frac{i(i+1)}{2}=-i^{2}+i+ih=i\cdot(h+1-i).
\]
A maximumot $i=\frac{h+1}{2}$ helyen kapjuk, ami tulajdonképpen a
pálya közepe. Ha $h=2k-1$ alakú, akkor $i=k$ és $L_{k}=k^{2}$ a
maximum lépésszám.

Ha $p\neq q$, akkor legyen $p<q$. Ekkor a karakterisztikus egyenlet
két gyöke $r_{1}=\frac{p}{q}$ és $r_{2}=1$. Legyen $c=\frac{p}{q}$
és ekkor $d_{i}=a\cdot c^{i}+b$ alakba írható. Az egyik feltétel
ismét az $L_{h+1}=\sum\limits_{i=1}^{h+1}$ összefüggésből adódik:
\[
L_{h+1}=a\cdot c\cdot\frac{1-c^{h+1}}{1-c}+b\cdot(h+1)=0.
\]
Továbbá a másik egyenlethez ismét meghatározzuk az első néhány tagot:
\[
L_{0}=0,\phantom{i}L_{1}=d_{1}=ac+b,\phantom{i}L_{2}=L_{1}+d^{2}=ac+b+ac^{2}+b=ac(c+1)+2b.
\]
A rekurzív alakot felhasználva 
\[
L_{2}=\frac{L_{1}-pL_{0}-1}{q}=\frac{L_{1}-1}{q}=\frac{ac+b-1}{q},\phantom{i}\textrm{tehát}\phantom{i}ac(c+1)+2b=\frac{ac+b-1}{q}.
\]
Az így kapott egyenletrendszer 
\[
\left\{ \begin{array}{l}
a\cdot c\cdot\frac{1-c^{h+1}}{1-c}+b\cdot(h+1)=0\\
qac(c+1)+2bq=ac+b-1,
\end{array}\right.
\]
melynek megoldása $b=-\frac{1}{q-p}$, $a=\frac{h+1}{p(1-c^{h+1})}$
és $L_{i}$ általános alakja: 
\[
L_{i}=ib+ac\frac{1-c^{i}}{1-c}=-\frac{i}{q-p}+\frac{c(h+1)}{p(1-c^{h+1)}}\cdot\frac{1-c^{i}}{1-c}=-\frac{i}{q-p}+\frac{h+1}{q-p}\cdot\frac{1-c^{i}}{1-c^{h+1}}.
\]
Például egy $h=10$ hosszúságú pályán, ha $p=\frac{1}{3}$, $q=\frac{2}{3}$,
akkor megközelítőleg az $L_{i}$ szélsőértékhelye 3, valamint a felvett
szélsőérték 20.
\end{solution}
\begin{extraproblem}[Seres Brigitta-Alexandra]
Adott $a>0$ és $p\in\mathbb{N}$, $p\geq2$ szám segítségével értelmezzük
az $(a_{n})_{n\in\mathbb{N}}$ sorozatot a következő módon: 
\[
a_{0}=a,\quad a_{n+1}=a_{n}+\frac{1}{a_{n}^{p-1}+1},\forall n\in\mathbb{N}.
\]
a.) Konvergens-e az $(a_{n})_{n\geq1}$ sorozat?\\
 b.) Számítsd ki a következő határértéket 
\[
\lim\limits_{n\to\infty}\frac{a_{n}}{\sqrt[p]{n}}.
\]
\begin{flushright}
\textit{(EMMV 2013, XI. osztály, 4. feladat)} 
\par\end{flushright}
\end{extraproblem}

\begin{solution}
a.) Tegyük fel, hogy a sorozat konvergens, ekkor létezik $l\in\mathbb{R}$
határték, úgy hogy $\lim\limits_{n\to\infty}{a_{n}}=l$. Így az is
igaz, hogy $\lim\limits_{n\to\infty}{a_{n+1}}=l$. Határértékre térve
a rekurziós összefüggésben, kapjuk hogy 
\[
l=l+\frac{1}{l^{p-1}+1},\quad\text{vagyis}\quad\frac{1}{l^{p-1}+1}=0.
\]
Egy tört akkor és csakis akkor lehet egyenlő nullával, ha a számlálója
nulla, ami esetünkben nem áll fenn, így az összefüggés nem lehet igaz.
Mivel egy hamis összefüggéshez jutottunk abból a feltevésből, hogy
a megadott sorozat konvergens, kijelenthető, hogy $(a_{n})_{n\geq1}$
sorozat nem konvergens.

Mivel $a_{n}>0,\forall n\in\mathbb{N}$ és $p\in\mathbb{N}$, $p\geq2$
esetén $a_{n}^{p-1}+1>0$, így $a_{n+1}-a_{n}=\frac{1}{a_{n}^{p-1}+1}>0$,
vagyis $(a_{n})_{n\geq1}$ szigorúan növekvő sorozat.

Tehát $(a_{n})_{n\geq1}$ sorozat nem konvergens, ugyanakkor szigorúan
növekvő, így létezik határértéke: $\lim\limits_{n\to\infty}{a_{n}}=+\infty.$\\
 b.) A kiszámolandó határértékben a gyökvonás rendje "zavaró", így
azt figyelembe véve kellene egy vele ekvivalens, de könnyebben megközelíthető
alakra hozni a kiszámolandó határértéket. Innen jön az ötlet a következő
átalakításokra (a.)-ból tudjuk, hogy $a_{n}>0,\forall n\in\mathbb{N}$,
így $a_{n}^{p}>0,\forall p\in\mathbb{N},p\geq2$): 
\[
\lim_{n\rightarrow\infty}\frac{a_{n}}{\sqrt[p]{n}}=\lim_{n\rightarrow\infty}\sqrt[p]{\frac{a_{n}^{p}}{n}}=\sqrt[p]{\lim_{n\rightarrow\infty}\frac{a_{n}^{p}}{n}}=\sqrt[p]{s}.
\]
Mivel teljesülnek a Ceasáro--Stolz-tétel feltételei ($b_{n}=n$ sorozat
esetén $(b_{n})_{n\in\mathbb{N}}$ szigorúan növekvő és $\lim_{n\rightarrow\infty}b_{n}=+\infty$),
$s$ határérték kiszámítására alkalmazzuk: 
\[
\begin{aligned}h=\lim_{n\rightarrow\infty} & \frac{a_{n+1}^{p}-a_{n}^{p}}{n+1-n}=\lim_{n\rightarrow\infty}\left[\left(a_{n}+\frac{1}{a_{n}^{p-1}+1}\right)^{p}-a_{n}^{p}\right].\end{aligned}
\]

\vspace{2mm}
 Felhasználva a binomiális tételt, kapjuk hogy 
\[
\begin{aligned}h= & \lim_{n\rightarrow\infty}\left[\sum_{k=0}^{p}C_{p}^{k}a_{n}^{p-k}\frac{1}{\left(a_{n}^{p-1}+1\right)^{k}}-a_{n}^{p}\right]\\
= & \lim_{n\rightarrow\infty}\left[C_{p}^{0}a_{n}^{p-0}\cdot1+\sum_{k=1}^{p}C_{p}^{k}a_{n}^{p-k}\frac{1}{\left(a_{n}^{p-1}+1\right)^{k}}-a_{n}^{p}\right]\\
= & \lim_{n\rightarrow\infty}\sum_{k=1}^{p}C_{p}^{k}a_{n}^{p-k}\frac{1}{\left(a_{n}^{p-1}+1\right)^{k}}.
\end{aligned}
\]

Kibontva a határértékben levő összeget, majd határértékre térve és
felhasználva azt hogy $\lim\limits_{n\to\infty}{a_{n}^{p}}=+\infty,\forall p\in\mathbb{N},p\geq2$,
kapjuk hogy 
\[
\begin{aligned}h= & \lim_{n\rightarrow\infty}\left(C_{p}^{1}\frac{a_{n}^{p-1}}{a_{n}^{p-1}+1}+C_{p}^{2}\frac{a_{n}^{p-2}}{\left(a_{n}^{p-1}+1\right)^{2}}+\ldots+C_{p}^{p}\frac{1}{\left(a_{n}^{p-1}+1\right)^{p}}\right)\\
= & C_{p}^{1}=p.
\end{aligned}
\]
Ekkor létezik a $h$ határérték és $h=p$, így Cezáro--Stolz-tétel
alapján létezik $s$ határérték is, ahol $s=h=p.$ Innen következik,
hogy a keresett határérték 
\[
\lim_{n\rightarrow\infty}\frac{a_{n}}{\sqrt[p]{n}}=\sqrt[p]{p}.
\]
\end{solution}
\begin{extraproblem}[Seres Brigitta-Alexandra]
Az $(a_{n})_{n\in\mathbb{N}^{*}}$ valós számsorozatot az $a_{n+1}+1=a\cdot a_{n}$,
$n\geq1$ és $a_{1}=\frac{a}{a-1}$, $a\in(1,\infty)$, rekurzióval
értelmezzük. 
\begin{enumerate}
\item[a)] Határozzuk meg a sorozat általános tagját! 
\item[b)] Igazoljuk, hogy 
\[
{\sum}_{k=0}^{n}\frac{(a-1)^{2}a_{2k+1}+1}{(a-1)a_{k+1}-1}=\frac{a_{n+1}}{a^{n-1}}\left(a_{n+1}-\frac{2}{a-1}\right)+\frac{2a-1}{(a-1)^{2}},
\]
bármely $n\in\mathbb{N}^{*}$ és $a>1$ esetén! 
\item[c)] Számítsuk ki a következő határértéket: 
\[
\lim_{n\to\infty}\frac{1}{n}\sum_{k=2}^{n+1}\sqrt[k]{a_{k}}.
\]
\end{enumerate}
\begin{flushright}
\textit{( Matlap 2024/03/L:3716)} 
\par\end{flushright}
\end{extraproblem}

\begin{solution}
a.) Az általános tag megsejtéséhez írjuk fel a sorozat néhány tagját:

\[
\begin{aligned}a_{1} & =\frac{a}{a-1}=\frac{a-1+1}{a-1}=1+\frac{1}{a-1}\\
a_{2} & =a\cdot\left(1+\frac{1}{a-1}\right)-1=a+\frac{a}{a-1}-1=a+\frac{a-a+1}{a-1}=a+\frac{1}{a-1}\\
a_{3} & =a\cdot\left(a+\frac{1}{a-1}\right)-1=a^{2}+\frac{a}{a-1}-1=a^{2}+\frac{a-a+1}{a-1}=a^{2}+\frac{1}{a-1}\\
a_{4} & =a\cdot\left(a^{2}+\frac{1}{a-1}\right)-1=a^{3}+\frac{a}{a-1}-1=a^{3}+\frac{a-a+1}{a-1}=a^{3}+\frac{1}{a-1}.
\end{aligned}
\]

A fenti tagok alapján megállapíthatjuk azt a sejtést, hogy a sorozat
általános tagja $a_{n}=a^{n-1}+\frac{1}{a-1}$ bármely $n\geq1$ esetén.
Ezt a matematikai indukció módszerével igazoljuk: $P(n):a_{n}=a^{n-1}+\frac{1}{a-1}$,
$\forall n\geq1$.

A kiszámított tagok alapján megállapíthatjuk, hogy $P(1)$, $P(2)$,
$P(3)$ és $P(4)$ igazak. Feltételezzük, hogy tetszőleges $k\geq1$
esetén $P(k)$ igaz és bizonyítsuk be, hogy $P(k+1)$ is teljesülni
fog. A rekurziós képletet használva

\[
\begin{aligned}a_{k+1} & =a\cdot a_{k}-1=a\cdot\left(a^{k-1}+\frac{1}{a-1}\right)-1\\
 & =a^{k}+\frac{a}{a-1}-1=a^{k}+\frac{a-a+1}{a-1}=a^{k}+\frac{1}{a-1}.
\end{aligned}
\]

Tehát $P(k+1)$ is igaz. A matematikai indukció elve alapján kijelenthető,
hogy a sorozat általános tagja $a_{n}=a^{n-1}+\frac{1}{a-1}$ bármely
$n\geq1$ esetén.\\

b.) Behelyettesítjük az a) alpontban megkapott általános képletet,
egyszerűsítünk, használjuk a mértani haladvány összegképletét, majd
ezt átrendezve megkapjuk a kért alakot: 
\begin{align*}
\sum\limits_{k=0}^{n}\frac{(a-1)^{2}a_{2k+1}+1}{(a-1)a_{k+1}-1} & =\sum\limits_{k=0}^{n}\frac{(a-1)^{2}\left(a^{2k}+\frac{1}{a-1}\right)+1}{(a-1)\left(a^{k}+\frac{1}{a-1}\right)-1}\\
 & =\sum\limits_{k=0}^{n}\frac{(a-1)^{2}a^{2k}+a-1+1}{(a-1)a^{k}+1-1}\\
 & =\sum\limits_{k=0}^{n}\frac{(a-1)^{2}a^{2k}+a}{(a-1)a^{k}}=(a-1)\sum\limits_{k=0}^{n}a^{k}+\frac{a}{a-1}\sum\limits_{k=0}^{n}\frac{1}{a^{k}}\\
 & =(a-1)\cdot\frac{a^{n+1}-1}{a-1}+\frac{a}{a-1}\cdot\frac{\frac{1}{a^{n+1}}-1}{\frac{1}{a}-1}\\
 & =a^{n+1}-1+\frac{a}{a-1}\cdot\frac{1-a^{n+1}}{a^{n+1}}\cdot\frac{a}{1-a}\\
 & =a^{n+1}-1-\frac{a^{2}}{(a-1)^{2}}\left(\frac{1}{a^{n+1}}-1\right)\\
 & =\frac{1}{a^{n-1}}\left(a^{2n}-\frac{1}{(a-1)^{2}}\right)+\frac{a^{2}-(a-1)^{2}}{(a-1)^{2}}\\
 & =\frac{1}{a^{n-1}}\left(a^{n}+\frac{1}{a-1}\right)\left(a^{n}-\frac{1}{a-1}\right)+\frac{2a-1}{(a-1)^{2}}\\
 & =\frac{a_{n+1}}{a^{n-1}}\left(a_{n+1}-\frac{2}{a-1}\right)+\frac{2a-1}{(a-1)^{2}}.
\end{align*}
\\
 c.) A ${\displaystyle \lim\limits_{n\to\infty}\frac{1}{n}\sum\limits_{k=2}^{n+1}\sqrt[k]{a_{k}}}$
határértéket a Cesaro--Stolz-tétel alkalmazásával számoljuk ki, amit
megtehetünk, mert a $b_{n}=n$, $n\geq1$ sorozat szigorúan növekvő
és $\lim\limits_{n\to\infty}n=+\infty$: 
\[
\begin{aligned}\lim\limits_{n\to\infty}\frac{1}{n}\sum\limits_{k=2}^{n+1}\sqrt[k]{a_{k}} & =\lim\limits_{n\to\infty}\frac{\sum\limits_{k=2}^{n+1}\sqrt[k]{a_{k}}}{n}=\lim\limits_{n\to\infty}\frac{\sum\limits_{k=2}^{n+2}\sqrt[k]{a_{k}}-\sum\limits_{k=2}^{n+1}\sqrt[k]{a_{k}}}{n+1-n}\\
 & =\lim\limits_{n\to\infty}\sqrt[n+2]{a_{n+2}}.
\end{aligned}
\]

A kapott határértéket a Cauchy--d'Alambert-tétellel ($a_{n+1}>0,\forall n\geq1$)
számoljuk tovább behelyettesítve az általános tag képletét: 
\[
\begin{aligned}\lim\limits_{n\to\infty}\sqrt[n+2]{a_{n+2}}= & \lim\limits_{n\to\infty}\frac{a_{n+3}}{a_{n+2}}=\lim\limits_{n\to\infty}\frac{a^{n+2}+\frac{1}{a-1}}{a^{n+1}+\frac{1}{a-1}}=\lim\limits_{n\to\infty}\frac{a+\frac{1}{a-1}\cdot\frac{1}{a^{n+1}}}{1+\frac{1}{a-1}\cdot\frac{1}{a^{n+1}}}\\
= & \frac{a+0}{1+0}=a.
\end{aligned}
\]
Tehát a keresett határérték 
\[
\lim\limits_{n\to\infty}\frac{1}{n}\sum\limits_{k=2}^{n+1}\sqrt[k]{a_{k}}=a.
\]
\end{solution}
\begin{extraproblem}[Gergely Verona, Száfta Antal]
Hány olyan $\{a,b,c\}$ három különböző pozitív egész számot tartalmazó
halmaz létezik, amelyre teljesül, hogy: 
\[
a\cdot b\cdot c=11\cdot21\cdot31\cdot41\cdot51\cdot61
\]
\end{extraproblem}

\begin{solution}
Először is faktorizáljuk az adott számokat: 
\begin{align*}
21 & =3\cdot7\\
51 & =3\cdot17
\end{align*}

Így a teljes szorzat prímtényezőkre bontva: 
\[
3\cdot3\cdot7\cdot11\cdot17\cdot31\cdot41\cdot61
\]

Feladatunk az, hogy e 8 tényezőt három nem üres halmazra osszuk úgy,
hogy az azok szorzataként kapott három szám különböző legyen.

Egy ilyen felosztás megfelel annak, hogy a 8 tényezőt három nem üres
részhalmazba rendezzük. A nem üres halmazokra való felosztások számát
a másodfajú Stirling-számok adják, azaz: 
\[
S(8,3)=966
\]

De mivel két azonos tényezőnk is van (a két darab 3-as), bizonyos
eseteket többször számolunk. Akkor lép fel duplikáció, amikor a két
3-as külön halmazba kerül. Ezt komplementer módon számoljuk: 
\[
\text{Túlbecslés: }\frac{S(8,3)-S(7,3)-1}{2}=\frac{966-301-1}{2}=332
\]

Tehát a három nem üres halmaz esetében helyes felosztások száma: 
\[
966-332=634
\]

Hasonlóan vizsgáljuk azokat az eseteket is, amikor két vagy csak egy
halmaz nem üres. Ezekhez a Stirling-számok: 
\begin{align*}
S(8,2) & =266, & S(7,2) & =63\\
S(8,1) & =1, & S(7,1) & =1
\end{align*}

Így a korrekciók után ezek az esetek: 
\begin{align*}
S(8,2)-\frac{S(8,2)-S(7,2)}{2} & =266-\frac{203}{2}=164.5\\
S(8,1)-\frac{S(8,1)-S(7,1)}{2} & =1-0=1
\end{align*}

A teljes lehetséges felosztások száma tehát: 
\[
\frac{S(8,3)+S(8,2)+S(8,1)+S(7,3)+S(7,2)+S(7,1)+1}{2}=
\]
\[
\frac{966+266+1+301+63+1+1}{2}=\frac{1599}{2}=730
\]

Ez azonban tartalmaz néhány olyan esetet is, ahol az $a,b,c$ számok
nem különbözőek, például $(1,1,x)$ vagy $(3,3,x)$. Ezek száma 2,
így a helyes válasz: 
\[
\boxed{728}
\]
\end{solution}

