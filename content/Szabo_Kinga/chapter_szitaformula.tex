
\chapter{A szitaformula (a bennfoglalás és kizárás elve)}\label{chap:szita}
\begin{description}
{\large \item [{Szerző:}] Szabó Kinga (Didaktikai mesteri -- Matematika, II. év)}
\end{description}

\begin{problem}
Egy osztály tanulói közül matematika óra elején kiküldünk 2 diákot.
A többiek karamelles és cseresznyés cukorkákat kapnak. Választhatnak,
hogy ki milyent szeretne venni. Egy gyerek 1 fajta cukorkából, csak
egyet vehet, de lesz olyan diák is, aki egyik cukorkából sem kér.
Visszajön a két diák és el kell dönteniük, hogy összesen hány gyerek
kapott cukorkát. Ha 15-en vettek karamelleset, 20-an cseresznyéset,
és mindkét cukorkából 5-ön vettek, akkor hány gyerek vett el a cukorkából?
\vspace{3mm}

\textbf{Megjegyzés:} Érdemes megpróbálni élőben, cukorkákkal, a gyerekek
felállhatnak és halmazokat alkothatnak. Különböző színű kötelekkel
el el lehet különíteni a halmazokat. \vspace{3mm}
\end{problem}
\begin{solution}
Köröket rajzolunk a táblára és rajz alapján megoldjuk. Megpróbáljuk,
halmazokkal, matematikailag leírni a feladatot.

Legyen a karamelles cukorkát választó gyerekek halmaza a K halmaz,
és a cseresznyéset választók halmaza pedig a C.

$|K|=15,|C|=20$ és $|K\cap C|=5$. Ekkor $|K\cup C|=|K|+|C|-|K\cap C|=15+20-5=30$.-Szitaformula
\end{solution}
\subsubsection*{Szitaformula (2 halmaz esetén):}

$A$ és $B$ nem üres halmazok, ekkor: 
\[
|A\cup B|=|A|+|B|-|A\cap B|
\]
\begin{problem}
Vizsgáljuk az előző feladatot ha lesz egy harmadik,
citromos izű cukorka is. Citromos cukorkát 4 gyerek vesz el, citromos
és cseresznyés cukorkát 2, citromos és karamelles cukorkát 3 míg citromos,
karamelles, és cseresznyés cukorkát 1 gyerek vesz el. \vspace{5mm}
\end{problem}
\begin{solution}
A citromos cukorkát jelöljük I-vel: $|I|=4,|I\cap C|=2,|I\cap K|=3,|I\cap K\cap C|=1.$
\[
|I\cup K\cup C|=|I|+|K|+|C|-|I\cap K|-|I\cap C|-|K\cap C|+|I\cap K\cap C|=4+15+20-3-2-5+1=30
\]
\end{solution}

\begin{theorem}{thm:altalanos-szita}
(Szita-formula általánosan): $A_{1},A_{2},\dots,A_{n}$ véges nem
üres halmazokra az alábbi formula érvényes:

\begin{equation}
\left|\bigcup_{i=1}^{n}A_{i}\right|=\sum_{i=1}^{n}|A_{i}|-\sum_{1\leq i<j\leq n}|A_{i}\cap A_{j}|+\sum_{1\leq i<j<k\leq n}|A_{i}\cap A_{j}\cap A_{k}|-\dots+(-1)^{n+1}\left|\bigcap_{i=1}^{n}A_{i}\right|
\end{equation}

\begin{equation}
\left|\bigcap_{i=1}^{n}A_{i}\right|=\sum_{i=1}^{n}|A_{i}|-\sum_{1\leq i<j\leq n}|A_{i}\cup A_{j}|+\sum_{1\leq i<j<k\leq n}|A_{i}\cup A_{j}\cup A_{k}|-\dots+(-1)^{n+1}\left|\bigcup_{i=1}^{n}A_{i}\right|
\end{equation}

ahol:

\[
\bigcup_{i=1}^{n}A_{i}=A_{1}\cup A_{2}\cup\dots\cup A_{n},\quad\bigcap_{i=1}^{n}A_{i}=A_{1}\cap A_{2}\cap\dots\cap A_{n}.
\]
\end{theorem}

\begin{proof}
$n=2$ esetén igaz a következő képlet:

\begin{equation}
|A_{1}\cup A_{2}|=|A_{1}|+|A_{2}|-|A_{1}\cap A_{2}|.
\end{equation}

Feltételezzük, hogy az összefüggés fennáll $n-1$ halmazra:

\begin{equation}
\left|\bigcup_{i=1}^{n-1}A_{i}\right|=\sum_{i=1}^{n-1}|A_{i}|-\sum_{1\leq i<j\leq n-1}|A_{i}\cap A_{j}|+\dots+(-1)^{n}\left|\bigcap_{i=1}^{n-1}A_{i}\right|.
\end{equation}

Az alábbi formulát használjuk:

\begin{equation}
\bigcup_{i=1}^{n}A_{i}=\left(\bigcup_{i=1}^{n-1}A_{i}\right)\cup A_{n}.
\end{equation}

Az $(n=2)$ esetén igaz képlet alkalmazásával és az indukciós feltevés
alapján:

\begin{align*}
\left|\bigcup_{i=1}^{n}A_{i}\right| & =\left(\bigcup_{i=1}^{n-1}A_{i}\right)\cup A_{n}=\left|\bigcup_{i=1}^{n-1}A_{i}\right|+|A_{n}|-\left|\left(\bigcup_{i=1}^{n-1}A_{i}\right)\cap A_{n}\right|=\\
 & =\sum_{i=1}^{n-1}|A_{i}|-\sum_{1\leq i<j\leq n-1}|A_{i}\cap A_{j}|+\dots+(-1)^{n}\left|\bigcap_{i=1}^{n-1}A_{i}\right|\\
 & \quad+|A_{n}|-\left|\left(\bigcup_{i=1}^{n-1}A_{i}\right)\cap A_{n}\right|.
\end{align*}

Ez éppen azt mutatja, hogy az állítás $n$-re is teljesül, tehát a
bizonyítás befejeződött.

Az iménti számolást még a következő alakban írhatjuk fel:

\begin{equation}
\left(\bigcup_{i=1}^{n-1}A_{i}\right)\cap A_{n}=\bigcup_{i=1}^{n-1}(A_{i}\cap A_{n}).
\end{equation}

Ezt behelyettesítve kapjuk a kért (5.0.1) összefüggést.

A (5.0.2) bizonyítása teljesen hasonló módon történik, elégséges felcserélni
az $\cup$ és $\cap$ műveleteket. 
\end{proof}
\begin{theorem}{thm:egyenlotlensegek}
Egyenlőtlenségek:

Adott $A_{1},A_{2},\dots,A_{n}$ véges, nem üres halmaz esetén, ha
$m$ páros és $m<n$, akkor:

\begin{equation}
\sum_{i=1}^{m}|A_{i}|-\sum_{1\leq i<j\leq m}|A_{i}\cap A_{j}|+\dots+(-1)^{m+1}\left|\bigcap_{i=1}^{m}A_{i}\right|\leq\left|\bigcup_{i=1}^{n}A_{i}\right|.
\end{equation}

Ha pedig $m$ páratlan és $m<n$, akkor:

\begin{equation}
\sum_{i=1}^{m}|A_{i}|-\sum_{1\leq i<j\leq m}|A_{i}\cap A_{j}|+\dots+(-1)^{m+1}\left|\bigcap_{i=1}^{m}A_{i}\right|\geq\left|\bigcup_{i=1}^{n}A_{i}\right|.
\end{equation}
\end{theorem}

A bizonyítás hasonló az előző tétel bizonyításához.
\begin{theorem}{thm:logikai-szita}
(Logikai szita):

Legyen $S$ egy $N$ elemű halmaz, és legyenek $A_{1},A_{2},\dots,A_{k}$
nem szükségképpen különböző, nem üres részhalmazai $S$-nek. Bármely
$M\subseteq\{1,2,\dots,k\}$ esetén legyen:

\begin{equation}
N(M)=\left|\{s\in S\mid s\in\bigcap_{i\in M}A_{i}\}\right|,
\end{equation}

\begin{equation}
N_{j}=\sum_{|M|=j}N(M),\quad0\leq j\leq k.
\end{equation}

Azon $S$-beli elemek száma, amelyek egyetlen $A_{i}$-ben sem találhatók:

\begin{equation}
N-N_{1}+N_{2}-N_{3}+\dots+(-1)^{k}N_{k}.
\end{equation}
\end{theorem}

\begin{proof}
Két esetet különböztetünk meg: 
\begin{itemize}
\item Ha $x\in S$ és $x$ nem eleme egyetlen $A_{i}$-nek sem, akkor az
$x$-et csak egyszer az $N$-ben számoljuk. 
\item Ha $x\in S$ és $x$ pontosan $\ell$ darab $A_{i}$ halmazban található,
akkor az $N$-ben egyszer, $N_{1}$-ben $\binom{\ell}{1}$-szer, $N_{2}$-ben
$\binom{\ell}{2}$-szer, ..., $N_{\ell}$-ben $\binom{\ell}{\ell}$-szer
szerepel. Így az $x$ súlya: 
\begin{equation}
1-\binom{\ell}{1}+\binom{\ell}{2}-\dots+(-1)^{\ell}\binom{\ell}{\ell}=(1-1)^{\ell}=0.
\end{equation}
\end{itemize}
Ez azt jelenti, hogy $x$-et nem számoltuk az $S\setminus\left(\bigcup_{i=1}^{k}A_{i}\right)$
elemei közé. 
\end{proof}
\vspace{5mm}

\begin{problem}
Tegyük fel, hogy van egy $S$ halmazunk, amely 10 elemből áll: 
\[
S=\{1,2,3,4,5,6,7,8,9,10\}.
\]
Három részhalmazt definiálunk: $A_{1}$ tartalmazza a 2-vel osztható
számokat, $A_{2}$ tartalmazza a 3-mal osztható számokat, $A_{3}$
tartalmazza az 5-tel osztható számokat. Hány olyan elem van $S$-ben,
amely egyik $A_{i}$-ben sem szerepel? \vspace{5mm}
\end{problem}
\begin{solution} 
\begin{itemize}
\item $A_{1}$ tartalmazza a 2-vel osztható számokat: 
\[
A_{1}=\{2,4,6,8,10\}
\]
\item $A_{2}$ tartalmazza a 3-mal osztható számokat: 
\[
A_{2}=\{3,6,9\}
\]
\item $A_{3}$ tartalmazza az 5-tel osztható számokat: 
\[
A_{3}=\{5,10\}
\]
\end{itemize}
\[
|A_{1}|=5,\quad|A_{2}|=3,\quad|A_{3}|=2
\]

\[
A_{1}\cap A_{2}=\{6\}\Rightarrow|A_{1}\cap A_{2}|=1
\]

\[
A_{1}\cap A_{3}=\{10\}\Rightarrow|A_{1}\cap A_{3}|=1
\]

\[
A_{2}\cap A_{3}=\emptyset\Rightarrow|A_{2}\cap A_{3}|=0
\]

\[
A_{1}\cap A_{2}\cap A_{3}=\emptyset\Rightarrow|A_{1}\cap A_{2}\cap A_{3}|0
\]

A logikai szita tétel szerint az elemek száma, amelyek egyik halmazban
sem szerepelnek:

\[
|S-(A_{1}\cup A_{2}\cup A_{3})|=N-N_{1}+N_{2}-N_{3}=
\]

\[
=|S|-(|A_{1}|+|A_{2}|+|A_{3}|)+(|A_{1}\cap A_{2}|+|A_{1}\cap A_{3}|+|A_{2}\cap A_{3}|)-|A_{1}\cap A_{2}\cap A_{3}|=
\]

\[
=10-(5+3+2)+(1+1+0)-0=
\]

\[
=10-10+2-0=2
\]
\end{solution}

\begin{problem}[Szürjektív függvények száma]
Legyenek $X=\{x_{1},x_{2},\ldots,x_{m}\},Y=\{y_{1},y_{2},\ldots,y_{n}\}$
halmazok, $m\geq n.$ Határozd meg az $f:X\rightarrow Y$ szürjektív
függvények számát.
\end{problem}
\begin{solution}
Ha $f:X\rightarrow Y$ szürjektív $\Rightarrow\forall y\in Y\exists x\in X:f(x)=y$

Legyen $M=Y^{X},M=Y^{X}=\{f|f:X\rightarrow Y\}$\\

\[
A_{i}=\{f|f:X\rightarrow Y,Y_{i}\neq Imf\},1\leq i\leq n.
\]

\[
A_{1},A_{2},\ldots,A_{n}\subseteq M.
\]

\[
S(m,n)=|M-(\bigcap_{i=1}^{n}A_{i})|=\sum_{I\subseteq\{1,n\}}(-1)^{|I|}|A_{I}|^{i=1}=
\]
\[
=M-\sum_{i=1}^{n}|A_{i}|+\sum_{1\leq i<j\leq n}|A_{i}\cap A_{j}|-\sum_{1\leq i<j<k\leq n}|A_{i}\cap A_{j}\cap A_{k}|+\ldots+(-1)^{n}|A_{1}\cap A_{2}\cap\ldots\cap A_{n}|=
\]
\[
=n^{m}-C_{n}^{1}\cdot(n-1)^{m}+C_{n}^{2}\cdot(n-2)^{m}+\ldots+(-1)^{n-1}\cdot C_{n}^{n-1}\cdot1^{m}=\sum_{k=0}^{m}(-1)^{k}\cdot C_{n}^{k}\cdot(n-k)^{m}
\]
\end{solution}

\section*{Házi feladatok}
\begin{problem}
Egy osztályban angolul, franciául, németül rendre 16, 6, 7 tanuló
tanul. Négyen angolul is és németül is, hárman németül és franciául,
ketten franciául és angolul is tanulnak. Egy tanul mindhárom nyelvet
tanulja. Hányan vannak az osztályban, ha mindenki tanul legalább egy
nyelvet? Hányan tanulnak csak angolul?\\
 
\end{problem}

\begin{solution}
Legyen $A$ az angolul, $F$ a franciául, és $N$ a németül tanuló
tanulók halmaza.

\[
|A|=16,|F|=6,|N|=7,|A\cap N|=4,|N\cap F|=3,|F\cap A|=2,|A\cap F\cap N|=1
\]

Tanulók száma szitaformula segítségével: 
\[
|A\cup F\cup N|=|A|+|F|+|N|-|A\cap N|-|N\cap F|-|F\cap A|+|A\cap F\cap N|=16+6+7-4-3-2+1=21
\]
Csak angolul tanuló diákok száma: 
\[
|A|-|A\cap F|-|A\cap N|+|A\cap F\cap N|=16-2-4+1=11
\]
\end{solution}
\begin{problem}
Egy felmérés során 100 embert megkérdeztek, hogy milyen forrásból
szerzik a híreket. A következő eredmény született: tévéből 65, rádióból:
38, újságból 39, tévéből és rádióból: 27, tévéből és újságból : 20,
rádióból és újságból: 9, tévéből, rádióból és újságból: 6. Hányan
nem szerzik a híreket egyik forrásból sem? Hányan vannak, akik csupán
egy forrásból szerzik a híreket a három közül?\\
 
\end{problem}

\begin{solution}
Legyen $T$ a tévéből, $R$ a rádióból, és $U$ az újságból híreket
szerzők halmaza.

\[
|T|=65,|R|=38,|U|=39,|T\cap R|=27,|T\cap U|=20,|R\cap U|=9,|T\cap R\cap U|=6
\]

\[
|T\cup R\cup U|=|T|+|R|+|U|-|T\cap R|-|T\cap U|-|R\cap U|+|T\cap R\cap U|=65+38+39-27-20-9+6=92
\]

\[
100-92=8
\]

Csak a tévéből szerzik a híreket: 
\[
|T|-|T\cap R|-|T\cap U|+|T\cap R\cap U|=65-27-20+6=24
\]

Csak a rádióból szerzik a híreket: 
\[
|R|-|T\cap R|-|R\cap U|+|T\cap R\cap U|=38-27-9+6=8
\]

Csak az újságból szerzik a híreket: 
\[
|U|-|T\cap U|-|R\cap U|+|T\cap R\cap U|=39-20-9+6=16
\]
\end{solution}
\begin{problem}
Egy sportegyesületben úszni, futni és kerékpározni tanulnak a gyerekek.
Úszni 18-an, futni 12-en, kerékpározni 10-en tanulnak. Öten úsznak
és futnak is, négyen futnak és kerékpároznak is, hárman pedig úsznak
és kerékpároznak is. Két gyerek mindhárom sportot űzi. Hány gyerek
jár az egyesületbe, ha mindenki legalább egy sportot tanul? Hányan
tanulnak csak úszni? 
\end{problem}

\begin{solution}
Legyen $U$ az úszni, $F$ a futni, $K$ a kerékpározó gyerekek száma.

\[
|U|=18,|F|=12,|K|=10,|U\cap F|=5,|F\cap K|=4,|U\cap K|=3,|U\cap F\cap K|=2
\]

\[
|U\cup F\cup K|=|U|+|F|+|K|-|U\cap F|-|U\cap K|-|F\cap K|+|U\cap F\cap K|=18+12+10-5-4-3+2=30
\]

Tehát $30$ gyerek jár a sportegyesületbe.\\

Csak úszni tanuló gyerekek: 
\[
|U|-|U\cap F|-|U\cap K|+|U\cap F\cap K|=18-4-3+2=13
\]
\end{solution}
\begin{problem}
Hányféleképpen ültethetünk le egy sorba három angolt három franciát
és három törököt úgy, hogy három azonos nemzetiségű ne üljön egymás
mellé?
\end{problem}

\begin{solution}
\[
9!-C_{3}^{1}\cdot7!\cdot3!+C_{3}^{2}\cdot5!\cdot3!\cdot3!-C_{3}^{3}\cdot3!\cdot3!\cdot3!\cdot3!=362880-90720+12960-1296=283824
\]
\end{solution}
\begin{problem}
Aladár levelet írt 10 barátjának. Pont mikor borítékba akarta rakni
a leveleket elaludt a villany. A leveleket ekkor véletlenszerűen rakta
be a borítékokba. Hányféle olyan elrendezés van, hogy senki ne a neki
szóló levelet kapja meg?\\
 
\end{problem}

\begin{solution}
\[
10!-C_{10}^{1}\cdot9!+C_{10}^{2}\cdot8!-C_{10}^{3}\cdot7!+C_{10}^{4}\cdot6!-C_{10}^{5}\cdot5!+C_{10}^{6}\cdot4!-C_{10}^{7}\cdot3!+C_{10}^{8}\cdot2!-C_{10}^{9}\cdot1!+C_{10}^{10}\cdot1!=
\]
\[
=10!-10!+45\cdot8!-15\cdot8!+30\cdot7!-6\cdot7!+210\cdot4!-30\cdot4!+45\cdot2-5\cdot2+1=
\]
\[
=30\cdot8!+24\cdot7!+180\cdot4!+81=1334961
\]
\end{solution}
\begin{problem}
Hányféleképpen ülhet le 8 házaspár egy kerekasztal köré úgy, hogy
minden nő két férfi közé ül, de egyik nő sem ül a férje jobboldalára?\\
 
\end{problem}

\begin{solution}
Az összes ültetések száma: $7!\cdot8!$\\

Ha legalább egy hölgy a férje jobbján ül: $C_{8}^{1}\cdot7!\cdot7!.$\\

És így tovább... ekkor a helyes ültetések száma szitaformula alapján
az alábbi módon számolható:

\[
7!\cdot8!-C_{8}^{1}\cdot7!\cdot7!+C_{8}^{2}\cdot7!\cdot6!-C_{8}^{3}\cdot7!\cdot5!+C_{8}^{4}\cdot7!\cdot4!-C_{8}^{5}\cdot7!\cdot3!+C_{8}^{6}\cdot7!\cdot2!-C_{8}^{7}\cdot7!\cdot1!+C_{8}^{8}\cdot7!\cdot0!=
\]

\[
7!\cdot(8!-8!+4\cdot7!-1120+1680-336+56-8+1)=7!\cdot20433
\]
\end{solution}
\begin{problem}
Hányféleképpen tudunk eltenni 20 különböző papírlapot egy piros, egy
kék, egy zöld és egy sárga irattartóba úgy, hogy mindegyik irattartóba
kerüljön legalább egy papír és a dosszién belül a sorrend közömbös?
\end{problem}

\begin{solution}
Az összes elrendezések száma: $20^{4},$ hiszen minden papírlap esetén
$4$ irattartó közül választhatunk.\\
 Az olyan elrendezések száma, ahol (legalább) egy irattartó üresen
marad: $C_{4}^{1}\cdot20^{3}.$\\
 Az olyan elrendezések száma, ahol (legalább) két irattartó üresen
marad: $C_{4}^{2}\cdot20^{2}.$\\
 Az olyan elrendezések száma, ahol három irattartó üresen marad: $C_{4}^{3}\cdot20.$\\

Így az összes helyes elrendezések száma: 
\[
20^{4}-C_{4}^{1}\cdot20^{3}+C_{4}^{2}\cdot20^{2}-C_{4}^{3}\cdot20=160000-32000+2400-80=130320
\]
\end{solution}
\begin{problem}
Egy zsákban 10 pár cipő van. Hányféleképpen vehetünk ki belőle 6 darabot
úgy, hogy ne legyen köztük egy pár sem?\\
\end{problem}

\begin{solution}
Logikai szitaformulával:

\[
C_{20}^{6}-C_{10}^{1}\cdot C_{18}^{4}+C_{10}^{2}\cdot C_{16}^{2}-C_{10}^{3}\cdot C_{14}^{0}=13440
\]
\end{solution}

\section*{Nehezebb feladatok}
\begin{extraproblem}[Szabó Kinga]
Az osztályba $n$ gyerek jár. Karácsonykor kihúzzák egymás nevét:
mindenki pontosan egyvalakiét. Akkor sikeres egy húzás, ha mindenkinél
van cetli, és senki sem saját magát húzta. Mennyi a lehetséges sikeres
húzások száma?
\end{extraproblem}

\begin{solution}
Megjegyzés: A húzásra gondolhatunk úgy, hogy az osztálynapló sorrendjében
minden diáknak van egy sorszáma, és minden ilyen sorszámhoz azt a
sorszámot rendeljük, akit húzott. Ez a számok egy permutációját adja.
A fixpont jelentené azt, hogy valaki önmagát húzta.\\

1. lépés: Határozzuk meg az összes esetet ($H$ halmaz) és a rossz
esetek kategóriáit ($A_{i}$).\\

Az összes kiosztások száma $|H|=n!$, ha a sikertelen húzásokat is
számoljuk. Ez a szorzási szabályból következik: Az első gyerek $n$-féle
cetlit húzhat, a következő már csak $n-1$-et, és egyesével fogynak
a kihúzható cetlik minden egyéni húzás után. A rossz húzás jellemzője,
hogy van olyan gyerek, aki magát húzta. Jelölje $A_{i}$ az olyan
cetlikiosztások eseteit (vagyis az esetek halmazát) amikor az $i$.
gyerek önmagát húzta. Így annyiféle rossz eset fordul elő, ahányan
vannak az osztályban.\\

Jelölje $H$ az összes kiosztások halmazát. Ezzel az 1. lépést elvégeztük.
Valóban szükséges lesz szitálni, mivel könnyű látni, hogy vannak olyan
kiosztások amik többféle szempontból is rosszak, legalább két $A_{i}$
$A_{j}$ halmazban is számolnánk őket, de mi minden rossz esetet csak
egyszer akarunk az összes esetből levonni, vagyis $H-(A_{1}\cup\dots A_{n})$
méretére vagyunk kíváncsiak.\\

2. lépés: Határozzuk meg az $|A_{i}|,|A_{i}\cap A_{j}|$, illetve
általában a $k$-s metszetek elemszámait. $|A_{i}|=(n-1)!$, hiszen
ha az $i$. diák a saját cetlijét húzza, a többiek szabadon húzhatnak
a maradék $n-1$ cetli közül.\\

Hasonlóan bármelyik$A_{i}\cap A_{j}$ kettes metszet elemszáma $(n-2)!$,
hiszen az $i$. és $j$. diák cetlijének kivétele után szabadon oszthatunk
cetliket. Ezt általánosíthatjuk: ha rögzített $k$ ember (pl. $1,2,\dots,k$)
önmagát húzza, a többiekről pedig nincsen megkötésünk, akkor egy $k$-s
metszet elemszámát kell megszámolnunk $(A_{1}\cap A_{2}\cap\dots\cap A_{k})$,
és erre a $(n-k)!$ eredmény adódik.\\

3. lépés: Írjuk fel a szita-formulát. Eszerint a sikeres húzások keresett
száma: 
\[
|H-(A_{1}\cup\dots\cup A_{n})|=
\]
\[
=n!-C_{n}^{1}\cdot(n-1)!+C_{n}^{2}\cdot(n-2)!-\dots+(-1)^{k}\cdot C_{n}^{k}\cdot(n-k)!\pm\dots+(-1)^{n}\cdot C_{n}^{n}.
\]
\end{solution}
\begin{extraproblem}[Csurka-Molnár Hanna]
Határozzuk meg azoknak az $500$-nál nem nagyobb természetes számoknak
a számát, amelyek oszthatóak valamely $1$-nél nagyobb számjeggyel! 
\end{extraproblem}

\begin{solution}
Ha egy szám osztható egy összetett számmal, akkor osztható a prímtényezős
felbontásában szereplő prímszámokkal is. Ezért az egynél nagyobb számjegyek
közül elegendő a prímeket figyelembe vennünk a feladat megoldásánál,
mert ha egy összetett számjeggyel osztható a szám, akkor osztható
lesz valamelyik prímszámmal is.

Tehát az $500$-nál nem nagyobb, $2,3,5$, vagy $7$-el osztható természetes
számokat szeretnénk megszámolni. Tekintsük azon $A_{i}$ halmazokat,
amelyek az $i$-vel osztható, $500$-nál kisebb számokat tartalmazzák,
ahol $i\in\{2,3,5,7\}$. 
\begin{align*}
|A_{i}| & =\lfloor\frac{500}{i}\rfloor+1\\
|A_{i}\cap A_{j}| & =\lfloor\frac{500}{i\cdot j}\rfloor+1\\
|A_{i}\cap A_{j}\cap A_{k}| & =\lfloor\frac{500}{i\cdot j\cdot k}\rfloor+1\\
|A_{i}\cap A_{j}\cap A_{k}\cap A_{l}| & =\lfloor\frac{500}{i\cdot j\cdot k\cdot l}\rfloor+1\\
\end{align*}
ahol $i,j,k,l$ páronként különböző egyjegyű prímszámok.

A logikai szitaformula alapján az $500$-nál kisebb természetes számoknak
a száma, amelyek oszthatóak valamely $1$-nél nagyobb számjeggyel:
\begin{align*}
N & =(\lfloor\frac{500}{2}\rfloor+\lfloor\frac{500}{3}\rfloor+\lfloor\frac{500}{5}\rfloor+\lfloor\frac{500}{7}\rfloor+4)\\
 & -(\lfloor\frac{500}{2\cdot3}\rfloor+\lfloor\frac{500}{2\cdot5}\rfloor+\lfloor\frac{500}{2\cdot7}\rfloor+\lfloor\frac{500}{3\cdot5}\rfloor+\lfloor\frac{500}{3\cdot7}\rfloor+\lfloor\frac{500}{5\cdot7}\rfloor+6)\\
 & +(\lfloor\frac{500}{2\cdot3\cdot5}\rfloor+\lfloor\frac{500}{2\cdot3\cdot7}\rfloor+\lfloor\frac{500}{2\cdot5\cdot7}\rfloor+\lfloor\frac{500}{3\cdot5\cdot7}\rfloor+4)-(\lfloor\frac{500}{2\cdot3\cdot5\cdot7}\rfloor+1)\\
 & =(250+166+100+71+4)-(83+50+35+33+23+14+6)+(16+11+7+4+4)-3\\
 & =591-244+42-3\\
 & =386
\end{align*}
\end{solution}
\begin{extraproblem}[Czofa Vivien]
Adott az $\{1,2,\dots,100\}$ halmaz. Hány szám osztható 2-vel vagy
3-mal vagy 5-tel? 
\end{extraproblem}

\begin{solution}
A halmazban összesen 100 darab szám van. Három halmazra kell használni
a \textbf{szitaformulát}. Legyen az $A=$ 2-vel osztható számok, $B=$
3-mal osztható számok, $C=$ 5-tel osztható számok.

\[
|A\cup B\cup C|=|A|+|B|+|C|-|A\cap B|-|A\cap C|-|B\cap C|+|A\cap B\cap C|
\]

vagyis 
\[
|A|=|\{2,4,6,\dots,100\}|=\left\lfloor \frac{100}{2}\right\rfloor =50
\]
\[
|B|=|\{3,6,9,\dots,99\}|=\left\lfloor \frac{100}{3}\right\rfloor =33
\]
\[
|C|=|\{5,10,15,\dots,100\}|=\left\lfloor \frac{100}{5}\right\rfloor =20
\]
\[
|A\cap B|=|\{6,12,18,\dots,96\}|=\left\lfloor \frac{100}{6}\right\rfloor =16
\]
\[
|A\cap C|=|\{10,20,30,\dots,100\}|=\left\lfloor \frac{100}{10}\right\rfloor =10
\]
\[
|B\cap C|=|\{15,30,45,\dots,90\}|=\left\lfloor \frac{100}{15}\right\rfloor =6
\]
\[
|A\cap B\cap C|=|\{30,60,90\}|=\left\lfloor \frac{100}{30}\right\rfloor =3
\]

Tehát a válasz: 
\[
|A\cup B\cup C|=50+33+20-16-10-6+3=74
\]

Ez azt jelenti, hogy akkor $100-74=26$ szám \emph{nem} osztható se
2-vel, se 3-mal, se 5-tel.
\end{solution}
\begin{extraproblem}[Czofa Vivien, Kiss Andrea-Tímea, Seres Brigitta-Alexandra]
\textit{\emph{A Kerekasztal körül lévő 12 székre fel van írva a lovagok
neve. Ők azonban ezt figyelmen kívül hagyva, véletlenszerűen ülnek
le. Mennyi a valószínűsége annak, hogy senki sem ül a saját székén?
}}\textit{(II. Országos Magyar Matematikai Olimpia, II. forduló, XI-XII.
osztály)}
\end{extraproblem}

\begin{solution}
A lehetséges esetek száma $12!$, ennyiféleképpen tudjuk elhelyezni
a lovagokat a megjelölt székeken. Kiszámoljuk a kedvező esetek számát
a logikai \textbf{szitaformula} segítségével. Az összes ültetés számából
kivonjuk azoknak az ültetéseknek a számát, amikor pontosan egyvalaki
ül a helyén, majd hozzáadjuk azoknak a számát, amikor pontosan két
lovag ül a helyén és így tovább.

Így a kedvező esetek száma:

\[
S=12!-\binom{12}{1}\cdot11!+\binom{12}{2}\cdot10!-\binom{12}{3}\cdot9!+\dots+(-1)^{12}\cdot\binom{12}{12}\cdot0!=
\]

\[
=12!\left(1-\frac{1}{1!}+\frac{1}{2!}-\frac{1}{3!}+\dots+(-1)^{12}\cdot\frac{1}{12!}\right)
\]

Ezek alapján a keresett valószínűség:

\[
P=1-\frac{1}{1!}+\frac{1}{2!}-\frac{1}{3!}+\dots+(-1)^{12}\cdot\frac{1}{12!}
\]

\textbf{Megjegyzés.} Minél több lovag van, a valószínűség annál jobban
közelít $\frac{1}{e}=\sum_{k=0}^{\infty}\frac{(-1)^{k}}{k!}$-hez.
\end{solution}
\begin{extraproblem}[Fábián Nóra]
Négy kör úgy helyezkedik el, ahogyan az ábrán látható. A körökön
belül létrejött 10 tartományba úgy kell beírni a 0, 1, . . . 9 számokat,
hogy az egyes körökön belüli számok összege egyenlő legyen egymással.
Legfeljebb mekkora lehet ez az összeg?
\begin{center}
\includegraphics[width=0.3\textwidth]{\string"content/Szabo_Kinga/image1\string".png} 
\par\end{center}
\end{extraproblem}
\begin{solution}
Jelölje a körök egyenlő összegét $x$, az egyes tartományokba írt
számokat pedig a lenti ábrán a megfelelő helyre írt betű.
\begin{center}
\includegraphics[width=0.3\textwidth]{\string"content/Szabo_Kinga/image2\string".png}
\par\end{center}
Összeadom körönként azokat a területeket, amik rajta kívül esnek,
ez $45-x$ ($45=0+1+2+3+4+5+6+7+8+9$). Ha ezeket az (egyenlő) számokat
sikerül csökkenteni, akkor nő a körökön belüli összeg.

\[
45-x=i+a+j=i+f+c+d+a=i+f+h+g+j=a+b+d+g+j
\]

Innen leolvashatjuk az alábbiakat:

\[
i=b+d+g=f+h+g=j=f+c+d.
\]

Mivel azt szeretnénk, hogy $45-x$ minimális legyen, ezért

\[
i+a+j=(b+d+g)+(f+h+g)+(f+c+d)
\]

összeget akarjuk minimalizálni. Mivel ebben az összegben $g,f,d$
szerepel kétszer, így ők lesznek a legkisebbek.

Legyen $g,f,d=0,1,2$ és $b,h,c=3,4,5$ így

\[
i+a+j\geq6+7+8
\]

De ekkor

\[
i+a+j\geq(b+d+g)+(f+h+g)+(f+c+d)
\]

Vagyis számokkal behelyettesítve:

\[
6+7+8>2\cdot0+2\cdot1+2\cdot2+3+4+5
\]

Ami egyszerűbben írva:

\[
21>18.
\]

Ha $b,h,c$-t eggyel növeljük, akkor $i+a+j$ eggyel csökken. Ekkor

\[
20>19
\]

Tehát még ez sem jó. $45-x$-et min. 2-vel kell csökkenteni. Erre
a következő levezetésnél látunk majd megoldásokat. \textbf{Az egyes
körökbe írt számok:}

\[
x=a+b+c+d+e
\]
\[
x=c+e+f+h+i
\]
\[
x=e+d+h+g+j
\]
\[
x=b+c+d+e+f+g+h.
\]

Vonjuk ki az első három sor összegéből az utolsót!

\[
2x=a+c+d+2e+h+i+j.
\]

Így $2x$ értéke nem haladhatja meg

\[
2\cdot9+8+7+6+5+4+3=51
\]

értékét, azaz $x$ -- lévén egész szám -- legfeljebb 25. Ez az érték
el is érhető, a következő ábrán két példát mutatunk rá.
\begin{center}
\includegraphics[width=0.7\textwidth]{\string"content/Szabo_Kinga/image\string".png}
\par\end{center}
\end{solution}
\begin{extraproblem}[Gál Tamara]
Egy 52 lapos franciakártya-csomagot szétosztunk 4 játékos között.
Hány olyan leosztás van, amelyben minden játékosnak jut legalább egy
treff? (A francia kártyában 4-féle színű lap található, amely színek
egyike a treff, és minden színből 13 lap fordul elő.) 
\end{extraproblem}

\begin{solution}
Álljon a $H$ alaphalmaz a kártyacsomag 4 játékos közötti összes lehetséges
szétosztásából, továbbá legyenek ennek $A_{1},A_{2},A_{3},A_{4}$
részhalmazai azon szétosztások, amelyekben az 1., 2., 3., illetve
4. játékosnak nem jut treff. Feladatunk ekkor $\left|H\backslash\left(A_{1}\cup A_{2}\cup A_{3}\cup A_{4}\right)\right|$
meghatározása, amelyre a szita-módszert fogjuk alkalmazni.

Tudjuk, hogy $|H|=4^{52}$, hiszen mind az 52 lapot egymástól függetlenül
a 4 játékos bármelyikének adhatjuk. Azt is tudjuk, hogy $\left|A_{1}\right|=\left|A_{2}\right|=\left|A_{3}\right|=\left|A_{4}\right|=3^{13}\cdot4^{39}$,
mert ezen szétosztások esetében a 13 treffet mindig csak 3 játékos
valamelyike kaphatja (például $A_{1}$ esetében csak a 2., 3. vagy
4. játékos), a fennmaradó 39 lapot pedig a 4 játékos bármelyike.\\
 $\left|A_{1}\cap A_{2}\right|=\left|A_{1}\cap A_{3}\right|=\left|A_{1}\cap A_{4}\right|=\left|A_{2}\cap A_{3}\right|=\left|A_{2}\cap A_{4}\right|=\left|A_{3}\cap A_{4}\right|=2^{13}\cdot4^{39}$,
hiszen ekkor a 13 treffet csak 2 játékos valamelyike kaphatja (például
$A_{1}\cap A_{2}$ esetében csak a 3. vagy 4. játékos), a fennmaradó
39 lapot pedig továbbra is bárki.\\
 $\left|A_{1}\cap A_{2}\cap A_{3}\right|=\left|A_{1}\cap A_{2}\cap A_{4}\right|=\left|A_{1}\cap A_{3}\cap A_{4}\right|=\left|A_{2}\cap A_{3}\cap A_{4}\right|=1^{13}\cdot4^{39}$,
hiszen ekkor a 13 treffet mindig csak egy játékos kaphatja (például
$A_{1}\cap A_{2}\cap A_{3}$ esetében csak a 4. játékos), a fennmaradó
39 lapot pedig továbbra is bárki.

Végül $\left|A_{1}\cap A_{2}\cap A_{3}\cap A_{4}\right|=0$, hiszen
ekkor egyik játékos sem kaphatna treffet, ilyen szétosztás pedig nyilvánvalóan
nem létezik.

A szita-módszer szerint $\left|H\backslash\left(A_{1}\cup A_{2}\cup A_{3}\cup A_{4}\right)\right|=$

\[
=|H|-\left(\left|A_{1}\right|+\ldots\right)+\left(\left|A_{1}\cap A_{2}\right|+\ldots\right)-\left(\left|A_{1}\cap A_{2}\cap A_{3}\right|+\ldots\right)+\left|A_{1}\cap A_{2}\cap A_{3}\cap A_{4}\right|
\]

ahol a ... rendre az összes ugyanannyi halmazból álló metszet felsorolását
jelenti. A korábbi számításokat behelyettesítve a keresett mennyiség
$4^{52}-4\cdot3^{13}\cdot4^{39}+6\cdot2^{13}\cdot4^{39}-4\cdot1^{13}\cdot4^{39}+0=$
$=4^{39}\cdot\left(4^{13}-4\cdot3^{13}+6\cdot2^{13}-4\right)$, tehát
ennyi a megfelelő leosztások száma. (A végeredmény közelítő értéke
$1,837\cdot10^{31}$.) 
\end{solution}
\begin{extraproblem}[Gál Tamara]
10 ember színházba ment, ahol mindegyikük leadta a kabátját a ruhatárba
(a kabátok mind különbözőek). A szórakozott ruhatáros összecserélte
a kiadott sorszámokat, és az előadás végén úgy adta vissza a kabátokat
(minden embernek egyet-egyet), hogy senki sem a saját kabátját kapta
vissza. Hányféle különböző módon kaphatták vissza a kabátokat? (A
visszaadás sorrendje nem számít.) 
\end{extraproblem}

\begin{solution}
Álljon a $H$ alaphalmaz a 10 kabát 10 ember közötti összes lehetséges
szétosztásából, továbbá legyenek ennek $A_{1},A_{2},\ldots,A_{10}$
részhalmazai azon szétosztások, amelyekben az 1., $2.,\ldots$, illetve
10. ember a saját kabátját kapja vissza. Feladatunk ekkor $\left|H\backslash\left(A_{1}\cup A_{2}\cup\ldots\cup A_{10}\right)\right|$
meghatározása, amelyre az előző feladatban is látott szita-módszert
fogjuk alkalmazni.

Tudjuk, hogy $|H|=10$ !, továbbá $\left|A_{i}\right|=9$ ! (ahol
$1\leq i\leq10$ ), hiszen ha az $i$-edik ember a saját kabátját
kapja, akkor a fennmaradó 9 kabátot a többi 9 ember között 9!-féleképpen
lehet kiosztani. Hasonlóképpen $\left|A_{i}\cap A_{j}\right|=8$ !
(ahol $1\leq i,j\leq10$ és $i\neq j$ ), illetve bármely $k$ részhalmaz
metszetének elemszáma $(10-k)$ ! (ahol $1\leq k\leq10$ ). Mivel
$k$ részhalmazt $C_{10}^{k}$-féleképpen lehet kiválasztani, így
a szita-formulát felírva a keresett lehetőségek száma a következő:\\
 $10!-10\cdot9!+C_{10}^{2}\cdot8!-C_{10}^{3}\cdot7!+C_{10}^{4}\cdot6!-C_{10}^{5}\cdot5!+C_{10}^{6}\cdot4!-C_{10}^{7}\cdot3!+C_{10}^{8}\cdot2!-C_{10}^{9}\cdot1!+C_{10}^{10}\cdot0!$\\
 (ahol $0!=1$ alapján az utolsó tag jelöli a 10 halmaz közös metszetét,
vagyis azt az 1 lehetőséget, amikor mindenki a saját kabátját kapja
vissza).

Felhasználva a $C_{10}^{k}\cdot(10-k)!=\frac{10!}{k!\cdot(10-k)!}\cdot(10-k)!=\frac{10!}{k!}$
összefüggést, az előző eredmény felírható $10!\left(\frac{1}{0!}-\frac{1}{1!}+\frac{1}{2!}-\frac{1}{3!}+\frac{1}{4!}-\frac{1}{5!}+\frac{1}{6!}-\frac{1}{7!}+\frac{1}{8!}-\frac{1}{9!}+\frac{1}{10!}\right)$
alakban is, amelynek pontos értéke 1334961. (A zárójelben álló összeg
értéke egyébként azt mutatja meg, hogy a 10 ! összes esetnek hányadrésze
megfelelő számunkra.) Tehát 1334961 -féleképpen kaphatták vissza a
kabátokat. 
\end{solution}
\begin{extraproblem}[Gál Tamara]
5 matematikatanár 30 érettségi dolgozatot szeretne elosztani javításra
egymás között úgy, hogy mindegyikük legalább egyet kijavítson. Hányféleképpen
tehetik ezt meg? 
\end{extraproblem}

\begin{solution}
Álljon a $H$ alaphalmaz a 30 érettségi dolgozat 5 tanár közötti összes
lehetséges szétosztásából, továbbá legyenek ennek $A_{1},A_{2},A_{3},A_{4},A_{5}$
részhalmazai azon szétosztások, amelyekben az $1.,2$., 3., 4., illetve
5. tanárnak nem jut dolgozat. Feladatunk ekkor $\left|H\backslash\left(A_{1}\cup A_{2}\cup A_{3}\cup A_{4}\cup A_{5}\right)\right|$
meghatározása, amelyre a szita-módszert fogjuk alkalmazni.

Tudjuk, hogy $|H|=5^{30}$, hiszen a 30 dolgozat mindegyikét az 5
tanár bármelyikének adhatjuk. Továbbá $\left|A_{i}\right|=4^{30}$
(ahol $1\leq i\leq5$ ), hiszen ha az $i$-edik tanárnak nem jut dolgozat,
akkor minden dolgozatot 4 tanár valamelyikének adhatunk. Hasonlóképpen
$\left|A_{i}\cap A_{j}\right|=3^{30}$ (ahol $1\leq i,j\leq5$ és
$i\neq j$ ), illetve bármely $k$ részhalmaz metszetének elemszáma
$(5-k)^{30}$ (ahol $1\leq k\leq5$ ). Mivel $k$ részhalmazt $C_{5}^{k}$-féleképpen
lehet kiválasztani, így a szita-formulát felírva a keresett lehetőségek
száma $5^{30}-C_{5}^{1}\cdot4^{30}+C_{5}^{2}\cdot3^{30}-C_{5}^{3}\cdot2^{30}+C_{5}^{4}\cdot1^{30}-C_{5}^{5}\cdot0^{30}$
(ahol az utolsó tag jelöli az 5 halmaz metszetét, amely nyilvánvalóan
üres, hiszen ekkor egyik tanár se kaphatna dolgozatot).

A lehetséges szétosztások száma tehát $5^{30}-5\cdot4^{30}+10\cdot3^{30}-10\cdot2^{30}+5$.\\
 (A végeredmény közelítő értéke $9,256\cdot10^{20}$.)
\end{solution}
\begin{extraproblem}[Gergely Verona]
Egy osztály 42 leánytanulójának három kedvence van: egy színész (Sz),
egy vívóbajnok (V) és egy korcsolyabajnok (K). A leányok gyüjtik kedvenceik
aláírásos fényképeit. 6 leánynak már mindháromtól van ilyen fényképe.
Az osztálynak együttesen 21 Sz-képe van, 20 V-és 18 K-képe. Arra a
kérdésre, hogy kiknek van meg Sz képe is, V képe is, 7-en jelentkeztek,
hasonlóan Sz és K -ra 10-en, V és K -ra 11-en. Van-e ezek szerint
olyan leány, akinek még egyik kedvenctől sincs aláírott fényképe,
és ha igen, hány ilyen van? \emph{(KöMaL, 1963) }
\end{extraproblem}

\begin{solution}
Ha azt feltételezzük, hogy minden leánynak legfeljebb egy kedvencétől
lenne képe, akkor a képpel még nem rendelkezők számát úgy számolhatjuk
ki, hogy a képek összmennyiségéből kivonjuk az osztály létszámát.
Ebben az esetben a következő negatív számot kapnánk:

\[
42-(21+20+18)=-17
\]

Ez azért érthető, mert többször is kivontuk azok számát, akik több
képpel rendelkeznek. Például kétszer vontuk ki azokét, akiknek Sz
és V képe is megvan. Az ilyen kétszeres kivonások miatt hibák keletkezhetnek,
amit úgy javíthatunk ki, hogy hozzáadjuk azok számát, akik mindhárom
(Sz-V, Sz-K, V-K) képpárokra jelentkeztek:

\[
42-(21+20+18)+(7+10+11)
\]

Háromszor vontuk ki azok számát, akik mindhárom kedvencüktől rendelkeznek
képpel. Ők már mindhárom kérdésre válaszoltak, tehát a számukat háromszor
visszaadtuk. Mivel ezek a 6 lányok szerepeltek mindhárom csoportban,
ezt a számot most már le kell vonnunk a végeredményből.

Ezért az egyetlen képpel sem rendelkező leányok száma:

\[
42-(21+20+18)+(7+10+11)-6=5.
\]
\end{solution}
\begin{extraproblem}[Kis Aranka-Enikő]
Egy urnában $k$-féle színű golyó van, mindegyik színből ugyanannyi
darab. Egyenként húzunk a golyókból úgy, hogy minden húzás után visszatesszük
a kihúzott golyót, és minden húzásnál bármelyik golyó ugyanolyan valószínűséggel
kerül kihúzásra.
\begin{itemize}
\item[\foreignlanguage{english}{(a)}] Mennyi annak a $q_{n}$ valószínűsége, hogy legalább $n$ húzás kellett
ahhoz, hogy minden szín előforduljon? 
\item[\foreignlanguage{english}{(b)}] Mennyi annak a $p_{n}$ valószínűsége, hogy $n$ húzás során minden
szín előfordult, és ez az $n$-edik húzásnál következik be először
(vagyis az első $(n-1)$ húzás során csak $(k-1)$ szín fordult elő)? 
\end{itemize}
\end{extraproblem}

\begin{solution}
\textbf{(a)} Az, hogy legalább $n$ húzás kell, hogy minden szín előforduljon,
pontosan azt jelenti, hogy $n-1$ húzás után még nem volt minden szín.
Jelölje $A_{i,n-1}=A_{i}$ azt az eseményt, hogy az első $n-1$ húzás
során nem volt $i$ színű golyó.

Ekkor az az esemény, hogy $n-1$ húzás után nem volt minden szín,
pontosan azt jelenti, hogy az $A_{1},A_{2},\dots,A_{k}$ események
közül legalább egy bekövetkezett, azaz: 
\[
C_{n}:=\{\text{legalább }n\text{ húzás kell}\}=\bigcup_{i=1}^{k}A_{i}.
\]

Az $A_{i}$ események nem kizáróak, ezért az unió valószínűségét szita-formulával
határozhatjuk meg. Eszerint: 
\[
P\left(\bigcup_{i=1}^{k}A_{i}\right)=\sum_{j=1}^{k}(-1)^{j+1}\binom{k}{j}P(A_{1}\cap\dots\cap A_{j}),
\]
ahol az utolsó egyenlőségnél felhasználtuk, hogy a $j$-es metszetek
valószínűségei megegyeznek (például ugyanakkora valószínűséggel nem
volt sem piros, sem kék, mint sárga meg zöld).

A metszetek valószínűségeit könnyű meghatározni. Valóban: 
\[
P(A_{1})=\frac{(k-1)^{n-1}}{k^{n-1}},
\]
hiszen az összes eset $k^{n-1}$, mert az $n-1$ húzás során mindig
$k$-féle golyót kaphatunk, és a kedvező esetek száma $(k-1)^{n-1}$,
hiszen 1-es színű golyót nem húztunk.

Hasonlóan: 
\[
P(A_{1}\cap A_{2})=\frac{(k-2)^{n-1}}{k^{n-1}},
\]
hiszen ekkor sem 1-es, sem 2-es színű golyót nem húzhattunk.

Általánosan: 
\[
P(A_{1}\cap\dots\cap A_{j})=\frac{(k-j)^{n-1}}{k^{n-1}}.
\]

Ezt visszahelyettesítve: 
\[
P\left(\bigcup_{j=1}^{k}A_{j}\right)=\sum_{j=1}^{k}(-1)^{j+1}\binom{k}{j}\cdot\frac{(k-j)^{n-1}}{k^{n-1}}.
\]

\textbf{(b)} Legyen $D_{n}$ az az esemény, hogy pontosan $n$ húzás
kellett. Könnyű belátni, hogy: 
\[
D_{n}=C_{n}\setminus C_{n+1},
\]
hiszen ha pontosan $n$ kellett, akkor legalább $n$ kellett, de nem
kellett $n+1$. Nyilván $C_{n}\supset C_{n+1}$, tehát: 
\[
P(D_{n})=P(C_{n})-P(C_{n+1}).
\]
\end{solution}
\begin{extraproblem}[Kis Brigitta]
Egy művész 10 különböző festményt készít, amelyeket 5 különböző galériában
szeretne kiállítani. Minden festményt pontosan egy galériában kell
elhelyeznie. Hányféleképpen oszthatja el a festményeket a galériák
között, ha minden galériában legalább egy festményt el kell helyeznie? 
\end{extraproblem}

\begin{solution}
Először meghatározzuk az összes lehetséges elosztást, ha nem lenne
a "minden galériában legalább egy festmény" feltétel. Mivel minden
festményt 5 galéria közül választhatunk, az összes elosztás száma:

\[
5^{10}
\]

A szitaformula alkalmazásához levonjuk azokat az eseteket, amikor
legalább egy galéria üres.
\begin{itemize}
\item Egy galéria üres: A maradék 4 galériába kell elhelyezni a 10 festményt:

\[
4^{10}
\]

Mivel 5 galéria van, az ilyen esetek száma:

\[
5\cdot4^{10}
\]

\item Két galéria üres: A maradék 3 galériába kell elhelyezni a 10 festményt:

\[
3^{10}
\]

Az üresen maradó két galériát $\binom{5}{2}=10$ módon választhatjuk
ki, így az összes ilyen eset száma:

\[
10\cdot3^{10}
\]

\item Három galéria üres: A maradék 2 galériába kell elhelyezni a 10 festményt:

\[
2^{10}
\]

Az üresen maradó három galériát $\binom{5}{3}=10$ módon választhatjuk
ki, így az összes ilyen eset száma:

\[
10\cdot2^{10}
\]

\item Négy galéria üres: Az összes festményt az egyetlen maradék galériába
kell rakni:

\[
1^{10}=1
\]

Az egyetlen maradék galériát $\binom{5}{4}=5$ módon választhatjuk
ki, így:

\[
5\cdot1^{10}=5
\]

\end{itemize}
A végső képlet:

\[
5^{10}-5\cdot4^{10}+10\cdot3^{10}-10\cdot2^{10}+5\cdot1^{10}
\]
\end{solution}
\begin{extraproblem}[Kis Brigitta]
Egy iskola 15 diákját 6 különböző tanulmányi csoportba osztják be.
Minden diák pontosan egy csoportba kerül. Hányféleképpen lehet az
elosztást végrehajtani, ha minden csoportba legalább egy diákot be
kell osztani? 
\end{extraproblem}

\begin{solution}
Az összes lehetséges elosztás, ha nem lenne megkötés:

\[
6^{15}
\]

Levesszük azokat az eseteket, amikor legalább egy csoport üres.
\begin{itemize}
\item Egy csoport üres: A maradék 5 csoportba osztjuk a 15 diákot:

\[
5^{15}
\]

Azt, hogy melyik csoport legyen üres, $\binom{6}{1}=6$ módon választhatjuk
ki, így:

\[
6\cdot5^{15}
\]

\item Két csoport üres: A maradék 4 csoportba osztjuk a 15 diákot:

\[
4^{15}
\]

Az üresen maradó két csoportot $\binom{6}{2}=15$ módon választhatjuk
ki, így:

\[
15\cdot4^{15}
\]

\item Három csoport üres: A maradék 3 csoportba osztjuk a 15 diákot:

\[
3^{15}
\]

Az üresen maradó három csoportot $\binom{6}{3}=20$ módon választhatjuk
ki, így:

\[
20\cdot3^{15}
\]

\item Négy csoport üres: A maradék 2 csoportba osztjuk a 15 diákot:

\[
2^{15}
\]

Az üresen maradó négy csoportot $\binom{6}{4}=15$ módon választhatjuk
ki, így:

\[
15\cdot2^{15}
\]

\item Öt csoport üres: Az összes diák az egyetlen maradék csoportba kerül:

\[
1^{15}=1
\]

Az egyetlen maradék csoportot $\binom{6}{5}=6$ módon választhatjuk
ki, így:

\[
6\cdot1^{15}=6
\]

\end{itemize}
A végső képlet:

\[
6^{15}-6\cdot5^{15}+15\cdot4^{15}-20\cdot3^{15}+15\cdot2^{15}-6\cdot1^{15}
\]
\end{solution}
\begin{extraproblem}[Lukács Andor]
Hány olyan pozitív egész szám van, amely nem haladja meg a $2025$-öt,
és osztható $3$-mal vagy $4$-gyel, de nem osztható $5$-tel? 
\end{extraproblem}

\begin{solution}
Ha $A$ jelöli a $3$-mal osztható számok halmazát, $B$ a $4$-gyel
osztható számok halmazát, $C$ pedig az $5$-tel osztható számok halmazát
a megadott intervallumból, akkor 
\begin{itemize}
\item $|A|=\left\lfloor \frac{2025}{3}\right\rfloor =675$; 
\item $|B|=\left\lfloor \frac{2025}{4}\right\rfloor =506$. 
\end{itemize}
Azoknak a számoknak a halmaza, amelyek $3$-mal és $4$-gyel is oszthatók,
a $12$-vel osztható számok halmaza, tehát 
\[
|A\cap B|=\left\lfloor \frac{2025}{12}\right\rfloor =168.
\]
A szitaformula alapján tehát 
\[
|A\cup B|=|A|+|B|-|A\cap B|=675+506-168=1013.
\]
A feladat azt kéri, hogy számoljuk meg az $(A\cup B)\setminus C$
halmaz elemeit, ehhez felhasználjuk a 
\[
|(A\cup B)\setminus C|=|A\cup B|-|(A\cup B)\cap C|
\]
azonosságot, tehát elég meghatározni a 
\[
|(A\cup B)\cap C|=|(A\cap C)\cup(B\cap C)|=|A\cap C|+|B\cap C|-|A\cap B\cap C|
\]
számot. Viszont 
\begin{itemize}
\item $|A\cap C|=\left\lfloor \frac{2025}{15}\right\rfloor =135$; 
\item $|B\cap C|=\left\lfloor \frac{2025}{20}\right\rfloor =101$; 
\item $|A\cap B\cap C|=\left\lfloor \frac{2025}{60}\right\rfloor =33$, 
\end{itemize}
így 
\[
|(A\cup B)\cap C|=|A\cap C|+|B\cap C|-|A\cap B\cap C|=135+101-33=203.
\]
Végül tehát 
\[
|(A\cup B)\setminus C|=|A\cup B|-|(A\cup B)\cap C|=1013-203=810.
\]
\end{solution}
\begin{extraproblem}[Miklós Dóra]
Egy osztály 35 tanulója közül 12 szereti a matematikát, 14 a fizikát,
13 a kémiát és 10 a történelmet. Ha valaki több tárgyat is szeret
a felsoroltak közül, akkor az összes természettudományi tárgyat szereti.
Hat olyan gyerek van, aki az összes természettudományi tárgyat szereti,
és négy olyan gyerek van, aki az összes tárgyat szereti. Hányan nem
szeretik egyiket sem a négy tárgy közül? 
\end{extraproblem}

\begin{solution}
A feladatot a szitaformula segítségével oldjuk meg. Jelölje $M$,
$F$, $K$ és $T$ a megfelelő tantárgyat szerető tanulók halmazát.
Annak meghatározásához, hogy hányan nem szeretik egyik tantárgyat
sem előbb meghatározzuk, hogy hányan szeretik valamelyik tantárgyat,
vagyis az $M\cup F\cup K\cup T$ halmaz számosságára vagyunk kíváncsiak.
A szitaformula képletét az alapján alkalmazzuk, hogy figyelembe vesszük
a feladat feltételei szerint az alábbi összefüggéseket: 
\[
|M\cap F|=|M\cap K|=|F\cap K|=|M\cap F\cap K|=6
\]
\[
|T\cap M|=|T\cap F|=|T\cap K|=|T\cap M\cap F\cap K|=4
\]
\[
|T\cap M\cap F|=|T\cap M\cap K|=|T\cap F\cap K|=|T\cap M\cap F\cap K|=4.
\]
Ezek alapján a szitaformula az alábbiak szerint írható fel: 
\begin{align*}
|M\cup F\cup K\cup T| & =|M|+|F|+|K|+|T|-3\cdot|M\cap F\cap K|-3\cdot|M\cap F\cap K\cap T|\\
 & +|M\cap F\cap K|+3\cdot|M\cap F\cap K\cap T|-|M\cap F\cap K\cap T|=\\
 & =|M|+|F|+|K|+|T|-2\cdot|M\cap F\cap K|-|M\cap F\cap K\cap T|=\\
 & =12+14+13+10-2\cdot6-4=33.
\end{align*}
Tehát összesen $35-33=2$ diák van, aki egyik tantárgyat sem szereti. 
\end{solution}
\begin{extraproblem}[Miklós Dóra]
Hányféleképpen jelölhetünk ki egy konvex, $n$ oldalú $(n\geq6)$
sokszög csúcsai közül hármat úgy, hogy ezek egyike se legyen szomszédos
semelyik másikkal? 
\end{extraproblem}

\begin{solution}
Mivel $n>3$ ezért mindhárom pont nem lehet szomszédos, így három
esetet tudunk megkülönböztetni: 
\begin{itemize}
\item egyik pont se szomszédos egy másikkal; 
\item pontosan két pont szomszédos egymással; 
\item a három pont egymás mellett helyezkedik el, így két szomszédos pár
van köztük. 
\end{itemize}
Azon kiválasztások számát, amikor nincsenek szomszédos pontok úgy
fogjuk meghatározni, hogy az összes kiválasztásból kivonjuk azoknak
a számát, ahol legalább két pont szomszédos egymással.

Összesen $C_{n}^{3}$ féleképpen tudunk kiválasztani $n$-ből $3$
pontot. Vizsgáljuk meg azon esetek számát, amikor vannak szomszédos
pontok. Azok száma, ahol legalább két pont szomszédos $n\cdot C_{n-2}^{1}$,
vagyis $n(n-2)$, mivel két szomszédos pont $n$ féleképpen választható
ki és a harmadik pont bármelyik lehet a maradék $n-2$-ből. Azon esetek
száma pedig, ahol két szomszédos pontpár van ugyancsak $n$, mivel
ekkor a három pont egymásutáni. Így azon kiválasztások száma, ahol
vannak szomszédos pontok: 
\[
n\cdot(n-2)-n=n\cdot(n-3).
\]
Továbbá azok száma, ahol nincsenek szomszédos pontok: 
\begin{align*}
C_{n}^{3}-n\cdot(n-3) & =\frac{(n-2)(n-1)n}{6}-n\cdot(n-3)=\\
 & =\frac{n\cdot(n^{2}-3n+2-6n+18)}{6}=\\
 & =\frac{n\cdot(n^{2}-9n+20)}{6}.
\end{align*}
\end{solution}
\begin{extraproblem}[Péter Róbert]
Hány olyan pozitív egész szám van $1$ és $1000$ között, amely nem
osztható sem $4$-gyel, sem $6$-tal, sem $10$-zel, sem $15$-tel? 
\end{extraproblem}

\begin{solution}
Legyen $A,B,C,D$ rendre azoknak a számoknak a halmaza, amelyek oszthatók
$4$-gyel, $6$-tal, $10$-zel, illetve $15$-tel.

1. Egyes halmazok elemszáma:

\[
|A|=\left\lfloor \frac{1000}{4}\right\rfloor =250,\quad|B|=\left\lfloor \frac{1000}{6}\right\rfloor =166,\quad|C|=\left\lfloor \frac{1000}{10}\right\rfloor =100,\quad|D|=\left\lfloor \frac{1000}{15}\right\rfloor =66
\]

2. Páros metszetek:

\[
\begin{aligned}|A\cap B| & =\left\lfloor \frac{1000}{\mathrm{lkkt}(4,6)}\right\rfloor =\left\lfloor \frac{1000}{12}\right\rfloor =83\\
|A\cap C| & =\left\lfloor \frac{1000}{20}\right\rfloor =50\\
|A\cap D| & =\left\lfloor \frac{1000}{60}\right\rfloor =16\\
|B\cap C| & =\left\lfloor \frac{1000}{30}\right\rfloor =33\\
|B\cap D| & =\left\lfloor \frac{1000}{30}\right\rfloor =33\\
|C\cap D| & =\left\lfloor \frac{1000}{30}\right\rfloor =33
\end{aligned}
\]

3. Hármas metszetek:

\[
\begin{aligned}|A\cap B\cap C| & =\left\lfloor \frac{1000}{\mathrm{lkkt}(4,6,10)}\right\rfloor =\left\lfloor \frac{1000}{60}\right\rfloor =16\\
|A\cap B\cap D| & =\left\lfloor \frac{1000}{\mathrm{lkkt}(4,6,15)}\right\rfloor =\left\lfloor \frac{1000}{60}\right\rfloor =16\\
|A\cap C\cap D| & =\left\lfloor \frac{1000}{\mathrm{lkkt}(4,10,15)}\right\rfloor =\left\lfloor \frac{1000}{60}\right\rfloor =16\\
|B\cap C\cap D| & =\left\lfloor \frac{1000}{\mathrm{lkkt}(6,10,15)}\right\rfloor =\left\lfloor \frac{1000}{30}\right\rfloor =33
\end{aligned}
\]

4. Négyes metszet:

\[
|A\cap B\cap C\cap D|=\left\lfloor \frac{1000}{\mathrm{lkkt}(4,6,10,15)}\right\rfloor =\left\lfloor \frac{1000}{60}\right\rfloor =16
\]

5. Szitaformula alkalmazása:

\[
\begin{aligned}|A\cup B\cup C\cup D| & =(250+166+100+66)-(83+50+16+33+33+33)\\
 & \quad+(16+16+16+33)-16\\
 & =582-248+81-16=399
\end{aligned}
\]

A keresett számok száma: 
\[
1000-399=601
\]

Tehát 601 olyan szám van 1 és 1000 között, amely nem osztható sem
4-gyel, sem 6-tal, sem 10-zel, sem 15-tel. 
\end{solution}
\begin{extraproblem}[Sógor Bence]
Egy vállalatnak mind a 2025 alkalmazottja számára biztosítja az ebédet.
Mindenki leves és főfogás közül választ legalább egyet. A levest kérők
számának aránya 32 és 50 százalék között mozog, míg főfogást az alkalmazottak
80-92 százaléka szokott kérni. Mennyi $M-m$ értéke, ahol $M$ a teljes
ebédet kérők maximális száma, míg $m$ a teljes ebédet kérők minimális
száma? 
\end{extraproblem}

\begin{solution}
Jelöljük $L$-lel a levest kérők halmazát, míg $F$-fel a főfogást
kérők halmazát. Mivel levest az alkalmazottak 32-50 százaléka szokott
kérni ezért tudjuk, hogy $\frac{32}{100}\leq\frac{|L|}{2025}\leq\frac{50}{100}$.
Innen $648\leq|L|\leq1012,5$, ahol $|L|$ egy egész szám. Hasonlóan
$\frac{80}{100}\leq\frac{|F|}{2025}\leq\frac{92}{100}$, ahonnan $1620\leq|F|\leq1863$.
A teljes ebédet kérők száma $|L\cap F|$. Ez szitaformula alapján
$|L\cap F|=|L|+|F|-|F\cup L|$. Ennek az értéke akkor maximális vagy
minimális, amikor $|L|$ és $|F|$ értéke maximális vagy minimális.
Tehát $M=1012+1863-2025=850$ és $m=1620+648-2025=243$. Ezek alapján
$M-m=850-243=607$. \\
Feladat 1
\end{solution}
\begin{extraproblem}[Száfta Antal]
Legyen adott egy $n\times n$-es rács, ahol a bal alsó sarok koordinátái
$(0,0)$, a jobb felső saroké $(n,n)$.

Hány olyan rácsút vezet $(0,0)$-ból $(n,n)$-be, amely csak jobbra
vagy felfelé haladhat, és nem halad át egyetlen olyan rácsponton sem,
amely osztható $2$-vel, $3$-mal vagy $5$-tel, azaz a pont koordinátái
$(x,y)$ teljesítik, hogy $x\cdot y$ osztható $2$-vel, $3$-mal
vagy $5$-tel.
\end{extraproblem}

\begin{solution}
~

\underline{Összes út:}\\

Mivel csak jobbra vagy felfelé léphetünk, a $(0,0)\to(n,n)$ pont
közötti összes út száma: 
\[
C_{2n}^{n}.
\]

\underline{Tiltott pontok:}\\

Jelölje: 
\[
A_{2}=\{(x,y)\mid x\cdot y\text{ osztható }2\text{-vel}\},
\]
\[
A_{3}=\{(x,y)\mid x\cdot y\text{ osztható }3\text{-mal}\},
\]
\[
A_{5}=\{(x,y)\mid x\cdot y\text{ osztható }5\text{-tel}\}.
\]

\underline{Szitaformula alkalmazása:}\\

Legyen $S$ a tiltott pontokon áthaladó utak száma. A szitaformula
szerint: 
\[
S=S_{2}+S_{3}+S_{5}-S_{2,3}-S_{2,5}-S_{3,5}+S_{2,3,5},
\]
ahol $S_{2}$ az olyan utak száma, amelyek érintenek legalább egy
$A_{2}$-beli pontot, stb.

\underline{Kizárás, bennfoglalás}\\

A $S_{2}$ értékét úgy kapjuk, hogy kiszámoljuk az összes olyan utat,
amely érint legalább egy olyan pontot, ahol $x\cdot y$ osztható $2$-vel.
Ez ismert módszerrel történik: 
\begin{itemize}
\item A tiltottság miatt a rácsteret "blokkokra" kell osztani. 
\item Az egyes $S_{k}$ értékek kiszámítása az utak darabolásával történik. 
\end{itemize}
A végső válasz: 
\[
\boxed{C_{2n}^{n}-S},
\]
ahol $S$ a szitaformulával kiszámolt érték.
\end{solution}
\begin{extraproblem}[Száfta Antal]
 Legyen $S=\{1,2,3,\dots,N\}$. Számoljuk meg azon $k$-elemű részhalmazokat
$S$-ben, amelyek nem tartalmaznak 2-vel, 3-mal vagy 5-tel osztható
elemeket.
\end{extraproblem}

\begin{solution}
Jelöljük: 
\begin{itemize}
\item $A_{2}=\{x\in S\mid2\mid x\}$ 
\item $A_{3}=\{x\in S\mid3\mid x\}$ 
\item $A_{5}=\{x\in S\mid5\mid x\}$ 
\end{itemize}
A keresett elemszám: az $S$ olyan $k$-elemű részhalmazainak száma,
amelyek $A_{2}\cup A_{3}\cup A_{5}$ komplementerében vannak.

Legyen $M$ a $S$-beli számok száma, amelyek \emph{nem} oszthatók
2-vel, 3-mal vagy 5-tel.

Ezek azok a számok, amelyek prímtényezőik között nincs 2, 3, vagy
5.

Legyen $r$ ezek száma. Ekkor a válasz: 
\[
C_{r}^{k}
\]

A $r$ értékét szitaformulával határozzuk meg:

\[
r=N-|A_{2}\cup A_{3}\cup A_{5}|
\]

A szitaformula szerint: 
\[
|A_{2}\cup A_{3}\cup A_{5}|=|A_{2}|+|A_{3}|+|A_{5}|-|A_{2}\cap A_{3}|-|A_{2}\cap A_{5}|-|A_{3}\cap A_{5}|+|A_{2}\cap A_{3}\cap A_{5}|
\]

Mivel: 
\[
|A_{d}|=\left\lfloor \frac{N}{d}\right\rfloor 
\]

Ezért: 
\[
r=N-\left(\left\lfloor \frac{N}{2}\right\rfloor +\left\lfloor \frac{N}{3}\right\rfloor +\left\lfloor \frac{N}{5}\right\rfloor -\left\lfloor \frac{N}{6}\right\rfloor -\left\lfloor \frac{N}{10}\right\rfloor -\left\lfloor \frac{N}{15}\right\rfloor +\left\lfloor \frac{N}{30}\right\rfloor \right)
\]

\underline{Végső válasz:}\\

A keresett részhalmazok száma: 
\[
\boxed{C_{r}^{k}}
\]
ahol $r$ az előzőek szerint számolt érték. 
\end{solution}
\begin{extraproblem}[Szélyes Klaudia]
 Hány szökőév van az 1000 és 4004 közötti évek között (mindkettő
beleértve)? Egy év szökőév, ha: 
\begin{itemize}
\item osztható 4-gyel, de nem osztható 100-zal, \textbf{vagy} 
\item osztható 400-zal. 
\end{itemize}
\end{extraproblem}

\begin{solution}
Az 1000 és 4004 közötti (zárt) intervallumban lévő évek száma: 
\[
4004-1000+1=3005.
\]

\medskip{}

\textbf{1. lépés:} Az összes olyan év megszámlálása, amely osztható
4-gyel:

Legkisebb ilyen év: 1000 \\
 Legnagyobb ilyen év: 4004

\[
\frac{4004-1000}{4}+1=\frac{3004}{4}+1=751+1=752
\]

\textbf{2. lépés:} Kivonjuk azoknak az éveknek a számát, amelyek oszthatók
100-zal, de \textbf{nem} oszthatók 400-zal (mert ezek nem szökőévek):

Osztható 100-zal: $1000,1100,\dots,4000$

Ez számtani sorozat, első tag: $a_{1}=1000$, utolsó tag: $a_{n}=4000$,
differencia: $d=100$

\[
n=\frac{4000-1000}{100}+1=31
\]

Osztható 400-zal: $1200,1600,\dots,4000$

\[
n=\frac{4000-1200}{400}+1=\frac{2800}{400}+1=7+1=8
\]

Tehát a 100-zal osztható, de 400-zal nem osztható évek száma: $31-8=23$

\medskip{}

\textbf{Végső eredmény:} 
\[
\text{Szökőévek száma}=752-23=\boxed{729}
\]
\end{solution}
\begin{extraproblem}[Szélyes Klaudia]
Egy 100 hallgatóból álló évfolyamon:
\begin{itemize}
\item 50 hallgató jár matematika órára, 
\item 40 hallgató jár informatika órára, 
\item 35 hallgató jár kémia órára, 
\item 12 hallgató jár matematika \textbf{és} kémia órára is, 
\item 10 hallgató jár matematika \textbf{és} informatika órára is, 
\item 11 hallgató jár informatika \textbf{és} kémia órára is, 
\item 5 hallgató jár \textbf{mindhárom} órára. 
\end{itemize}
Kérdés:
\begin{enumerate}
\item Hány hallgató jár legalább az egyik órára? 
\item Hány hallgató nem jár egyik órára sem? 
\end{enumerate}
\end{extraproblem}

\begin{solution}
Legyenek a következők:
\begin{itemize}
\item $|M|=50$ (matematika), 
\item $|I|=40$ (informatika), 
\item $|K|=35$ (kémia), 
\item $|M\cap K|=12$, 
\item $|M\cap I|=10$, 
\item $|I\cap K|=11$, 
\item $|M\cap I\cap K|=5$. 
\end{itemize}
\end{solution}
\medskip{}

\textbf{1. kérdés: Legalább egy órára járók száma}

A háromhalmazos unió képlete: 
\[
|M\cup I\cup K|=|M|+|I|+|K|-|M\cap I|-|M\cap K|-|I\cap K|+|M\cap I\cap K|
\]

Behelyettesítve: 
\[
|M\cup I\cup K|=50+40+35-10-12-11+5=125-33+5=\boxed{97}
\]

\medskip{}

\textbf{2. kérdés: Egyik órára sem járók száma}

\[
100-97=\boxed{3}
\]

