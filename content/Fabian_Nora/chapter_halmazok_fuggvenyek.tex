
\chapter{Halmazok és függvények}\label{chap:halmazok}
\begin{description}
	{\large \item [{Szerző:}] Fábián Nóra (Korszerű módszerek a matematikatanításban, Didaktikai mesteri -- Matematika, 
		II. év)}
	
\end{description}
\begin{theorem}%%
	ALegyen $X$ egy véges halmaz, elemeit jelölje $1,2,\dots,n$ és $P(X)=\set{A\mid A\subseteq X}$
	az $X$ részhalmazainak halmazát. Akkor $\left|P(X)\right|=2^{n}$,
	ahol $n=\left|X\right|$. 
\end{theorem}
\begin{proof}
	Sajátos esetek vizsgálata: \\
	$X=\{1\}\Rightarrow P(X)=\{\emptyset,\{1\}\}\Rightarrow|P(X)|=2$
	\\
	$X=\{1,2\}\Rightarrow P(X)=\{\emptyset,\{1\},\{2\},\{1,2\}\}\Rightarrow|P(X)|=4$
	\\
	$X=\{1,2,3\}\Rightarrow P(X)=\{\emptyset,\{1\},\{2\},\{3\},\{1,2\},\{1,3\},\{2,3\},\{1,2,3\}\}\Rightarrow|P(X)|=8$
	\\
	$\Rightarrow$ Sejtés: $|P(X)|=2^{n}$
	
	Sejtés igazolása, döntésfával: 
	\begin{figure}[h]
		\centering \includegraphics[width=1\textwidth]{\string"content/Fabian_Nora/abra1\string".PNG}
		\caption{Döntésfa}
		\label{abra} 
	\end{figure}
	
	A döntésfán végighaladva követhetjük, hogy milyen elemek kerülnek
	a részhalmazokba. Ezzel a módszerrel az összes részhalmazt meghatározhatjuk.
	Mivel minden szinten két lehetőségünk van (benne van a halmazban vagy
	nincs benne) és a szinteken egymástól független döntéseket hozunk,
	így észrevehetjük, hogy az összes részhalmaz számát megadhatjuk, mint
	$2^{n}$, ahol $n$ az $A$ halmaz számossága volt.
	
	Sejtés igazolása más módszerrel: \\
	Legyen $B_{n}=\{(x_{1},x_{2},\ldots,x_{n})|x_{i}\in\{0,1\},i=\overline{1,n}\}$-
	$n$ hosszú bináris sorozatok halmaza. $|B_{n}|=?$ \\
	Tudjuk, hogy $|B_{n}|=\underbrace{2\cdot2\cdot\ldots\cdot2}_{\text{n-szer}}=2^{n}$
	({*}) \\
	Legyen $f:P(X)\rightarrow B_{n}$ $f(A)=(x_{1},x_{2},\ldots,x_{n})$,
	ahol
	
	\[
	x_{i}=\left\{ \begin{array}{ll}
		0 & \text{ha }i\in A\\
		1 & \text{ha }i\not\in A
	\end{array}\right.\forall i\in\{1,2,\ldots,n\}
	\]
	\\
	Igazoljuk, hogy $f$ bijektív leképezés.
	
	(1) Injektivitás: $A_{1},A_{2}\subseteq X,A_{1}\neq A_{2}\Rightarrow\exists i\in\{1,2,\ldots,n\}$
	úgy, hogy $i\in A_{1}$ és $i\not\in A_{2}$ vagy $i\not\in A_{1}$
	és $i\in A_{2}\Rightarrow x_{i}\neq y_{i}$, ahol $f(A_{1})=(x_{1},x_{2},\ldots,x_{n})$
	és $f(A_{2})=(y_{1},y_{2},\ldots,y_{n})\Rightarrow f(A_{1})\neq f(A_{2})\Rightarrow f$
	injektív.
	
	Injektivitás igazolása másképp: Tegyük fel, hogy $A_{1},A_{2}\subseteq X$
	úgy, hogy $f(A_{1})=f(A_{2})$, ahol $(x_{1},x_{2},\ldots,x_{n})=f(A_{1})$
	és $(y_{1},y_{2},\ldots,y_{n})=f(A_{2})\Rightarrow$ \\
	$\Rightarrow(x_{1},x_{2},\ldots,x_{n})=(y_{1},y_{2},\ldots,y_{n})\Leftrightarrow x_{i}=y_{i},\forall i\in\{1,2,\ldots,n\}\Leftrightarrow$
	\\
	$\Leftrightarrow i\in A_{1}$ és $i\in A_{2}$ vagy $i\not\in A_{1}$
	és $i\not\in A_{2},\forall i\in\{1,2,\ldots,n\}$, ami azt jelenti,
	hogy $i\in A_{1}\Leftrightarrow i\in A_{2}$, ami alapján $A_{1}=A_{2}\Rightarrow f$
	injektív
	
	(2) Szürjektivitás: Legyen $(x_{1},x_{2},\ldots,x_{n})\in B_{n}$
	tetszőleges. Akkor $\exists!A\in P(X)$ úgy, hogy $f(A)=(x_{1},x_{2},\ldots,x_{n})$,
	ahol $A=\{i|x_{i}=1,i\in\{1,2,\ldots,n\}\}\Rightarrow f$ szürjektív
	.\\
	(1), (2) $\Rightarrow f$ bijektív $\Rightarrow|B_{n}|=|P(X)|\xLeftrightarrow{(*)}|P(X)|=2^{n}$ 
\end{proof}

\section*{Házi feladatok}
\begin{problem}
	Bizonyítsuk be a következő állítást: egy $mn+1$ hosszú, különböző
	egész számokból álló $u=(u_{1},u_{2},\dots,u_{mn+1})$ sorozatnak
	vagy létezik $m$-nél hosszabb csökkenő részsorozata, vagy $n$-nél
	hosszabb növekvő részsorozata van.
	
	Például legyen $m=3$, $n=4$ és az $u=(1,-2,7,3,2,4,5,6,14,10,8,9,13)$
	sorozat. Ez a sorozat nem tartalmaz $3$-nál hosszabb csökkenő részsorozatot
	(a leghosszabb csökkenő részsorozatai a $(7,3,2)$ és a $(14,10,8)$,
	de tartalmaz egy négytagúnál hosszabb növekvő részsorozatot: $(1,2,4,5,6,8,9,13)$ 
\end{problem}
\begin{solution}
	Legyen $(l_{n_{i}},l_{cs_{i}})$ az $u_{i}$ elemből induló leghosszabb
	növekvő, illetve leghosszabb csökkenő részsorozat hossza.
	
	Red.~ad abs.: Feltételezzük, hogy $\forall i\in\{1,2,\ldots,n\cdot m+1\}$
	esetén $l_{n_{i}}\leq n$ és $l_{cs_{i}}\leq m$. \\
	Legyen 
	\[
	f:\{1,2,3,\ldots,n\cdot m+1\}\rightarrow\{1,2,3,\ldots,n\}\times\{1,2,3,\ldots,m\},f(i)=(l_{n_{1}},l_{cs_{i}})
	\]
	jól definiált függvény. Ez a függvény nem injektív, mert \\
	$|\{1,2,\ldots,n\cdot m+1\}|=n\cdot m+1\not\leq n\cdot m=|\{1,2,3,\ldots,n\}\times\{1,2,3,\ldots,m\}|$
	(1) \\
	Legyen $i<j,$ $i,j\in\{1,2,\ldots,m\cdot n+1\}$ \\
	Igazoljuk, hogy $f(i)\neq f(j)$.
	
	$u_{1},u_{2},\ldots,u_{i}=a,\ldots,u_{j}=b,\ldots u_{n\cdot m+1}$
	\\
	
	\[
	a\neq b\left\{ \begin{array}{lll}
		a<b & \Rightarrow l_{n_{i}}>l_{n_{j}}, & \text{mert az }u_{j}\text{-ből induló növekvő/csökkenő }\\
		a>b & \Rightarrow l_{cs_{i}}>l_{cs_{j}}, & \text{részsorozatotki lehet egészíteni }u_{i}\text{-vel}
	\end{array}\right.
	\]
	\\
	$\Rightarrow f(i)\neq f(j)$ \\
	Így ezek alapján $f$ injektív, mert bármely $i\neq j$ esetén $f(i)\neq f(j)$
	(2) \\
	Az (1) és (2) ellentmond egymásnak, így a red.~ad abs.~módszer
	alapján következik, hogy a bizonyítandó állítás igaz. 
\end{solution}
\begin{problem}
	Bizonyítsuk be, hogy egy véges halmaznak ugyanannyi páros elemszámú
	részhalmaza van, mint páratlan elemszámú. 
\end{problem}
\begin{solution}
	Legyen a véges halmazunk $X$, melynek számossága $n$, tehát $|X|=n$.
	\\
	Ha $n$ páros, akkor: 
	{\def\arraystretch{1.2}
	\[\begin{array}{ccc}
		\text{1 elemű részhalmazok száma: }C_{n}^{1} & & \text{0 elemű részhalmazok száma: }C_{n}^{0}\\
		\text{3 elemű részhalmazok száma: }C_{n}^{3} & & \text{2 elemű részhalmazok száma: }C_{n}^{2}\\
		\text{5 elemű részhalmazok száma: }C_{n}^{5} & & \text{4 elemű részhalmazok száma: }C_{n}^{4}\\
		\vdots & & \vdots\\
		\text{\ensuremath{n-1} elemű részhalmazok száma: }C_{n}^{n-1} & & \text{\ensuremath{n} elemű részhalmazok száma: }C_{n}^{n}\\
		\hrulefill & & \hrulefill\\
		C_{n}^{1}+C_{n}^{3}+C_{n}^{5}+\ldots+C_{n}^{n-1} & + & C_{n}^{2}+C_{n}^{4}+C_{n}^{6}+\ldots+C_{n}^{n}
	\end{array}\]
}
	Az $X$ halmaznak csak akkor van ugyanannyi páros részhalmaza van,
	mint páratlan, ha:
	\[
	C_{n}^{1}+C_{n}^{3}+C_{n}^{5}+\ldots+C_{n}^{n-1}=C_{n}^{2}+C_{n}^{4}+C_{n}^{6}+\ldots+C_{n}^{n}\Leftrightarrow
	\]
	
	\[
	-C_{n}^{1}+C_{n}^{2}-C_{n}^{3}+C_{n}^{4}-C_{n}^{5}+C_{n}^{6}+\ldots+C_{n}^{n}-C_{n}^{n-1}=0\Leftrightarrow
	\]
	
	\[
	(-1)^{1}\cdot C_{n}^{1}+(-1)^{2}\cdot C_{n}^{2}+(-1)^{3}\cdot C_{n}^{3}+(-1)^{4}\cdot+\ldots+(-1)^{n-1}\cdot C_{n}^{n-1}+(-1)^{n}\cdot C_{n}^{n}=0\Leftrightarrow
	\]
	
	\[
	(-1+1)^{n}=0\Leftrightarrow0=0
	\]
	\\
	Ha $n$ páratlan, akkor is teljesen hasonló egyenletet kapunk. Ezzel
	beláttuk, hogy valóban egy véges halmaznak ugyanannyi páros elemszámú
	részhalmaza van, mint páratlan elemszámú. 
\end{solution}
\begin{problem}
	Hány olyan pozitív szám van, ami nem haladja meg a 2001-et és a 3-nak
	vagy 4-nek többszöröse, de nem többszöröse 5-nek? 
\end{problem}
\begin{solution}
	Legyen
	
	\[
	A=\{x|x<2001\text{ és }3|x\}\Rightarrow|A|=667
	\]
	
	\[
	B=\{x|x<2001\text{ és }4|x\}\Rightarrow|B|=500
	\]
	
	\[
	C=\{x|x<2001\text{ és }5x\}\Rightarrow|C|=400
	\]
	
	\[
	D=\{x|x<2001\text{ és }3|x\text{ és }4|x\}=\{x|x<2001\text{ és }12|x\}\Rightarrow|D|=166
	\]
	
	\[
	E=\{x|x<2001\text{ és }3|x\text{ és }5|x\}=\{x|x<2001\text{ és }15|x\}\Rightarrow|E|=133
	\]
	
	\[
	F=\{x|x<2001\text{ és }4|x\text{ és }5|x\}=\{x|x<2001\text{ és }20|x\}\Rightarrow|F|=100
	\]
	
	\[
	G=\{x|x<2001\text{ és }3|x\text{ és }4|x\text{ és }5|x\}=\{x|x<2001\text{ és }60|x\}\Rightarrow|G|=33
	\]
	A következő ábrán a számok azt jelölik, hogy az hányszor számoltuk
	bele a halmaz/részhalmaz elemeit. A számolásaink végén el kell jutnunk
	oda, hogy csak az alábbi ábrán látható részeket számoljuk össze. 
	\begin{figure}[h]
		\centering \includegraphics[width=0.5\linewidth]{\string"content/Fabian_Nora/abra2\string".PNG}
		\label{fig:enter-label} 
	\end{figure}
	
	Egy lehetséges összeszámolási mód: 
	\begin{center}
		\includegraphics[width=1\textwidth]{\string"content/Fabian_Nora/abra3\string".PNG} 
		\par\end{center}
	$|A|+|B|-|D|-|E|-|F|+|G|=667+500-166-133-100+33=801$ 
\end{solution}
\begin{problem}
	Igazoljuk a következő összefüggéseket az $A,B$ és $C$ halmazokkal: 
	\begin{enumerate}
		\item[a.)] ${\displaystyle {A\cap(B\cup C)=(A\cap B)\cup(A\cap C)}}$ 
		\item[b.)] ${\displaystyle {A\cup(B\cap C)=(A\cup B)\cap(A\cup C)}}$ 
	\end{enumerate}
\end{problem}
\begin{solution}
	$a.)$ Bizonyítsuk be, hogy $A\cap(B\cup C)=(A\cap B)\cup(A\cap C)$.
	
	Bal oldal tartalmazza a jobb oldalt: \\
	Legyen $x\in A\cap(B\cup C)$. Ekkor $x\in A$ és $x\in B\cup C$.
	Mivel $x\in B\cup C$, ezért $x\in B$ vagy $x\in C$. Ha $x\in B$,
	akkor $x\in A$ és $x\in B$, vagyis $x\in A\cap B$. Ha $x\in C$,
	akkor $x\in A$ és $x\in C$, vagyis $x\in A\cap C$. Tehát $x\in(A\cap B)\cup(A\cap C)$.
	
	Jobb oldal tartalmazza a bal oldalt: Legyen $x\in(A\cap B)\cup(A\cap C)$.
	Ez azt jelenti, hogy $x\in A\cap B$ vagy $x\in A\cap C$. Ha $x\in A\cap B$,
	akkor $x\in A$ és $x\in B$. Ha $x\in A\cap C$, akkor $x\in A$
	és $x\in C$. Mivel mindkét esetben $x\in A$ és $x\in B$ vagy $x\in C$,
	ezért $x\in A$ és $x\in B\cup C$. Így $x\in A\cap(B\cup C)$.
	
	Tehát $A\cap(B\cup C)=(A\cap B)\cup(A\cap C)$. \\
	$b.)$ Bizonyítsuk be, hogy $A\cup(B\cap C)=(A\cup B)\cap(A\cup C)$.
	
	Bal oldal tartalmazza a jobb oldalt: \\
	Legyen $x\in A\cup(B\cap C)$. Ekkor $x\in A$ vagy $x\in B\cap C$.
	Ha $x\in A$, akkor $x\in A\cup B$ és $x\in A\cup C$, így $x\in(A\cup B)\cap(A\cup C)$.
	Ha $x\in B\cap C$, akkor $x\in B$ és $x\in C$. Mivel $x\in B$,
	ezért $x\in A\cup B$, és mivel $x\in C$, ezért $x\in A\cup C$.
	Ezért $x\in(A\cup B)\cap(A\cup C)$.
	
	Jobb oldal tartalmazza a bal oldalt: \\
	Legyen $x\in(A\cup B)\cap(A\cup C)$. Ekkor $x\in A\cup B$ és $x\in A\cup C$.
	Ha $x\in A$, akkor nyilvánvalóan $x\in A\cup(B\cap C)$. Ha $x\notin A$,
	akkor $x\in B$ (mivel $x\in A\cup B$) és $x\in C$ (mivel $x\in A\cup C$).
	Ekkor $x\in B\cap C$, így $x\in A\cup(B\cap C)$.
	
	Tehát $A\cup(B\cap C)=(A\cup B)\cap(A\cup C)$. 
\end{solution}
\begin{problem}
	Hány elemű az a halmaz, amelynek legalább ezerrel több részhalmaza
	van, mint ahány eleme? 
\end{problem}
\begin{solution}
	Jelölje $x$ a keresett halmaz elemeinek a számát. Ennek a halmaznak
	$2^{x}$ számú részhalmaza van. \\
	A feladat kérdése a következő egyenlőtlenségben írható fel: 
	\[
	x^{2}>x+1000.
	\]
	\\
	Ezt az egyenlőtlenséget hagyományos középiskolai algebrai eszközökkel
	nem tudjuk megoldani. Ha azonban végigfutunk $2$ egész kitevőjű hatványain,
	megtaláljuk azt a kitevőt, amelyre a fenti egyenlőtlenség teljesül:
	\\
	Mivel $2^{10}=1024$, így $2^{10}=1024>10+1000=1014$. \\
	Tehát minden olyan halmaz, amelynek legalább 10 eleme van, annak legalább
	1000-el több részhalmaza van, mint amennyi az elemeinek száma. 
\end{solution}
\begin{problem}
	Tekintsük az $A=\{1,2,3,4,5,6,7,8,9,10\}$ halmazt. Hány olyan részhalmaza
	van az $A$ halmaznak, amely: 
	\begin{enumerate}
		\item[a.)] tartalmazza a halmaz egy kijelölt elemét? 
		\item[b.)] tartalmazza az 1 és 2 elemeket (egyszerre)? 
		\item[c.)] tartalmazza az 1 és 2 elemek közül legalább az egyiket? 
		\item[d.)] tartalmazza az 1 és 2 elemek közül legfeljebb az egyiket? 
		\item[e.)] amelyben az elemek szorzata osztható 3-mal? 
	\end{enumerate}
\end{problem}
\begin{solution}
	$a.)$ Legyen a kijelölt elem az $1$-es. Számoljuk meg azokat a halmazokat,
	amelyek az $1$-et nem tartalmazzák, ehhez vegyük ki a halmazból az
	1-et. \\
	Ekkor a $\{2,3...,10\}$ halmazt kapjuk. Ennek a halmaznak $9$ eleme
	van, vagyis $2^{9}=512$ darab részhalmaza. Ha ezekhez a halmazokhoz
	hozzávesszük az 1-et is, akkor megkapjuk az összes olyan részhalmazt,
	amely tartalmazza az 1-et elemként. \\
	Tehát $512$ olyan részhalmaz van, amely tartalmazza a halmaz egy
	kijelölt elemét.
	
	$b.)$ Ugyanaz a taktika, mint az $a.)$-ban. Kivesszük az 1 és a
	2 elemeket, ekkor a $\{3,4,...,10\}$ halmazt kapjuk, amelynek 8 eleme,
	így $2^{8}=256$ részhalmaza van. Ezekhez vegyük hozzá az 1-et és
	a 2-t, így az összes részhalmazt megkapjuk, amely tartalmazza a halmaz
	két kijelölt elemét egyszerre. \\
	Tehát 256 darab részhalmaz van, amely az 1-et és a 2-t is tartalmazza.
	
	$c.)$ Azt tudjuk, hogy 512 olyan részhalmaz van, amely az 1-et tartalmazza,
	és nyilván ugyanennyien vannak a 2-t tartalmazók is, ezzel összesen
	1024 halmazt számoltunk össze. Azonban azokat a részhalmazokat így
	kétszer számoltuk meg, amelyek mindkettőt tartalmazzák egyszerre,
	így őket le kell vonnunk, hogy csak egyszer legyenek megszámolva.
	Ezekből a $b.)$ szerint 256 darab van.\\
	Tehát $1024-256=768$ olyan halmaz van, amely legalább az egyiket
	tartalmazza.
	
	$d.)$ A legfeljebb itt azt jelenti, hogy az 1 és a 2 közül vagy csak
	az 1-et, vagy csak a 2-t, vagy egyiket sem tartalmazza. Ez esetben
	a $b.)$ szerint 256 olyan részhalmaz van, amely egyiket sem tartalmazza.
	Ha ezekbe a halmazokba belerakjuk csak az 1-et, akkor az összes 1-et
	de 2-t nem tartalmazó halmazt megkapjuk, ha pedig csak a 2-t, akkor
	az összes 2-t de 1-et nem tartalmazó részhalmazt kapjuk meg. \\
	Tehát 256+256+256=768 olyan halmaz van, amely az 1 és a 2 közül legfeljebb
	az egyiket tartalmazza.
	
	$e.)$ Ha a számok szorzata osztható 3-mal, akkor a szorzat tartalmaz
	3-mal osztható számot. A számok között a 3;6;9 számok oszthatóak 3-mal,
	így ezek közül legalább az egyiket tartalmaznia kell a halmaznak.
	Azonban egyszerűbb megszámolni, hogy hány olyan van, amely ezeket
	nem tartalmazza. Kivéve a számokat, a halmazban 7 szám marad, így
	ezek $2^{7}=128$ részhalmazt alkotnak. Ezeken kívül az összes tartalmazza
	a 3;6;9 szám valamelyikét. \\
	Tehát $1024-128=896$ olyan részhalmaz van, amelyekben az elemek szorzata
	3-mal osztható. 
\end{solution}
\begin{problem}
	Válaszoljunk a következő kérdésekre:
	\begin{enumerate}
		\item[a.)] Legyen $A=\{1,2,3\},B=\{1,2,3,4\}$ halmazok és $f$ reláció egy
		függvény úgy, hogy $f:A\rightarrow B$. Hány darab ilyen $A$-ból
		$B$-be menő leképezés lehetséges? 
		\item[b.)] Ha $A$ és B véges halmazok úgy, hogy $|A|=n$ és $|B|=m$, akkor
		mennyi az $f\rightarrow A:B$ leképezések száma? 
	\end{enumerate}
\end{problem}
\begin{solution}
	$a.)$ Mivel $f$ egy függvény $A$-ból $B$-be, ezért minden $a\in A$
	elemhez egyértelműen hozzárendelünk egy $b\in B$ elemet. Mivel $|A|=3$
	és $|B|=4$, ezért minden egyes $a\in A$ elemhez $4$ féle $B$ halmazbeli
	elemet rendelhetünk.
	
	A függvények számát tehát: $4^{3}=64.$ \\
	Így összesen $64$ különböző leképezés lehetséges.
	
	\textbf{Általános eset:} Legyenek $A$ és $B$ véges halmazok úgy,
	hogy $|A|=n$ és $|B|=m$. Mivel minden $a\in A$ elemhez pontosan
	egy $b\in B$ elem van hozzárendelve, és minden elem $m$ féle értéket
	vehet fel, ezért a lehetséges függvények száma: $m^{n}.$ \\
	Következtetés: Ha $A$ egy $n$ elemű halmaz, és $B$ egy $m$ elemű
	halmaz, akkor az $A$-ból $B$-be menő függvények száma: 
	\[
	|B|^{|A|}=m^{n}.
	\]
\end{solution}
\begin{problem}
	Legyenek $A=\{1,2,3\}$, $B=\{a,b,c,d\}$ halmazok és $f:A\rightarrow B$
	egy függvény. Ebben az esetben mennyi az $A$-ból $B$-be menő \textbf{injektív}
	leképezések száma? Fogalamzzuk meg általánosn is, ha $A$ és $B$
	véges halmazok úgy, hogy $|A|=n\text{ és }|B|=m$ ! 
\end{problem}
\begin{solution}
	Egy függvény injektív, ha minden elemhez különböző képet rendel. Így
	van egy feltételünk, miszerint $n\leq m$. Ennek megfelelően:
	
	$\begin{rcases}\begin{array}{cc}
			\text{1-eshez rendelhetünk} & 4\text{ elemet}\\
			\text{2-eshez rendelhetünk} & 3\text{ elemet}\\
			\text{3-eshez rendelhetünk} & 2\text{ elemet}
	\end{array}\end{rcases}\Rightarrow4\cdot3\cdot2=24$ injektív leképezés lehetséges \\
	Általánosan: \\
	$\begin{rcases}\begin{array}{cc}
			\text{1-eshez rendelhetünk} & m\text{ elemet}\\
			\text{2-eshez rendelhetünk} & m-1\text{ elemet}\\
			\text{3-eshez rendelhetünk} & m-2\text{ elemet}\\
			\dotfill & \dotfill\\
			\text{n.-hez rendelhetünk} & m-n+1\text{ elemet}
	\end{array}\end{rcases}\Rightarrow\begin{array}{cc}
		m\cdot(m-1)\cdot\ldots\cdot(m-n+1)=\\
		={\displaystyle {\frac{m!}{(m-n)!}=V_{m}^{n}}}\\
		\text{ injektív leképezés lehetséges}
	\end{array}$ 
\end{solution}
\begin{problem}
	Határozzuk meg az $A$-ból $B$-be menő szürjektív függvények számát,
	ha $|A|=n$ és $|B|=m$. 
\end{problem}
\begin{solution}
	Vonjuk ki az $A$-ból $B$-be menő összes leképezés számából azoknak
	a leképezéseknek a számát, amelyek nem szürjektívek. \\
	Első lépésben vonjuk le azokat, amelyekben $B$-ből csak elemnek nincs
	ősképe. Mintha csak $(m-1)$ elemünk lenne a $B$ halmazban. Ilyen
	függvény összesen \textbf{$m\cdot(m-1)^{n}$} van. (Mindegyik elem
	esetén csak $(m-1)^{n}$) függvény van és ezekből $m$ darab van.)
	\\
	-Amikor 2 elemnek nincs ősképe, akkor összesen $(m-2)^{n}\cdot C_{m}^{2}$
	függvény képezhető \\
	-Hasonlóan, amikor 3 elemnek nincs ősképe, akkor $(m-3)^{n}\cdot C_{m}^{3}$
	függvény képezhető. \\
	-Folytatva a gondolatmenetet $k$ elem esetén pedig $(m-k)^{n}\cdot C_{m}^{k}$
	\\
	-Végül pedig $(m-1)$ elem esetén pedig $(m-m+1)^{n}\cdot C_{m}^{m-1}=m$
	függvény képezhető.
	
	Viszont, ha ezeket összeadjuk és levonjuk az összes függvény számából,
	amely képezhető, akkor lesznek olyan függvények, amit egynél többször
	vonunk le. Így ezeket hozzá kell adni az összeghez, hogy megkapjuk
	az összes szürjektív függvényt. \\
	$\Rightarrow m^{n}-{\displaystyle {\sum_{k=1}^{m-1}}C_{m}^{k}\cdot(m-k)^{n}\cdot(-1)^{k}\Rightarrow}$
	\\
	Összesen tehát ${\displaystyle {\sum_{k=0}^{m}}C_{m}^{k}\ \cdot(m-k)^{n}\cdot(-1)^{k}}$
	szürjektív leképezés van $A$-ból $B$-be. 
\end{solution}
\begin{problem}
	Határozzuk meg az $A$-ból $B$-be menő bijektív függvények számát,
	ha $|A|=n$ és $|B|=m$ 
\end{problem}
\begin{solution}
	Egy függvény akkor bijektív, ha injektí\'{v} és szürjektív is. Így
	ismét van egy feltételünk, miszerint $n\leq m$, mert injektív és
	$n\geq m$, mert szürjektív így $n=m$. \\
	$\begin{rcases}\begin{array}{cc}
			\text{1-eshez rendelhetünk} & m\text{ elemet}\\
			\text{2-eshez rendelhetünk} & m-1\text{ elemet}\\
			\dotfill & \dotfill\\
			\text{n.-hez rendelhetünk} & m-n+1\text{ elemet}\\
			\text{ és }m=n
	\end{array}\end{rcases}\Rightarrow\begin{array}{cc}
		m\cdot(m-1)\cdot\ldots\cdot(m-n+1)=\\
		=m\cdot(m-1)\cdot\ldots\cdot1=n!\\
		\text{ bijektív leképezés lehetséges}
	\end{array}$ 
\end{solution}
\begin{problem}
	Legyen $H$ a sík háromszögeinek halmaza és $S$ pedig a sík pontjainak
	halmaza. Tekintjük a $H$-ból $S$-be menő, $f$ leképezést, $f:H\rightarrow S$.
	Ez a leképezés egy háromszöghöz hozzárendeli a köré írt körének középpontját.
	Vizsgáljuk meg, hogy ez a leképezés injektív-e, szürjektív-e, bijektív-e? 
\end{problem}
\begin{solution}
	A leképezés nem injektív, ugyanis különböző háromszögekhez tartozhat
	ugyanaz a pont a síkon. Például ha kiválasztok egy elemet $S$-ből,
	és valamekkora (nem 0) sugárral kört rajzolok a pont körül, akkor
	a körvonalon kiválaszthatok többféleképpen is 3 pontot úgy, hogy háromszöget
	alkossanak.
	
	A leképezés szürjektív, mert bárhogy választok egy pontot S-ből a
	korábban leírt módszerrel, találhatok hozzá egy háromszöget a $H$
	elemeiből.
	
	Mivel a leképezés nem injektív, így nem bijektív.
\end{solution}
\begin{problem}
	Döntsük el, hogy az alábbi állítások közül melyik igaz és melyik hamis! 
	\begin{enumerate}
		\item[a.)] $f:\mathbb{N}\rightarrow\mathbb{N}:$ $f(n)=|n-3|+1$ leképezés injektív. 
		\item[b.)] $f:\mathbb{Z}\rightarrow\mathbb{Z}^{2}$ $f(x)=(x-1,1)$ leképezés
		szürjektív. 
		\item[c.)] $f:\mathbb{Z}\times\mathbb{Z}\rightarrow\mathbb{N}$ $f(x,y)=|x-y|$
		leképezés szürjektív. 
	\end{enumerate}
\end{problem}
\begin{solution}
	$a.)$ $f:\mathbb{N}\rightarrow\mathbb{N}:$ $f(n)=|n-3|+1$ leképezés
	injektív.
	
	$\begin{rcases}f(1)=|1-3|+1=3\end{rcases}\Rightarrow f(1)=f(5)$,
	de $1\neq5$ így $f$ nem injekív. \\
	$b.)$ $f:\mathbb{Z}\rightarrow\mathbb{Z}^{2}$ $f(x)=(x-1,1)$ leképezés
	szürjektív.
	
	A leképezés szerint a $\mathbb{Z}^{2}$ elemei közül csak azok állnak
	elő, amelyeknek a második komponense $1$. Így például a $(1,2)\in\mathbb{Z}^{2}$
	elemnek nincs ősképe. Tehát a leképezés nem szürjektív. \\
	$c.)$ $f:\mathbb{Z}\times\mathbb{Z}\rightarrow\mathbb{N}$ $f(x,y)=|x-y|$
	leképezés szürjektív.
	
	Vegyük például az összes olyan $\mathbb{Z\times Z}$-beli elemet,
	amely $(n,0)$ alakú, ahol $n\in\mathbb{N}$, így $f(n,0)=|n-0|=n$
	vagyis az összes természetes számnak lesz ősképe. Tehát a leképezés
	szürjektív. 
\end{solution}

\section*{Nehezebb feladatok}
\begin{extraproblem}[Csapó Hajnalka]
	Hány olyan $(A_{1},A_{2},A_{3})$ rendezett halmazhármas van, amelyre
	$A_{1}\cup A_{2}\cup A_{3}=\{1,2,\dots,n\}$ és $A_{1}\cap A_{2}\cap A_{3}=\emptyset$? 
\end{extraproblem}
\begin{solution}
	Tekintsük az alábbi táblázatot. 
	\begin{center}
		\begin{tabular}{|c|c|c|c|c|c|}
			\hline 
			& $1$ & $2$ & $3$ & $\dots$ & $n$\tabularnewline
			\hline 
			$A_{1}$ &  &  &  &  & \tabularnewline
			\hline 
			$A_{2}$ &  &  &  &  & \tabularnewline
			\hline 
			$A_{3}$ &  &  &  &  & \tabularnewline
			\hline 
		\end{tabular}
		\par\end{center}
	A fenti táblázatban $a_{ij}=1$, ha $j\in A_{i}$ és $a_{ij}=0$,
	ha $j\notin A_{i}$. A feltételek szerint a $(0,0,0)$ és $(1,1,1)$
	számhármasok egyik oszlopban sem jelenhetnek meg, így minden egyes
	halmazhármashoz egyértelműen hozzárendelhetünk egy ilyen táblázatot
	(mátrixot). Tehát a halmazhármasok száma az ilyen mátrixok számával
	egyezik meg. A mátrixok minden egyes oszlopát $2^{3}-2=6$ féleképpen
	tölthetjük ki, így $6^{n}$ ilyen mátrix létezik.
	
	\textbf{Megjegyzés.} Ha ugyanez a feladat $m$ halmaz esetén, akkor
	$(2^{m}-2)^{n}$ ilyen halmaz $m$-es létezik.
\end{solution}
\begin{extraproblem}[Csapó Hajnalka]
	Hány olyan $(A_{1},A_{2},A_{3})$ halmazhármas létezik, amelyre $A_{i}\neq\emptyset$,
	$(i=1\dots3)$, $A_{1}\cup A_{2}=A_{2}\cup A_{3}=A_{3}\cup A_{1}=\{1,2,\dots,n\}$
	és $A_{1}\cap A_{2}\cap A_{3}=\emptyset$? 
\end{extraproblem}
\begin{solution}
	Tekintsük a $B_{i}=A\setminus A_{i}$ halmazokat, ahol $A=\{1,2,\dots,n\}$.
	A feltételek alapján $B_{1}\cap B_{2}=B_{2}\cap B_{3}=B_{3}\cap B_{1}=\emptyset$
	és $B_{1}\cup B_{2}\cup B_{3}=A$. Nyilvánvalóan minden $(A_{1},A_{2},A_{3})$
	halmazhármasnak egyértelműen megfelel egy $(B_{1},B_{2},B_{3})$ halmazhármas,
	ezért ezeket fogjuk megszámolni. Minden ilyen $(B_{1},B_{2},B_{3})$
	hármashoz hozzárendelhetünk egy $(a_{1},a_{2},\dots,a_{n})$ szám
	$n$-est a következőképpen: 
	\[
	a_{i}=\left\{ \begin{array}{rl}
		1, & \mathrm{ha}\ \ i\in B_{1}\\
		2, & \mathrm{ha}\ \ i\in B_{2}\\
		3, & \mathrm{ha}\ \ i\in B_{3}
	\end{array}\right.
	\]
	Mivel az $(1,1,\dots,1)$, $(2,2,\dots,2)$ és $(3,3,\dots,3)$ hármasokat
	nem kaphatjuk, hiszen $B_{i}\neq A$, $(i=1,2,3)$, ezért összesen
	$3^{n}-3$ ilyen halmazhármas létezik.
	
	2. megoldás:
	
	Az előbbiekhez hasonlóan tekintsük az alábbi táblázatot. 
	\begin{center}
		\begin{tabular}{|c|c|c|c|c|c|}
			\hline 
			& $1$ & $2$ & $3$ & $\dots$ & $n$\tabularnewline
			\hline 
			$B_{1}$ &  &  &  &  & \tabularnewline
			\hline 
			$B_{2}$ &  &  &  &  & \tabularnewline
			\hline 
			$B_{3}$ &  &  &  &  & \tabularnewline
			\hline 
		\end{tabular}
		\par\end{center}
	A feltételek szerint minden oszlopban egy 1-es és két 0-s jelenik
	meg. Így minden $(B_{1},B_{2},B_{3})$ halmazhármashoz egyértelműen
	hozzárendelhetünk egy ilyen táblázatot. Ilyen táblázat $3^{n}$ létezik,
	de egyik sorban sem jelenhet meg csupa 1-es, tehát $3^{n}-3$ ilyen
	halmazhármas létezik létezik.
\end{solution}
\begin{extraproblem}[Csurka-Molnár Hanna]
	Egy matematikaversenyen $3$ feladatot tűztek ki. A $30$ induló
	közül az első feladatot $19$-en, a másodikat $15$-en, a harmadikat
	$18$-an oldották meg hibátlanul. Az első és második feladatra $7$,
	a második és harmadik feladatra $10$, az első és harmadik feladatra
	$9$ tanuló adott helyes megoldást. Mindhárom feladat megoldása $3$
	diáknak sikerült. Hányan nem tudtak egyetlen feladatot sem megoldani? 
\end{extraproblem}
\begin{solution}
	Legyenek a következő halmazok: 
	\begin{itemize}
		\item $A$ az első feladatot helyesen megoldó diákok halmaza, 
		\item $B$ a második feladatot helyesen megoldó diákok halmaza, 
		\item $C$ a harmadik feladatot helyesen megoldó diákok halmaza. 
	\end{itemize}
	A megadott adatok szerint: 
	\[
	|A|=19,\quad|B|=15,\quad|C|=18,
	\]
	\[
	|A\cap B|=7,\quad|B\cap C|=10,\quad|A\cap C|=9,\quad|A\cap B\cap C|=3.
	\]
	
	Az szitaformulát alkalmazva az alábbi képlettel számítjuk ki, hogy
	hány diák oldott meg legalább egy feladatot:
	
	\[
	|A\cup B\cup C|=|A|+|B|+|C|-|A\cap B|-|B\cap C|-|A\cap C|+|A\cap B\cap C|
	\]
	
	Behelyettesítve a megadott értékeket:
	
	\[
	|A\cup B\cup C|=19+15+18-7-10-9+3=29.
	\]
	
	Mivel összesen $30$ diák vett részt a versenyen, azok száma, akik
	egyetlen feladatot sem tudtak megoldani:
	
	\[
	30-29=1.
	\]
	
	\textbf{Válasz:} Egyetlen diák nem tudott egy feladatot sem megoldani. 
\end{solution}
\begin{extraproblem}[Gergely Verona]
	Jelölje $r_{n}$ az $\{1;2;\cdots;n\}$ halmaz azon részhalmazainak
	számát, amelyek nem tartalmaznak szomszédos számokat, ahol az $1$-et
	és az $n$-et is szomszédosnak tekintjük. Határozzuk meg $r_{16}$
	értékét.
	
	(OKTV 2010/2011; II. kategória, döntő) 
\end{extraproblem}
\begin{solution}
	Vizsgáljuk meg kisebb $n$-re.
	
	Ha $n=1$ akkor $\{1\}$ a feltételnek megfelelő részhalmazok $\emptyset$
	és $\{1\}$, tehát $r_{1}=2$.
	
	Ha $n=2$ akkor $\{1,2\}$ a feltételnek megfelelő részhalmazok $\emptyset$,
	$\{1\}$ és $\{2\}$, tehát $r_{2}=3$.
	
	Ha $n=3$ akkor $\{1,2,3\}$ a feltételnek megfelelő részhalmazok
	$\emptyset$, $\{1\}$, $\{2\}$ és $\{3\}$, tehát $r_{3}=4$.
	
	Ha $n=4$ akkor $\{1,2,3,4\}$ a feltételnek megfelelő részhalmazok
	$\emptyset$, $\{1\}$, $\{2\}$, $\{3\}$, $\{4\}$, $\{1,3\}$ és
	$\{2,4\}$ tehát $r_{4}=7$.\\
	
	Sejtés: Ha $n\geq2$ akkor $r_{n+2}=r_{n+1}+r_{n}$.\\
	
	Három részre bontjuk az $n+2$ elemű halmaz megfelelő részhalmazait:
	\begin{enumerate}
		\item Ha a részhalmaz nem tartalmazza $n+2$-t, és sem $1$-et, sem $n+1$-et,
		akkor éppen az első $n+1$ elem megfelelő részhalmazait kapjuk, amelyek
		száma $r_{n+1}$.
		\item Ha a részhalmaz nem tartalmazza $n+2$-t, de tartalmazza $1$-et és
		$n+1$-et is, akkor nem lehet benne $n$. Ezekből a részhalmazokból,
		ha kivesszük $n+1$-et, akkor az első $n$ elem megfelelő részhalmazai
		közül azok maradnak, amelyek tartalmazzák $1$-et. Ezek száma pontosan
		$r_{n}$.
		\item Ha a részhalmaz tartalmazza $n+2$-t, akkor nem tartalmazhatja sem
		$1$-et, sem $n+1$-et. Ezekből a részhalmazokból, ha kivesszük $n+2$-t,
		akkor az első $n$ elem megfelelő részhalmazai közül azok maradnak,
		amelyek nem tartalmazzák $1$-et. Ezek száma pontosan $r_{n}$.
	\end{enumerate}
	Ezzel beláttuk az $r_{n+2}=r_{n+1}+r_{n}$ összefüggést.\\
	
	Ezek alapján: $r_{5}=r_{4}+r_{3}=7+4=11$,$r_{6}=r_{5}+r_{4}=11+7=18$,$r_{7}=r_{6}+r_{5}=18+11=29$,$\dots$,$r_{16}=r_{15}+r_{14}=1364+843=2207$
\end{solution}
\begin{extraproblem}[Gergely Verona]
	Legyen $A=\{1;2;3;4;5\}$ és $B=\{1;2;3\}$. Az $f(x)$ függvény
	értelmezési tartománya $A$, és minden $A$-beli $x$ esetén $f(x)\in A$.
	Hány $f(x)$ függvényre teljesül, hogy $\{f(f(x))|x\in A\}=B$?
	
	(OKTV 2011/2012; II. kategória, döntő) 
\end{extraproblem}
\begin{solution}
	Legyen $f(1)=4$, mivel $f(f(x))$ értékkészlete $B$, ezért $\exists y\in A$
	úgy, hogy $f(f(y))=1$. Ha $f(y)=a,a\in A$ akkor $f(f(a))=f(1)=4\notin B\Rightarrow$
	a keresett $f$ függvény az $1,2,3$-hoz nem rendelhet $4,5$-öt.
	
	$f(4)=4$ nem lehetséges mert ekkor $f(f(4))=4\notin B$, hasonlóan
	$f(5)=5$ sem lehetséges.\\
	
	Ha $f(1),f(2),f(3)$ értékei az $1,2,3$ számok valamilyen permutációja
	akkor 
	\begin{enumerate}
		\item $f(4)$ és $f(5)$ is $B$-beli értéket vesz fel, egymástól függetlenül
		vehetnek fel 3-3 különböző értéket. Összesen $3!\cdot3\cdot3=54$
		darab ilyen függvény van. 
		\item $f(4)$ és $f(5)$ közül legalább az egyik nem $B$-beli, vagyis $f(4)=5$
		vagy $f(5)=4$ ekkor $f(4)=5$ esetén $f(f(4))\in B$ miatt $f(5)\in B$,
		míg $f(5)=4$ esetén $f(4)\in B$.Ekkor $2$ lehetőség van arra, hogy
		$4$ vagy $5$ képe legyen $B$-beli, majd a választott kép $3$ különböző
		értéket vehet fel. Összesen $3!\cdot2\cdot3=36$ darab ilyen függvény
		van. 
	\end{enumerate}
	Vizsgáljuk meg azt az esetet amikor $f(1),f(2),f(3)$ értékei az $1,2,3$
	számok valamilyen permutációja úgy, hogy lehetnek azonos értékek.
	
	Ha $f(1)=f(2)=f(3)=k$ ahol $k\in\{1,2,3\}$ ez azt jelenti, hogy
	az $f(f(x))$ értékkészlete csak egy elemet tartalmaz ami ellentmond
	a feladat feltételeinek.
	
	Legyen két érték azonos(jel. a) (pl $f(1)=f(2)$), egy különböző(jel.
	k), akkor ha: 
	\begin{enumerate}
		\item $f(4)$ és $f(5)$ is $B$-beli elem, de ez nem lehetséges mivel $f(f(x))$
		értékkészlete csak $2$ számból állna. 
		\item $f(4)$ vagy $f(5)$ $B$-beli elem úgy, hogy különbözik az idáig
		választott értékektől (jel. b), mert csak így lesz $f(f(x))$ értékkészlete
		a teljes $B$. Az a értékre $3$ lehetőségünk van, k értéke $2$-féle
		lehet. Továbbá $3$ -féle választásunk van, hogy az $1,2,3$ számok
		közül melyikhez rendeljük a-t, és 2-féle választásunk, hogy a 4-hez
		vagy az 5-höz rendeljük b-t. Összesen $3\cdot2\cdot3\cdot2=36$ darab
		ilyen függvény van. 
	\end{enumerate}
	A feladat feltételeit $54+36+36=126$ függvény teljesíti. 
\end{solution}
\begin{extraproblem}[Kis Aranka-Enikő]
	Legyen a következő feladat:
	\begin{enumerate}
		\item Határozd meg az $f:\mathbb{R}\setminus\{-1,0,1\}\to\mathbb{R}\setminus\{-1,0,1\}$
		szürjektív függvényt, ha 
		\[
		f(x)\cdot f(f(x))+f(f(x))=f(x)-1,
		\]
		minden $x\in\mathbb{R}\setminus\{-1,0,1\}$ esetén!
		\item Értelmezzük a $h_{n}=f\circ f\circ\cdots\circ f$ ($n$-szer $f$)
		függvényeket, ahol $n\in\mathbb{N}^{*}$. Határozd meg a $h_{2024}$
		függvényt! 
	\end{enumerate}
\end{extraproblem}
\begin{solution}
	A kérdésekre adott válasz:
	\begin{enumerate}
		\item Az adott $f(x)\cdot f(f(x))+f(f(x))=f(x)-1$ egyenlőségből kiindulva
		kapjuk, hogy 
		\[
		f(f(x))(f(x)+1)=f(x)-1.
		\]
		Mivel a függvény értékkészlete az $\mathbb{R}\setminus\{-1,0,1\}$
		halmaz, ezért $f(x)\neq-1$, tehát eloszthatunk $(f(x)+1)$-el, így
		\[
		f(f(x))=\frac{f(x)-1}{f(x)+1}.
		\]
		Jelöljük $y=f(x)$, ekkor 
		\[
		f(y)=\frac{y-1}{y+1},\quad\forall y\in\text{Im}(f).
		\]
		Tehát a keresett függvény az $f:\mathbb{R}\setminus\{-1,0,1\}\to\mathbb{R}\setminus\{-1,0,1\}$,
		ahol 
		\[
		f(x)=\frac{x-1}{x+1}.
		\]
		Ellenőrizzük az $f$ szürjektivitását: minden $y\in\mathbb{R}\setminus\{-1,0,1\}$
		esetén létezik $x\in\mathbb{R}\setminus\{-1,0,1\}$ úgy, hogy 
		\[
		f(x)=y\iff\frac{x-1}{x+1}=y,
		\]
		ahonnan 
		\[
		x=\frac{1+y}{1-y},
		\]
		ez pedig létezik az $\mathbb{R}\setminus\{-1,0,1\}$ halmazon. Ha
		\[
		\frac{1+y}{1-y}=-1\iff1+y=-1+y\iff1=-1,
		\]
		ami lehetetlen. Ha 
		\[
		\frac{1+y}{1-y}=0\iff1+y=0\iff y=-1,
		\]
		ami szintén lehetetlen.
		\item Számítsuk ki: 
		\begin{itemize}
			\item $n=1$ esetén $h_{1}(x)=f(x)=\frac{x-1}{x+1}$, 
			\item $n=2$ esetén $h_{2}(x)=(f\circ f)(x)=-\frac{1}{x}$, 
			\item $n=3$ esetén $h_{3}(x)=(f\circ f\circ f)(x)=-\frac{1}{f(x)}=\frac{x+1}{1-x}$, 
			\item $n=4$ esetén $h_{4}(x)=(f\circ f\circ f\circ f)(x)=x$. 
		\end{itemize}
		Ahonnan következik, hogy $n=2024$ esetén $h_{2024}(x)=h_{4}(x)=x$. 
		
	\end{enumerate}
\end{solution}
\begin{extraproblem}[Kis Brigitta]
	Legyen $A=\{1,2,3,4,5,6,7,8,9,10\}$ egy halmaz, és $f:A\to A$ egy
	olyan függvény, amelyre teljesül, hogy:
	\begin{itemize}
		\item $f(f(n))=n$ minden $n\in A$-ra 
		\item $f(1)=2$ és $f(2)=1$. 
	\end{itemize}
	Hányféle olyan $f$ függvény létezik, amely megfelel a fenti feltételeknek? 
\end{extraproblem}
\begin{solution}
	Vegyük sorra a feltételeket:
	\begin{itemize}
		\item Az első feltétel azt mondja, hogy $f(f(n))=n$ minden $n\in A$-ra.
		Ez azt jelenti, hogy minden elemhez egyértelműen egy másik elem tartozik,
		és ha egyszer alkalmazzuk $f$-et, majd újra $f$-et, akkor visszajutunk
		az eredeti elemhez. 
		\item A második feltétel biztosítja, hogy $f(1)=2$ és $f(2)=1$, tehát
		a 1-es és 2-es elemek kölcsönösen hozzárendelve vannak. 
		\item Az $f(f(n))=n$ azt is jelenti, hogy ha $f(n)=m$, akkor $f(m)=n$.
		Ezért minden más párt (3, 4), (5, 6), (7, 8), (9, 10) szimmetrikusan
		kell kezelni, tehát a maradék elemeket úgy kell rendelni, hogy párosan
		rendeljük őket egymáshoz. 
		\item A maradék számok: $\{3,4,5,6,7,8,9,10\}$. Mivel minden elemhez egy
		másik elem tartozik, az összes lehetséges párt a következő módon kell
		kiválasztani: 
		\item A 3-at választhatjuk a 4-hez, a 5-öt a 6-hoz, a 7-et a 8-hoz, és a
		9-et a 10-hez. 
		\item Mivel 4 párt kell kialakítani, a 4 pár kiválasztásának száma a $4!$
		lehetséges permutációjából adódik. 
	\end{itemize}
	Tehát a megoldás (az összes lehetséges függvények száma): $4!=24$.
\end{solution}
\begin{extraproblem}[Kiss Andrea-Tímea]
	Határozd meg az $f:\mathbb{N}\rightarrow\mathbb{N}$ függvényeket,
	amelyekre $f(1)=1$, $f(2n)<6f(n)$ és $3f(n)f(2n+1)=f(2n)(3f(n)+1)$,
	bármely $n\in\mathbb{N}$ esetén. 
\end{extraproblem}
\begin{solution}
	A fenti összefüggéseket felírva $n=1$ esetén azt kapjuk, hogy 
	\[
	\left\{ \begin{array}{l}
		3f(1)f(3)=f(2)(3f(1)+1)\\
		f(1)=1
	\end{array}\right.\Rightarrow3f(3)=4f(2)
	\]
	\[
	\Rightarrow\left\{ \begin{array}{l}
		3\phantom{x}|\phantom{x}4f(2)\\
		lnko(3,4)=1
	\end{array}\right.\Rightarrow3\phantom{x}|\phantom{x}f(2).
	\]
	Tudjuk, hogy $f(2)<6f(1)=6$, így $f(2)\in\{0,3\}$.
	
	Ha $f(2)=0$, akkor $3f(2)f(5)=f(4)(3f(2)+1)$, vagyis $f(4)=0$.
	Hasonlóan, ha feltételeznénk, hogy egy $k\in\mathbb{N}^{*}$ esetén
	$f(k)=0$, akkor abból következne, hogy $f(2k)=0$, de ekkor az $f(2k)<6f(k)$
	feltétel nem teljesülne. Ez azt jelenti, hogy $\forall n\in\mathbb{N}^{*}$
	esetén $f(n)\neq0$. Tehát $f(2)=3$.
	
	\[
	\left\{ \begin{array}{l}
		3f(n)f(2n+1)=f(2n)(3f(n)+1),\forall n\in\mathbb{N}\\
		f(k)\in\mathbb{N},\forall k\in\mathbb{N}
	\end{array}\right.
	\]
	\[
	\Rightarrow\left\{ \begin{array}{l}
		3f(n)\phantom{x}|\phantom{x}f(2n)(3f(n)+1)\\
		lnko(3f(n),3f(n)+1)=1
	\end{array}\right.\Rightarrow3f(n)\phantom{x}|\phantom{x}f(2n),
	\]
	ami alapján létezik olyan $l\in\mathbb{N}$, hogy $f(2n)=3lf(n)$.
	Viszont tudjuk, hogy $f(2n)<6f(n)$, ami alapján $l\in\{0,1\}$. Mivel
	$f(k)\neq0$, $\forall k\in\mathbb{N}$, ezért $l\neq0$. Tehát $l=1$
	és így $f(2n)=3f(n)$, $\forall n\in\mathbb{N}$.
	
	\[
	\left\{ \begin{array}{l}
		3f(n)f(2n+1)=f(2n)(3f(n)+1)\\
		f(2n)=3f(n)
	\end{array}\right.\Rightarrow3f(n)f(2n+1)=3f(n)(3f(n)+1)
	\]
	\[
	\Rightarrow\left\{ \begin{array}{l}
		f(n)(f(2n+1)-3f(n)-1)=0,\forall n\in\mathbb{N}.\\
		f(n)\neq0,\forall n\in\mathbb{N}^{*}.
	\end{array}\right.\Rightarrow f(2n+1)=3f(n)+1,\forall n\in\mathbb{N}^{*}.
	\]
	
	Vizsgáljuk meg az $n=0$ esetet. Ha $n=0$, akkor $3f(0)f(1)=f(0)(3f(0)+1)$
	$\Leftrightarrow$ $f(0)(2-3f(0))=0$, $f(0)\in\mathbb{N}$ $\Rightarrow$
	$f(0)=0$.
	
	Tehát azt kaptuk, hogy $f:\mathbb{N}\rightarrow\mathbb{N}$ olyan
	függvény, amelyre $f(0)=0$, $f(1)=1$ és $f(2n)=3f(n)$, illetve
	$f(2n+1)=3f(n)+1$, $\forall n\in\mathbb{N}^{*}$.
	
	Ahhoz, hogy a függvényt explicit módon is meg tudjuk adni, vizsgáljunk
	meg pár példát.
	
	Például $f(81)=f(2\cdot40+1)=3f(40)+1=3f(2\cdot20)+1=3^{2}f(20)+1=3^{2}f(2\cdot10)+1=3^{3}f(10)+1=3^{3}f(2\cdot5)+1=3^{4}f(5)+1=3^{4}f(2\cdot2+1)+1=3^{4}(3f(2)+1)+1=3^{5}f(2)+3^{4}+1=3^{5}f(2\cdot1)+3^{4}+1=3^{6}f(1)+3^{4}+1=3^{6}+3^{4}+1$,
	viszont $81=2\cdot40+1=2^{2}\cdot20+1=2^{3}\cdot10+1=2^{4}\cdot5+1=2^{4}(2\cdot2+1)+1=2^{6}+2^{4}+1$.
	
	Tehát, ha $n=2^{\alpha_{1}}+2^{\alpha_{2}}+\cdots+2^{\alpha_{k}}$,
	ahol $\alpha_{1}>\alpha_{2}>\cdots>\alpha_{k}$, $\alpha_{i}\in\mathbb{N}$
	(ami mindig felírható), akkor $f(n)=3^{\alpha_{1}}+3^{\alpha_{2}}+\cdots+3^{\alpha_{k}}$.
	
	Így $f:\mathbb{N}\rightarrow\mathbb{N}$, ahol 
	\[
	f(n)=\left\{ \begin{array}{ll}
		0, & n=0;\\
		3^{\alpha_{1}}+3^{\alpha_{2}}+\cdots+3^{\alpha_{k}}, & n=2^{\alpha_{1}}+2^{\alpha_{2}}+\cdots+2^{\alpha_{k}},\alpha_{1}>\alpha_{2}>\cdots>\alpha_{k},\alpha_{i}\in\mathbb{N}.
	\end{array}\right.
	\]
\end{solution}
\begin{extraproblem}[Lukács Andor]
	Egy $1000$ oldalú konvex sokszög belsejében felveszünk $n$ pontot.
	Jelöljük $\mathcal{H}$-val a sokszög csúcsaiból és a felvett pontokból
	álló halmazt. A sokszöget osszuk fel páronként diszjunkt belsejű \emph{üres}
	háromszögekre úgy, hogy a háromszögek csúcsa a $\mathcal{H}$-ból
	legyenek. Egy háromszöget \emph{üres} háromszögnek nevezünk, ha sem
	a belsejében, sem az oldalainak belsejében nem tartalmaz $\mathcal{H}$-beli
	pontot.
	\begin{itemize}
		\item[(a)] Létezik-e olyan $n$ érték, amelyre a felbontás $2025$ üres háromszögből
		áll? 
		\item[(b)] Milyen $k$ értékekre létezik olyan $n>0$, amelyre a felbontás $k$
		darab üres háromszögből áll? 
	\end{itemize}
	\begin{flushright}
		(OMMO, 2025, 2. forduló, IX. osztály) 
		\par\end{flushright}
\end{extraproblem}
\begin{solution}
	A következő módon járhatunk el:
	\begin{itemize}
		\item[(a)] Jelöljük $A_{1}$, $A_{2}$, \ldots , $A_{1000}$-rel a sokszög csúcsait
		és $B_{1}$, $B_{2}$, \ldots , $B_{n}$-nel a belső pontokat. Kiszámítjuk
		a felbontásban szereplő üres háromszögek szögeinek összege kétféleképpen.
		Ha a felbontás $2025$ háromszögből áll, akkor ez a szögösszeg $2025\cdot180^{\circ}$.
		Másrészt, az üres háromszögek szögeinek összege egyenlő a sokszög
		csúcsai körül, illetve a belső pontok körül létrejött szögek összegével.
		A sokszög $A_{i}$ csúcsa körül létrejött szögek összege éppen az
		$A_{i}$ szög. Így a sokszög csúcsai körül létrejött szögek összege
		a sokszög szögeinek összegével egyenlő, azaz $180^{\circ}\cdot(1000-2)$.
		Egy $B_{i}$ pont körül létrejött szögek összege $360^{\circ}$, tehát
		a $B_{1},B_{2},\ldots,B_{n}$ pontok körül létrejött szögek összege
		$n\cdot360^{\circ}$. A fentiek alapján, ha létezik $2025$ üres háromszögből
		álló felbontás, akkor 
		\[
		2025\cdot180^{\circ}=998\cdot180^{\circ}+n\cdot360^{\circ},
		\]
		ahonnan $2n+998=2025$. Az egyenlőség bal oldala páros, a jobb oldala
		páratlan, tehát nem létezik ilyen $n$ érték. 
		\item[(b)] A fenti gondolatmenet alapján 
		\[
		k\cdot180^{\circ}=998\cdot180^{\circ}+n\cdot360^{\circ},
		\]
		vagyis 
		\[
		2n+998=k.
		\]
		Ennek az egyenletnek minden 1000-nél nagyobb vagy egyenlő páros $k$
		értékre van megoldása. 
	\end{itemize}
\end{solution}
\begin{extraproblem}[Miklós Dóra]
	(Moszkvai Matematika Olimpia, 1969) A 0,1 számok segítségével felírunk
	$2^{n-1}$ $n$-jegyű különböző számsorozatot. Ha a felírt sorozatok
	közül bármelyik hármat összehasonlítjuk, akkor azokra teljesül, hogy
	létezik olyan $1\leq p\leq n$ pozitív egész szám, amelyre a vizsgált
	sorozatok $p$-dik jegye 1-es. Mutassuk meg, hogy ekkor létezik olyan
	$1\leq k\leq n$ pozitív egész szám is, amelyre az összes sorozat
	$k$-dik jegye 1-es és ez egyértelmű! 
\end{extraproblem}
\begin{solution}
	Jelöljük $S$-sel azt a halmazt, amely tartalmazza az összes $n$-jegyű
	csak a 0, 1-ből álló számsorozatot: 
	\[
	S=\{X|X=(x_{1},x_{2},\dots,x_{n}),\textrm{ ahol }x_{i}\in\{0,1\},i\in\{1,2,\dots,n\}\}.
	\]
	Továbbá vezessük be ezen a halmazon a szorzás műveletet az alábbiak
	szerint: 
	\[
	X=(x_{1},\dots,x_{n}),Y=(y_{1}.\dots,y_{n})\in S:X\cdot Y=(x_{1}\cdot y_{1},x_{2}\cdot y_{2},\dots,x_{n}\cdot y_{n})\in S.
	\]
	Bevezetünk még egy a konjugált műveletet is: 
	\[
	X=(x_{1},\dots,x_{n})\in S:\overline{X}=(\overline{x_{1}},\overline{x_{2}},\dots,\overline{x_{n}}),\textrm{ ahol }
	\]
	\[
	\forall i\in\{1,2,\dots,n\},\phantom{i}\overline{x_{i}}=\left\{ \begin{array}{ll}
		1, & \textrm{ha }x_{i}=0\\
		0, & \textrm{ha }x_{i}=1.
	\end{array}\right.
	\]
	A továbbiakban jelöljük $S_{0}$-val azt a halmazt, mely részhalmaza
	$S$-nek és eleget tesz a feltételeknek. Vizsgáljuk ennek a halmaznak
	a tulajdonságait.
	
	A feltételek alapján elmondhatjuk, hogy tetszőleges $X,Y,Z\in S_{0}$
	esetén $X\cdot Y\cdot Z\neq(0,0,\dots,0)$. Ugyanakkor ebből az is
	következik, hogy $(0,0,\dots,0)\notin S_{0}$, valamint az is, hogy
	bármely két $S_{0}$-beli tag szorzata sem lesz egyenlő $(0,0,\dots,0)$-val.
	Ezt a gondolatot folytatva, mivel a szerkesztés alapján bármely $X\in S$
	esetén $X\cdot\overline{X}=(0,0,\dots,0)$, ezért $X$ és $\overline{X}$
	közül legfennebb az egyik lehet benne $S_{0}$-ban. Tudjuk, hogy $|S_{0}|=2^{n-1}$,
	vagyis pontosan fele annyi eleme van, mint $S$-nek, ezért az $(X,\overline{X})$
	párokból pontosan az egyik benne is kell legyen $S_{0}$-ban ahhoz,
	hogy ne jussunk ellentmondáshoz.
	
	A következő észrevétel a szorzásra fog vonatkozni. Igazolni szeretnénk,
	hogy tetszőleges $X,Y\in S_{0}$ esetén $X\cdot Y$ is az $S_{0}$-nak
	eleme. A bizonyítást reductio ad absurdum végezzük el. Feltételezzük,
	hogy $X\cdot Y\notin S_{0}$. Ebből arra következtethetünk, hogy $\overline{X\cdot Y}\in S_{0}$.
	Vegyük a következő három elemét az $S_{0}$-nak: $X$, $Y$, $\overline{X\cdot Y}$.
	Az $S_{0}$ tulajdonsága miatt tudjuk, hogy $X\cdot Y\cdot\overline{X\cdot Y}\neq(0,0,\dots,0)$,
	de ez ellentmondáshoz vezet, mert 
	\[
	X\cdot Y\cdot\overline{X\cdot Y}=(X\cdot Y)\cdot\overline{X\cdot Y}=(0,0,\dots,0)
	\]
	a konjugáltak szorzása miatt. Tehát $X\cdot Y\in S_{0}$ valóban igaz.
	
	Most pedig, ha $S_{0}=\{X_{1},X_{2},\dots,X_{2^{n-1}}\}$ alakba írjuk
	fel, akkor vizsgáljuk az $X=X_{1}\cdot X_{2}\cdot...\cdot X_{2^{n-1}}$
	elemet. Elmondhatjuk, hogy $X\in S_{0}$ és ezért $X\neq(0,0,\dots,0)$.
	Ebből az következik, hogy létezik legalább egy olyan eleme az $X$
	sorozatnak, ami 1-es, de a szorzás miatt ez azt is jelenti, hogy $S_{0}$
	minden tagjának 1-esnek kell lennie abban a pozicióban. Tehát létezik
	olyan $k\in\{1,...,n\}$ pozició, ahol $S_{0}$ minden számsorozatában
	1-es áll. Utolsó lépésként azt látjuk be reductio ad absurdum módszerrel,
	hogy mi történik, ha több ilyen érték is van. Feltételezzük, hogy
	két ilyen érték is van, ami azt fogja jelenteni, hogy minden elem
	ugyanabban a két pozicióban 1-et vesz fel. Emiatt $S_{0}$ elemei
	csak a többi $n-2$ elemben térnek el, de így $S_{0}$-nak maximum
	csak $2^{n-2}$ tagja lehetne. Ez ellentmond annak, hogy $|S_{0}|=2^{n-1}$,
	tehát valóban egyértelmű az a $k$ érték. 
\end{solution}
\begin{extraproblem}[Seres Brigitta-Alexandra]
	Határozzuk meg az összes olyan $f:\mathbb{R}\rightarrow\mathbb{R}$
	függvényt, mely teljesíti a következő összefüggést 
	\[
	f(xf(x)+f(y))=(f(x))^{2}+y,\phantom{aa}\forall x,y\in\mathbb{R}.
	\]
	\begin{flushright}
		\textit{(XIV. Balkán Olimpia, 1997)} 
		\par\end{flushright}
\end{extraproblem}
\begin{solution}
	Helyettesítsük be $x=0$, így kapjuk, hogy $f(f(y))=(f(0))^{2}+y,\forall y\in\mathbb{R}$.
	Az $f(0)=a$ jelölést használjuk és az adott összefüggésbe $x$ helyett
	$f(x)$-t helyettesítünk, és felhasználva az előbbi eredményt:
	
	\begin{equation}
		\centering\begin{aligned}f(f(x)\cdot f(f(x))+f(y))= & f\left(f(x)\left(x+a^{2}\right)+f(y)\right)=\\
			= & {[f(f(x))]^{2}+y=\left(x+a^{2}\right)^{2}+y.}
		\end{aligned}
		\label{seresbrigi1}
	\end{equation}
	
	Mivel $f(f(y))=(f(0))^{2}+y,\forall y\in\mathbb{R}$ $\Rightarrow f$
	minden valós értéket felvesz, tehát létezik olyan $n\in\mathbb{R}$,
	hogy $f(n)=0$. Ezt helyettesítve az eredeti összefüggésbe kapjuk,
	hogy $f(f(y))=y,\phantom{a}\forall y\in\mathrm{R}$. Figyelembe véve
	az $f(f(y))=(f(0))^{2}+y,\forall y\in\mathbb{R}$ összefüggést, következik,
	hogy $f(0)=0$ (vagyis $n=0=a$ ).
	
	Ekkor a (\ref{seresbrigi1})-es egyenlőségek alapján 
	\[
	x^{2}+y=f(xf(x)+f(y))=(f(x))^{2}+y,\quad\forall y\in\mathbb{R}.
	\]
	
	Látható, hogy $(f(x))^{2}=x^{2},\phantom{a}\forall x\in\mathbb{R}$.
	Tehát létezik olyan $A\subseteq\mathbb{R}$, amelyre 
	\[
	f(x)=\left\{ \begin{array}{cc}
		x, & x\in A\\
		-x, & x\in\mathbb{R}\backslash A.
	\end{array}\right.
	\]
	
	Tegyük fel, hogy az $A\backslash\{0\}$ és az $\mathbb{R}^{*}\backslash A$
	halmaz nem üres, tehát létezik $x_{1}\in A\backslash\{0\}$ és $x_{2}\in\mathbb{R}^{*}\backslash A$.
	Ha ezt a két értéket helyettesítjük az eredeti összefüggésbe kapjuk,
	hogy 
	\[
	f(x_{1}f(x_{1})+f(x_{2}))=(f(x_{1}))^{2}+x_{2}\Rightarrow f\left(x_{1}^{2}-x_{2}\right)=x_{1}^{2}+x_{2}.
	\]
	Ha $x_{1}^{2}-x_{2}\in A\backslash\{0\}$, akkor $f\left(x_{1}^{2}-x_{2}\right)=x_{1}^{2}-x_{2}=x_{1}^{2}+x_{2}$,
	vagyis $x_{2}=-x_{2}$, amiből $x_{2}=0$. Ha $x_{1}^{2}-x_{2}\in\mathbb{R}^{*}\backslash A$,
	akkor $f\left(x_{1}^{2}-x_{2}\right)=-x_{1}^{2}+x_{2}=x_{1}^{2}+x_{2}$,
	vagyis $x_{1}^{2}=-x_{1}^{2}$, amiből $x_{1}=0$. De ezen esetek
	nem lehetségesek, mert $x_{1,2}\neq0$, hiszen $x_{1}\in A\backslash\{0\}$
	és $x_{2}\in\mathbb{R}^{*}\backslash A$.
	
	Tehát csak az $f(x)=x$ és $f(x)=-x$ függvények teljesítik a megadott
	egyenletet.
\end{solution}
\begin{extraproblem}[Seres Brigitta-Alexandra]
	Az $f:\mathbb{N}^{*}\rightarrow\mathbb{N}^{*}$ szigorúan növekvő
	függvény teljesíti a következő egyenlőtlenséget 
	\[
	f(1)+f(2)+\ldots+f(n)\geq\frac{f(n)\cdot(1+f(n))}{2},\phantom{a}\forall n\in\mathbb{N}^{*}.
	\]
	Bizonyítsd be, hogy $f(n)=n,\phantom{a}\forall n\in\mathbb{N}^{*}.$
	\begin{flushright}
		\textit{(András Szilárd, Kovács Lajos: Függvényegyenletek, 119 oldal/18.feladat)} 
		\par\end{flushright}
\end{extraproblem}
\begin{solution}
	\textbf{n=1} esetén teljesül, hogy $f(1)\geq\frac{f(1)(1+f(1))}{2}$,
	amit végigosztva $f(1)\neq0$ (mert az értékkészlete a függvénynek
	nem tartalmazza a nullát), kapjuk hogy $2\geq1+f(1)$, így $f(1)\leq1.$
	De a függvény értékkészlete $\mathbb{N}^{*}$, vagyis minden $n\in\mathbb{N}^{*}$
	elemre igaz, hogy $f(n)\geq1$, így $f(1)\geq1$ is. Az előbbi két
	állításból következik, hogy $f(1)=1.$
	
	\textbf{n=2} esetén a feltétel és előbbi eredmény szerint 
	\begin{align*}
		f(1)+f(2)=1+f(2) & \geq\frac{(1+f(2))f(2)}{2}\Rightarrow\frac{(1+f(2))f(2)-2(1+f(2))}{2}\leq0\\
		& \Rightarrow(f(2)+1)(f(2)-2)\leq0.
	\end{align*}
	Mivel $f(2)+1>0$, így $f(2)\leq2$ szükséges teljesüljön. De $f$
	szigorúan növekvő és $f(1)=1$, tehát $f(2)\geq2$. Ez csak akkor
	teljesül, ha $f(2)=2$.
	
	A matematikai indukció segítségével igazoljuk, hogy $f(n)=n,\forall n\in\mathbb{N}^{*}.$
	
	Fennebb beláttuk, hogy $n=1$-re és $n=2$-re teljesül az indukciós
	feltevés. Tegyük fel, hogy $f(k)=k,\quad\forall k\leq n$, és ekkor
	lássuk be azt, hogy $f(n+1)=n+1$ is teljesül.
	
	Az adott egyenlőtlenséget $(n+1)$-re írva fel majd $k\leq n$-re
	az $f(k)=k$ értékeket behelyettesítve (indukciós feltevést használjuk
	fel), kapjuk hogy 
	\begin{align*}
		\centering & f(1)+f(2)+\ldots+f(n)+f(n+1)=1+2+\ldots+n+f(n+1)\\
		& =\frac{n(n+1)}{2}+f(n+1)\geq\frac{f(n+1)(1+f(n+1))}{2}.
	\end{align*}
	Innen 
	\begin{align*}
		\frac{f(n+1)(1+f(n+1))}{2}-\frac{n(n+1)+2f(n+1)}{2} & \leq0\\
		\Rightarrow f(n+1)+f^{2}(n+1)-n(n+1)-2f(n+1) & \leq0.
	\end{align*}
	Vagyis $f^{2}(n+1)-f(n+1)-n(n+1)\leq0$, ahonnan 
	\[
	(f(n+1)+n)(f(n+1)-(n+1))\leq0.
	\]
	Mivel $f(n+1)\geq1$, ezért $f(n+1)+n\geq1>0$, így következik, hogy
	$f(n+1)\leq n+1$. Másrészt a függvény szigorúan növekvő, ezért $f(n+1)>f(n)=n$,
	tehát $f(n+1)\geq n+1$. Az előbbi két egyenlőtlenség alapján $f(n+1)=n+1$
	valóban teljesül.
	
	Tehát a matematikai indukció elve alapján beláttuk, hogy $f(n)=n,\quad\forall n\in N^{*}$
	az egyetlen függvény ami teljesíti a feladat kérelmeit, így ez a függvény
	a feladat egyetlen megoldása. 
\end{solution}
\begin{extraproblem}[Sógor Bence]
	Létezik-e két olyan $A,B$ nem negatív számokból álló végtelen elemű
	halmaz, hogy bármely nem negatív egész szám felírható egyértelműen
	$A$ és $B$ egy-egy elemének összegeként. (Kürschák József Matematikai
	Tanulóverseny 1966. évi 3. feladata)
\end{extraproblem}
\begin{solution}
	Létezik két ilyen halmaz:
	
	Legyen $A$ azon elemek halmaza, amelyek jobbról balra minden páros
	pozíción nullást tartalmaznak. (pl. 10504)
	
	Legyen $B$ azon elemek halmaza, amelyek jobbról balra minden páros
	pozíción nullást tartalmaznak. (pl. 3070)
	
	A nulla legyen eleme mindkét halmaznak.\\
	
	Mind a két halmaznak jól láthatóan végtelen sok eleme van.
	
	A nulla felírható egyértelműen a két nulla összegeként.
	
	Az összes többi egész szám felbontható egy $A$-beli és egy $B$-beli
	elem összegére.
	
	Például, ha 
	\[
	c=\overline{c_{n}c_{n-1}\dots c_{3}c_{2}c_{1}}\in\mathbb{N},\text{ akkor }a=\overline{\dots c_{3}0c_{1}}\in A,b=\overline{\dots c_{4}0c_{2}0},a+b=c
	\]
	
	Feltételezzük, hogy két összeg egyenlő. Mivel a páratlan helyen levő
	értékekért csak az $A$-beli szám felel, ezért az összeadások $A$-beli
	tagjai megegyeznek. Hasonlóan a $B$-beli tagok is megegyeznek. Tehát
	$A$ és $B$ halmazok teljesítik a feladat feltételeit.
\end{solution}
\begin{extraproblem}[Szabó Kinga]
	Adjunk meg olyan, 100-nál kisebb pozitív egészekből álló halmazt,
	amelyre teljesül, hogy a belőlük készíthető összegek mind különbözőek.
	(Az összeg lehet egytagú is, és vehetjük az összes szám összegét is.)
	Legfeljebb hány számból állhat ez a halmaz? (Róka Sándor: Válogatás
	Erdős Pál kedvenc feladataiból) 
\end{extraproblem}
\begin{solution}
	Nézzük, 100-ig hogyan tudunk számokat választani úgy, hogy a belőlük
	képzett összegek különbözőek legyenek.
	
	7 számot választhatunk: $1,2,4,8,16,32,64$.
	
	Más lehetőségek: 
	\[
	1,3,5,11,23,48,99
	\]
	
	\[
	1,3,5,10,21,41,85
	\]
	
	\[
	2,3,6,12,25,50,99
	\]
	
	\[
	20,31,37,40,42,43,44
	\]
	
	Választhatunk 8 számot is?
	
	A 2-hatványok sorozata gazdaságos sorozat, egy adott értéket nem kaphatunk
	meg kétféle összeg eredményeként. Vegyük az $1,2,4,8,16,32$ számok
	3-szorosát. Ez is gazdaságos sorozat, ha az eredeti az volt. A belőlük
	kapható összegek a $3\cdot63=189$-nél nem nagyobb 3-mal osztható
	számok.
	
	Vegyük a sorozathoz a 95 és 97 számokat, az így kapott 8 szám is gazdaságos
	sorozat. A már meglévő összegeken kívül a következő új összegeket
	kaphatjuk.
	
	A 97 szerepel tagként, a 95 pedig nem. Ezek az összegek 3-mal osztva
	2 maradékot adnak.
	
	Mind a két új szám, a 95 és 97 is szerepel bennük tagként. Az ilyen
	összegek oszthatók 3-mal, és nem kisebbek $95+97=192$-nél, és ezek
	az összegek is páronként különbözőek.
	
	Tehát van 8 ilyen szám: 3, 6, 12, 24, 48, 95, 96, 97.
	
	Van másik, jobb számhalmaz, ahol a legnagyobb szám kisebb 97-nél:
	40, 60, 71, 77, 80, 82, 83, 84.
	
	Van-e 9, van-e 10 szám?
	
	10 szám nincs.
	
	Ha lenne 10 db legfeljebb kétjegyű pozitív egészből álló halmaz, akkor
	a belőlük képezhető összegek száma kisebb $10\cdot99=990$-nél. Azonban
	10 számból $2^{10}-1=1023$ összeg készíthető, így a skatulyaelv miatt
	van két összeg, amelynek ugyanaz az értéke.
	
	9 számot sem tudunk választani.
	
	Az $a_{1}<a_{2}<\dots<a_{9}$ pozitív egészekre teljesül, hogy a belőlük
	készíthető (legalább egy, legfeljebb kilenc különböző tagból álló)
	összegek mind különbözőek.
	
	Bizonyítsuk be, hogy $a_{9}>100$.
	
	Tegyük fel, hogy $a_{9}\leq100$.
	
	Belátjuk, hogy az $a_{9}$-et nem használó összegek között van kettő,
	amelyek különbsége $a_{9}$.
	
	Legyen $S$ azoknak a legfeljebb öttagú összegeknek a halmaza, amelyek
	az $a_{1},a_{2},\dots,a_{8}$ számokból készíthetők, és nem kisebbek
	$a_{4}$-nél.
	
	\[
	\binom{8}{1}+\binom{8}{2}+\binom{8}{3}+\binom{8}{4}+\binom{8}{5}=218
	\]
	
	Összesen 218 legfeljebb öttagú összeg van, ezek közül csak azok lehetnek
	kisebbek $a_{4}$-nél, amelyekben csak $a_{1},a_{2}$ és $a_{3}$
	szerepel, összesen 7 darab.
	
	Tehát az összegekből legalább $218-7=211$ akkora, mint $a_{4}$.
	
	Az összegek között $a_{9}\leq100$ miatt van három olyan, amelyek
	$a_{9}$-cel osztva ugyanazt a maradékot adják.
	
	A legnagyobb összeg $a_{4}+a_{5}+a_{6}+a_{7}+a_{8}<a_{4}+4\cdot a_{9}.$.
	
	Az $S$ összegei közül a legkisebb $a_{4}$, ezért bármely két összeg
	különbsége kisebb $4\cdot a_{9}$-nél, így a három azonos maradékú
	összeg között van kettő, amelyek különbsége $a_{9}$.
	
	A két összegből a kisebbhez $a_{9}$-et adva a nagyobbat kapjuk. Itt
	az ellentmondás.
\end{solution}
\begin{extraproblem}[Száfta Antal]
	Legyen $A_{1},A_{2},\dots,A_{n}$ egy halmazrendszer, ahol minden
	$A_{i}$ részhalmaza egy véges $X$ halmaznak. Tegyük fel, hogy az
	alábbi két feltétel teljesül:
	\begin{enumerate}
		\item Minden $A_{i}$ halmaz legfeljebb $k$ elemet tartalmaz, azaz $|A_{i}|\leq k$.
		\item Bármely két halmaz metszetének elemszáma legfeljebb $t$, azaz $|A_{i}\cap A_{j}|\leq t$
		minden $i\neq j$ esetén.
	\end{enumerate}
	Bizonyítsuk be, hogy az összes halmaz együttes elemszámára fennáll
	az alábbi egyenlőtlenség:
	
	\[
	\left|\bigcup_{i=1}^{n}A_{i}\right|\geq\frac{nk}{k-t}.
	\]
\end{extraproblem}
\begin{solution}
	Legyen $S=\bigcup_{i=1}^{n}A_{i}$, azaz az összes halmaz egyesítése.
	Célunk annak bizonyítása, hogy 
	\[
	|S|\geq\frac{nk}{k-t}.
	\]
	\begin{enumerate}
		\item Elemösszegzési módszer:\\
		Tekintsük az alábbi kétféle megszámlálási módot:\\
		\emph{Első megszámlálás}: Minden egyes $A_{i}$ legfeljebb $k$ elemet
		tartalmaz, tehát az összes halmaz együttvéve összesen legfeljebb $nk$
		elemet vesz figyelembe. \\
		\emph{Második megszámlálás}: Minden egyes $x\in S$ elemet annyiszor
		számolunk meg, ahány különböző $A_{i}$-ben előfordul. Jelöljük ezt
		az előfordulási számot $d_{x}$-szel: 
		\[
		d_{x}=|\{i\mid x\in A_{i}\}|.
		\]
		\item A metszési korlát alkalmazása:\\
		Tegyük fel, hogy egy konkrét $x\in S$ szerepel az $A_{i_{1}},A_{i_{2}},\dots,A_{i_{m}}$
		halmazokban, vagyis $d_{x}=m$. Mivel minden két halmaz átfedése legfeljebb
		$t$ lehet, ezért ha összeadjuk az összes halmaz elemszámát, akkor:
		\[
		\sum_{x\in S}d_{x}=\sum_{i=1}^{n}|A_{i}|\leq nk.
		\]
		Másrészt az egyes elemek előfordulási száma nem lehet tetszőlegesen
		nagy. Egy jól ismert kombinatorikai becslés alapján:
		\[
		\sum_{x\in S}d_{x}^{2}\leq nkd_{x}.
		\]
		A Cauchy-egyenlőtlenség szerint:
		\[
		|S|\geq\frac{\left(\sum_{x\in S}d_{x}\right)^{2}}{\sum_{x\in S}d_{x}^{2}}.
		\]
		A fentiek felhasználásával:
		\[
		|S|\geq\frac{(nk)^{2}}{nkd_{x}}=\frac{nk}{d_{x}}.
		\]
		Mivel $d_{x}\leq k-t$, így
		\[
		|S|\geq\frac{nk}{k-t}.
		\]
		\emph{Konklúzió:} Megmutattuk, hogy az egyesített halmaz elemszáma
		alsó korlátként követi az
		\[
		|S|\geq\frac{nk}{k-t}
		\]
		összefüggést, ezzel bizonyítva az állítást.
	\end{enumerate}
\end{solution}
\begin{extraproblem}[Gál Tamara]
	Jelölje $r_{n}$ az $\{1;2;\ldots;n\}$ halmaz azon részhalmazainak
	számát, amelyek nem tartalmaznak szomszédos számokat, ahol az 1-et
	és az n-et is szomszédosnak tekintjük. Határozzuk meg $r_{16}$ értékét.
	(OKTV 2010/2011; II. kategória, döntő)
\end{extraproblem}
\begin{solution}
	Nevezzük a feladat feltételeit teljesítő részhalmazokat megfelelőnek.
	Először kisebb n-ekre meghatározzuk $r_{n}$ értékét.\\
	$r_{1}=2$, a megfelelő részhalmazok $\varnothing$ és $\{1\}$.\\
	$r_{2}=3$, a megfelelő részhalmazok $\varnothing,\{1\}$ és $\{2\}$.\\
	$r_{3}=4$, a megfelelő részhalmazok $\varnothing,\{1\},\{2\}$ és
	$\{3\}$.\\
	$r_{4}=7$, a megfelelő részhalmazok $\varnothing,\{1\},\{2\},\{3\},\{4\},\{1;3\}$
	és $\{2;4\}$.\\
	$r_{5}=11$, a megfelelő részhalmazok $\varnothing,\{1\},\{2\},\{3\},\{4\},\{5\},\{1;3\},\{2;4\},\{3;5\},\{1;4\}$
	és $\{2;5\}$.
	
	Az első néhány értékből azt sejthetjük, hogy $n\geq2$ esetén $r_{n+2}=r_{n+1}+r_{n}$.
	Ehhez három részre bontjuk az $n+2$ elemű halmaz megfelelő részhalmazait:
	\begin{itemize}
		\item Ha a részhalmaz nem tartalmazza $n+2$-t, továbbá nem tartalmazza
		egyszerre 1-et és $n+1$-et, akkor éppen az első $n+1$ elem megfelelő
		részhalmazait kapjuk, ezek száma $r_{n+1}$. 
		\item Ha a részhalmaz nem tartalmazza $n+2-t$, de tartalmazza 1-et és $n+1$-et
		is, akkor biztosan nem tartalmazza $n$-et. Ha elhagyjuk ezen részhalmazokból
		$n+1$-et, akkor megkapjuk az első $n$ elem olyan megfelelő részhalmazait,
		amelyek tartalmazzák 1-et. 
		\item Ha a részhalmaz tartalmazza $n+2$-t, akkor nem tartalmazza 1-et és
		$n+1$-et. Ha elhagyjuk ezen részhalmazokból $n+2-t$, akkor megkapjuk
		az első $n$ elem olyan megfelelő részhalmazait, amelyek nem tartalmazzák
		1-et. Az előző esettel együtt ezen megfelelő részhalmazok száma $r_{n}$. 
	\end{itemize}
	Ezzel beláttuk az $r_{n+2}=r_{n+1}+r_{n}$ összefüggést. A kapott
	rekurzió alapján sorra kiszámíthatjuk az $r_{n}$ értékeket:
	\begin{center}
		\begin{tabular}{|c|c|c|c|c|c|c|c|c|c|c|c|c|c|}
			\hline 
			$n$ & 4 & 5 & 6 & 7 & 8 & 9 & 10 & 11 & 12 & 13 & 14 & 15 & 16\tabularnewline
			\hline 
			$r_{n}$ & 7 & 11 & 18 & 29 & 47 & 76 & 123 & 199 & 322 & 521 & 843 & 1364 & 2207\tabularnewline
			\hline 
		\end{tabular}
		\par\end{center}
	Vagyis $r_{16}=2207$.\\
	
\end{solution}
\begin{extraproblem}[Gál Tamara]
	Legyen $A=\{1;2;3;4;5\}$ és $B=\{1;2;3\}$. Az $f(x)$ függvény
	értelmezési tartománya $A$, és minden $A$-beli $x$ esetén $f(x)\in A$.
	Hány $f(x)$ függvényre teljesül, hogy $\{f(f(x))\mid x\in A\}=B$
	? (OKTV 2011/2012; II. kategória, döntő) 
\end{extraproblem}
\begin{solution}
	Először megmutatjuk, hogy a keresett $f$ függvény az 1, 2, 3 számok
	egyikéhez sem rendelhet sem 4-et, sem 5-öt. Ezt az $f(1)=4$ példán
	szemléltetjük. Mivel $f(f(x))$ értékkészlete $B$, ezért van olyan
	$y\in A$, amelyre $f(f(y))=1$. Legyen ekkor $f(y)=z$ (ahol $z\in A$
	), de így $f(f(z))=f(1)=4$ lenne, ami nem eleme $B$-nek, és így
	nem lehetséges. Hasonlóan bizonyítható az álítás többi része is.
	
	Most megvizsgáljuk, hogy mi lehet $f(4)$ és $f(5)$ értéke. Nem lehet
	$f(4)=4$, hiszen ekkor $f(f(4))=4$ lenne, ami nem eleme $B$-nek.
	Ugyanígy $f(5)=5$ sem lehetséges. Innentől két esetet különböztetünk
	meg:\\
	(a) $f(4)$ és $f(5)$ is $B$-beli elem.\\
	(b) $f(4)$ és $f(5)$ közül (legalább) az egyik nem $B$-beli, azaz
	$f(4)=5$ vagy $f(5)=4$. Ekkor $f(4)=5$ esetén $f(f(4))\in B$ miatt
	$f(5)\in B$; míg $f(5)=4$ esetén $f(4)\in B$.
	
	A továbbiakban a keresett függvényeket három csoportra osztva számoljuk
	össze.
	\begin{enumerate}
		\item csoport: $f(1),f(2)$ és $f(3)$ értéke az $1,2,3$ számok valamely
		permutációja, továbbá $f(4)$ és $f(5)$ értéke (a) szerint $B$-beli.
		Ekkor a permutáció 3!-féle lehet, míg $f(4)$ és $f(5)$ értéke egymástól
		függetlenül $3-3$-féle lehet, tehát ilyen függvényből $3!3\cdot3=54$
		darab van. 
		\item csoport: $f(1),f(2)$ és $f(3)$ értéke az $1,2,3$ számok valamely
		permutációja, továbbá $f(4)$ és $f(5)$ értéke (b) szerint alakul.
		Ekkor a permutáció 3!-féle lehet. Ezután ki kell választanunk, hogy
		4 vagy 5 képe legyen $B$-beli (erre 2 lehetőségünk van), majd a kép
		értéke 3 -féle lehet. Tehát az ilyen függvények száma $3!\cdot2\cdot3=36$. 
		\item csoport: $f(1),f(2)$ és $f(3)$ értéke úgy kerül ki az $1,2,3$ számok
		közül, hogy vannak közöttük azonos értékek. Először megmutatjuk, hogy
		nem lehet mindhárom érték azonos. Ha ugyanis $f(1)=f(2)=f(3)$ lenne,
		akkor $f(f(x))$ értékkészletében (a) esetén csak egy, (b) esetén
		csak két szám állna. Vagyis a 3. csoportban $f(1),f(2)$ és $f(3)$
		értéke közül kettő azonos (jelölje ezt\\
		$k$ ), egy pedig ezektől különböző (jelölje ezt $l$ ). Az (a) eset
		nem lehetséges, hiszen ekkor $f(f(x))$ érékkészlete csak két számból
		állna. A (b) esetben $f(4)$ és $f(5)$ valamelyike $B$-beli, mégpedig
		$B$-nek a $k$-tól és $l$-től különböző $m$ eleme, hiszen csak
		így lesz $f(f(x))$ értékkészlete a teljes $B$. Az ilyen függvények
		esetén $l$ értéke 3 -féle lehet, $m$ értéke 2 -féle. Továbbá 3-féle
		választásunk van, hogy az 1, 2, 3 számok közül melyikhez rendeljük
		l-et, és 2-féle választásunk, hogy a (b) esetben a 4-hez vagy az 5
		-höz rendeljük $m$-et. Vagyis ebbe a csoportba összesen $3\cdot2\cdot3\cdot2=36$
		függvény tartozik. 
	\end{enumerate}
	Így összesen $54+36+36=126$ függvény teljesíti a feladat feltételeit. 
\end{solution}
\begin{extraproblem}[Czofa Vivien]
	Adott az $A=\left\{ 3^{x}\,\middle|\,3^{n+1}<3^{x}\leq3^{n+6},\ x\in\mathbb{N}\right\} $,
	a 3 bizonyos hatványainak halmaza.  
\begin{enumerate}
	\item Hány eleme van az $A$ halmaznak? 
	\item Határozzuk meg $n$-et, ha az $A$ halmaz növekvő sorrendbe helyezett
	elemei közül a harmadik szám 2187. 
\end{enumerate}
\textit{(XXI. Vályi Gyula emlékverseny -- Marosvásárhely -- 2015)}
\end{extraproblem}
\begin{solution}
	~
	\begin{enumerate}
		\item A $3^{x}$ akkor szigorúan nagyobb a $3^{n+1}$-nél, ha $x$ szigorúan
		nagyobb $n+1$-nél. Ugyanakkor $3^{x}\leq3^{n+6}$, tehát $x\leq n+6$.
		Ez azt jelenti, hogy $x$ a következő értékeket veheti fel: 
		\[
		x\in\{n+2,n+3,n+4,n+5,n+6\}
		\]
		Tehát az $A$ halmaznak 5 eleme van.
		\item A halmaz harmadik eleme a $3^{n+4}$, a 2187 pedig a 3-nak a hetedik
		hatványa, ebből következik, hogy $n+4=7$, vagyis $n=3$. 
	\end{enumerate}
\end{solution}
\begin{extraproblem}[Czofa Vivien]
	Adott az $A=\{1,2,\dots,99,100\}$ halmaz. Határozzuk meg, hány 5040-nel
	egyenlő összeget alkothat az $A$ halmaz elemeiből! \textit{(Matlap
		2012/2. szám -- V. osztály)}
\end{extraproblem}
\begin{solution}
	Adjuk össze az $A$ halmaz elemeit: 
	\[
	1+2+3+\cdots+100=\frac{101\cdot100}{2}=5050
	\]
	
	Ahhoz, hogy az összeg 5040 legyen, a teljes összegből el kell vennünk
	10-et. Hányféleképpen alakulhat ki a 10-es összeg? A következő esetekben:
	
	\begin{align*}
		10,\quad & 1+9,\quad2+8,\quad3+7,\quad4+6,\\
		& 1+2+7,\quad1+3+6,\quad1+4+5,\quad2+3+5,\\
		& 1+2+3+4
	\end{align*}
	
	Tehát a válaszunk: 10 lehetséges eset van arra, hogy az $A$ halmaz
	elemeiből 5040-nel egyenlő összeget kapjunk.
\end{solution}
\begin{extraproblem}[Domokos Ábel]
	Egy $n$ elemű halmaznak legfeljebb hány részhalmaza adható meg úgy,
	hogy bármelyik kettő metszete nem üres halmaz legyen?
\end{extraproblem}
\begin{solution}
	Legyen $A$ egy $n$-elemű halmaz, ekkor $|P(A)|=2^{n}$.
	
	Legyen $X\in P(A)$ tetszőleges. Ekkor létezik $C_{A}X$(X komplementere),
	melyre teljesül, hogy $X\cap C_{A}X=\emptyset$. Ezeket párba rendezhetjük
	és kapjuk, hogy A elemei $2^{n-1}$ párba rendezhetők úgy, hogy a
	párbeli elemek diszjunktak egymással.
	
	Lássuk be, hogy legfeljebb $2^{n-1}$ halmaz választható ki a feladat
	feltételeinek megfelelően.
	
	Először bebizonyítjuk, hogy létezik egy ilyen választás. Legyen $a\in A$.
	Ekkor válasszuk ki az $a$-t tartalmazó $A$ részhalmazait, ezekből
	$2^{n-1}$ különböző halmaz van. Bármely két ilyen részhalmaz tartalmazza
	az $a$-t, tehát metszetük nem üres. Kész vagyunk a szerkesztéssel.
	
	Lássuk be, hogy nem létezik $2^{n-1}$-nél nagyobb elemű megfelelő
	választás. \\
	Red. ad. abs: tegyük fel, hogy mégis létezik. Mivel $P(A)$ elemei
	$2^{n-1}$ párba rendezhetők, ezért ha több, mint $2^{n-1}$ különböző
	részhalmazt választunk, akkor skatulyaelv alapján biztos, hogy kiválasztunk
	két elemet, amelyek párban vannak. Ezek diszjunktak lesznek, azonban
	ezzel ellentmondásba ütköztünk, mivel azt feltételeztük a választásról,
	hogy diszjunkt halmazokat választottunk ki.
	
	Tehát beláttuk, hogy legfeljebb $2^{n-1}$ különböző halmazt lehet
	kiválasztani $P(A)$-ból.
\end{solution}

