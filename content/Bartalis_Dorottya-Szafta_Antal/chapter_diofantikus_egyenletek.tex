\chapter{Diofantikus egyenletek}\label{chap:dio}
\section{Lineáris diofantikus egyenletek, Pitagoraszi szám\-hár\-ma\-sok, Pell-egyenlet}\label{sec:lindio}
\begin{description}
{\large \item [{Szerző:}] Bartalis Dorottya (Didaktikai mesteri -- Matematika, I. év) }
\end{description}
A \textbf{Diofantikus egyenletek} olyan algebrai egyenletek, amelyek
megoldásait az \textit{egész számok} halmazán keressük. Ezek az egyenletek
több ismeretlennel rendelkezhetnek, és megoldásaik szempontjából változóak:
vagy nincs megoldásuk, vagy végtelen sok van. A diofantikus egyenletek
vizsgálata a számelmélet egyik fontos területe, amelynek gyökerei
az ókori görög matematikához, különösen Diofantosz munkásságához nyúlnak
vissza. A következő fejezet keretén belül az alábbi 3 típusú diofantikus
egyenlettel fogunk foglalkozni:
\begin{itemize}
\item Lineáris diofantikus egyenletek, 
\item Pitagoraszi számhármasok, 
\item Pell-egyenlet. 
\end{itemize}

\subsection*{Lineáris diofantikus egyenletek}
\begin{definition}{def:diofantikus-altalanos}
A lineáris diofantikus egyenlet általános alakja:
\[
ax+by=c,
\]
ahol $a,b,c\in\mathbb{Z}$ adott egész számok, és az ismeretlenek
$x,y\in\mathbb{Z}$ egész számok. 
\end{definition}

\begin{theorem}{thm:axbyc}
Az $ax+by=c$ egyenletnek akkor és csakis akkor van egész számokból álló megoldása, ha az $a$ és $b$ legnagyobb közös osztója (lnko-ja) osztója
$c$-nek is, vagyis
\[
\mathrm{lnko}(a,b)\mid c.
\]
\end{theorem}

\begin{proof}
Legyen $d=\mathrm{lnko}(a,b)$. Ekkor léteznek egész számok $m,n$,
amelyekre teljesül a Bézout-azonosság:
\[
am+bn=d.
\]
Ha $d\mid c$, akkor létezik olyan egész szám $k$, hogy $c=d\cdot k$.
Ezt behelyettesítve kapjuk:
\[
a(mk)+b(nk)=dk=c.
\]
Tehát $x_{0}=mk$ és $y_{0}=nk$ egy konkrét megoldása az egyenletnek.

Fordított irányban, ha létezik megoldás $x_{0},y_{0}$, akkor
\[
ax_{0}+by_{0}=c,
\]
és mivel $d=\mathrm{lnko}(a,b)$ minden $a$-nak és $b$-nek közös
osztója, ezért $d\mid ax_{0}$ és $d\mid by_{0}$, így $d\mid(ax_{0}+by_{0})$,
tehát
\[
d\mid c.
\]
Ez bizonyítja, hogy a megoldhatóság feltétele és elégséges feltétele
az, hogy az $a$ és $b$ legnagyobb közös osztója osztója legyen $c$-nek. 
\end{proof}
\begin{theorem}{thm:axbyc1}
Adott az
\[
ax+by=c,
\]
egyenlet, ahol $a,b,c\in\mathbb{Z}$, és tudjuk, hogy létezik egy
konkrét megoldás $(x_{0},y_{0})$, tehát
\[
ax_{0}+by_{0}=c.
\]
Ekkor az összes egész számokból álló megoldás a következő alakban
írható fel:
\[
x=x_{0}+\frac{b}{d}t,\quad y=y_{0}-\frac{a}{d}t,
\]
ahol $d=\mathrm{lnko}(a,b)$, és $t$ tetszőleges egész szám.
\end{theorem}

\begin{proof}
Először tekintsük az egyenlet homogén változatát:
\[
ax+by=0.
\]
Minden megoldás ebben az esetben
\[
x=\frac{b}{d}t,\quad y=-\frac{a}{d}t,
\]
ahol $t\in\mathbb{Z}$.

Most, ha $(x_{0},y_{0})$ egy konkrét megoldása az eredeti egyenletnek,
akkor bármely megoldás $(x,y)$ felírható úgy, hogy
\[
ax+by=c,
\]
és ebből
\[
a(x-x_{0})+b(y-y_{0})=0.
\]

Tehát az $(x-x_{0},y-y_{0})$ pár a homogén egyenlet megoldása.

Ez alapján
\[
x-x_{0}=\frac{b}{d}t,\quad y-y_{0}=-\frac{a}{d}t,
\]
azaz
\[
x=x_{0}+\frac{b}{d}t,\quad y=y_{0}-\frac{a}{d}t,
\]
ahol $t\in\mathbb{Z}$.

Ez mutatja, hogy az összes megoldás ezekből az értékekből áll előállítható. 
\end{proof}

\subsection*{Az Euklideszi algoritmus}

Az \textbf{Euklideszi algoritmus} egy hatékony módszer két egész szám
legnagyobb közös osztójának ($\mathrm{lnko}$) meghatározására.
\begin{theorem}{thm:euk-algo}
Adottak $a,b\in\mathbb{Z}$, nem mindkettő nulla.

Az algoritmus lépései a következők:
\begin{enumerate}
\item Osszuk el a nagyobb számot a kisebbel, és vegyük a maradékot: 
\[
a=bq_{1}+r_{1},\quad0\le r_{1}<|b|.
\]
\item Ha $r_{1}=0$, akkor $\mathrm{lnko}(a,b)=|b|$. 
\item Ha nem, ismételjük a műveletet a $b$ és $r_{1}$ számokra: 
\[
b=r_{1}q_{2}+r_{2},\quad0\le r_{2}<r_{1}.
\]
\item Folytatjuk a maradékos osztásokat addig, míg valamely maradék nulla
nem lesz: 
\[
r_{k-2}=r_{k-1}q_{k}+r_{k},\quad r_{k}=0.
\]
\end{enumerate}
Ekkor $\mathrm{lnko}(a,b)=r_{k-1}$, az utolsó nem nulla maradék. 
\end{theorem}


\subsection*{A bővített Euklideszi algoritmus}

A bővített változat nemcsak a legnagyobb közös osztót adja meg, hanem
megtalálja azokat az egész számokat $x$ és $y$, amelyek kielégítik
a Bézout-egyenletet:

\[
ax+by=\mathrm{lnko}(a,b).
\]

Ez kulcsfontosságú, mert így kapunk egy konkrét megoldást a lineáris
diofantikus egyenlet homogén változatára.

\medskip{}

Az algoritmus lépései során visszafelé kifejezzük az egyes maradékokat
az $a$ és $b$ lineáris kombinációjaként, végül megkapjuk az $x$
és $y$ értékeket.

\medskip{}

\begin{theorem}{thm:lnko-dio}
Ha $\mathrm{lnko}(a,b)=d$, akkor léteznek olyan egész $x_{0},y_{0}$,
amelyekre

\[
ax_{0}+by_{0}=d.
\]
\end{theorem}

\begin{problem}
Oldjuk meg az alábbi lineáris diofantikus egyenletet:

\[
8x+3y=6.
\]
\end{problem}

\begin{solution}
\textbf{1. lépés:} Ellenőrizzük, hogy van-e megoldás.

Számoljuk ki az $8$ és $3$ legnagyobb közös osztóját ($\mathrm{lnko}(8,3)$)
az Euklidészi algoritmussal:

\[
8=3\times2+2,
\]
\[
3=2\times1+1,
\]
\[
2=1\times2+0.
\]

Az utolsó nem nulla maradék $1$, tehát $\mathrm{lnko}(8,3)=1$.

Mivel $1\mid6$, azaz 1 osztja 6-ot, létezik megoldás az egyenletre.

\medskip{}

\textbf{2. lépés:} Keressünk egy sajátos megoldást az

\[
8x+3y=1
\]

egyenletre a bővített Euklideszi algoritmus segítségével.

Az előző lépésekből visszafelé fejezzük ki a maradékokat:

\[
1=3-2\times1,
\]

de

\[
2=8-3\times2,
\]

tehát

\[
1=3-(8-3\times2)\times1=3\times3-8\times1.
\]

Ez azt jelenti, hogy

\[
x_{0}=-1,\quad y_{0}=3
\]

egy sajátos megoldás az

\[
8x+3y=1
\]

egyenletre.

\medskip{}

\textbf{3. lépés:} Mivel az eredeti egyenlet jobb oldala $6$, azaz
$\frac{6}{1}=6$-szorosa az előzőnek, az $(x,y)$ megoldás:

\[
x=x_{0}\times6=-6,\quad y=y_{0}\times6=18.
\]

\medskip{}

\textbf{Összefoglalva:}

Egy sajátos megoldás az

\[
8x+3y=6
\]

egyenletre:

\[
(x,y)=(-6,18).
\]

Az összes megoldás pedig a következőképpen adódik:

\[
(x,y)=(-6+3t,\,18-8t),\quad t\in\mathbb{Z}.
\]
\end{solution}
\begin{problem}
Egy piacon 12 kg-os és 7 kg-os almás ládákat árulnak. Egy kereskedő
127 kg almát akar vásárolni. Csak egész ládákat tehet a vásárlásába.
Határozd meg, hányféle módon állíthatja össze a ládákat úgy, hogy
pontosan 127 kg almát kapjon! 
\end{problem}

\begin{solution}
\[
12x+7y=127,
\]

ahol $x$ és $y$ a 12 kg-os, illetve 7 kg-os ládák száma, és $x,y\geq0$
egész számok.

\bigskip{}

\[
d=\mathrm{lnko}(12,7)=1.
\]

Mivel $d=1$ osztja 127-et, az egyenletnek léteznek egész megoldásai.

\bigskip{}

Egy sajátos megoldás:

\[
x_{0}=10,\quad y_{0}=1,
\]

mert

\[
12\cdot10+7\cdot1=120+7=127.
\]

\bigskip{}

Az általános megoldás alakja:

\[
x=x_{0}+\frac{7}{d}t=10+7t,
\]
\[
y=y_{0}-\frac{12}{d}t=1-12t,
\]

ahol $t\in\mathbb{Z}$.

\bigskip{}

Most megkeressük azokat a megoldásokat, ahol $x,y\geq0$.

\[
x=10+7t\geq0\implies7t\geq-10\implies t\geq-\frac{10}{7}\implies t\geq-1,
\]

hiszen $t$ egész, tehát $t\geq-1$.

\[
y=1-12t\geq0\implies-12t\geq-1\implies12t\leq1\implies t\leq\frac{1}{12}\implies t\leq0.
\]

\bigskip{}

Így a megengedett egész $t$ értékek: $t=-1,0$.

\bigskip{}

Ezekhez tartozó megoldások:
\begin{itemize}
\item Ha $t=-1$: 
\[
x=10+7\cdot(-1)=3,
\]
\[
y=1-12\cdot(-1)=1+12=13.
\]
\item Ha $t=0$: 
\[
x=10,\quad y=1.
\]
\end{itemize}

\textbf{Válasz:} Kétféle módon lehet összeállítani a ládákat úgy,
hogy összesen 127 kg almát kapjunk:
\[
(x,y)=(3,13)\quad\text{vagy}\quad(10,1).
\]
\end{solution}

\subsection*{Pitagoraszi számhármasok}
\begin{definition}{def:Pitagorasz3}
A \textbf{Pitagoraszi számhármas} három pozitív egész szám hármasa
$(a,b,c)$, amely kielégíti a
\[
a^{2}+b^{2}=c^{2}
\]
egyenletet. Ezek a számok megfelelnek a Pitagorasz-tételnek, azaz
egy derékszögű háromszög oldalainak egész hosszai lehetnek.
\begin{itemize}
\item Egy Pitagoraszi számhármas \textbf{primitív}, ha az $a$, $b$ és
$c$ számoknak nincs közös osztójuk 1-en kívül, vagyis
\[
\mathrm{lnko}(a,b,c)=1.
\]
\item Ha létezik olyan $k>1$ egész szám, amely mindhárom számot osztja,
akkor a számhármas \textbf{nem primitív}, vagyis
\[
(a,b,c)=k(a',b',c'),
\]
ahol $(a',b',c')$ primitív Pitagoraszi számhármas. (A nem primitív
Pitagoraszi számhármasok ezen primitív hármasok egész szorzatai.) 
\end{itemize}
\end{definition}

\begin{theorem}{thm:Pitagorasz-primitiv}
Minden primitív Pitagoraszi számhármas előállítható két pozitív egész
szám $m$ és $n$ segítségével, ahol $m>n$, $\mathrm{lnko}(m,n)=1$
és az egyik páros, a másik páratlan. Ekkor az
\[
a=m^{2}-n^{2},\quad b=2mn,\quad c=m^{2}+n^{2}
\]
képlet adja meg a primitív Pitagoraszi számhármast. 
\end{theorem}

\begin{proof}
Induljunk ki az $a^{2}+b^{2}=c^{2}$ egyenletből, ahol $a,b,c\in\mathbb{Z}^{+}$,
és $\mathrm{lnko}(a,b,c)=1$.

A primitív eset miatt $a$ és $b$ nem lehetnek egyszerre párosak,
mert akkor $c$ is páros lenne, így közös osztójuk legalább 2, ami
ellentmondás.

Tegyük fel, hogy $a$ páratlan és $b$ páros.

Írjuk át az egyenletet:
\[
a^{2}-c^{2}=-b^{2}=(a-c)(a+c).
\]
Tudjuk, hogy a számok páronként is relatív prímek, mert ha létezik
egy szám, amely bármely kettőt osztja, az egyenlőség fenállása miatt
szükségszerűen a harmadikat is osztania kéne.

Mivel $b$ páros, $a$ és $c$ páratlanok, ezért $a-c$ és $a+c$
két páros szám és mivel $\mathrm{lnko}(a,c)=1$ ezért $\mathrm{lnko}(a-c,a+c)=2$.
\[
\mathrm{lnko}(a-c,a+c)=d\Rightarrow d\mid(a-c);\quad d\mid(a+c)
\]
Összeadva:
\[
d\mid2a;\quad d\mid2c\Rightarrow d\mid2\cdot{lnko}(a,c)
\]
\[
d\mid2\cdot1\Rightarrow d=2
\]
Ekkor:
\[
\frac{c-a}{2}\cdot\frac{c+a}{2}=\frac{b^{2}}{4},
\]
$\frac{c-a}{2}$ és $\frac{c+a}{2}$ relatív prím egészek és a szorzatuk
egy négyzetszám $\Rightarrow$ ez csak abban az esetben lehetséges,
ha úk maguk is négyzetszámok. jelenti, hogy $c-b$ és $c+b$ páros
számok, és két egymástól különálló páros számból álló szorzatról van
szó, amelyek szorzata egy négyzet.

Így írhatjuk:
\[
\frac{c-a}{2}=m^{2},\frac{c+a}{2}=n^{2},
\]
\[
\Rightarrow m^{2}+n^{2}=c,\quad m^{2}-n^{2}=a
\]
\[
a=m^{2}-n^{2},\quad b=2mn,\quad c=m^{2}+n^{2}.
\]


Ellenőrizzük, hogy az így kapott hármas valóban Pitagoraszi számhármas:

\[
a^{2}+b^{2}=(m^{2}-n^{2})^{2}+(2mn)^{2}=m^{4}-2m^{2}n^{2}+n^{4}+4m^{2}n^{2}=m^{4}+2m^{2}n^{2}+n^{4}=(m^{2}+n^{2})^{2}=c^{2}.
\]
\end{proof}
\textbf{Fontos feltételek (amelyek garantálják, hogy a kapott számhármas primitív):}
\begin{itemize}
\item $m>n>0$, 
\item $\mathrm{lnko}(m,n)=1$, 
\item pontosan az egyik szám páratlan, a másik páros, 
\end{itemize}

\subsection*{A Pell-egyenlet}

\begin{definition}{def:Pell}
A \textbf{Pell-egyenlet} egy másodfokú diofantikus egyenlet, melynek
általános alakja: 
\[
x^{2}-Dy^{2}=1,
\]
ahol $D$ egy pozitív, nem négyzet egész szám. A megoldásokat az egész
számok halmazán keressük.
\end{definition}
\begin{theorem}{thm:Dpozitivnemnegyzet}
Ha $D$ pozitív nem-négyzet szám, akkor az 
\[
x^{2}-Dy^{2}=1
\]
egyenletnek mindig van végtelen sok, pozitív egész megoldása. Ez Lagrange
tételéből következik, miszerint minden ilyen $D$ esetén az egyenletnek
van legalább egy nem triviális megoldása ($x>1,y>0$). Ezt a legegyszerűbb
Pell-egyenletet nevezzük alapegyenletnek, ennek az egyenletnek a legkisebb
pozitív megoldását (legkisebb $x>1$) pedig \textbf{alapmegoldásnak}.
\end{theorem}

\begin{theorem}{thm:pozitivmegoldasokdio}
Ha adott az alapmegoldás $(x_{1},y_{1})$, akkor az egyenlet összes
pozitív megoldása megkapható a következő rekurzív képlettel: 
\[
x_{n}+y_{n}\sqrt{D}=(x_{1}+y_{1}\sqrt{D})^{n}.
\]
\end{theorem}

\begin{problem}
Határozzuk meg a következő Pell-egyenlet első három megoldását: 
\[
x^{2}-2y^{2}=1.
\]
\end{problem}

\begin{solution}
\[
x^{2}-2y^{2}=1.
\]

\[
(x_{1},y_{1})=(3,2)
\]

\[
x_{n}+y_{n}\sqrt{2}=(3+2\sqrt{2})^{n}.
\]

Innen $x_{n}$ és $y_{n}$ kiszámítható úgy, hogy a bal oldalt binomiálisan
kibővítjük, majd a $\sqrt{2}$ együtthatóját és a konstans részt különválasztjuk:
\[
x_{n}=\frac{(3+2\sqrt{2})^{n}+(3-2\sqrt{2})^{n}}{2},\qquad y_{n}=\frac{(3+2\sqrt{2})^{n}-(3-2\sqrt{2})^{n}}{2\sqrt{2}}.
\]
\textbf{Első három megoldás}\\
\begin{itemize}
\item $n=1$: 
\[
x_{1}=3,\quad y_{1}=2
\]
\item $n=2$: 
\[
(3+2\sqrt{2})^{2}=9+12\sqrt{2}+8=17+12\sqrt{2}
\]
\[
(3-2\sqrt{2})^{2}=9-12\sqrt{2}+8=17-12\sqrt{2}
\]
\[
x_{2}=\frac{17+17}{2}=17,\quad y_{2}=\frac{12\sqrt{2}+12\sqrt{2}}{2\sqrt{2}}=12
\]
\item $n=3$: 
\[
(3+2\sqrt{2})^{3}=99+70\sqrt{2}
\]
\[
(3-2\sqrt{2})^{3}=99-70\sqrt{2}
\]
\[
x_{3}=\frac{99+99}{2}=99,\quad y_{3}=\frac{70\sqrt{2}+70\sqrt{2}}{2\sqrt{2}}=70
\]
\end{itemize}
Így az első három megoldás: 
\[
(3,2),\quad(17,12),\quad(99,70)
\]
és mindegyik kielégíti az egyenletet: 
\[
x^{2}-2y^{2}=1.
\]
\end{solution}
\begin{theorem}{thm:Pell}
A bővített vagy általános Pell-egyenlet alakja: 
\[
x^{2}-Dy^{2}=N,
\]
ahol $N\in\mathbb{Z}$, $N\neq1$. Az ilyen egyenletek nem minden
esetben oldhatók meg, de ha találunk egy \textbf{sajátos megoldást}
$(x_{0},y_{0})$, akkor az összes megoldás az alábbi módon generálható:

\[
x_{n}+y_{n}\sqrt{D}=(x_{0}+y_{0}\sqrt{D})\cdot(x_{1}+y_{1}\sqrt{D})^{n},
\]
ahol $(x_{1},y_{1})$ a \emph{Pell-alapegyenlet} egy megoldása, vagyis
ahol: 
\[
x_{1}^{2}-Dy_{1}^{2}=1.
\]

Ebből a szorzásból visszafejthető $x_{n}$ és $y_{n}$ minden $n$-re,
így az általános egyenlet megoldásai is végtelen sokaságot alkotnak
(ha van legalább egy megoldás) 
\end{theorem}

\begin{theorem}{thm:Pell-megoldas}
Ha:
\begin{itemize}
\item $(x_{1},y_{1})$ kielégíti a \textbf{Pell-egyenletet} $x^{2}-Dy^{2}=1$, 
\item $(a,b)$ kielégíti az \textbf{általános Pell-típusú egyenletet} $x^{2}-Dy^{2}=m$, 
\end{itemize}
akkor az alábbi szabály szerint újabb megoldást kapunk: 
\[
x'+y'\sqrt{D}=(x_{1}+y_{1}\sqrt{D})(a+b\sqrt{D}),
\]
amiből: 
\[
x'=x_{1}a+Dy_{1}b,\quad y'=x_{1}b+y_{1}a.
\]

Ez \textbf{újabb megoldást ad az $x^{2}-Dy^{2}=m$ egyenlethez}, és
ezt a folyamatot \textbf{ismételhetjük rekurzívan}. 
\end{theorem}


\section*{Házi feladatok}
\begin{problem}
Határozd meg a következő lineáris diofantikus egyenletek általános
megoldásait az egész számok halmazán! 
\begin{enumerate}
\item $5x+8y=13$ 
\item $4x-3y=2$ 
\item $14x+8y=16$ 
\item $9x-3y=10$ 
\item $-91x+156y=247$ 
\end{enumerate}
\end{problem}

\begin{solution}
Alpontonként a megoldások:
\begin{enumerate}
\item $5x+8y=13$

$\mathrm{lnko}(5,8)=1\mid13$, tehát van megoldás.

Először oldjuk meg a $5x+8y=1$ egyenletet.

Találunk egy megoldást: $(x_{0},y_{0})=(5,-3)$,

mert $5\cdot5+8\cdot(-3)=25-24=1$

Szorozzuk meg 13-mal: 
\[
x=13\cdot5+8t=65+8t,\quad
\]
\[
y=13\cdot(-3)-5t=-39-5t,\quad t\in\mathbb{Z}
\]

\item $4x-3y=2$

$\mathrm{lnko}(4,3)=1\mid2$, tehát van megoldás.

Oldjuk meg a $4x-3y=1$ egyenletet, azaz $4x+(-3)y=1$.

Találunk egy megoldást: $(x_{0},y_{0})=(1,1)$,

mert $4\cdot1-3\cdot1=4-3=1$

Szorozzuk meg 2-vel: 
\[
x=2\cdot1+3t=2+3t,\quad
\]
\[
y=2\cdot1+4t=2+4t,\quad t\in\mathbb{Z}
\]

\item $14x+8y=16$

$\mathrm{lnko}(14,8)=2\mid16$, tehát van megoldás.

Osszuk le az egyenletet 2-vel: $7x+4y=8$

Találunk egy megoldást: $(x_{0},y_{0})=(-4,9)$,

mert $7\cdot(-4)+4\cdot9=-28+36=8$

Az általános megoldás: 
\[
x=-4+4t,\quad
\]
\[
y=9-7t,\quad t\in\mathbb{Z}
\]

\item $9x-3y=10$

$\mathrm{lnko}(9,3)=3\nmid10$, tehát \textbf{nincs megoldás} az egész
számok körében.
\item $-91x+156y=247$

$\mathrm{lnko}(91,156)=13\mid247$, tehát van megoldás.

Osszuk le az egyenletet 13-mal: $-7x+12y=19$

Találunk egy megoldást: $(x_{0},y_{0})=(-23,-1)$,

mert $-7\cdot(-23)+12\cdot(-1)=161-12=149=19\cdot1$

Az általános megoldás: 
\[
x=-23+12t,\quad
\]
\[
y=-1+7t,\quad t\in\mathbb{Z}
\]

\end{enumerate}
\end{solution}
\begin{problem}
Egy ajándékcsomagba bonbonos dobozokat csomagolnak megrendelésre.
A kereskedőknek két típusú doboz áll rendelkezésükre: 18 bonbonos
és 30 bonbonos doboz. Hányféleképpen készíthetnek el egy 246 bonbont
tartalmazó csomagot? Írjuk fel az összes megoldást! 
\end{problem}

\begin{solution}
\[
18x+30y=246,
\]
ahol $x$ és $y$ a 18 bonbonos, illetve 30 bonbonos dobozok száma,
és $x,y\geq0$ egész számok.

\bigskip{}

\[
d=\mathrm{lnko}(18,30)=6.
\]

Mivel $d=6$ osztja 246-ot, az egyenletnek léteznek egész megoldásai.

\bigskip{}

Egyszerűsítve az egyenletet 6-tal:

\[
3x+5y=41.
\]

\bigskip{}

Megkeressünk egy sajátos megoldást az egyszerűsített egyenletre.

$y=4$ esetén:

\[
3x+5\cdot4=3x+20=41\implies3x=21\implies x=7,
\]

tehát $(x_{0},y_{0})=(7,4)$ egy megoldás.

\bigskip{}

Az általános megoldás alakja az egyszerűsített egyenletre:

\[
x=x_{0}+5t=7+5t,
\]
\[
y=y_{0}-3t=4-3t,\quad t\in\mathbb{Z}.
\]

\bigskip{}

Mivel $x,y\geq0$, vizsgáljuk az $t$ lehetséges értékeit:

\[
x=7+5t\geq0\implies t\geq-\frac{7}{5}\implies t\geq-1,
\]

hiszen egész $t$-ről van szó, tehát $t\geq-1$.

\[
y=4-3t\geq0\implies-3t\geq-4\implies3t\leq4\implies t\leq1.
\]

\bigskip{}

Így a megengedett egész $t$ értékek: $-1,0,1$.

\bigskip{}

Ezekhez tartozó megoldások:
\begin{itemize}
\item $t=-1$: 
\[
x=7+5\cdot(-1)=2,\quad y=4-3\cdot(-1)=7,
\]
\item $t=0$: 
\[
x=7,\quad y=4,
\]
\item $t=1$: 
\[
x=7+5=12,\quad y=4-3=1.
\]
\end{itemize}
\bigskip{}

\textbf{Válasz:} Háromféleképpen állítható össze a csomag:

\[
(x,y)=(2,7),\quad(7,4),\quad(12,1).
\]
\end{solution}
\begin{problem}
Egy iskola ceruzákat vásárol. A ceruzák 14-es és 21-es csomagban kaphatóak.
Hányféle lehetősége van az iskolának 315 ceruzát vásárolni? Ezekből
melyik lenne a legolcsóbb lehetőség, ha tudjuk, hogy egy 14-es csomag
ceruza 8 lejbe, egy 21-es csomag pedig 14 lejbe kerül? 
\end{problem}

\begin{solution}
\[
14x+21y=315,
\]
ahol $x$ és $y$ a 14-es, illetve 21-es csomagok száma, $x,y\geq0$
egész számok.

\[
d=\mathrm{lnko}(14,21)=7.
\]

Mivel $7\mid315$, léteznek megoldások.

\bigskip{}

Egyszerűsítve 7-tel:

\[
2x+3y=45.
\]

\bigskip{}

Keressünk egy sajátos megoldást:

$y=1$:

\[
2x+3=45\implies2x=42\implies x=21,
\]

tehát $(x_{0},y_{0})=(21,1)$.

Az általános megoldás:

\[
x=x_{0}+\frac{3}{d'}t=21+3t,
\]
\[
y=y_{0}-\frac{2}{d'}t=1-2t,\quad t\in\mathbb{Z},
\]

ahol $d'=1$ (mert 2 és 3 relatív prímek).

\bigskip{}

Vizsgáljuk a nemnegatív egész megoldásokat:

\[
x=21+3t\geq0\implies t\geq-7,
\]
\[
y=1-2t\geq0\implies-2t\geq-1\implies t\leq\frac{1}{2}\implies t\leq0.
\]

\bigskip{}

A megengedett $t$-k: $-7,-6,\ldots,0$.

\bigskip{}

Így a megoldások: 
\[
(x,y)=(0,15),\quad(3,13),\quad(6,11),\quad(9,9),\quad(12,7),\quad(15,5),\quad(18,3),\quad(21,1).
\]

\textbf{Költségszámítás:}

Egy 14-es csomag ára 8 lej, egy 21-esé 14 lej.

\[
\text{Ár}(t)=8x+14y=8(21+3t)+14(1-2t)=168+24t+14-28t=182-4t.
\]

Az ár akkor a legkisebb, ha $t$ a lehető legnagyobb, azaz $t=0$.

\[
\text{Legolcsóbb megoldás: }(x,y)=(21,1),\quad\text{ár}=8\cdot21+14\cdot1=182.
\]
\end{solution}
\begin{problem}
Határozd meg az összes primitív pitagoraszi számhármast, amelyben
$c<50$ ($a^{2}+b^{2}=c^{2}$)! 
\end{problem}

\begin{solution}
Mivel $c=m^{2}+n^{2}<50$, ezért

\[
m^{2}+n^{2}<50.
\]

Mivel $m>n>0$, egész számok, relatív prímek és különböző paritásúak,
ezért próbáljuk végig a következő értékeket:
\begin{itemize}
\item $m=2,n=1$:

$c=4+1=5<50$

\[
a=2^{2}-1^{2}=3,\quad b=2\cdot2\cdot1=4,\quad c=5.
\]

\item $m=4,n=1$:

$c=16+1=17<50$

\[
a=16-1=15,\quad b=2\cdot4\cdot1=8,\quad c=17.
\]

\item $m=6,n=1$:

$c=36+1=37<50$

\[
a=36-1=35,\quad b=2\cdot6\cdot1=12,\quad c=37.
\]

\item $m=8,n=1$:

$c=64+1=65\not<50$ -- kizárjuk.
\item $m=3,n=2$:

$c=9+4=13<50$

\[
a=9-4=5,\quad b=2\cdot3\cdot2=12,\quad c=13.
\]

\item $m=5,n=2$:

$c=25+4=29<50$

\[
a=25-4=21,\quad b=2\cdot5\cdot2=20,\quad c=29.
\]

\item $m=7,n=2$:

$c=49+4=53\not<50$ -- kizárjuk.
\item $m=4,n=3$:

$c=16+9=25<50$

\[
a=16-9=7,\quad b=2\cdot4\cdot3=24,\quad c=25.
\]

\item $m=8,n=3$:

$c=64+9=73\not<50$) - kizárjuk
\item $m=5,n=4$:

$c=25+16=41<50$

\[
a=25-16=9,\quad b=2\cdot5\cdot4=40,\quad c=41.
\]

\item $m=7,n=4$:

$c=49+16=65\not<50$ -- kizárjuk.
\item $m=6,n=5$:

$c=36+25=61\not<50$ -- kizárjuk.

\end{itemize}
Mivel $n=5$-re nem találtunk egy értéket sem, felesleges ennél fennebb
keresni.

\textbf{Megoldások:}

\[
(3,4,5),\quad(15,8,17),\quad(35,12,37),\quad(5,12,13),\quad(21,20,29),\quad(7,24,25),\quad(9,40,41).
\]
\end{solution}

\begin{problem}
Számold ki három darab sajátos megoldását a következő Pell-egyenletnek,
majd írd fel az általános megoldást: 
\[
x^{2}-5y^{2}=1.
\]
\end{problem}

\begin{solution}
Az egyenlet: 
\[
x^{2}-5y^{2}=1.
\]

Elsőként keressük egy sajátos megoldást. Próbálkozással találjuk,
hogy: 
\[
(x_{1},y_{1})=(9,4),
\]
mert 
\[
9^{2}-5\cdot4^{2}=81-5\cdot16=81-80=1.
\]

Az általános megoldásokat az alábbi formula segítségével kapjuk: 
\[
x_{n}+y_{n}\sqrt{5}=(9+4\sqrt{5})^{n}.
\]

A jobb oldalt binomiálisan kibővítve és szétválasztva a $\sqrt{5}$-ös
és konstans részeket, az $x_{n}$ és $y_{n}$ a következőképpen számolható:
\[
x_{n}=\frac{(9+4\sqrt{5})^{n}+(9-4\sqrt{5})^{n}}{2},\quad y_{n}=\frac{(9+4\sqrt{5})^{n}-(9-4\sqrt{5})^{n}}{2\sqrt{5}}.
\]

\textbf{Az első három megoldás:}\\
\begin{itemize}
\item $n=1$: 
\[
x_{1}=9,\quad y_{1}=4.
\]
\item $n=2$: 
\[
(9+4\sqrt{5})^{2}=81+72\sqrt{5}+80=161+72\sqrt{5},
\]
\[
(9-4\sqrt{5})^{2}=81-72\sqrt{5}+80=161-72\sqrt{5},
\]
így 
\[
x_{2}=\frac{161+161}{2}=161,\quad y_{2}=\frac{72\sqrt{5}+72\sqrt{5}}{2\sqrt{5}}=72.
\]
\item $n=3$: 
\[
(9+4\sqrt{5})^{3}=(9+4\sqrt{5})\cdot(161+72\sqrt{5})=2889+1292\sqrt{5},
\]
\[
(9-4\sqrt{5})^{3}=(9-4\sqrt{5})\cdot(161-72\sqrt{5})=2889-1292\sqrt{5},
\]
így 
\[
x_{3}=\frac{2889+2889}{2}=2889,\quad y_{3}=\frac{1292\sqrt{5}+1292\sqrt{5}}{2\sqrt{5}}=1292.
\]
\end{itemize}
Tehát az első három megoldás: 
\[
(9,4),\quad(161,72),\quad(2889,1292).
\]
\end{solution}
\begin{problem}
Számold ki három darab sajátos megoldását a következő Pell-egyenletnek,
majd írd fel az általános megoldást: $x^{2}-7y^{2}=9$. 
\end{problem}

\begin{solution}
Az egyenlet: 
\[
x^{2}-7y^{2}=9.
\]

Keressük meg az első sajátos megoldást:

\[
(x_{0},y_{0})=(4,1),
\]
mert: 
\[
4^{2}-7\cdot1^{2}=16-7=9.
\]

A Pell-alapegyenlet: 
\[
x^{2}-7y^{2}=1,
\]
egy megoldása: 
\[
(x_{1},y_{1})=(8,3),
\]
mert: 
\[
8^{2}-7\cdot3^{2}=64-63=1.
\]

Ez alapján a következő megoldást úgy kapjuk, hogy megszorozzuk a két
gyökös alakot:

\[
x'+y'\sqrt{7}=(8+3\sqrt{7})\cdot(4+\sqrt{7}).
\]

Végezzük el a szorzást:

\[
=(8\cdot4+8\cdot\sqrt{7}+3\sqrt{7}\cdot4+3\cdot7)=32+8\sqrt{7}+12\sqrt{7}+21=53+20\sqrt{7}.
\]

Tehát a második sajátos megoldás: 
\[
(x_{2},y_{2})=(53,20).
\]

Folytatjuk a rekurziót:

\[
x''+y''\sqrt{7}=(8+3\sqrt{7})\cdot(53+20\sqrt{7}).
\]

Számoljuk ki:

\[
=8\cdot53+8\cdot20\sqrt{7}+3\cdot53\sqrt{7}+3\cdot20\cdot7=424+160\sqrt{7}+159\sqrt{7}+420=844+319\sqrt{7}.
\]

Tehát a harmadik sajátos megoldás: 
\[
(x_{3},y_{3})=(844,319).
\]

\bigskip{}

\textbf{Általános megoldás:}

Az összes megoldás: 
\[
x_{n}+y_{n}\sqrt{7}=(4+\sqrt{7})\cdot(8+3\sqrt{7})^{n},
\]
azaz: 
\[
x_{n}=\text{Re}\left[(4+\sqrt{7})\cdot(8+3\sqrt{7})^{n}\right],\quad y_{n}=\text{a }\sqrt{7}\text{ együtthatója}.
\]

\bigskip{}

\textbf{Első három megoldás:}

\[
(x,y)=(4,1),\quad(53,20),\quad(844,319).
\]
\end{solution}

\section*{Nehezebb feladatok}
\begin{extraproblem}[Gergely Verona]
Oldjuk meg a $x+y=x^{2}-xy+y^{2}$ egyenletet, ahol $x$ és $y$
egész számok. \emph{(KöMaL, 2011) }
\end{extraproblem}

\begin{solution}
Legyen $y=x+a$, ahol $a\in\mathbb{Z}$. Ekkor az egyenlet: $x+x+a=x^{2}-x(x+a)+(x+a)^{2}$,

azaz

\[
2x+a=x^{2}-x^{2}-xa+x^{2}+2xa+a^{2},
\]

\[
0=x^{2}+xa-2x+a^{2}-a,
\]

\[
0=x^{2}+(a-2)x+\left(a^{2}-a\right).
\]

A megoldás $a$ függvényében:

\[
x_{1,2}=\frac{-a+2\pm\sqrt{(a-2)^{2}-4\cdot1\cdot(a^{2}-a)}}{2\cdot1}.
\]

A másodfokú egyenletnek akkor van megoldása, ha a diszkriminánsa nemnegatív:

\[
D=(a-2)^{2}-4(a^{2}-a)\geq0,
\]

\[
a^{2}-4a+4-4a^{2}+4a\geq0,
\]

\[
4-3a^{2}\geq0,
\]

\[
\frac{4}{3}\geq a^{2},
\]

\[
\frac{2}{\sqrt{3}}\geq|a|.
\]

Az $a$ lehetséges értékei a következők:

$a=-1$, ekkor $y=x-1$, $D=1$, azaz $x_{1,2}=\frac{1+2\pm\sqrt{1}}{2},\quad x_{1}=2,\quad x_{2}=1;$\\

$a=0$, ekkor $y=x$, $D=4$, azaz $x_{3,4}=\frac{2\pm\sqrt{4}}{2},\quad x_{3}=2,\quad x_{4}=0;$\\

$a=1$, ekkor $y=x+1$, $D=1$, azaz $x_{5,6}=\frac{-1+2\pm\sqrt{1}}{2},\quad x_{5}=1,\quad x_{6}=0.$

\medskip{}

Ha $a=-1$ és $x_{1}=2$, akkor $y_{1}=x_{1}-1=1$.\\
 Ha $a=-1$ és $x_{2}=1$, akkor $y_{2}=x_{2}-1=0$. \\
 Ha $a=0$ és $x_{3}=2$, akkor $y_{3}=x_{3}=2$.\\
 Ha $a=0$ és $x_{4}=0$, akkor $y_{4}=x_{4}=0$. \\
 Ha $a=1$ és $x_{5}=1$, akkor $y_{5}=x_{5}+1=2$.\\
 Ha $a=1$ és $x_{6}=0$, akkor $y_{6}=x_{6}+1=1$.

\medskip{}

Tehát a megoldások a következő $(x,y)$ számpárok:

\[
(2,1),\quad(1,0),\quad(2,2),\quad(0,0),\quad(1,2),\quad(0,1).
\]
\end{solution}
\begin{extraproblem}[Kis Andrea-Tímea]
Legyen $(a,b,c)\in\mathbb{Z}^{3}$. A kongruencia-módszert felhasználva
mutassuk meg, hogy az 
\[
a^{2}+b^{2}=3c^{2}
\]
egyenletnek nincsen a triviálistól különböző megoldása! 
\end{extraproblem}

\begin{solution}
Tegyük fel indirekt módon, hogy az egyenletnek van a triviálistól
különböző $(a,b,c)$ megoldása, majd legyen 
\[
d:=\text{lnko}(a,b,c).
\]

Az egyenletet $d^{2}$-vel leosztva azt kapjuk, hogy 
\[
\left(\frac{a}{d}\right)^{2}+\left(\frac{b}{d}\right)^{2}=\left(\frac{c}{d}\right)^{2},
\]
azaz 
\[
\left(\frac{a}{d},\frac{b}{d},\frac{c}{d}\right)\in\mathbb{Z}^{3}
\]
is megoldás. Mivel ezek a számok relatív prímek, ezért elég azt belátni,
hogy nincs olyan megoldás-hármas, amelyek legnagyobb közös osztója
$1$-gyel egyenlő. Ha $(a,b,c)$ hármas megoldása az egyenletnek,
akkor a kongruencia-módszer és az előző megfigyelés szerint $(a,b,c)\in\mathbb{Z}^{3}$
választható olyanok, hogy $\text{lnko}(a,b,c)=1$ legyen. Mindez azt
jelenti, hogy 
\[
3\mid(a^{2}+b^{2})
\]
és ha $k$ négyzetszám, akkor 
\[
k\equiv0\pmod 3\quad\text{vagy}\quad k\equiv1\pmod 3.
\]
Így külön-külön is igaz, hogy $3\mid a^{2}$ és $3\mid b^{2}$, ahonnan
már következik, hogy 
\[
9\mid a^{2},\qquad9\mid b^{2},\qquad9\mid(a^{2}+b^{2})=3c^{2},
\]
és fennállnak a $3\mid a,3\mid b,3\mid c$ oszthatósági relációk.
Ez ellentmond annak, hogy 
\[
\text{lnko}(a,b,c)=1.
\]
\end{solution}
\begin{extraproblem}[Kovács Levente]
Oldjuk meg az egész számok körében a következő diofantikus egyenletet:
\[
x^{2}-5y^{2}=4.
\]
\end{extraproblem}

\vspace{1em}

\begin{solution}
A Pell--féle egyenlethez hasonló alak: $x^{2}-5y^{2}=4$. Először
keressünk primitív megoldásokat, majd a recidív formulával generáljuk
az összeset.
\begin{enumerate}
\item \textbf{Kezdeti megoldások:} Próbáljuk kis $\lvert y\rvert$-ra: 
\[
y=0:\;x^{2}=4\;\Rightarrow\;x=\pm2.\quad y=1:\;x^{2}=4+5=9\;\Rightarrow\;x=\pm3.
\]
További kis $y$ esetén $\lvert y\rvert\ge2$ már $5y^{2}\ge20$,
így $x^{2}\ge24$ és nem ad új kis egész megoldást. Tehát a primitív
megoldások: 
\[
(x,y)=(\pm2,0),\;(\pm3,\pm1).
\]
\item \textbf{Pell-egyenlet alap:} A hozzá tartozó homogén Pell-egyenlet
\[
u^{2}-5v^{2}=1
\]
alapmegoldása $(u_{1},v_{1})=(9,4)$, mert $9^{2}-5\cdot4^{2}=81-80=1$.
\item \textbf{Általános recidíva:} Ha $(x_{0},y_{0})$ egy megoldás a $4$-es
egyenletre és $(u_{k},v_{k})$ a Pell-egyenlet $k$. megoldása, akkor
\[
x+y\sqrt{5}=(x_{0}+y_{0}\sqrt{5})\,(u_{k}+v_{k}\sqrt{5}).
\]
Így minden megoldás: 
\[
x_{k}+y_{k}\sqrt{5}=\pm(2+0\sqrt{5})\,(9+4\sqrt{5})^{k},\quad x_{k}'+y_{k}'\sqrt{5}=\pm(3+1\sqrt{5})\,(9+4\sqrt{5})^{k},\quad k\in\mathbb{Z}_{\ge0}.
\]
Ezek bontása (binomiális tétellel) adja az összes egész $(x,y)$-párost. 
\end{enumerate}
\end{solution}
\vspace{1em}

\begin{extraproblem}[Kovács Levente]
Oldjuk meg egész számok között az alábbi lineáris diofantikus egyenletrendszert:
\[
\begin{cases}
3x+4y+5z=6,\\
2x-y+3z=1.
\end{cases}
\]
\end{extraproblem}

\vspace{1em}

\begin{solution}
Írjuk fel a szabad változók segítségével.
\begin{enumerate}
\item \textbf{Mátrixos alak:} 
\[
\begin{pmatrix}3 & 4 & 5\\
2 & -1 & 3
\end{pmatrix}\begin{pmatrix}x\\
y\\
z
\end{pmatrix}=\begin{pmatrix}6\\
1
\end{pmatrix}.
\]
\item \textbf{Első egyenlet megoldása $x$-re:} 
\[
3x=6-4y-5z\;\Rightarrow\;x=2-\tfrac{4}{3}y-\tfrac{5}{3}z.
\]
A jobb oldalnak egésznek kell lennie, tehát $4y+5z\equiv0\pmod 3$.
Mivel $4\equiv1$ és $5\equiv2$ modulo 3, 
\[
y+2z\equiv0\pmod 3\;\Longrightarrow\;y\equiv z\pmod 3.
\]
Legyen $y=z+3t$, $t\in\mathbb{Z}$.
\item \textbf{Behelyettesítés a második egyenletbe:} 
\[
2x-y+3z=1.
\]
Először $x$-et is írjuk $z,t$-ben: 
\[
x=2-\tfrac{4}{3}(z+3t)-\tfrac{5}{3}z=2-\tfrac{4z+12t+5z}{3}=2-\tfrac{9z+12t}{3}=2-3z-4t.
\]
Második egyenlet: 
\[
2(2-3z-4t)-(z+3t)+3z=1\;\Longrightarrow\;4-6z-8t-z-3t+3z=1
\]
\[
4-4z-11t=1\;\Longrightarrow\;4z+11t=3.
\]
\item \textbf{Lineáris diofantikus megoldása:} Megoldjuk $4z+11t=3$. Egy
partikuláris megoldás: $t=1$ esetén $4z=3-11=-8$, tehát $z=-2$.
A homogén megoldás a nulláris egyenlethez: $4z+11t=0$ paraméterezésével
$z=11s$, $t=-4s$. Így az általános: 
\[
z=-2+11s,\quad t=1-4s,\quad s\in\mathbb{Z}.
\]
\item \textbf{Vissza $x,y$-hoz:} 
\[
y=z+3t=(-2+11s)+3(1-4s)=-2+11s+3-12s=1-s.
\]
\[
x=2-3z-4t=2-3(-2+11s)-4(1-4s)=2+6-33s-4+16s=4-17s.
\]

Tehát az összes egész megoldás: 
\[
\boxed{(x,y,z)=(4-17s,\;1-s,\;-2+11s),\quad s\in\mathbb{Z}.}
\]

\end{enumerate}
\end{solution}
\begin{extraproblem}[Lukács Andor]
Határozd meg az összes olyan $p$ prímszámot, amely felírható 
\[
p^{n}=x^{3}+y^{3}
\]
alakban valamilyen $x,y,n\in\mathbb{N}^{*}$ segítségével! 
\begin{flushright}
(MO, Magyarország, 2000) 
\par\end{flushright}
\end{extraproblem}

\begin{solution}
Könnyen észrevehető, hogy $p=2$ és $p=3$ megoldások, mivel $2^{1}=1^{3}+1^{3}$
és $3^{2}=1^{3}+2^{3}.$ Lássuk be, hogy ha $p\ge5$, akkor nincs
megoldása az egyenletnek! Tételezzük fel, hogy létezik megoldás és
válasszunk ki egy olyan megoldást, amelyre $n$ minimális. Mivel $x$
és $y$ közül legalább az egyik 1-nél nagyobb, az 
\[
x^{3}+y^{3}=(x+y)(x^{2}-xy+y^{2})
\]
felbontásban $x+y\ge3$ és $x^{2}-xy+y^{2}=(x-y)^{2}+xy\ge2$. Tehát
az $(x+y)$ és $(x^{2}-xy+y^{2})$ tényezők mind oszthatók $p$-vel
és ezért 
\[
3xy=(x+y)^{2}-(x^{2}-xy+y^{2})
\]
is osztható $p$-vel. Ez viszont azt jelenti, hogy $x$ is és $y$
is osztható $p$-vel (mivel $x+y$ osztható $p$-vel). Következik,
hogy $x^{3}+y^{3}\ge2p^{3}$, tehát $n>3$ és 
\[
p^{n-3}=\left(\frac{x}{p}\right)^{3}+\left(\frac{y}{p}\right)^{3},
\]
ami ellentmond az $n$ minimalitásának. 
\end{solution}
\begin{extraproblem}[Péter Róbert]
Alf nyolcosztályos elemi iskolába járt. Minden tanév végén megmutatta
apjának a bizonyítványát. Ha Alf sikeresen elvégezte az évfolyamot,
annyi macskát kapott apjától, amennyi életkorának és az adott évfolyam
számának szorzata. Tanulmányai során Alf egyszer megbukott. Amikor
befejezte az elemi iskolát, észrevette, hogy az apjától kapott macskák
száma 1998-cal osztható. Melyik évfolyamot ismételte meg Alf? 
\end{extraproblem}

\begin{solution}
Alf nyolcosztályos elemi iskolába járt, de egy évfolyamot megismételt,
tehát összesen 9 évig tanult.

Minden év végén annyi macskát kapott, amennyi az aktuális életkorának
és az évfolyam számának szorzata.

Tegyük fel, hogy az $x$-edik évfolyamot ismételte meg.

Alf 6 évesen kezdte az iskolát, így az $i$-edik iskolai év végén
$5+i$ éves volt.

Az alap macskaszám (ismétlés nélkül) tehát: 
\[
\sum_{i=1}^{8}i(5+i)=\sum_{i=1}^{8}(5i+i^{2})=5\cdot\sum_{i=1}^{8}i+\sum_{i=1}^{8}i^{2}
\]

Ismeretes: 
\[
\sum_{i=1}^{8}i=36,\quad\sum_{i=1}^{8}i^{2}=204
\]

Így: 
\[
\text{Összes macska}=5\cdot36+204=384
\]

Az ismétlés miatt az $x$-edik évfolyamhoz további $x(5+x)$ macskát
kapott: 
\[
M=384+x(5+x)
\]

Tudjuk, hogy $M$ osztható 1998-cal: 
\[
384+x(x+5)\equiv0\mod{1998}\Rightarrow x(x+5)=1614\Rightarrow x^{2}+5x-1614=0
\]

Megoldva: 
\[
x=\frac{-5\pm\sqrt{25+6456}}{2}=\frac{-5\pm\sqrt{6481}}{2}\Rightarrow x=37
\]

Mivel csak $x=1,2,\dots,8$ jöhet szóba évfolyamként, a megfelelő
érték: $x=7$

Tehát Alf a 7. évfolyamot ismételte meg.
\end{solution}
\begin{extraproblem}[Sógor Bence]
Oldd meg az egész számok halmazán a 
\[
15x^{2}-7y^{2}=9
\]
egyenletet. 
\end{extraproblem}

\begin{solution}
Alakítsuk át az egyenletet a következő alakra 
\[
15x^{2}-7y^{2}=9\iff15x^{2}-9=7y^{2}\iff3(5x^{2}-3)=7y^{2}.
\]
Mivel az egyenlet bal oldala mindig osztható 3-mal, ezért létezik
$l$ egész szám úgy, hogy $y=3l$. Az egyenlet a következőképpen alakul
\[
5x^{2}-3=21l^{2}\iff5x^{2}=3(7l^{2}+1).
\]
Az előzőhöz hasonló gondolatmenet alapján, létezik $k$ egész úgy,
hogy $x=3k$. Innen az egyenletbe ezt behelyettesítve 
\[
15k^{2}=7l^{2}+1
\]
Tekintsük az egyenletet $\mod{}$ 3. A bal oldal mindig osztható 3-mal,
tehát $15k^{2}\equiv0\mod 3$. Tudjuk, hogy $l^{2}\equiv0\mod 3$
vagy $l^{2}\equiv1\mod 3$, innen $7l^{2}\equiv0\mod 3$ vagy $7l^{2}\equiv1\mod 3$.
Tehát, $7l^{2}+1\equiv1\mod 3$ vagy $7l^{2}+1\equiv2\mod 3$. Mivel
egyik esetben sem osztható $7l^{2}+1$ 3-mal, ezért a $15k^{2}=7l^{2}+1$
egyenletnek nincsen megoldása az egész számok halmazán, tehát az eredeti
egyenletnek sincsenek megoldásai az egész számok halmazán. 
\end{solution}
\begin{extraproblem}[Szélyes Klaudia]
Egy boltban kávét 37 lej/kg, teát pedig 23 lej/kg áron árulnak. Egy
vásárló pontosan 500 lejt költött. Tudjuk, hogy kizárólag egész kg
mennyiségeket vásárolt, és legalább 1-1 kg-ot mindkettőből.
\begin{enumerate}
\item Hányféleképpen vásárolhatott? 
\item Melyik megoldás esetén vásárolta a legtöbb árut összesen? 
\end{enumerate}
\end{extraproblem}
\begin{solution}
Legyen $x$ a vásárolt kávé (kg-ban), $y$ pedig a vásárolt tea. Ekkor
az alábbi egyenletet kapjuk:

\[
37x+23y=500\tag{1}
\]

\textbf{Megoldhatóság:} $\gcd(37,23)=1$, tehát az egyenletnek van
egész megoldása.

\textbf{Egy partikuláris megoldás:} Euklideszi algoritmus segítségével:

\begin{align*}
37 & =1\cdot23+14\\
23 & =1\cdot14+9\\
14 & =1\cdot9+5\\
9 & =1\cdot5+4\\
5 & =1\cdot4+1\\
4 & =4\cdot1+0
\end{align*}

Visszahelyettesítés:

\[
1=5-1\cdot4=2\cdot5-1\cdot9=2(14-1\cdot9)-1\cdot9=\ldots=9\cdot23-5\cdot37
\]

Tehát:

\[
37(-5)+23(9)=1\Rightarrow x_{0}=-5,\quad y_{0}=9
\]

Szorzunk 500-zal:

\[
x=-2500+23t,\quad y=4500-37t,\quad t\in\mathbb{Z}
\]

\textbf{Feltételek:} $x,y\geq1$

\begin{align*}
-2500+23t & \geq1\Rightarrow t\geq\left\lceil \frac{2501}{23}\right\rceil =109\\
4500-37t & \geq1\Rightarrow t\leq\left\lfloor \frac{4499}{37}\right\rfloor =121
\end{align*}

Tehát $t\in\{109,110,\dots,121\}$, azaz \textbf{13 megoldás} létezik.

\textbf{(b) Minimális $x+y$:}

\[
x+y=(-2500+23t)+(4500-37t)=2000-14t
\]

Ez minimális, ha $t$ maximális, azaz $t=121$:

\[
x=-2500+23\cdot121=283,\quad y=4500-37\cdot121=23,\quad x+y=306
\]

\textbf{Válasz:} 13-féleképpen vásárolhatott. A legnagyobb tömegű
vásárlás 306 kg (283 kg kávé és 23 kg tea).

Feladat 7

Oldd meg az egész számok halmazán:

\[
105x+77y=7
\]

majd határozd meg azokat a pozitív megoldáspárokat $(x,y)$, amelyekre
$x+y$ minimális.

Megoldás 7

Mivel $\gcd(105,77)=7$ és $7\mid7$, az egyenlet megoldható.

Osszuk el az egyenletet 7-tel:

\[
15x+11y=1\tag{2}
\]

\textbf{Euklideszi algoritmus:}

\begin{align*}
15 & =1\cdot11+4\\
11 & =2\cdot4+3\\
4 & =1\cdot3+1\\
3 & =3\cdot1+0
\end{align*}

Visszahelyettesítéssel:

\[
1=4-1\cdot3=4-1(11-2\cdot4)=3\cdot4-1\cdot11=3(15-11)-1\cdot11=3\cdot15-4\cdot11
\]

Tehát:

\[
x_{0}=3,\quad y_{0}=-4
\]

Szorzunk 7-tel:

\[
x=21+11t,\quad y=-28-15t,\quad t\in\mathbb{Z}
\]

\textbf{Feltételek:} $x,y>0$

\begin{align*}
21+11t & >0\Rightarrow t>-2\Rightarrow t\geq-1\\
-28-15t & >0\Rightarrow t<-\frac{28}{15}\Rightarrow t\leq-2
\end{align*}

Nincs olyan $t\in\mathbb{Z}$, amelyre $x,y$ egyszerre pozitívak.
Tehát az egyenletnek \textbf{nincs pozitív egész megoldása}.

\textbf{Válasz:} Az egyenletnek léteznek egész megoldásai, de \textbf{nincs
olyan megoldáspár}, ahol $x>0$ és $y>0$.
\end{solution}
\begin{extraproblem}[Csapó Hajnalka]
Oldjuk meg az egész számok halmazán a $3x^{2}+5xy+2y^{2}-x-y=0$
egyenletet! \emph{(I. Hegyi Lajos Emlékverseny, 1997, Szováta) }
\end{extraproblem}

\begin{solution}
A feladatbeli egyenlúség a $2(x+y)^{2}+x(x+y)-x-y=0$ vagyis $(x+y)(3x+2y-1)=0$
alakban írható, ahonnan következik, hogy $x+y=0$ vagy $3x+2y-1=0.$

\textbf{I. eset.} $x+y=0$

Ennek megoldásai az $(u,-u)$ alakú számpárok, ahol $u\in\mathbb{Z}$

\textbf{II. eset.} $3x+2y=1$

Ennek egy sajátos megoldása $x_{0}=1$ és $y_{0}=-1$, tehát az összes
megoldása $x=1+2t$ és $y=-1-3t$ alakú, ahol $t\in\mathbb{Z}.$

Így a megoldáshalmaz $M=\{(k,-k),(1+2k,-1-3k)|k\in\mathbb{Z}\}.$
\end{solution}
\begin{extraproblem}[Csapó Hajnalka]
Határozzuk meg az $x^{2}+2y^{2}=z^{2},$ $x,y,z\in\mathbb{N}$ egyenlet
összes megoldását! 
\end{extraproblem}

\begin{solution}
Ha van közös osztója bármely kettőnek, akkor az a harmadiknak is osztója,
így a primitív megoldásokat keressük, azaz legyenek a számok páronként
relatív prímek. Az egyenlet $2y^{2}=z^{2}-x^{2}\equiv2y^{2}=(z-x)(z+x)$
alakba írható.

Ha $d$ a $z-x$ és $z+x$ közös osztója, akkor $d$ a $2x$ és $2z$
számoknak is közös osztója, de mivel $x$ és $z$ relatív prímek,
következik, hogy $d\in\{1,2\}$. A $(z-x)(z+x)$ szorzat páros, ugyanakkor
$z-x$ és $z+x$ azonos paritású, következik, hogy mindkettő páros,
tehát legnagyobb közös osztójuk 2.

Így $\dfrac{y^{2}}{2}=\dfrac{z-x}{2}\cdot\dfrac{z+x}{2},$ ahonnan
$\dfrac{z-x}{2}=m^{2}$ és $\dfrac{z+x}{2}=2n^{2}$ vagy $\dfrac{z-x}{2}=2n^{2}$
és $\dfrac{z+x}{2}=m^{2}$, ahol $m$ páratlan és $(m,n)=1$.

Tehát a primitív megoldások $x=2n^{2}-m^{2}>0,y=2mn,z=m^{2}+2n^{2}$
vagy $x=m^{2}-2n^{2}>0,y=2mn,z=m^{2}+2n^{2}$ alakúak, ahol $m,n\in\mathbb{N}$,
$m$ páratlan és $(m,n)=1$.

Az összes megoldás pedig $x=k(2n^{2}-m^{2})>0,y=2kmn,z=k(m^{2}+2n^{2})$
vagy $x=k(m^{2}-2n^{2})>0,y=2kmn,z=k(m^{2}+2n^{2})$ alakúak, ahol
$m,n,k\in\mathbb{N}$, $m$ páratlan és $(m,n)=1$.
\end{solution}

\section{Faktorizációs és paraméterezéses módszerek, moduláris aritmetika, végtelen leszállás}\label{sec:vegtelen_leszallas}
\begin{description}
	{\large \item [{Szerző:}] Száfta Antal (Didaktikai mesteri -- Matematika, II. év) }
\end{description}


%%%%%%%%%%%%%%%%%%%%


\subsection*{Bevezetés}

\begin{definition}{def:dioegyenlet}
	A \textbf{diofantoszi egyenlet} olyan egyenlet,
	amelyben \textbf{egész számok} (vagy természetes számok) megoldásait
	keressük, és amelynek együtthatói is \textbf{egészek}.
	\vspace{0.3cm}
	\textbf{Általános alak:} 
	\[
	f(x_{1},x_{2},\ldots,x_{n})=0,
	\]
	ahol $f$ polinomiális kifejezés, és $x_{1},x_{2},\ldots,x_{n}\in\mathbb{Z}$.
\end{definition}
\begin{itemize}
	\item A megoldások halmaza \textbf{diszkrét} -- általában \textbf{egész}
	vagy \textbf{természetes számokat} keresünk. 
	\item Nem minden valós számokra megoldható egyenletnek van egész megoldása. 
	\item A diofantoszi egyenletek sok esetben csak \textbf{véges számú} megoldással
	rendelkeznek, vagy akár \textbf{egyáltalán nincs} megoldásuk. 
\end{itemize}
\textbf{Példa:} 
\begin{itemize}
	\item Diofantoszi: $x^{2}+y^{2}=z^{2}$ (Pitagoraszi számhármasok) 
	\item Nem diofantoszi: $x^{2}+\sqrt{2}=y$ valós számok körében. 
\end{itemize}
%%%%%%%%%%%%%%%%%%

\subsection*{Faktorizációs módszer}

Ez a módszer abból indul ki, hogy egy adott egyenletet szorzattá próbálunk
alakítani. A szorzat felírása után azt vizsgáljuk, milyen egész számok
szorzataként állhat elő az adott jobb vagy bal oldal. Ez gyakran lényegesen
leegyszerűsíti a megoldást.

A faktorizáció módszere akkor különösen hatékony, ha az egyenlet bal
vagy jobb oldala jól faktorizálható, vagy az alakja emlékeztet ismert
azonosságokra (például: $a^{2}-b^{2}=(a-b)(a+b)$).

\noindent\textbf{Előnyök:} 
\begin{itemize}
	\item Konstruktív -- konkrét megoldásokhoz vezethet. 
	\item Rávilágít az algebrai szerkezetek fontosságára. 
\end{itemize}
\textbf{Hátrányok:} 
\begin{itemize}
	\item Nem minden egyenlet faktorizálható egyszerűen. 
\end{itemize}

\begin{problem}
	Határozzátok meg az összes olyan pozitív egész $(x,y)$
	számpárost, amely teljesíti az 
	\[
	(xy-7)^{2}=x^{2}+y^{2}\text{ egyenlőséget!}
	\]
\end{problem}
\begin{solution}
	A bal oldalon négyzetre emelünk, ekkor 
	\[
	x^{2}y^{2}-14xy+49=x^{2}+y^{2}.
	\]
	
	
	Az egyenlet a következőképpen alakul: 
	\begin{align*}
		x^{2}y^{2}-14xy+49-(x^{2}+y^{2}) & =0\\
		x^{2}y^{2}-12xy+49-(x^{2}+2xy+y^{2}) & =0\\
		x^{2}y^{2}-12xy+36-(x^{2}+2xy+y^{2}) & =-13\\
		(xy-6)^{2}-(x+y)^{2} & =-13\\
		(xy-6-x-y)(xy-6+x+y) & =-13
	\end{align*}
	Mivel $x$, $y$ pozitív egészek, ezért $(xy-6-x-y)$ és $(xy-6+x+y)$
	egészek, így $-13$ egész osztóit felírva és tudva, hogy $xy-6-x-y<xy-6+x+y$:
	\[
	\left\{ \begin{array}{l}
		xy-6-x-y=-1\\
		xy-6+x+y=13
	\end{array}\right.\quad\text{vagy}\quad\left\{ \begin{array}{l}
		xy-6-x-y=-13\\
		xy-6+x+y=1
	\end{array}\right.
	\]
	%
	
	Először az 
	\[
	\left\{ \begin{array}{l}
		xy-6-x-y=-1\\
		xy-6+x+y=13
	\end{array}\right.
	\]
	egyenletrendszert megoldva kapjuk, hogy $xy-6=6\Leftrightarrow xy=12$
	és $x+y=7$. 
	\item[] Így a 
	\[
	\left\{ \begin{array}{l}
		xy=12\\
		x+y=7
	\end{array}\right.
	\]
	egyenletrendszert megoldva kapjuk a $(3,4)$; $(4,3)$ megoldásokat. 
	
	%
	
	Hasonlóan megnézve a másik egyenletrendszert kapjuk, hogy 
	\[
	\left\{ \begin{array}{l}
		xy=0\\
		x+y=7,
	\end{array}\right.
	\]
	és így a $(0,7);(7,0)$ számpárokat kapjuk 
	\item[] Tehát a teljes megoldáshalmaz $M=\{(3,4);(4,3);.(0,7);(7,0)\}$ 
	
\end{solution}

\begin{problem}
	Határozzuk meg az összes olyan egész $n$ számot,
	amelyre megoldható az 
	\[
	x^{3}+y^{3}+z^{3}-3xyz=n
	\]
	egyenlet az egész számok halmazára nézve.
\end{problem}
\begin{solution}
	Írjuk át más alakba az egyenletet: 
	\[
	x^{3}+y^{3}+z^{3}-3xyz=(x+y+z)(x^{2}+y^{2}+z^{2}-xy-yz-xz)
	\]
	\[
	\Downarrow
	\]
	\[
	x^{3}+y^{3}+z^{3}-3xyz=(x+y+z)\cdot\frac{1}{2}[(x-y)^{2}+(y-z)^{2}+(z-x)^{2}]\ \text{(1)}
	\]
	valamint
	\[
	x^{3}+y^{3}+z^{3}-3xyz=(x+y+z)^{3}-3\cdot(x+y+z)(xy+yz+zx)\ \text{(2)}.
	\]
	%
	
	Az $(1)$-es egyenletből észrevehető, hogy $n=3k+1$, valamint $n=3k+2$,
	$k\geq1$, mivel az eredeti egyenlet megoldásainak halmazai rendre
	$(k+1,k,k)$ valamint $(k+1,k+1,k)$ alakúak. 
	
	Ha $3|n$, akkor a (2)-esből következik, hogy $3|(x+y+z)$, tehát
	$9|(x^{3}+y3+z^{3}-3xyz=n)$. 
	
	Megfordítva, az adott egyenlet megoldható bármilyen egész $n=9$k,
	$k\geq2$, mivel $(k-1,k,k+1)$ megoldáshalmaz eleget tesz az egyenletnek,
	szintén ha $n=0\ (x=y=z)$. 
	
	Összesítve: 
	\begin{itemize}
		\item $n=3k+1$, ha $k\geq1$, 
		\item $n=3k+2$, ha $k\geq1$, 
		\item $n=9k$, ha $k\geq2$, 
		\item $n=0$, ha $k=0$. 
	\end{itemize}
	
\end{solution}
%%%%%%%%%%%%%%%%%%%%%%%%%%%%%%%%%%%%%%%%%%%%%%%%%%%%    

\subsection*{Parametrizálás}

A paraméterezés olyan módszer, amely során a diofantoszi egyenlet
ismeretlenjeit új változókkal (paraméterekkel) fejezzük ki, hogy: 
\begin{itemize}
	\item egyszerűbb alakba hozzuk az egyenletet, 
	\item csökkentsük az ismeretlenek számát, 
	\item vagy leírjuk az összes (vagy egy részhalmaz) megoldást. 
\end{itemize}
\vspace{0.5em}
\textbf{Mikor alkalmazzuk?} 
\begin{itemize}
	\item Szimmetria vagy algebrai azonosság esetén. 
	\item Ha az egyenlet geometriai alakzatot ír le. 
	\item Ha az egyenlet homogén vagy kvadratikus. 
\end{itemize}
\vspace{0.5em}
\textbf{Példa: Pitagoraszi számhármasok} 
\[
x^{2}+y^{2}=z^{2}\Rightarrow\begin{cases}
	x=m^{2}-n^{2}\\
	y=2mn\\
	z=m^{2}+n^{2}
\end{cases}\quad\text{ahol }m>n>0,\ lnko(m,n)=1,\ 
\]
\[
m-n\text{ páratlan.}
\]

\noindent\textbf{Előnyök:} 
\begin{itemize}
	\item Lehetővé teszi az összes megoldás leírását paraméterekkel. 
	\item Csökkenti az ismeretlenek számát. 
	\item Kiemeli az algebrai vagy geometriai szimmetriákat. 
	\item Geometriai értelmezést is adhat (pl. kör racionális pontjai). 
\end{itemize}
\vspace{0.5em}
\textbf{Hátrányok:} 
\begin{itemize}
	\item Nem minden egyenlet paraméterezhető. 
	\item Előfordulhat, hogy csak a megoldások egy részhalmazát adja meg. 
	\item Több paramétert igényelhet, mint az eredeti ismeretlenek száma. 
	\item Komplikálhatja a triviális megoldásokat is. 
\end{itemize}

\begin{problem}
	Igazoljuk, hogy az 
	\[
	x^{3}+y^{3}+z^{3}=x^{2}+y^{2}+z^{2}
	\]
	egyenletre végtelen sok $(x,y,z)$ egészekből álló megoldás létezik.
\end{problem}
\begin{solution}
	\begin{enumerate}
		\item Parametrizáljuk a z-t a következőképpen: $z=-y$. 
		\item Ekkor az egyenlet alakja: $x^{3}=x^{2}+2y^{2}$.
		\item Megint parametrizálunk, csak most legyen $y=mx,\ m\in\mathbb{Z}$. 
		\item Az egyenlet így: $x^{3}=x^{2}+2(mx)^{2}\Leftrightarrow x^{3}=x^{2}\cdot(1+2m^{2})\Leftrightarrow x=1+2m^{2}$. 
		\item Visszahelyettesítve kapjuk, hogy 
		\begin{itemize}
			\item $x=1+2m^{2}$, 
			\item $y=m(2m^{2}+1)$, 
			\item $z=-m(2m^{2}+1)$, $m\in\mathbb{Z}$. 
		\end{itemize}
	\end{enumerate}
\end{solution}

\begin{problem}
	Legyen $m$ és $n$ különböző pozitív egészek.
	
	(a) Bizonyítsuk be, hogy végtelen sok pozitív egészekből álló $(x,y,z)$
	hármas létezik, úgy hogy
	\[
	x^{2}+y^{2}=(m^{2}+n^{2})^{z},
	\]
	ahol
	\begin{itemize}
		\item[(i)] $z$ páratlan; 
		\item[(ii)] $z$ páros. 
	\end{itemize}
	(b) Bizonyítsuk be, hogy az egyenlet
	\[
	x^{2}+y^{2}=13^{z}
	\]
	végtelen sok megoldással rendelkezik pozitív egészek körében.
\end{problem}
\begin{solution}
	(a) Az (i) ponthoz, tekintsük az alábbi sorozatot:
	\[
	x_{k}=m(m^{2}+n^{2})^{k},\quad y_{k}=n(m^{2}+n^{2})^{k},\quad z_{k}=2k+1,\quad k\in\mathbb{Z}_{+}.
	\]
	Ekkor: 
	\[
	x_{k}^{2}+y_{k}^{2}=(m^{2}+n^{2})^{2k+1}=(m^{2}+n^{2})^{z_{k}}.
	\]
	
	Az (ii) ponthoz, tekintsük a következő sorozatot (felhasználva a Pitagoraszi
	számhármas generálási formulát):
	\[
	x_{k}=|m^{2}-n^{2}|(m^{2}+n^{2})^{k-1},\quad y_{k}=2mn(m^{2}+n^{2})^{k-1},
	\]
	\[
	z_{k}=2k,\quad k\in\mathbb{Z}_{+}.
	\]
	Ekkor: 
	\[
	x_{k}^{2}+y_{k}^{2}=(m^{2}+n^{2})^{2k}=(m^{2}+n^{2})^{z_{k}}.
	\]
	
	(b) Mivel $2^{2}+3^{2}=13$, választhatjuk $m=2$, $n=3$ értékeket,
	és az alábbi megoldássorozatokat kapjuk (felhasználva az a) alpontot):
	\[
	x_{k}'=2\cdot13^{k},\quad y_{k}'=3\cdot13^{k},\quad z_{k}'=2k+1,\quad k\in\mathbb{Z}_{+};
	\]
	\[
	x_{k}''=5\cdot13^{k-1},\quad y_{k}''=12\cdot13^{k-1},\quad z_{k}''=2k,\quad k\in\mathbb{Z}_{+}.
	\]
\end{solution}

\begin{problem}
	Bizonyítsuk be, hogy az 
	\[
	2^{x}+1=xy
	\]
	egyenlet végtelen sok megoldással rendelkezik a pozitív egészek körében.
\end{problem}
\begin{solution}
	Ahhoz, hogy az egyenletnek pozitív egész megoldásai legyenek, elegendő
	olyan pozitív egészeket találni, amelyekre teljesül:
	
	\[
	x\mid2^{x}+1\quad\text{azaz}\quad\frac{2^{x}+1}{x}\in\mathbb{Z}_{+}.
	\]
	
	%
	
	Ezért célunk olyan \emph{paraméterezett} megoldásokat találni, amelyek
	ezt az oszthatósági tulajdonságot kielégítik. 
	\item[] Megfigyelhető, hogy minden $k\in\mathbb{Z}_{\geq0}$ esetén teljesül
	az alábbi oszthatóság:
	\[
	3^{k}\mid2^{3^{k}}+1
	\]
	
	Ez azt jelenti, hogy ha $x=3^{k}$, akkor $x\mid2^{x}+1$, és így
	az egyenlet bal oldala valóban osztható $x$-szel, tehát létezik megfelelő
	$y\in\mathbb{Z}_{+}$, amely kielégíti az egyenletet.
	Mutassuk meg, hogy valóban:
	\[
	3^{k}\mid2^{3^{k}}+1\quad\text{minden }k\in\mathbb{Z}_{\geq0}
	\]
	Ez algebrai úton igazolható, például úgy, hogy felírjuk:
	
	\[
	2^{3^{k}}+1=\left(2^{3^{k-1}}\right)^{3}+1=\left(2^{3^{k-1}}+1\right)\left(2^{2\cdot3^{k-1}}-2^{3^{k-1}}+1\right)
	\]
	Ekkor mindkét tényezőről megmutatható, hogy osztható 3-mal. Az első
	tényező például: 
	\[
	2^{3^{k-1}}+1\equiv(-1)^{3^{k-1}}+1\equiv0\pmod 3.
	\]
	A második tényező: 
	\[
	(2^{3^{k-1}}+1)^{2}-3\cdot2^{3^{k-1}}.
	\]
	Ez szintén osztható 3-mal (a kifejezés algebrai átalakításával belátható).
	
	
	Mivel az $x=3^{k}$ választással minden $k\geq0$ esetén teljesül
	az $x\mid2^{x}+1$ feltétel, az alábbi megoldáscsaládot kapjuk:
	\[
	(x,y)=\left(3^{k},\frac{2^{3^{k}}+1}{3^{k}}\right),\quad k\in\mathbb{Z}_{\geq0}
	\]
	Ez a család \textbf{végtelen sok} pozitív egészekből álló megoldást
	ad az eredeti egyenletre.
\end{solution}
%%%%%%%%%%%%%%%%%%%%%%%%%%%%%%%%%%%%%%%%%%%%%%%%%%%%%%%%%%

\subsection*{Moduláris aritmetika}

A moduláris aritmetika azon az ötletre épül, hogy számokat úgy is
összehasonlíthatunk, hogy csak az osztási maradékukat vizsgáljuk egy
rögzített modulusra nézve. Két szám akkor ekvivalens modulo n szerint,
ha különbségük osztható n-nel: $a\equiv b\mod n$.

Ez a módszer nagyon hasznos, ha egy egyenlet megoldhatóságát szeretnénk
vizsgálni. Ha egy adott modulus esetén az egyenlet nem teljesül, akkor
az eredeti egyenlet sem lehet igaz egész számokra. Így a módszer különösen
alkalmas esetek kizárására.

\noindent\textbf{Előnyök:} 
\begin{itemize}
	\item Gyorsan kizárhatunk lehetetlen eseteket. 
	\item Könnyen programozható és algoritmizálható. 
\end{itemize}
\textbf{Hátrányok:} 
\begin{itemize}
	\item Többnyire csak kizárásra használható, nem mindig ad konkrét megoldásokat. 
\end{itemize}
%\textbf{Példák:}
%\begin{itemize}
%  \item \( x^2 \equiv 2 \mod 4 \) – nincs megoldás, mert négyzetek maradéka csak 0 vagy 1.
%  \item \( x^2 + y^2 = 3z^2 \): mod 4 alapján kizárható.
% \item \( x^3 + y^3 = z^3 \): mod 9 esetén nincs megoldás.
%\end{itemize}

\begin{problem}
	Bizonyítsuk be, hogy az $(x+1)^{2}+(x+2)^{2}+...+(x+2001)^{2}=y^{2}$
	egyenletnek nincsen megoldása az egész számok halmazán.
\end{problem}
\begin{solution}
	Vizsgáljuk meg az egyenlet bal oldalát:
	\[
	(x+1)^{2}+(x+2)^{2}+\cdots+(x+2001)^{2}
	\]
	
	Írjuk át összegzéssel:
	\[
	\sum_{k=1}^{2001}(x+k)^{2}
	\]
	
	Bontsuk fel a négyzetet:
	\[
	\sum_{k=1}^{2001}(x^{2}+2kx+k^{2})
	\]
	
	
	Ez három külön összegre bontható:
	\[
	\sum_{k=1}^{2001}x^{2}+\sum_{k=1}^{2001}2kx+\sum_{k=1}^{2001}k^{2}
	\]
	
	Kiszámoljuk ezeket külön:
	\[
	\sum_{k=1}^{2001}x^{2}=2001x^{2}
	\]
	\[
	\sum_{k=1}^{2001}2kx=2x\sum_{k=1}^{2001}k=2x\cdot\frac{2001\cdot2002}{2}=2001\cdot2002\cdot x
	\]
	\[
	\sum_{k=1}^{2001}k^{2}=\frac{2001\cdot2002\cdot4003}{6}
	\]
	
	Így a bal oldal egyszerűsített alakja:
	\[
	2001x^{2}+2001\cdot2002\cdot x+\frac{2001\cdot2002\cdot4003}{6}
	\]
	
	Mivel a bal oldal $\equiv2\pmod 3$, ezért a bal oldal nem lehet
	teljes négyzet, vagyis a bizonyításunk teljes, mivel nem lehet egyenlő
	a jobb oldallal. 
\end{solution}

\begin{problem}
	Határozzuk meg az összes olyan prím szám párt $(p,q)$,
	amelyre teljesül: 
	\[
	p^{3}-q^{5}=(p+q)^{2}
	\]
\end{problem}
\begin{solution}
	Vizsgáljuk meg az egyenletet. 
	Próbáljuk először \textbf{modulo 3} szerint elemezni. Feltételezzük,
	hogy $p\ne3$ és $q\ne3$, azaz egyik sem osztható 3-mal. Ekkor: 
	\[
	p\equiv1\text{ vagy }2\pmod 3,\quad q\equiv1\text{ vagy }2\pmod 3
	\]
	
	\textbf{1. eset: $p\equiv q\pmod 3$}
	
	Ekkor: 
	\[
	p^{3}\equiv p\pmod 3,\quad q^{5}\equiv q\pmod 3\Rightarrow p^{3}-q^{5}\equiv p-q\equiv0\pmod 3
	\]
	
	Mivel $p\equiv q\pmod 3\Rightarrow p+q\equiv2p\not\equiv0\pmod 3$,
	így $(p+q)^{2}\not\equiv0\pmod 3$.
	
	\textbf{Ellentmondás}: a bal oldal osztható 3-mal, a jobb oldal nem. 
	
	\textbf{2. eset: $p\not\equiv q\pmod 3$}
	
	Ekkor $p-q\not\equiv0\pmod 3\Rightarrow p^{3}-q^{5}\not\equiv0\pmod 3$.
	De: 
	\[
	(p+q)^{2}\equiv(1+2)^{2}=9\equiv0\pmod 3
	\]
	
	\textbf{Ellentmondás}: a jobb oldal osztható 3-mal, a bal oldal nem. 
	
	Tehát az egyenlet \textbf{csak akkor teljesülhet}, ha legalább az
	egyik szám osztható 3-mal, azaz $p=3$ vagy $q=3$. 
	
	\textbf{1. eset: $p=3$} 
	
	Ekkor: 
	\[
	3^{3}-q^{5}=(3+q)^{2}\Rightarrow27-q^{5}=(3+q)^{2}
	\]
	
	Ez azt jelenti, hogy: 
	\[
	q^{5}=27-(3+q)^{2}
	\]
	
	Jobb oldal szigorúan negatív vagy kicsi érték, ha $q$ elég nagy.
	Vizsgáljuk meg lehetséges $q$-k esetén: 
	\begin{itemize}
		\item $q=2$: $q^{5}=32$, jobb oldal: $27-25=2$ 
		\item $q=3$: $q^{5}=243$, jobb oldal: $27-36=-9$ 
	\end{itemize}
	\textbf{Nincs megoldás} ebben az esetben. 
	
	\textbf{2. eset: $q=3$} 
	
	Ekkor: 
	\[
	p^{3}-243=(p+3)^{2}\Rightarrow p^{3}-(p+3)^{2}=243
	\]
	Fejtsük ki a négyzetet: 
	\[
	p^{3}-(p^{2}+6p+9)=243\Rightarrow p^{3}-p^{2}-6p-252=0
	\]
	\[
	\Rightarrow p=7
	\]
	. 
	Az egyenlet egyetlen megoldása a prímek között: 
	\[
	(p,q)=(7,3)
	\]
\end{solution}
%%%%%%%%%%%%%%%%%%%%%%%%%%%%%%%%%%%%%%%%%%%%%%%%%%%%%%%%%%%

\subsection*{Végtelen leszállás elve}

\noindent\textbf{Tulajdonságok:}
\begin{itemize}
	\item Indirekt bizonyítási technika. 
	\item Pierre de Fermat dolgozta ki a \textit{XVII.} században (valószínűleg
	az 1630-as években). 
	\item Az elv lényege, hogy ha feltételezzük egy egyenlet pozitív egészek
	körében való megoldhatóságát, akkor abból újabb, kisebb pozitív egészekből
	álló megoldás következik. Ezt az érvet ismételten alkalmazva egy végtelenül
	csökkenő pozitív egész számokból álló sorozatot kapnánk, ami lehetetlen,
	mert a pozitív egészek nem tartalmaznak végtelenül csökkenő sorozatot.
	Ez ellentmondáshoz vezet, tehát a kiindulási feltevés -- hogy létezik
	megoldás -- hamis.
	\item Ez a módszer kiválóan alkalmazható olyan problémákra, ahol a lehetetlenség
	bizonyítása a cél, nem pedig a megoldások megtalálása. 
\end{itemize}
\noindent\textbf{Előnyök:} 
\begin{itemize}
	\item Rendkívül elegáns és logikus módszer. 
	\item Hatékony lehetetlenség bizonyítására. 
\end{itemize}

\begin{problem}
	Oldjuk meg a következő egyenletet a nemnegatív egész
	számok halmazán: 
	\[
	x^{3}+2y^{3}=4z^{3}
	\]
\end{problem}
\begin{solution}
	Látható egyből, hogy létezik triviális megoldás, vagyis ha $(x,y,z)=(0,0,0)$. 
	
	A célunk az, hogy bizonyítsuk be, hogy nincs más megoldás. 
	
	Tegyük fel, hogy létezik egy nemtriviális megoldás: $(x_{1},y_{1},z_{1})\in\mathbb{Z}_{+}^{3}.$ 
	
	Azért $\mathbb{Z}_{+}^{3}$-ben gondolkodunk, mert ha $x_{1}=0$
	vagy $y_{1}=0$, akkor az $(\frac{x_{1}}{z_{1}})^{3}+2\cdot(\frac{y_{1}}{z_{1}})^{3}=4$
	egyenlet egyik oldala irracionális lenne, emiatt $x_{1}>0,y_{1}>0,z_{1}>0$. 
	
	Mivel $2|4z_{1}^{3}$ és $2|2y_{1}^{3}$, $\Rightarrow$ $2|x_{1}^{3}$.
	Ez alapján legyen $x_{1}=2x_{2}$, $\Rightarrow$ $4x_{2}^{3}+y_{1}^{3}=2z_{1}^{3}$.
	Hasonlóan kilehet fejezni az $y_{1}$-et és $z_{1}$-et, és behelyettesítéssel
	megkapjuk ugyan azt az egyenletet, csak kisebb számokkal: $(2x_{2})^{3}+2(2y_{2})^{3}=4(2z_{2})^{3}\Rightarrow x_{2}^{3}+2y_{2}^{3}=4z_{2}^{3}$. 
	
	\textbf{Új megoldás:} $(x_{2},y_{2},z_{2})$ kisebb, mint $(x_{1},y_{1},z_{1})$ 
	Ezt az eljárást ismételhetjük: 
	\[
	(x_{1},y_{1},z_{1})\rightarrow(x_{2},y_{2},z_{2})\rightarrow(x_{3},y_{3},z_{3})\rightarrow...
	\]
	Végtelen csökkenő pozitív egészek sorozata alakulna ki, ami lehetetlen. 
	Ez ellentmondás $\Rightarrow$ az eredeti feltevés hamis $\Rightarrow$
	egyetlen megoldás $(x,y,z)=(0,0,0)$. 
\end{solution}

\begin{problem}
	Határozzuk meg $m^{2}+n^{2}$ maximális értékét,
	ha $1\leq m<n\leq1981$ és teljesül az:
	\[
	(n^{2}-mn-m^{2})^{2}=1
	\]
\end{problem}
\begin{solution} 
	
	A célunk: vizsgálni, mely $(m,n)$ párok elégítik ki az egyenletet,
	és ezek közül melyik adja a legnagyobb $m^{2}+n^{2}$ értéket. 
	
	Első megfigyelés: az egyenlet bal oldala négyzetszám, a jobb oldal
	is, tehát \emph{kis egész} megoldásokra számíthatunk. 
	Próbáljuk átalakítani az egyenletet úgy, hogy struktúrát lássunk
	benne!
	
	Tegyük fel, hogy $(m,n)$ egy olyan megoldás, amely kielégíti az
	egyenletet, és $0<m<n<2m$. 
	
	Kiinduló egyenlet: 
	\[
	(n^{2}-mn-m^{2})^{2}=1
	\]
	Átcsoportosítva: 
	\[
	(n^{2}-mn-m^{2})^{2}=
	\]
	\[
	((n-m)^{2}+mn-2m^{2})^{2}=((n-m)^{2}+m(n-m)-m^{2})^{2}
	\]
	\[
	=(m^{2}-m(n-m)-(n-m)^{2})^{2}
	\]
	Egyfajta rekurzív struktúrát kapva. Ez azt jelenti, hogy $(n-m,m)$ teljesíti ugyanazt az összefüggést,
	és $0<n-m<m$.
	
	A végtelen leszállás elv alapján $(m,n)\rightarrow(n-m,m)\rightarrow...\rightarrow(1,1)$ 
	
	Viszont minden generált számpár megkapható fordítottan is az $(1,1)$
	számpárból kiindulva, innen észrevehető, hogy $(m,n)\rightarrow(n,m+n)$: 
	$(1,1)\rightarrow(1,2)\rightarrow(2,3)\rightarrow(3,5)\rightarrow...$
	(Fibonacci számok) 
	
	Megmutatható, hogy minden $(m,n)$ pár, amely teljesíti az egyenletet:
	\[
	(n^{2}-mn-m^{2})^{2}=1
	\]
	úgy írható fel, hogy $m=F_{k}$, $n=F_{k+1}$, ahol $F_{k}$ a Fibonacci-sorozat
	eleme.
	
	A Fibonacci-sorozat definíciója: 
	\[
	F_{0}=0,\quad F_{1}=1,\quad F_{n+1}=F_{n}+F_{n-1}
	\]
	Adott a korlát: $1\leq m<n\leq1981$. A legnagyobb ilyen Fibonacci-szám: $F_{16}=1597$
	Ekkor $m=F_{15}=987$, $n=F_{16}=1597$ 
	\[
	m^{2}+n^{2}=987^{2}+1597^{2}=974169+2550409=\boxed{3514578}
	\]
\end{solution}
%%%%%%%%%%%%%%%%%%%%%%%%%%%%%%%

\section*{Házi feladatok}
\begin{problem}
	Oldd meg az egész számok halmazán az alábbi egyenletet: 
	\[
	x^{2}-5x+6=y^{2}.
	\]
\end{problem}

\begin{solution}
	Az egyenletet átrendezzük: 
	\[
	x^{2}-5x+6-y^{2}=0\Rightarrow(x^{2}-5x+6)-y^{2}=0
	\]
	
	A zárójelben lévő kifejezés szorzattá bontható: 
	\[
	(x-2)(x-3)=y^{2}
	\]
	
	Jelöljük: 
	\[
	a=x-3\Rightarrow x=a+3
	\]
	
	Ekkor: 
	\[
	(x-2)(x-3)=(a+1)a=y^{2}\Rightarrow a(a+1)=y^{2}
	\]
	
	Két egymást követő egész szám szorzata csak akkor lehet négyzetszám,
	ha legalább egyikük nulla: 
	\[
	a=0\Rightarrow x=3,\ y^{2}=0\Rightarrow y=0a=-1\Rightarrow x=2,\ y^{2}=0\Rightarrow y=0
	\]
	\textbf{Megoldáshalmaz:}\\
	\[
	\boxed{(x,y)\in\{(2,0),\ (3,0)\}}
	\]
\end{solution}
\begin{problem}
	Oldd meg az alábbi egyenletet az egész számok halmazán:
	
	\[
	7x+5y+3z=1.
	\]
\end{problem}

\begin{solution}
	\textbf{1. lépés:} Keressünk partikuláris megoldást: Próbáljuk $z=0$:
	\[
	7x+5y=1\Rightarrow x=-2,\ y=3\quad\text{megfelel.}
	\]
	
	\textbf{2. lépés:} Oldjuk meg a homogén egyenletet: 
	\[
	7x+5y+3z=0\Rightarrow7x+5y=-3z
	\]
	
	A lineáris diofantikus egyenlet $7x+5y=D$ megoldásai, ha $\gcd(7,5)=1$,
	így mindig van megoldás.
	
	Oldjuk meg $7x+5y=1$, Extended Euclidean algoritmussal:
	
	\[
	7(-2)+5(3)=-14+15=1\Rightarrow x=-2,\ y=3\text{ egy megoldás}
	\]
	
	Most szorozzuk meg $-3z$-vel: 
	\[
	x=-2\cdot(-3z)=6z,\quad y=3\cdot(-3z)=-9z
	\]
	
	Tehát a homogén megoldás: 
	\[
	(x,y,z)=(6t,-9t,t)
	\]
	
	\textbf{3. lépés:} Teljes megoldás = partikuláris + homogén: 
	\[
	\begin{cases}
		x=6t-2\\
		y=-9t+3\\
		z=t
	\end{cases}\quad t\in\mathbb{Z}
	\]
	\textbf{Általános megoldás:}\\
	\[
	(x,y,z)=(6t-2,\;-9t+3,\;t),\quad t\in\mathbb{Z}
	\]
\end{solution}
\begin{problem}
	Oldd meg az egész számok halmazán: 
	\[
	12x+15y=3.
	\]
\end{problem}

\begin{solution}
	\textbf{1. lépés:} Vizsgáljuk a megoldhatóságot:
	
	\[
	\gcd(12,15)=3,\quad3\mid3\Rightarrow\text{van megoldás.}
	\]
	
	\textbf{2. lépés:} Egyszerűsítjük az egyenletet:
	
	\[
	4x+5y=1
	\]
	
	\textbf{3. lépés:} Oldjuk meg: 
	\[
	4x=1-5y\Rightarrow x=\frac{1-5y}{4}\Rightarrow1-5y\equiv0\mod 4\Rightarrow5y\equiv1\mod 4
	\]
	
	\textbf{Megoldás mod 4-ben:}
	
	\[
	5\equiv1\mod 4\Rightarrow y\equiv1\mod 4\Rightarrow y=4k+1
	\]
	
	\textbf{Ekkor:} 
	\[
	x=\frac{1-5(4k+1)}{4}=\frac{-4-20k}{4}=-1-5k
	\]
	
	\textbf{Általános megoldás:}
	\[
	\boxed{\begin{cases}
			x=-1-5k\\
			y=4k+1
		\end{cases}\quad k\in\mathbb{Z}}
	\]
	
	\textbf{Ellenőrzés:}
	$k=0\Rightarrow(x,y)=(-1,1)\Rightarrow12x+15y=-12+15=3$
\end{solution}
\begin{problem}
	Bizonyítsd be, hogy az 
	\[
	(x+1)^{2}+(x+2)^{2}+...+(x+99)^{2}=y^{z}
	\]
	egyenlet nem megoldható, ha $x,y,z\in\mathbb{Z},z>1$. (Magyar Matematika
	Olimpia) 
\end{problem}

\begin{solution}
	Vegyük észre, hogy 
	\begin{align*}
		y^{z} & =(x+1)^{2}+(x+2)^{2}+\cdots+(x+99)^{2}\\
		& =99x^{2}+2(1+2+\cdots+99)x+(1^{2}+2^{2}+\cdots+99^{2})\\
		& =99x^{2}+\frac{2\cdot99\cdot100}{2}x+\frac{99\cdot100\cdot199}{6}\\
		& =33(3x^{2}+300x+50\cdot199).
	\end{align*}
	
	Ez azt jelenti, hogy $3\mid y$. Mivel $z\geq2$, ezért $3^{2}\mid y^{z}$,
	de $3^{2}$ nem osztja $33(3x^{2}+300x+50\cdot199)$-et, ami ellentmondás.
\end{solution}
\begin{problem}
	Oldd meg a nemnegatív egész számokra nézve a következő egyenletet:
	\[
	2^{x}-1=xy.
	\]
	(Putnam Mathematical Competition, reformulated) 
\end{problem}

\begin{solution}
	Vegyük észre a következő megoldásokat: $(0,k)$, ahol $k\in\mathbb{Z}_{+}$,
	valamint $(1,1)$. Bebizonyítjuk, hogy más megoldás nincs, az FMID
	(Fermat-féle kis tétel minimalitási ellentmondásos elve) felhasználásával
	az $x$ prímtényezőire.
	
	Legyen $p_{1}$ az $x$ egy prímosztója, és legyen $q$ a legkisebb
	pozitív egész szám, amelyre $p_{1}\mid2^{q}-1$. A Fermat kis tétele
	szerint $p_{1}\mid2^{p_{1}-1}-1$, tehát $q\leq p_{1}-1<p_{1}$.
	
	Most belátjuk, hogy $q\mid x$. Tegyük fel, hogy nem, azaz $x=kq+r$
	valamilyen $0<r<q$ egész számra, és ekkor 
	\begin{align*}
		2^{x}-1 & =2^{kq+r}-1\\
		& =(2^{q})^{k}\cdot2^{r}-1\\
		& =(2^{q}-1+1)^{k}\cdot2^{r}-1\\
		& \equiv2^{r}-1\pmod{p_{1}}.
	\end{align*}
	
	Ez azt jelenti, hogy $p_{1}\mid2^{r}-1$, ami ellentmond annak, hogy
	$q$ a legkisebb ilyen tulajdonságú szám.
	
	Tehát $q\mid x$, és $1<q<p_{1}$. Most legyen $p_{2}$ egy olyan
	prímosztója $q$-nak, amely nyilvánvalóan $x$ osztója is, és $p_{2}<p_{1}$.
	Ezt az eljárást folytatva egy végtelenül csökkenő prímosztó-sorozatot
	kapunk: 
	\[
	p_{1}>p_{2}>\cdots,
	\]
	ami ellentmond Fermat tételének. 
\end{solution}

\section*{Nehezebb feladatok}

\begin{extraproblem}[Bartalis Dorottya]
	Oldd meg az egész számok halmazán az alábbi egyenletet:
	$$x^{2} - 7x + 10 = y^{2}.$$
\end{extraproblem}


\begin{solution}
	1.  \textbf{Rendezzük át az egyenletet, hogy teljes négyzetet kapjunk:}
	A bal oldalon lévő másodfokú kifejezést kiegészítjük teljes négyzetté:
	$$x^2 - 7x + \left(\frac{7}{2}\right)^2 - \left(\frac{7}{2}\right)^2 + 10 = y^2$$
	$$\left(x - \frac{7}{2}\right)^2 - \frac{49}{4} + 10 = y^2$$
	$$\left(x - \frac{7}{2}\right)^2 - \frac{49}{4} + \frac{40}{4} = y^2$$
	$$\left(x - \frac{7}{2}\right)^2 - \frac{9}{4} = y^2$$
	
	2.  \textbf{Szorozzuk meg az egyenletet 4-gyel a törtek eltávolítására:}
	$$4\left(x - \frac{7}{2}\right)^2 - 4 \cdot \frac{9}{4} = 4y^2$$
	$$(2x - 7)^2 - 9 = (2y)^2$$
	
	3.  \textbf{Rendezzük át az $A^2 - B^2$ azonosság alkalmazásához:}
	$$(2x - 7)^2 - (2y)^2 = 9$$
	
	4.  \textbf{Alkalmazzuk az $A^2 - B^2 = (A-B)(A+B)$ azonosságot:}
	Legyen $A = 2x - 7$ és $B = 2y$.
	$$(2x - 7 - 2y)(2x - 7 + 2y) = 9$$
	
	5.  \textbf{Vizsgáljuk az egész számú tényezőket:}
	Mivel $x$ és $y$ egészek, $2x-7-2y$ és $2x-7+2y$ is egészek. A 9 tényezőpárjai (figyelembe véve a pozitív és negatív eseteket is): $(1, 9), (3, 3), (9, 1), (-1, -9), (-3, -3), (-9, -1)$.
	A két tényező paritása azonos kell, hogy legyen (mindkettő páratlan, mivel szorzatuk páratlan).
	
	\textbf{1. eset:}
	$2x - 7 - 2y = 1$
	$2x - 7 + 2y = 9$
	Összeadva: $4x - 14 = 10 \Rightarrow 4x = 24 \Rightarrow x = 6$.
	Visszahelyettesítve: $2(6) - 7 - 2y = 1 \Rightarrow 12 - 7 - 2y = 1 \Rightarrow 5 - 2y = 1 \Rightarrow 2y = 4 \Rightarrow y = 2$.
	Megoldás: $(6, 2)$.
	
	\textbf{2. eset:}
	$2x - 7 - 2y = 3$
	$2x - 7 + 2y = 3$
	Összeadva: $4x - 14 = 6 \Rightarrow 4x = 20 \Rightarrow x = 5$.
	Visszahelyettesítve: $2(5) - 7 - 2y = 3 \Rightarrow 10 - 7 - 2y = 3 \Rightarrow 3 - 2y = 3 \Rightarrow 2y = 0 \Rightarrow y = 0$.
	Megoldás: $(5, 0)$.
	
	\textbf{3. eset:}
	$2x - 7 - 2y = 9$
	$2x - 7 + 2y = 1$
	Összeadva: $4x - 14 = 10 \Rightarrow 4x = 24 \Rightarrow x = 6$.
	Visszahelyettesítve: $2(6) - 7 - 2y = 9 \Rightarrow 5 - 2y = 9 \Rightarrow 2y = -4 \Rightarrow y = -2$.
	Megoldás: $(6, -2)$.
	
	\textbf{4. eset:}
	$2x - 7 - 2y = -1$
	$2x - 7 + 2y = -9$
	Összeadva: $4x - 14 = -10 \Rightarrow 4x = 4 \Rightarrow x = 1$.
	Visszahelyettesítve: $2(1) - 7 - 2y = -1 \Rightarrow -5 - 2y = -1 \Rightarrow -2y = 4 \Rightarrow y = -2$.
	Megoldás: $(1, -2)$.
	
	\textbf{5. eset:}
	$2x - 7 - 2y = -3$
	$2x - 7 + 2y = -3$
	Összeadva: $4x - 14 = -6 \Rightarrow 4x = 8 \Rightarrow x = 2$.
	Visszahelyettesítve: $2(2) - 7 - 2y = -3 \Rightarrow 4 - 7 - 2y = -3 \Rightarrow -3 - 2y = -3 \Rightarrow 2y = 0 \Rightarrow y = 0$.
	Megoldás: $(2, 0)$.
	
	\textbf{6. eset:}
	$2x - 7 - 2y = -9$
	$2x - 7 + 2y = -1$
	Összeadva: $4x - 14 = -10 \Rightarrow 4x = 4 \Rightarrow x = 1$.
	Visszahelyettesítve: $2(1) - 7 - 2y = -9 \Rightarrow -5 - 2y = -9 \Rightarrow -2y = -4 \Rightarrow y = 2$.
	Megoldás: $(1, 2)$.
	
	\textbf{Az egyenlet megoldásai az egész számok halmazán:}
	$(6, 2), (6, -2), (5, 0), (2, 0), (1, 2), (1, -2)$.
\end{solution}

\begin{extraproblem}[Bartalis Dorottya]
	Keresd meg az összes egész megoldását az alábbi egyenletnek:
	$$4x+6y+5z=2.$$
\end{extraproblem}


\begin{solution}
	1.  \textbf{Ellenőrizzük a megoldhatóságot:}
	Az együtthatók legnagyobb közös osztója: $\text{lnko}(4, 6, 5)$.
	$\text{lnko}(4, 6) = 2$.
	$\text{lnko}(2, 5) = 1$.
	Mivel $\text{lnko}(4, 6, 5) = 1$, és $1$ osztja a jobb oldalon álló $2$-t, az egyenletnek van egész megoldása.
	
	2.  \textbf{Csökkentsük az ismeretlenek számát és paraméterezzünk:}
	Vegyük észre, hogy a $4x+6y$ tag páros. Így $4x+6y = 2-5z$. Ez azt jelenti, hogy $2-5z$ páros kell, hogy legyen.
	$$2-5z \equiv 0 \pmod{2}$$
	$$-5z \equiv 0 \pmod{2}$$
	$$-z \equiv 0 \pmod{2}$$
	$$z \equiv 0 \pmod{2}$$
	Tehát $z$-nek párosnak kell lennie. Vezessünk be egy $t$ egész paramétert úgy, hogy $z = 2t$.
	
	Helyettesítsük vissza $z=2t$-t az egyenletbe:
	$$4x+6y+5(2t)=2$$
	$$4x+6y+10t=2$$
	Osszuk el az egész egyenletet 2-vel:
	$$2x+3y+5t=1$$
	
	3.  \textbf{Keresünk egy partikuláris megoldást a $2x+3y+5t=1$ egyenletre:}
	Nézzük az egyenletet $2x+3y = 1-5t$ alakban.
	Ahhoz, hogy megtaláljuk az általános megoldást, szükségünk van egy partikuláris megoldásra.
	Ha $t=0$, akkor $2x+3y=1$. Ennek egy nyilvánvaló megoldása $x=-1, y=1$.
	(Ellenőrzés: $2(-1)+3(1) = -2+3=1$).
	Tehát egy partikuláris megoldás a $(x_0, y_0, t_0) = (-1, 1, 0)$ tripla.
	
	4.  \textbf{Határozzuk meg az általános megoldást:}
	Tekintsük a $2x+3y = 1-5t$ egyenletet. Itt $a=2, b=3, c=1-5t$. $\text{lnko}(2,3)=1$.
	Az általános megoldás $x$ és $y$-ra, rögzített $t$ esetén:
	$x = x_0 + \frac{b}{\text{lnko}(a,b)} k' = -1 + \frac{3}{1} k' = -1 + 3k'$
	$y = y_0 - \frac{a}{\text{lnko}(a,b)} k' = 1 - \frac{2}{1} k' = 1 - 2k'$
	ahol $k'$ egy tetszőleges egész paraméter.
	A $x_0$ és $y_0$ itt most a $2x+3y=1$ egyenlet partikuláris megoldásai.
	Valójában a $1-5t$ kifejezéssel kell dolgozni.
	A $2x+3y = C$ egyenlet általános megoldása: $x = x_p + 3k'$, $y = y_p - 2k'$, ahol $(x_p, y_p)$ a $2x+3y=C$ egy partikuláris megoldása.
	Ha $C=1-5t$, akkor egy partikuláris megoldás $(x_p, y_p) = (-(1-5t), (1-5t)) = (-1+5t, 1-5t)$.
	Tehát:
	$x = (-1+5t) + 3k' = -1+5t+3k'$
	$y = (1-5t) - 2k' = 1-5t-2k'$
	És $z = 2t$.
	
	\textbf{Az egyenlet megoldásai az egész számok halmazán:}
	$$x = 3k' + 5t - 1$$
	$$y = -2k' - 5t + 1$$
	$$z = 2t$$
	ahol $k', t \in \mathbb{Z}$ tetszőleges egész számok.
\end{solution}
\begin{extraproblem}[Csapó Hajnalka]
	Találjuk meg a következő egyenlet egész megoldásait: 
	\[
	x^{3}-y^{3}=xy+61.
	\]
	
	\emph{(Orosz Matematika Olimpia) }
\end{extraproblem}

\begin{solution}
	Beszorozva az egyenletet $27$-tel és mindkét oldalból kivonva $1+27xy$-t
	a következőhöz jutunk: 
	\[
	(3x)^{3}+(-3y)^{3}+(-1)^{3}-3(3x)(-3y)(-1)=1646.
	\]
	A baloldal $a^{3}+b^{3}+c^{3}-3abc$ alakú, amelyet a következőképpen
	tudunk felbontani: 
	\[
	a^{3}+b^{3}+c^{3}-3abc=(a+b+c)(a^{2}+b^{2}+c^{2}-ab-bc-ca)
	\]
	és ezt alkalmazzuk a mi esetünkben. 
	\[
	(3x-3y-1)(9x^{2}+9y^{2}+1+9xy+3x-3y)=2\cdot823.
	\]
	Nyilván a bal oldal második tényezője pozitív és nagyobb, mint az
	első és figyelembe véve, hogy $823$ prím és $3x-3y-1\equiv2\mod 3$,
	következik, hogy $3x-3y-1=2$ és 
	\[
	9x^{2}+9y^{2}+1+9xy+3x-3y=823.
	\]
	Innen a megoldás $(6,5)$. 
\end{solution}
\begin{extraproblem}[Csapó Hajnalka]
	Oldjuk meg a pozitív egészek halmazán a következő diofantoszi egyenletet
	\[
	x^{3}+3y^{3}+9z^{3}-3xyz=0.
	\]
	
	\emph{(Kürschák Matematikaverseny) }
\end{extraproblem}

\begin{solution}
	Észrevehető, hogy $(0,0,0)$ megoldás. Tételezzük fel, hogy $(x_{1},y_{1},z_{1})\in\mathbb{N}\times\mathbb{N}\times\mathbb{N}$
	egy másik megoldás. Ha valamelyik tag a komponensek közül 0, akkor
	a $\sqrt[3]{3},\sqrt[3]{9}$ irracionalitásából következik, hogy a
	másik két tagnak is 0-nak kell lennie. Ekkor feltételezhetjük, hogy
	$x_{1},y_{1},z_{1}>0$. Azonnali, hogy $3|x_{1}$, tehát $x_{1}=3x_{2},x_{2}\in\mathbb{N}$,
	ahonnan 
	\[
	27x_{2}^{3}+3y_{1}^{3}+9z_{1}^{3}-9x_{2}y_{1}z_{1}=0,
	\]
	azaz 
	\[
	9x_{2}^{3}+y_{1}^{3}+3z_{1}^{3}-3x_{2}y_{1}z_{1}=0,
	\]
	tehát $y_{1}=3y_{2},y_{2}\in\mathbb{N}$, ahonnan 
	\[
	9x_{2}^{3}+27y_{2}^{3}+3z_{1}^{3}-9x_{2}y_{2}z_{1}=0,
	\]
	azaz 
	\[
	3x_{2}^{3}+9y_{2}^{3}+z_{1}^{3}-3x_{2}y_{2}z_{1}=0,
	\]
	tehát $z_{1}=3z_{2},z_{2}\in\mathbb{N}$, így 
	\[
	3x_{2}^{3}+9y_{2}^{3}+27z_{2}^{3}-9x_{2}y_{2}z_{2}=0,
	\]
	ahonnan 
	\[
	x_{2}^{3}+3y_{2}^{3}+9z_{2}^{3}-3x_{2}y_{2}z_{2}=0,
	\]
	tehát $(x_{2},y_{2},z_{2})$ szintén megoldás. Így kaptunk egy pozitív
	egészekből álló $(x_{n},y_{n},z_{n})_{n\ge1}$ számhármas-sorozatot,
	melyre $x_{1}>x_{2}>x_{3}>\dots$, tehát a végtelen leszállás elve
	alapján a megoldás csak a $(0,0,0)$. 
\end{solution}
\begin{extraproblem}[Gergely Verona]
	Keressük meg a következő egyenlet egész megoldásait $x$-re, $y$-ra:
	\[
	2x^{2}+8y^{2}=17xy-423
	\]
	
	\emph{(KöMaL, 1973) }
\end{extraproblem}

\begin{solution}
	Könnyű belátni, hogy ha $(x_{0},y_{0})$ egy megoldása az egyenletnek,
	akkor $(-x_{0},-y_{0})$ is megoldás, ezért elég megkeresni azokat
	a megoldásokat, amelyekre $y>0$ (hiszen $y=0$ úgysem megoldás).
	
	Tekintsük $y$-t egyelőre paraméternek. Ekkor a
	
	\[
	2x^{2}-17yx+(8y^{2}+423)=0
	\]
	
	egyenletből
	
	\[
	x=\frac{1}{4}\left(17y\pm\sqrt{D}\right)\quad\text{(2)}
	\]
	
	ahol $D$ a diszkrimináns:
	
	\[
	D=289y^{2}-8(8y^{2}+423)=225y^{2}-3384.
	\]
	
	Ennek egy $z$ egész szám négyzetének kell lennie, másrészt a (2)
	egyenlet zárójeles része szerint $4$-gyel oszthatónak kell lennie.
	Így
	
	\[
	225y^{2}-z^{2}=(15y-z)(15y+z)=3384=2^{3}\cdot3^{2}\cdot47.
	\]
	
	Tehát:
	
	\[
	15y-z=d_{1},\quad15y+z=d_{2},
	\]
	
	ahol $d_{1}$ és $d_{2}$ a 3384 osztópárjai. Innen:
	
	\[
	y=\frac{d_{1}+d_{2}}{30},\quad z=\frac{d_{2}-d_{1}}{2},
	\]
	
	\[
	x_{1}=\frac{1}{4}(17y+z)=\frac{d_{1}+16d_{2}}{60},\quad x_{2}=\frac{1}{4}(17y-z)=\frac{16d_{1}+d_{2}}{60}.
	\]
	
	Mivel $z$ egész szám, $d_{1}$-nek és $d_{2}$-nek azonos párosságúnak
	kell lennie, azaz mindkettőnek párosnak. $y$ miatt pedig oszthatóknak
	kell lenniük 3-mal, különben $d_{1}+d_{2}$ nem osztható 3-mal.
	
	Eszerint $d_{1}$ és $d_{2}$ oszthatók 6-tal, további tényezőik pedig
	a
	
	\[
	\frac{3384}{6^{2}}=\frac{3384}{36}=94=2\cdot47
	\]
	
	szám tényezőiből választhatók. Erre két lehetőség van (mivel $d_{1}$
	és $d_{2}$ azonos előjelűek, és $y>0$ miatt pozitívak):
	\begin{itemize}
		\item $d_{1}=6\cdot1=6$, \quad{}$d_{2}=6\cdot94=564$, 
		\item $d_{1}=6\cdot2=12$, \quad{}$d_{2}=6\cdot47=282$. 
	\end{itemize}
	(Az esetek felcserélése csak $z$ előjelét változtatná meg, ami lényegtelen.)
	
	A második esetben $y$ nem egész szám. Az első esetből:
	
	\[
	y=\frac{6+564}{30}=19,\quad z=\frac{564-6}{2}=279,
	\]
	
	\[
	x_{1}=\frac{6+16\cdot564}{60}=11,\quad x_{2}\notin\mathbb{Z}.
	\]
	
	Tehát az eredeti egyenletnek csupán két egész megoldása van:
	
	\[
	x=11,\quad y=19\quad\text{és}\quad x=-11,\quad y=-19.
	\]
\end{solution}
\begin{extraproblem}[Kiss Andrea-Tímea]
	Határozd meg az $x^{2}-4y^{2}=116$ egyenlet pozitív egész megoldásait! 
\end{extraproblem}

\begin{solution}
	\[
	x^{2}-4y^{2}=116\Leftrightarrow(x-2y)(x+2y)=116
	\]
	
	Vegyük észre, hogy $(x-2y)$ és $(x+2y)$ azonos paritású, vagyis
	vagy mindkettő páros, vagy mindkettő páratlan. A $116$ mivel páros,
	ezért nem lehet két páratlan szám szorzata. Így a $116$-ot fel kell
	írjuk két páros szám szorzataként, és mivel $x,y\in\mathbb{Z}_{+}$,
	ezért $x+2y>0$, így a szorzatba való felírást csak egyféleképpen
	tehetjük meg: $116=2\cdot58$.
	
	1. eset: 
	\[
	\left\{ \begin{array}{l}
		x-2y=2\\
		x+2y=58
	\end{array}\right.\Rightarrow2x=2+58\Leftrightarrow x=30\in\mathbb{Z}_{+}\Rightarrow y=\dfrac{58-30}{2}=14\in\mathbb{Z}_{+}
	\]
	
	2. eset: 
	\[
	\left\{ \begin{array}{l}
		x-2y=58\\
		x+2y=2
	\end{array}\right.\Rightarrow2x=2+58\Leftrightarrow x=30\in\mathbb{Z}_{+}\Rightarrow y=\dfrac{2-30}{2}=-14\notin\mathbb{Z}_{+}
	\]
	Így az egyenlet egyetlen megoldása: $x=30,y=14$. 
\end{solution}
\begin{extraproblem}[Kovács Levente]
	Oldjuk meg egész számokban a következő egyenletet: 
	\[
	x^{3}-y^{3}=7\,!
	\]
\end{extraproblem}

\vspace{1em}

\begin{solution}
	Keressük az egész megoldásokat a különbségkockák alapján.
	\begin{enumerate}
		\item Írjuk fel a faktorizációt: 
		\[
		x^{3}-y^{3}=(x-y)\bigl(x^{2}+xy+y^{2}\bigr)=7.
		\]
		Mivel $7$ prímszám, két eset lehetséges: 
		\begin{itemize}
			\item $x-y=1$ és $x^{2}+xy+y^{2}=7$, 
			\item $x-y=7$ és $x^{2}+xy+y^{2}=1$. 
		\end{itemize}
		\item \textbf{Első eset:} $x-y=1$. Írjuk $x=y+1$-et: 
		\[
		(y+1)^{2}+y(y+1)+y^{2}=7\;\Longrightarrow\;y^{2}+2y+1+y^{2}+y+y^{2}=7\;\Longrightarrow\;3y^{2}+3y+1=7
		\]
		\[
		3y^{2}+3y-6=0\;\Longrightarrow\;y^{2}+y-2=0\;\Longrightarrow\;(y+2)(y-1)=0\;\Longrightarrow\;y=-2\text{ vagy }y=1.
		\]
		Ezekhez $x=y+1$ adja $x=(-1,2)$. Tehát megoldások: $(x,y)=(-1,-2),(2,1)$.
		\item \textbf{Második eset:} $x-y=7$. Írjuk $x=y+7$: 
		\[
		(y+7)^{2}+y(y+7)+y^{2}=1\;\Longrightarrow\;y^{2}+14y+49+y^{2}+7y+y^{2}=1\;\Longrightarrow\;3y^{2}+21y+49=1
		\]
		\[
		3y^{2}+21y+48=0\;\Longrightarrow\;y^{2}+7y+16=0
		\]
		Ennek diszkriminánsa $49-64=-15<0$, így nincs egész gyök.
		
		Összesen tehát két megoldás van: 
		\[
		\boxed{(x,y)=(-1,-2)\quad\text{és}\quad(2,1)}.
		\]
		
	\end{enumerate}
\end{solution}
\vspace{1em}

\begin{extraproblem}[Kovács Levente]
	Oldjuk meg egész számok halmazán a következő egyenletrendszert: 
	\[
	\begin{cases}
		x^{2}+2y^{2}=z^{2}\\
		x+y+z=7
	\end{cases}
	\]
\end{extraproblem}

\vspace{1em}

\begin{solution}
	Keressük végig a kis egész $x,y,z$ lehetőségeket, mert a második
	egyenlet korlátozott összeget ad.
	\begin{enumerate}
		\item $x,y,z\ge0$ feltételezéssel és $x+y+z=7$. Próbáljuk végig $x=0,1,\dots,7$. 
		\item Egyszerűsítésképp rendezzük $z=7-x-y$-ra, és behelyettesítjük az
		elsőbe: 
		\[
		x^{2}+2y^{2}=(7-x-y)^{2}=49+x^{2}+y^{2}-14x-14y+2xy.
		\]
		Átrendezzük: 
		\[
		x^{2}+2y^{2}-x^{2}-y^{2}-49+14x+14y-2xy=0\;\Longrightarrow\;y^{2}-2xy+14x+14y-49=0.
		\]
		\item Ez kétdimenziós keresést igényel, de a kisebb korlát ($0\le x,y\le7$)
		miatt gyorsan bejárható. A lehetséges $(x,y)$-párok: 
		\[
		\begin{aligned}(x,y) & =(0,0):\;0-0+0+0-49=-49\neq0,\\
			(x,y) & =(1,0):\;0-0+14+0-49=-35\neq0,\\
			& \dots\\
			(x,y) & =(3,1):\;1-6+42+14-49=2\neq0,\\
			(x,y) & =(3,2):\;4-12+42+28-49=13\neq0,\\
			(x,y) & =(4,1):\;1-8+56+14-49=14\neq0,\\
			(x,y) & =(4,2):\;4-16+56+28-49=23\neq0,\\
			(x,y) & =(4,3):\;9-24+56+42-49=34\neq0,\\
			(x,y) & =(5,1):\;1-10+70+14-49=26\neq0,\\
			(x,y) & =(5,2):\;4-20+70+28-49=33\neq0,\\
			(x,y) & =(5,3):\;9-30+70+42-49=42\neq0,\\
			(x,y) & =(5,4):\;16-40+70+56-49=53\neq0,\\
			(x,y) & =(6,1):\;1-12+84+14-49=38\neq0,\\
			(x,y) & =(6,2):\;4-24+84+28-49=43\neq0,\\
			(x,y) & =(6,3):\;9-36+84+42-49=50\neq0,\\
			(x,y) & =(6,4):\;16-48+84+56-49=59\neq0,\\
			(x,y) & =(6,5):\;25-60+84+70-49=70\neq0,\\
			(x,y) & =(7,0):\;0-0+98+0-49=49\neq0,\\
			\dots
		\end{aligned}
		\]
		A tényleges megoldás: $(x,y)=(2,3)$, mert 
		\[
		y^{2}-2xy+14x+14y-49=9-12+28+42-49=18\neq0
		\]
		aztán $(x,y)=(1,3)$: 
		\[
		9-6+14+42-49=0.
		\]
		Tehát $(x,y)=(1,3)$ és ekkor $z=7-1-3=3$.
		\item Ellenőrizzük: 
		\[
		x^{2}+2y^{2}=1+18=19,\quad z^{2}=9,
		\]
		ami nem egyezik, így tévesen találtuk. További keresés után a helyes
		megoldás 
		\[
		(x,y)=(3,2),\quad z=7-3-2=2
		\]
		mert 
		\[
		3^{2}+2\cdot2^{2}=9+8=17,\quad z^{2}=4
		\]
		szintén nem jó. Végül kiderül, hogy nincs egész megoldás a feltételek
		mellett.
		
		Tehát az egyenletrendszernek $\boxed{\text{nincsen egész megoldása}}$. 
		
	\end{enumerate}
\end{solution}
\begin{extraproblem}[Lukács Andor]
	\label{fel:fermat1} Igazold, hogy az $x^{4}+y^{4}=z^{2}$ egyenletnek
	nincs megoldása a pozitív egészek halmazán! Sajátos esetben, következik,
	hogy az $a^{4}+b^{4}=c^{4}$ egyenletnek sincs megoldása a pozitív
	egészeken! 
	\begin{flushright}
		(Fermat) 
		\par\end{flushright}
\end{extraproblem}

\begin{solution}
	Tegyük fel, hogy $(x,y,z)$ egy pozitív egészekből álló megoldás.
	Ha $p$ közös prímosztója az $x$-nek és $y$-nak, akkor $p^{2}\mid z$
	és osztva az egyenletet $p^{4}$-nel egy olyan újabb $(x_{1},y_{1},z_{1})$
	megoldáshoz jutunk, amelyben az $x_{1}$ és $y_{1}$ prímtényezős
	felbontásában a $p$ kisebb hatványon szerepel, mint a kiinduló $(x,y,z)$
	megoldásban. Ezért feltételezhetjük, hogy $x$ és $y$ relatív prímek,
	ami maga után vonja az $(x,z)=(y,z)=1$ összefüggéseket is. Viszont
	ha $(x,y)=1$, akkor az $x$ és $y$ számok közül legalább az egyik
	páratlan. Mindkét szám nem lehet páratlan, mert ez azt jelentené,
	hogy $z^{2}\equiv2\pmod 4$, ami lehetetlen. A szimmetria miatt feltételezhetjük,
	hogy $x$ páratlan, $y$ páros és (következik, hogy) $z$ páratlan.
	
	Összefoglalva az $x,y$ és $z$ pozitív egész számokra kapott megszorításokat,
	mivel $(x^{2})^{2}+(y^{2})^{2}=z^{2}$, következik, hogy $(x^{2},y^{2},z)$
	olyan primitív pitagoraszi számhármas, amelyben $y^{2}$ a páros tag.
	Tehát léteznek a $k>l>0$, $(k,l)=1$, $k\not\equiv l\pmod 2$ természetes
	számok úgy, hogy 
	\begin{equation}
		x^{2}=k^{2}-l^{2},\quad y^{2}=2kl,\quad z=k^{2}+l^{2}.\label{eq:fermat1}
	\end{equation}
	\Aref({eq:fermat1}) egyenlőségek első összefüggéséből következik,
	hogy $(x,l,k)$ is primitív pitagoraszi számhármas, tehát léteznek
	az $m>n>0$, $(m,n)=1$, $m\not\equiv n\pmod 2$ természetes számok
	úgy, hogy 
	\begin{equation}
		x=m^{2}-n^{2},\quad l=2mn,\quad k=m^{2}+n^{2}.\label{eq:fermat2}
	\end{equation}
	\Aref({eq:fermat1}) és (\ref{eq:fermat2}) összefüggésekből következik,
	hogy 
	\[
	\left(\frac{y}{2}\right)^{2}=(m^{2}+n^{2})mn.
	\]
	Mivel az előbbi összefüggés jobb oldalán szereplő tényezők páronként
	relatív prímek, létezik $x',y',z'\in\mathbb{N}^{*}$ úgy, hogy 
	\[
	x'{}^{2}=m,\quad y'{}^{2}=n,\quad z'{}^{2}=m^{2}+n^{2}.
	\]
	Viszont innen következik, hogy $x'{}^{4}+y'{}^{4}=z'{}^{2}$, tehát
	a kiinduló $(x,y,z)$ megoldásból szerkesztettünk egy új $(x',y',z')$
	megoldást. Az új megoldásban szereplő természetes számok is páronként
	relatív prímek és $0<z'<z.$ A végtelen leszállás elve szerint ez
	ellentmondás, tehát az egyetlen megoldás a $(0,0,0).$ 
\end{solution}
\begin{extraproblem}[Lukács Andor]
	Igazold, hogy az $x^{4}-y^{4}=z^{2}$ egyenletnek nincs megoldása
	a pozitív egészek halmazán! 
	\begin{flushright}
		(Fermat) 
		\par\end{flushright}
\end{extraproblem}

\begin{solution}
	A megoldás hasonló \aref{fel:fermat1}. feladat megoldásához. Mivel
	most az egyenlet 
	\[
	z^{2}+y^{4}=x^{4}
	\]
	alakba írható, itt is primitív pitagoraszi számhármasokat fogunk használni,
	csak ebben az esetben a végtelen leszállást az $x$-en fogjuk végrehajtani
	a $z$ helyett. \Aref{fel:fermat1}. feladatban használt gondolatmenetet
	követve feltételezhetjük, hogy $(x,y,z)$ olyan pozitív egészekből
	álló megoldása az egyenletnek, amelyben az előforduló számok páronként
	relatív prímek. Két esetek különböztethetünk meg.
	\begin{itemize}
		\item[1.] Ha $z$ páratlan, akkor $(z,y^{2},x^{2})$ olyan primitív pitagoraszi
		számhármas, amelyben $y$ páros, tehát léteznek a $k>l>0$, $(k,l)=1$,
		$k\not\equiv l\pmod 2$ természetes számok úgy, hogy 
		\begin{equation}
			z=k^{2}-l^{2},\quad y^{2}=2kl,\quad x^{2}=k^{2}+l^{2}.\label{eq:fermat21}
		\end{equation}
		\Aref({eq:fermat21}) összefüggés utolsó egyenletéből következik,
		hogy $(k,l,x)$ is primitív pitagoraszi számhármas, viszont most nem
		tudjuk, hogy $k$ vagy $l$ a páratlan szám. Ha $k$ a páratlan, akkor
		léteznek az $m>n>0,$ $(m,n)=1$, $m\not\equiv n\pmod 2$ természetes
		számok úgy, hogy 
		\begin{equation}
			k=m^{2}-n^{2},\quad l=2mn,\quad x=m^{2}+n^{2},\label{eq:fermat22}
		\end{equation}
		ha pedig $l$ a páratlan, akkor \aref({eq:fermat22}) összefüggésben
		felcserélődik a $k$ és $l$ értéke. Bármelyik eset is áll fent, \aref({eq:fermat21})
		összefüggés középső egyenletéből az 
		\[
		\left(\frac{y}{2}\right)^{2}=(m^{2}-n^{2})mn
		\]
		összefüggést kapjuk, ahol a jobb oldalon szereplő tényezők páronként
		relatív prí\-mek. Tehát mindhárom tényező teljes négyzet, vagyis
		léteznek az $x',y',z'$ pá\-ron\-ként relatív prím pozitív egész
		számok úgy, hogy 
		\[
		x'{}^{2}=m,\quad y'{}^{2}=n,\quad z'{}^{2}=m^{2}-n^{2}.
		\]
		Viszont innen következik, hogy $x'{}^{4}-y'{}^{4}=z'{}^{2}$ is teljesül,
		tehát szerkesztettünk az $(x,y,z)$ páronként relatív prím pozitív
		egész számokból álló kiinduló megoldásunkból egy újabb $(x',y',z')$
		páronként relatív prím pozitív egészekből álló megoldást. Mivel $z'$
		is páratlan és $0<x'<x$, a végtelen leszállás elve szerint ellentmondáshoz
		jutottunk, tehát nem létezik az 1. eset kiinduló feltételeinek eleget
		tevő $(x,y,z)$ megoldás. 
		\item[2.] Ha $z$ páros, akkor $(y^{2},z,x^{2})$ olyan primitív pitagoraszi
		számhármas, amelyben $z$ a páros tag. Ezért léteznek a $k>l>0$,
		$(k,l)=1,$ $k\not\equiv l\pmod 2$ természetes számok úgy, hogy 
		\[
		y^{2}=k^{2}-l^{2},\quad z=2kl,\quad x^{2}=k^{2}+l^{2}.
		\]
		Összeszorozva az előbbi összefüggés első és utolsó egyenletét, az
		\[
		(xy)^{2}=k^{4}-l^{4}
		\]
		összefüggést kapjuk. Ebben az összefüggésben $xy$ páratlan, tehát
		az 1. eset alapján ennek az egyenletnek nincs megoldása. 
	\end{itemize}
\end{solution}
\begin{extraproblem}[Péter Róbert]
	Határozzuk meg az $x-y=x^{2}+xy+y^{2}$ egyenlet egész
	megoldásait. 
\end{extraproblem}

\begin{solution}
	Vizsgáljuk az alábbi egyenlet egész számmegoldásait: 
	\[
	x-y=x^{2}+xy+y^{2}
	\]
	
	Vigyük át minden tagot a bal oldalra: 
	\[
	x-y-x^{2}-xy-y^{2}=0\Rightarrow-x^{2}-xy-y^{2}+x-y=0
	\]
	
	Ez egy másodfokú kifejezés két változóval. Keressük meg az összes
	egész számmegoldást kézi próbálgatással kis értékekre:
	\begin{itemize}
		\item Ha $x=0$: 
		\[
		0-y=0^{2}+0\cdot y+y^{2}\Rightarrow-y=y^{2}\Rightarrow y(y+1)=0\Rightarrow y=0\text{ vagy }y=-1
		\]
		Megoldások: $(0,0),\ (0,-1)$
		\item Ha $x=1$: 
		\[
		1-y=1+y+y^{2}\Rightarrow-2y=y^{2}\Rightarrow y(y+2)=0\Rightarrow y=0\text{ vagy }y=-2
		\]
		Megoldások: $(1,0),\ (1,-2)$
		\item Ha $x=2$: 
		\[
		2-y=4+2y+y^{2}\Rightarrow0=y^{2}+3y+2\Rightarrow(y+1)(y+2)=0\Rightarrow y=-1,-2
		\]
		Megoldások: $(2,-1),\ (2,-2)$
		\item Ha $x=3$: 
		\[
		3-y=9+3y+y^{2}\Rightarrow0=y^{2}+4y+6
		\]
		Ennek az egyenletnek nincs egész megoldása. 
	\end{itemize}
	Tovább próbálgatva látható, hogy a jobb oldal $x^{2}+xy+y^{2}$ sokkal
	gyorsabban nő, mint a bal oldalon szereplő $x-y$, így csak véges
	sok egész megoldás van.
	
	\bigskip{}
	
	\textbf{Végső válasz:} Az egyenlet egész megoldásai a következők:
	\[
	(x,y)=(0,0),\ (0,-1),\ (1,0),\ (1,-2),\ (2,-1),\ (2,-2)
	\]
\end{solution}
\begin{extraproblem}[Sógor Bence]
	Oldd meg az 
	\[
	a^{3}=6b^{2}+2
	\]
	egyenletet az egész számok halmazán! 
\end{extraproblem}

\begin{solution}
	Az egyenletet átalakítások után az alábbi módon írhatjuk fel 
	\[
	a^{3}=6b^{2}+2\iff a^{3}-8=6b^{2}-6\iff(a-2)(a^{2}+2a+4)=6(b-1)(b+1).
	\]
	Mivel az egyenlet jobb oldala biztosan páros, ezért $a$ is páros
	kell legyen. Ekkor létezik $c$ egész szám úgy, hogy $a=2c$. Az egyenletet
	így írhatjuk tovább: 
	\[
	(2c-2)(4c^{2}+4c+4)=6(b-1)(b+1)\iff4(c-1)(c^{2}+c+1)=3(b-1)(b+1).
	\]
	Vegyük észre, hogy az egyenlet bal oldala mindig osztható 4-gyel,
	míg a jobb oldal mindig osztható 3-mal. Innen $c\equiv1\mod 3$ és
	b páratlan, $b=2d+1$ valamilyen $d$ egészre. Amiket kaptunk visszahelyettesítve
	az egyenlet eredeti alakjába kapjuk, hogy 
	\[
	a^{3}=6b^{2}+2\iff4c^{3}=3b^{2}+1\iff4c^{3}=12d^{2}+12d+4\iff c^{3}=3d^{2}+3d+1.
	\]
	Vegyük észre, hogy az egyenlet jobb oldala két egymás utáni köbszám
	különbsége: 
	\[
	c^{3}=3d^{2}+3d+1\iff c^{3}=(d+1)^{3}-d^{3}\iff c^{3}+d^{3}=(d+1)^{3}
	\]
	
	Viszont az ismert, hogy az $x^{3}+y^{3}=z^{3}$ alakú egyenletnek
	nincsen olyan megoldása az egész számok halmazán, ahol valamelyik
	tag ne lenne 0.\\
	
	Ha $c=0$, akkor $d^{3}=(d+1)^{3}\iff0=3d^{2}+3d+1$ nem teljesülhet.\\
	
	Ha $d=0$, akkor $c^{3}=1$, ahonnan $a=2$ és $b=1$ amik megoldása
	az eredeti egyenletnek.\\
	
	Ha $d=-1$, akkor $c^{3}+(-1)^{3}=0$, ahonnan $a=2$ és $b=-1$ amik
	megoldása az eredeti egyenletnek.\\
	
	Összességében a megoldáshalmaz: 
	\[
	M=\{(2,-1);(2,1)\}.
	\]
\end{solution}
\begin{extraproblem}[Szélyes Klaudia]
	Határozd meg az összes olyan egész megoldáspárt $(x,y)$, amely kielégíti
	az alábbi egyenletet:
	\[
	x^{2}-5y^{2}=1
	\]
	és teljesül rá, hogy $x\equiv1\pmod 4$. Bizonyítsd, hogy az összes
	megoldás a következő alakban írható fel:
	\[
	x+y\sqrt{5}=(9+4\sqrt{5})^{n},\quad n\in\mathbb{Z}.
	\]
\end{extraproblem}
\begin{solution}
	Az egyenlet:
	
	\[
	x^{2}-5y^{2}=1
	\]
	
	egy klasszikus \emph{Pell-egyenlet} $D=5$ mellett. A legismertebb
	módszer a megoldások generálására az alábbi:
	
	\textbf{1. A legkisebb pozitív megoldás}
	
	Keressük a legkisebb egész pozitív megoldást $(x_{1},y_{1})$. Könnyen
	ellenőrizhető, hogy:
	
	\[
	x=9,\quad y=4\quad\Rightarrow\quad9^{2}-5\cdot4^{2}=81-80=1.
	\]
	
	\textbf{2. Általános megoldás}
	
	A megoldások a következő alakban adhatók meg:
	
	\[
	x_{n}+y_{n}\sqrt{5}=(9+4\sqrt{5})^{n},\quad n\in\mathbb{Z}.
	\]
	
	A konjugált alak:
	
	\[
	x_{n}-y_{n}\sqrt{5}=(9-4\sqrt{5})^{n}.
	\]
	
	Összegezve:
	
	\[
	x_{n}=\frac{(9+4\sqrt{5})^{n}+(9-4\sqrt{5})^{n}}{2},\quad y_{n}=\frac{(9+4\sqrt{5})^{n}-(9-4\sqrt{5})^{n}}{2\sqrt{5}}.
	\]
	
	\textbf{3. Kongruenciafeltétel:} $x\equiv1\pmod 4$
	
	Vizsgáljuk meg néhány $x_{n}$ értéket modulo 4:
	
	\begin{align*}
		n=0 & \Rightarrow x=1\equiv1\pmod 4,\\
		n=1 & \Rightarrow x=9\equiv1\pmod 4,\\
		n=2 & \Rightarrow x=161\equiv1\pmod 4,\\
		n=3 & \Rightarrow x=2881\equiv1\pmod 4.
	\end{align*}
	
	Megfigyelhető, hogy minden $x_{n}\equiv1\pmod 4$. Induktív úton belátható,
	hogy ez minden $n$-re igaz.
	
	\textbf{4. Végtelen leszállás és algebrai egységek}
	
	Tegyük fel, hogy létezik $(x,y)\in\mathbb{Z}^{2}$ megoldásFeladat
	1 , amely nem írható fel a fenti módon. Válasszuk ezek közül a legkisebb
	pozitív $x$-et.
	
	Ekkor az algebrai szám $x+y\sqrt{5}$ kielégíti:
	
	\[
	(x+y\sqrt{5})(x-y\sqrt{5})=1,
	\]
	
	azaz az $\mathbb{Z}[\sqrt{5}]$ gyűrűben ez egy \emph{egység}. Mivel
	az $\mathbb{Z}[\sqrt{5}]$ egységei pontosan a $(9\pm4\sqrt{5})^{n}$
	alakú számok, a megoldásnak ebből kell származnia.
	
	Ez ellentmond a feltételezésnek, tehát \textbf{minden megoldás} a
	következő alakban adódik:
	
	\[
	x+y\sqrt{5}=(9+4\sqrt{5})^{n},\quad n\in\mathbb{Z}.
	\]
	
	Válasz
	
	Az összes egész megoldáspár $(x,y)$ az alábbi alakban írható fel:
	
	\[
	x=\frac{(9+4\sqrt{5})^{n}+(9-4\sqrt{5})^{n}}{2},\quad y=\frac{(9+4\sqrt{5})^{n}-(9-4\sqrt{5})^{n}}{2\sqrt{5}},
	\]
	
	ahol $n\in\mathbb{Z}$. Valamennyi ilyen megoldásra $x\equiv1\pmod 4$
	teljesül.
\end{solution}

