
\chapter{Ismétléses permutációk, ismétléses variációk, ismétléses kombinációk}\label{chap:ismetleses_perm}
\begin{description}
{\large \item [{Szerző:}] Gál Tamara (Didaktikai mesteri -- Matematika, II. év)}
\end{description}
\begin{problem}
 Mi a különbség az alábbi feladatpárokban?
 \begin{itemize}
 	\item[1/A)] Hányféleképpen lehet sorba rendezni 5 különböző színű golyót?
 	\item[1/B)] Hányféleképpen lehet sorba rakni egy fehér, két zöld és két kék
 	golyót? 
 	\item[2/A)] Hányféle különböző sorrendje van a MÉRTAN szó betűinek?
 	\item[2/B)] Hányféle különböző sorrendje van a MATEMATIKA szó betűinek?
 \end{itemize}
 
\end{problem}
Várható válasz: Az 1-es típusú kérdésekben minden elem csak egyszer
fordulhat elő, míg a 2-es típusú kérdésekben minden elem többször
is előfordulhat.
\begin{solution}
	\begin{itemize}
		\item[1/A)] Öt különböző színű golyót $5!=120$ féleképpen lehet sorba rendezni.
		\item[1/B)] Mi történik, ha az elemek nem különbözőek? Próbáljuk sorba rendezni
		az elemeket. Jelölje k a kék golyókat, f a fehér golyókat, z a zöld
		golyókat. Rögzítsük le a kék golyók helyét az első oszlopban, a többi
		oszlopot töltsük fel a maradék golyókkal.

%1/A) Öt különböző színű golyót $5!=120$ féleképpen lehet sorba rendezni.
%\\
% 1/B) Mi történik, ha az elemek nem különbözőek? Próbáljuk sorba rendezni
%az elemeket. Jelölje k a kék golyókat, f a fehér golyókat, z a zöld
%golyókat. Rögzítsük le a kék golyók helyét az első oszlopban, a többi
%oszlopot töltsük fel a maradék golyókkal.\\
% %
\begin{tabular}{l|l|l|l}
kk---  & kkfzz & kkzfz & kkzzf \tabularnewline
k-k--  & kfkzz & kzkfz & kzkzf \tabularnewline
k--k- & kfzkz & kzfkz & kzzkf\tabularnewline
k---k & kfzzk & kzfzk & kzzfk\tabularnewline
-kk-- & fkkzz & zkkfz & zkkzf\tabularnewline
-k-k- & fkzkz & zkfkz & zkzkf\tabularnewline
-k--k & fkzzk & zkfzk & zkzfk\tabularnewline
--kk- & fzkkz & zfkkz & zzkkf\tabularnewline
--k-k & fzkzk & zfkzk & zzkfk\tabularnewline
---kk & fzzkk & zfzkk & zzfkk\tabularnewline
\end{tabular}\\
 Összesen 30 féle sorrend lehetséges. 
 
 Na de hogyan csökkent le a 120
lehetőség 30-ra? Azonos színű golyók egymás közötti sorrendjeiket
egyformának tekinthetjük. Tehát van 5 helyünk amiből 2-re kék golyó
kerül, ez $C_{5}^{2}=10$ lehetőség. Egy helyre kerülhet fehér golyó
a maradék 3 hely közül, ez $C_{3}^{1}=3$ lehetősége marad. A maradék
két helyre pedig egyértelműen a zöld golyók kerülnek. Tehát összesen
$C_{5}^{2}\cdot C_{3}^{1}\cdot C_{2}^{2}=10\cdot3\cdot1=30$ lehetőség. 
	\end{itemize}
\end{solution}
\begin{definition}{ism_perm}
Ha az elemek között van olyan amelyik többször is előfordul, az elemek
egy sorba rendezését ismétléses permutációnak nevezzük.
\end{definition}

\begin{theorem}{ism_perm_theo}
Ha $n$ elem között $p_{1},p_{2},...,p_{k}$ darab megegyező van és
$p_{1}+p_{2}+...+p_{k}=n,$ akkor ezeket az elemeket $\frac{n!}{p_{1}!\cdot p_{2}!\cdot...\cdot p_{k}!}$
különböző módon lehet sorba rendezni. \cite{Moz2}
\end{theorem}

\begin{proof}
Az első $p_{1}$ egyforma elemet: $C_{n}^{p_{1}}$ különböző módon
lehet kiválasztani.
\[
C_{n}^{p_{1}}=\frac{(p_{1}+p_{2}+...+p_{n})!}{p_{1}!\cdot(p_{2}+...+p_{n})!}.
\]
A következő $C_{p_{2}+p_{3}+...+p_{n}}^{p_{2}}$ módon lehet kiválasztani:
$C_{p_{2}+p_{3}+...+p_{n}}^{p_{2}}=\frac{(p_{2}+...+p_{n})!}{p_{2}!\cdot(p_{3}+p_{4}+...+p_{n})!}.$\\
 Az utolsó $p_{k}$ elemet : $C_{p_{k}}^{p_{k}}$ féle módon lehet
kiválasztani: $C_{p_{k}}^{p_{k}}=\frac{p_{k}!}{p_{k}!}.$ 
\[
C_{n}^{p_{1}}\cdot C_{p_{2}+p_{3}+...+p_{n}}^{p_{2}}\cdot C_{(p_{3}+p_{4}+...+p_{n})}^{p_{3}}...\cdot C_{p_{k}}^{pk}\Leftrightarrow
\]
\[
\frac{(p_{1}+p_{2}+...+p_{n})!}{p_{1}!\cdot(p_{2}+...+p_{n})!}\cdot\frac{(p_{2}+...+p_{n})!}{p_{2}!\cdot(p_{3}+p_{4}+...+p_{n})!}\cdot...\cdot\frac{p_{k}}{p_{k}}=\frac{(p_{1}+p_{2}+...+p_{k})!}{p_{1}!\cdot p_{2}!\cdot...\cdot p_{k}!}=\frac{n!}{p_{1}!\cdot p_{2}!\cdot...\cdot p_{k}!}
\]
\end{proof}
\begin{solution}
	\begin{itemize}
		\item[2/A)] A Mértan szó 6 betűből áll tehát $6!$-féleképpen lehet sorba
		rendezni.
		\item[2/B)] A MATEMATIKA szóban van két M betű, három A betű és a két T
		betű. A lehetséges sorrendek száma:$\frac{10!}{(3!\cdot2!\cdot2!)}=151200.$ 
	\end{itemize}

\end{solution}
\begin{problem}
Mi a különbség az alábbi feladatpárokban?
\begin{itemize}
\item[1/A)] Az 1, 2, 3, 4 számok közül válasszunk ki kettőt és írjuk fel
ezeket az összes lehetséges sorrendben. Mennyi a lehetőségek száma?
\item[1/B)] Az 1, 2, 3, 4 számok közül válasszunk ki kettőt úgy, hogy ugyanazt
az elemet kétszer is vehetjük és írjuk fel ezeket az összes lehetséges
sorrendben. Mennyi a lehetőségek száma?
\item[1/A)] Egy 14 fős csoportban hányféleképpen lehet 4 különböző könyvet
kiosztani, ha mindenki 1 könyvet kaphat?
\item[2/B)] Egy 14 fős társaságban 4 könyvet osztunk szét. Hányféleképpen
tehetjük meg, ha minden könyv különböző, és mindenki több könyvet
is kaphat
\end{itemize}
\end{problem}
 
\begin{solution}
\begin{itemize}
\item[1/A)] A következőket kapjuk: 12 13 14 21 23 24 31 32 34 41 42 43 A
lehetőségek száma 12.
\item[1/B)] A következőket kapjuk: 11 12 13 14 21 22 23 24 31 32 33 34 41
42 43 44 A lehetőségek száma 16.
\item[2/A)] Az első könyvet 14 embernek adhatjuk. A második könyvet 13 embernek
oszthatjuk ki. A harmadik könyvet 12 embernek oszthatjuk ki, míg a
negyedik könyvet 11 embernek adhatjuk oda. Összesen 14·13·12·11 --
féleképpen oszthatjuk ki a könyveket.
\item[2/B)] Ha egy ember több könyvet is kaphat akkor mindegyik könyvet
14 embernek adhatnánk oda. Így a lehetőségek száma $14\cdot14\cdot14\cdot14=14^{4}$
lehetőség.
\end{itemize}

\end{solution}
\begin{definition}{ism_var}
Ha $n$-féle elemből a sorrend figyelembe vételével kiválasztunk $k$
darabot (egyféle elemből többet is választhatunk), az $n$-féle elemnek
egy $k$ tagú ismétléses variációját kapjuk. \cite{Moz2}
\end{definition}

\begin{theorem}{ism_var_theo}
$n-$féle elem $k$ tagú ismétléses variációinak a száma: $V_{n}^{k,i}=n^{k}.$ 
\end{theorem}

\begin{proof}
$n-$féle elemből kiválasztunk $k$ darabot mind a $k$ helyre n elemet
tudunk tenni, ez azt jelenti, hogy a lehetőségeink száma: $\underbrace{n\cdot n\cdot n\cdot...\cdot n}_{k\,\,\text{darab}}=n^{k}.$ 
\end{proof}
\begin{problem}
	Mi a különbség az alábbi feladatpárokban?
\begin{itemize}
\item[1/A)] Az 1, 2, 3, 4 számok közül válasszunk ki kettőt (két különbözőt)
és írjuk fel ezeket úgy, hogy nem vagyunk tekintettel a kiválasztott
elemek sorrendjére. Mennyi a lehetőségek száma?
\item[1/B)] Az 1, 2, 3, 4 számok közül válasszunk ki kettőt úgy, hogy ugyanazt
az elemet kétszer is vehetjük, de nem vagyunk tekintettel a kiválasztott
elemek sorrendjére. Mennyi a lehetőségek száma?
\item[2/A)] Egy 14 fős csoportban hányféleképpen lehet 4 egyforma könyvet
kiosztani, ha mindenki 1 könyvet kaphat? 
\item[2/B)] Egy 14 fős csoportban hányféleképpen lehet 4 egyforma könyvet
kiosztani, ha mindenki több könyvet is kaphat?
\end{itemize}
\end{problem}
\begin{solution}
\begin{itemize}
\item[1/A)] Hat lehetőség van, ezek a következők: 12 13 14 23 24 34
\item[1/B)] Az előző lehetőségekhez még pluszba kellenek nekünk az azonos
tagúak, a következőket kapjuk: 11 12 13 14 22 23 24 33 34 44 A lehetőségek
száma 10.
\item[1/A)] Az első könyvet 14 tanulónak adhatjuk. A második könyvet a maradék
13 tanulónak adhatjuk. És így tovább… Összesen 14·13·12·11 -- féleképpen
oszthatjuk ki a könyveket. Igen ám, de egyformák a könyvek. A kiválasztott
4 tanulót akárhogy állítjuk sorba, ugyanazt az esetet kapjuk, mert
ugyanaz a könyv lesz náluk. Az esetek száma annyiad részre csökken,
ahányféleképpen a kiválasztott négy tanulót sorba tudjuk rendezni.
$C_{14}^{4}=\frac{14!}{10!\cdot4!}=1001.$
\item[2/B)] Mindenki nevét írjuk rá egy-egy cédulára, és a cédulákat tegyük
bele egy kalapba. Ha valakinek kihúztuk a nevét, megkapja a könyvét,
és a céduláját visszatesszük a kalapba. Az utolsó húzás után már nem
kell visszatenni a cédulát. Tehát négy húzás után 3-szor kell visszatenni
a cédulát. Így ugyanannyi eset van, mintha 17 tanulóból kéne kiválasztani
négyet: $C_{17}^{4}=\frac{17!}{(17-4)!\cdot4!}=2380.$ 
\end{itemize}

\end{solution}
\begin{definition}{ism_komb}
Ha $n$-féle elemből választunk $k$ darabot úgy, hogy a választás
sorrendje nem számít és mindegyikféle elemből többet is választhatunk,
az $n$-féle elem $k$ tagú ismétléses kombinációját kapjuk.
\end{definition}

\begin{theorem}{ism_komb_theo}
$n$-féle elem $k$ tagú ismétléses kombinációinak száma: $C_{n}^{k,i}=C_{n+k-1}^{k}$ \cite{komb} 
\end{theorem}

\begin{proof}
Megmutatjuk, hogy bijektív leképezés létesíthető $n$ elem $k$-adosztályú
ismétléses kombinációi és $n+k-1$ elem $k$-adosztályú (ismétlés
nélküli) kombinációi között. Innen következni fog, hogy $C_{n}^{k,i}=C_{n+k-1}^{k}.$
Tekintsünk az $1,2,...,n$ elemek egy tetszőleges, rögzített $a_{i_{1}},a_{i_{2}},...,a_{i_{k}}$
ismétléses kombinációját, ahol $1\leq a_{i_{1}}\leq a_{i_{2}}\leq...\leq a_{i_{k}}\leq n$.
Adjuk hozzá az elemekhez rendre a $0,1,2,...,k-1$ számokat, azaz
legyen $a_{i_{1}},a_{i_{2}}+1,...,a_{i_{k}}+(k-1)$. Ez az $1,2,...,n+k-1$
elemeknek egy ismétlés nélküli kombinációja, mert itt $1\leq a_{i_{1}}<a_{i_{2}}+1<...<a_{i_{k}}+(k-1)\leq n+k-1.$
Minden ilyen ismétlés nélküli kombinációt megkapunk és pontosan egyszer.
Fordítva, ha $b_{i_{1}},b_{i_{2}},...,b_{i_{k}}$ az $1,2,...,n+k-1$
elemek egy ismétlés nélküli kombinációja, akkor $b_{i_{1}},b_{i_{2}}-1,...,b_{i_{k}}-(k-1)$
az $1,2,...,n$ elemek egy ismétléses kombinációja lesz. 
\end{proof}

\section*{Házi feladatok}
\begin{problem}[\cite{Moz2}]
Írjuk le a 334445 mellé az összes különböző hétjegyű számot, amely
a 334445 számjegyeinek felcserélésével keletkezett. Mennyi az összes
leírt számjegy összege? 
\end{problem}

\begin{solution}
Először a két darab 3-asból, három darab 4-esből és egy darab 5ösből
alkotható hatjegyű számok számát keressük. Ez a következő: 
\[
P_{6}^{2,3,1}=\frac{6!}{2!\cdot3!\cdot1!}=\frac{1\cdot2\cdot3\cdot4\cdot5\cdot6}{1\cdot2\cdot1\cdot2\cdot3\cdot1}=60
\]
Egy ilyen hatjegyű számban a számjegyek összege: 3+3+4+4+4+5=23, így
az összes leírt számjegy összege: $60\cdot23=1380.$ 
\end{solution}
\begin{problem}[\cite{Moz3}]
Melyik nagyobb: az (a,a,a,a,b,b,b) betűkészletből alkotható hétbetűs
szavak száma vagy ugyanebből a betűkészletből alkotható hatbetűs szavak
száma? 
\end{problem}

\begin{solution}
A héthosszúságú szavak száma: $P_{7}^{4,3}=\frac{7!}{4!\cdot3!}=\frac{1\cdot2\cdot3\cdot4\cdot5\cdot6\cdot7}{1\cdot2\cdot3\cdot4\cdot1\cdot2\cdot3}=5\cdot7=35.$
A hathosszúságú szavak száma: $P_{6}^{4,3}=\frac{6!}{4!\cdot3!}=\frac{1\cdot2\cdot3\cdot4\cdot5\cdot6}{1\cdot2\cdot3\cdot4\cdot1\cdot2\cdot3}=5.$
Tehát a hétbetűs szavak száma nagyobb, mint a hatbetűs szavak száma. 
\end{solution}
\begin{problem}[\cite{Moz2}]
Emese és Karcsi öt gyermeket szeretne. Fel is írták az alábbi neveket
egy táblára: Aladár, Eszter, Csilla, Tamás, Andrea, Kamilla, Gáspár,
Vivien, János, Csaba. Ezek közül mindig véletlenszerűen választanak
egyet az éppen születő csöppségnek (persze kétszer nem adják ugyanazt
a nevet). Hányféle sorrendben adhatják a neveket a megszületendő öt
gyermeknek? 
\end{problem}

\begin{solution}
Az öt gyermek megszülethet 5 lány, 4 lány és 1 fiú, 3 lány és 2 fiú,
2lány és 3 fiú, 1 lány és 4 fiú, 5 fiú variációban. Az első és az
utolsó esetben könnyű dolgunk van, mindkétszer $5!$ sorrendben adhatnak
nevet a megszületendő gyerekeknek. Vegyük bonyolultabb példának a
harmadik esetet. Ekkor a gyerekek között a lányok-fiúk sorrendje ismétléses
permutáció: $P_{5}^{3,2}=\frac{5!}{3!\cdot2!},$ azon belül a lányoknak
$\frac{5!}{(5-3)!},$ a fiúknak $\frac{5!}{(5-2)!}-$ féleképpen adhatnak
nevet. Ezek alapján minden esetet számba tudunk venni (a szimmetria
miatt elég az egyes eseteket kettővel szorozni), a végeredmény:
\[
2\cdot5!+2\cdot\frac{5!}{4!\cdot1!}\cdot\frac{5!}{(5-4)!}\cdot\frac{5!}{(5-1)!}+2\cdot\frac{5!}{3!\cdot2!}\cdot\frac{5!}{(5-3)!}\cdot\frac{5!}{(5-2)!}=30240.
\]
\end{solution}
\begin{problem}[\cite{Moz4}]
Gábor elfelejtette, hogy kémiából dolgozatot írnak.A 7 kérdés mindegyikére
A,B és C válasz közül lehetett választani egyet. Hányféleképpen tölthette
ki Gábor véletlenszerűen a tesztet?
\end{problem}

\begin{solution}
Ha nem hagyott üresen se, akkor minden válaszra van 3 lehetősége vagyis: $3\cdot3\cdot3\cdot3\cdot3\cdot3\cdot3=3^{7}.$ Ha üresen is hagyhatott
akkor minden válaszra 4 lehetősége van és akkor: $4^{7}$ féleképpen
tölthette ki. 
\end{solution}
\begin{problem}[\cite{Moz4}]
Egy Bolha üldögél egy négyzet mellett. Egyszer csak ráugrik a négyzet
egyik csúcsára, majd onnantól kezdve a csúcsokon ugrál még kilencszer.Hányféle
sorrendben érintheti a csúcsokat, ha:
\begin{enumerate}
\item bármikor bármelyikre ugorhat 
\item oda nem ugrik ahol éppen áll 
\item nem ugrik se oda ahol éppen áll sem pedig az eggyel korábbi állomáshelyére? 
\end{enumerate}

\end{problem}

\begin{solution}
\begin{enumerate}
\item  Bármikor bármelyik csúcsra ugorhat, minden esetben 4 lehetősége
van: $4^{10}.$
\item  Elsőnek bárhova, utána viszont már csak 3-3 helyre ugorhat: $4\cdot3^{9}$
\item Elsőnek bárhova ugorhat. A következő ugrása a mostani helyéről
elviszi, az utána levőkben pedig mindig kizárunk kettő csúcsot: $4\cdot3\cdot2^{8}$ 
\end{enumerate}

\end{solution}
\begin{problem}[\cite{Moz3}]
Hányféle a magyar zászlóhoz hasonló három sávos zászlót lehet készíteni
öt színből, ha a színek többször is előfordulhatnak, de azonos színek
nem lehetnek egymás mellett?
\end{problem}

\begin{solution}
Az első sáv kiszínezéséhez 5 lehetőségünk van. A második sáv nem lehet
az elsővel egyszínű erre 4 lehetőség jut. A harmadik sáv nem lehet
a másodikkal egyszínű , de lehet az elsővel azonos színű így erre
is 4 lehetőség van. Tehát a kiszínezéshez: $5\cdot4\cdot4=80$ lehetőségünk
van.
\end{solution}
\begin{problem}[\cite{Moz2}]
Egy kertészeti újságban ötféle növényt kínálnak akciós áron, petúniát,
muskátlit, rózsát, szarkalábat, levendulát. Eszter 12 növényt rendel.
Hányféleképpen teheti meg, ha egy növényből többet is rendelhet? 
\end{problem}

\begin{solution}
Valójában keressük 5 féle növény 12 darab ismétléses kombinációinak
a számát ami egyenlő: $C_{12+5-1}^{12}=C_{16}^{12}=\frac{16!}{12!\cdot4!}=\frac{13\cdot14\cdot15\cdot16}{1\cdot2\cdot3\cdot4}=1820.$
\end{solution}
\begin{problem}[\cite{Moz1}]
Egy dobozban 1-től 9-ig sorszámozott cédulák vannak. Kiemelünk belőlük
5 darabot, a kivett cédulát megnézzük majd visszatesszük. Hány olyan
húzás lehet, amiben legalább kettő 8-as van, ha a kivett számokat
üres cédulára írva egy másik dobozba helyezzük. 
\end{problem}

\begin{solution}
Érdemes áttérni az ellentett esetek összeszámlálására, itt ismétléses
kombinációkat kell felírnunk. Az összes esetekben kilenc számból választunk
ki ismétléssel ötöt: $C_{9+5-1}^{5}=C_{13}^{5}=\frac{13!}{8!\cdot5!}=1287.$
Ha nincs közöttük 8-as, akkor már csak nyolc számból választunk ötöt
ismétléssel: $C_{8+5-1}^{5}=C_{12}^{5}=\frac{12!}{7!\cdot5!}=792.$
Ha egy 8-as van a számok között, akkor a többi négyet a maradék nyolc
értékből választhatjuk, szintén ismétléssel: $C_{8+4-1}^{4}=C_{11}^{4}=\frac{11!}{7!\cdot4!}=330.$
A végeredmény: $C_{13}^{5}-C_{12}^{5}-C_{11}^{4}=1287-792-330=165.$
\end{solution}
\begin{problem}[\cite{Moz2}]
 Zsuzsi Párizsban 18 képeslapot vásárolt, melyeket 8 fajta képeslap
közül választott ki. Hányféleképpen tehette ezt meg? 
\end{problem}

\begin{solution}
Zsuzsi a képeslapokat 8 elem 18 darab ismétléses kombinációjával választhatta
ki: $C_{18+8-1}^{18}=C_{25}^{18}=\frac{25!}{7!\cdot18!}=480700$
\end{solution}

\section*{Nehezebb feladatok}
\begin{extraproblem}[Csurka-Molnár Hanna]
Egy $10$ fős társaságban $4$ könyvet osztunk szét. Hányféleképpen
tehetjük meg, ha a könyvek egyformák, és mindenki több könyvet is
kaphat? 
\end{extraproblem}

\begin{solution}
Az embereket sorba rendezve beékeljük a sorba a könyveket
az a személy után, akinek a könyvet adjuk. Mivel ez a sorozat csak
egy személlyel kezdődhet, könyvvel nem, figyelmen kívül hagyhatjuk
az első személyt. Így a lehetséges rendezések száma $(10-1+4)!=13!$.
A sorban csak az számít, hogy adott személy után hány könyv szerepel,
így a személyek egymáshoz viszonyított sorrendjét figyelmen kívül
kell hagynunk (ez $9!$ mivel az első személyt rögzítettük). Továbbá
a könyvek egyformák, ezért azok sorrendjét is figyelmen kívül hagyva
összesen $9!\cdot4!$ ismétlése lesz egy adott szétosztásnak a személyek
és könyvek összes lehetséges sorbarendezése között. Így a lehetséges
leosztások száma: 
\[
\frac{13!}{9!\cdot4!}=\frac{13\cdot12\cdot11\cdot10}{4\cdot3\cdot2\cdot1}=13\cdot11\cdot5=715.
\]

2.Megoldás: Vegyük sorra, hogy hányféleképpen lehet 4 könyvet szétosztani
\[
4=3+1=2+2=2+1+1=1+1+1+1
\]

Ha minden könyvet $1$ személy kap, a lehetőségek száma: 
\[
C_{10}^{1}=10.
\]
Ha $3$ könyv jut egy személyhez és a $4.$ könyv egy másikhoz, a
lehetőségek száma: 
\[
V_{10}^{2}=\frac{10!}{(10-2)!}=10\cdot9=90.
\]
Ha $2-2$ könyvet adunk oda két embernek, a lehetőségek száma: 
\[
C_{10}^{2}=\frac{10!}{(10-2)!\cdot2!}=\frac{10\cdot9}{2}=45.
\]
Ha egy ember kap $2$ könyvet és két ember kap $1-1$ könyvet, a lehetőségek
száma: 
\[
C_{10}^{1}\cdot C_{9}^{2}=10\cdot\frac{9!}{(9-2)!\cdot2!}=\frac{10\cdot9\cdot8}{2}=360.
\]
Ha $4$ ember kap $1-1$ könyvet, a lehetőségek száma: 
\[
C_{10}^{4}=\frac{10!}{(10-4)!\cdot4!}=\frac{10\cdot9\cdot8\cdot7}{4\cdot3\cdot2\cdot1}=10\cdot3\cdot7=210.
\]
Összesen tehát $10+90+45+360+210=715$ lehetőség van. 
\end{solution}
\begin{extraproblem}[Czofa Vivien]
\textit{\emph{Hányféleképpen lehet 15 szál virágot 5 lány között
szétosztani úgy, hogy mindenki kapjon legalább egy szálat? }}\textit{(Matlab
2017/01, L:2637, Deák Imre tanár, Székelyudvarhely)}
\end{extraproblem}

\begin{solution}
Első lépésben biztosítani kell, hogy mindenki kapjon egy-egy szál
virágot. Marad 10 szál virág.

A maradék 10 szál virág szétosztását ismétléses permutációval vagy
kombinációval számítjuk ki:

\[
P_{14}^{4,10}=\binom{10+4}{4}=\binom{14}{4}=\frac{14!}{10!\cdot4!}=\frac{11\cdot12\cdot13\cdot14}{1\cdot2\cdot3\cdot4}=1001.
\]
\end{solution}
\begin{extraproblem}[Czofa Vivien]
\textit{\emph{Hányféleképpen lehet sorrendbe állítani a RENDETLENÜL
szó betűit úgy, hogy ne álljon két E betű egymás mellett? (Minden
betűt pontosan egyszer használunk fel.)}}
\end{extraproblem}

\begin{solution}
\textbf{1. megoldás:} Először rendezzük el az E-től különböző betűket,
nyolc betűt, köztük két-két azonosat: \texttt{R D T U N N L L}. A
8 betűt $8!$ féle módon rendezhetjük sorba, de a két N betűt és a
két L betűt egymás között felcserélve nem kapunk új esetet (ismétléses
permutáció), ezért 
\[
\frac{8!}{2!\cdot2!}
\]
lehetőségünk van ezeknek a betűknek a sorba rendezéséhez.

Az E betűket az így kialakult szó elé, utána vagy a betűk közé, tehát
9 helyre tehetjük le. 9 helyből kell kiválasztanunk 3-at úgy, hogy
a sorrend nem számít. Ez $\binom{9}{3}$ lehetőség.

Ezért: 
\[
\frac{8!}{2!\cdot2!}\cdot\binom{9}{3}=10080\cdot84=846720\text{ esetet kapunk.}
\]

\medskip{}

\textbf{2. megoldás:} A \texttt{RENDETLENÜL} szó 11 betűből áll, ezek
között van három E betű, két N betű és két L betű. A 11 betűt $11!$
féle módon rendezhetjük sorba, a három E betűt, a két N betűt és a
két L betűt egymás között felcserélve nem kapunk új esetet (ismétléses
permutáció), ezért 
\[
\frac{11!}{3!\cdot2!\cdot2!}
\]
lehetőségünk van ezeknek a betűknek sorba rendezésére.

Ezek között azok az esetek, amelyekben egymás mellett szerepelnek
E betűk, számunkra rosszak, amelyeket le fogunk vonni.

Ha két E betű szerepel egymás mellett, akkor tekintsük ezeket egy
karakternek, a harmadik E betűt pedig sima betűnek. Most 10 karaktert
rendezünk sorba, köztük kettő-kettő azonos. Így ezen esetek száma:
\[
\frac{10!}{2!\cdot2!}
\]

Ha három E betű szomszédos, akkor azt három betűs egyetlen karakternek
tekinthetjük, tehát ezek számát majd vissza kell adnunk. Legyen most
\texttt{EEE} egyetlen jel, ilyen eset: 
\[
\frac{9!}{2!\cdot2!}
\]
van. A feladat feltételeinek megfelelő sorrendezések száma:

\[
\frac{11!}{3!\cdot2!\cdot2!}-\frac{10!}{2!\cdot2!}+\frac{9!}{2!\cdot2!}=846720.
\]
\end{solution}
\begin{extraproblem}[Gergely Verona]
Hányféle módon lehet felmenni egy 25 lépcsőfokból álló lépcsőn, ha
mindig csak 2-t vagy 3-at léphetünk felfelé? (Két feljutás különböző,
ha van legalább egy olyan lépcsőfok, amelyre az egyik feljutásban
rálépünk, de a másikban nem.) \emph{(Arany Dániel Matematikai Tanulóverseny
2014/2015; haladók, II. kategória, 1. forduló) }
\end{extraproblem}

\begin{solution}
Jelöljük $a$-val azt, hogy hányszor léptünk $2$-t, $b$-vel pedig
azt, hogy hányszor léptünk $3$-at. Ekkor teljesülnie kell a 
\[
2a+3b=25
\]
összefüggésnek, ahol $a$ és $b$ természetes számok.\\

Mivel $2a$ páros, ezért $3b$ páratlan, így $b$ is páratlan, emiatt
$b$ értéke csak $1,3,5$ vagy $7$ lehet. Az ezekhez tartozó $a$
értékek rendre $11,8,5$ és $2$.\\

A négy esethez a következő megoldások tartoznak: 
\begin{enumerate}
\item $b=1$ és $a=11$ esetén a $12$ lépésből $11$ egyforma, ez $P_{12}^{1,11}=\frac{12!}{11!\cdot1!}=12$
lehetőség. 
\item $b=3$ és $a=8$ esetén a $12$ lépésből $3$ és $8$ egyforma, ez
$P_{11}^{8,3}=\frac{12!}{8!\cdot3!}=\frac{8!\cdot9\cdot10\cdot11}{8!\cdot2\cdot3}=165$
lehetőség. 
\item $b=5$ és $a=5$ esetén a $12$ lépésből $5$ és $5$ egyforma, ez
$P_{10}^{5,5}=\frac{10!}{5!\cdot5!}=\frac{5!\cdot6\cdot7\cdot8\cdot9\cdot10}{5!\cdot2\cdot3\cdot4\cdot5}=252$
lehetőség. 
\item $b=7$ és $a=2$ esetén a $9$ lépésből $7$ és $2$ egyforma, ez
$P_{9}^{7,2}=\frac{9!}{7!\cdot2!}=\frac{7!\cdot8\cdot9}{7!\cdot2}=36$
lehetőség. 
\end{enumerate}
Összesen $12+165+252+36=465$ féleképpen juthatunk fel a lépcsőn. 
\end{solution}
\begin{extraproblem}[Kis Aranka-Enikő]
Legyen $A=\{1,2,\dots,k\}$, $B=\{1,2,\dots,n\}$. Hány $f:A\to B$
növekvő függvény létezik?
\end{extraproblem}

\begin{solution}
Legyen $f(1)=a_{1}\in B$, $f(2)=a_{2}\in B,\dots,f(k)=a_{k}\in B$.
Feltétel: $a_{1}\leq a_{2}\leq\dots\leq a_{k}$.

A lehetőségek száma, tehát az $f:A\to B$ növekvő függvények száma
minden $n,k\geq1$ esetén éppen

\[
\binom{n+k-1}{k}
\]

(a definíció szerint).

Ennek alapján az $\{1,2,\dots,n\}$ elemek $k$-adosztályú ismétléses
kombinációi úgy is definiálhatók, mint az $f:A\to B$ növekvő függvények.

Szokásos a következő jelölés is: ha $x$ valós szám és $k\geq1$ természetes
szám, akkor

\[
[x]_{k}=x(x+1)(x+2)\cdots(x+k-1).
\]

Így

\[
\binom{n+k-1}{k}=\frac{[n]_{k}}{k!}=\frac{n(n+1)(n+2)\cdots(n+k-1)}{k!}.
\]
\end{solution}
\begin{extraproblem}[Kis Aranka-Enikő]
Tekintsük azokat a dominókat, amelyek mindkét felén a pontok száma
0-tól 8-ig terjed. Ezeket a pontok számának megfelelően így azonosíthatjuk:
$xy$, ahol $0\leq x\leq y\leq8$.
\begin{enumerate}
\item Hány ilyen dominó van?
\item Hányféleképpen lehet a 45 ilyen dominó közül kettőt kiválasztani úgy,
hogy a két dominót egymás mellé lehessen tenni (azaz valamelyik felükön
a pontok száma azonos)?
\end{enumerate}
\end{extraproblem}

\begin{solution}
~
\begin{enumerate}
\item Ha $x=0$, akkor $y$ értékei $0,1,2,\dots,8$ lehetnek, ez 9 lehetőség.
Ha $x=1$, akkor $y$ értékei $1,2,\dots,8$ lehetnek, ez 8 lehetőség,
és így tovább, ha $x=8$, akkor csak $y=8$ lehet, ez 1 lehetőség.
Az összeadási szabály szerint a dominók száma:
\[
9+8+7+\dots+1=45.
\]
\item Válasszunk egy dominót. Ez lehet:

1. eset: "dupla" dominó, azaz $00$, $11$, $22$, $\dots$, $88$,
ezek száma 9, 2. eset: olyan dominó, amelyre $x<y$, ezek száma 36.

Az 1. esetben a második dominót 8-féleképpen választhatjuk. Például
$22$ esetén vehetjük a $02$, $12$, $23$, $24$, $25$, $26$,
$27$, $28$ jelzésűeket. A 2. esetben 8 + 8 = 16-féle párt választhatunk.
Például $27$ lehetséges párjai: $02$, $07$, $12$, $17$, $22$,
$23$, $24$, $25$, $26$, $28$, $37$, $47$, $57$, $67$, $77$,
$78$.

A szorzási szabály szerint:

Az 1. esetben a lehetőségek száma: $9\cdot8=72$, - A 2. esetben
a lehetőségek száma: $36\cdot16=576$.

Az összeadási szabály szerint az összes lehetőségek száma:

\[
72+576=648.
\]

Az előbbi számítások során különbséget tettünk aszerint, hogy a két
dominót milyen sorrendben húzzuk. Mivel minden összeillő dominó-pár
kétszer szerepel (pl. $35$ és $52$ majd $52$ és $35$), ha ezeket
nem tekintjük különbözőknek, akkor a lehetőségek száma az előbbi felére
csökken, azaz:

\[
\frac{648}{2}=324.
\]

\end{enumerate}
\end{solution}
\begin{extraproblem}[Kiss Andrea-Tímea]
Egy pénzfeldobás-sorozatban nyilvántarthatjuk, hogy hányszor követi
az egyik oldalt közvetlenül a másik oldal, például egy fej után írás
(FI), vagy egy fej után ismét egy fej (FF) következik, és így tovább.
Például a 
\[
FFIIFFFFIFFIIII
\]
15 dobásból álló sorozatban megfigyelhetjük, hogy van öt darab FF,
három darab FI, két darab IF és négy darab II.

Hány különböző, 15 pénzfeldobásból álló sorozat létezik, amely pontosan
két FF, három FI, négy IF és öt II szekvenciát tartalmaz? \emph{(AIME
1986)}
\end{extraproblem}

\begin{solution}
Tekintsük a pénzfeldobás-sorozatot írásokból és fejekből álló blokkok
sorozataként, ahol például egy fej blokk alatt azt értjük, hogy a
fejet közvetlenül egymás után valahányszor kidobjuk. Például egy FFFF
négy dobásból álló rész egy $\{F\}$ blokkott jelent, míg egy IIIIII
hat dobásból álló rész egy $\{I\}$ blokkott jelent.

Miképpen a sorozatunk három FI és négy IF szekvenciát kell tartalmazzon,
ezért a blokkok csak így következhetnek egymás után: 
\[
\{I\}\{F\}\{I\}\{F\}\{I\}\{F\}\{I\}\{F\}.
\]

Vizsgáljuk meg, hogy hogyan helyezzük el a fejeket és írásokat a blokkokban
aszerint, hogy két FF, öt II szekvencia legyen, valamint mindegyik
blokk legkevesebb egy dobást tartalmazzon. Ehhez feltételezzük, hogy
kezdetben minden blokk csak egy-egy fejet és írást tartalmaz. Így
még hiányzik $15-8=7$ pénzfeldobás-eredmény beillesztése a sorozatba.

Mivel két FF szekvencia van a sorozatban, ezért vagy egy $\{F\}$
blokkban van FFF, vagy két külön $\{F\}$ blokkban van FF. Mindkét
esetben 2 új fejet illesztettünk be a sorozatba, ezért a beillesztendő
írások száma 5 lesz. Az öt írást akárhogy illesszük be a $\{I\}$
blokkokba, minden esetben öt II szekvencia jön létre, ugyanis ahány
írást tartalmaz egy $\{I\}$ blokk, eggyel kevesebb II szekvencia
számolódik annál a blokknál. Ez azt jelenti, hogy minden új írás hozzáadása
a sorozathoz egy újabb II szekvenciát generál.

A probléma megoldásához azt kell vizsgáljuk, hogy a két fejet és öt
írást hányféleképpen tudunk elhelyezni a már meglévő blokkokban.

A két fejet a négy $\{F\}$ blokkba $C_{4+2-1}^{2}=C_{5}^{2}=\dfrac{5!}{2!\cdot3!}=10$-
féleképpen lehet elhelyezni, valamint az öt írást a négy $\{I\}$
blokkba $C_{4+5-1}^{5}=C_{8}^{5}=\dfrac{8!}{5!\cdot3!}=56$-féleképpen
lehet elhelyezni, ahol figyelembe vettük azt, hogy egy blokkot akár
többször választhatok, ezért számoltunk ismétléses kombinációval.

Ezek alapján $10\cdot56=560$-féle 15 dobásból álló sorozat lehetséges
a megadott feltételek mellett. 
\end{solution}
\begin{extraproblem}[Miklós Dóra]
Hányféleképpen juthatunk el a koordináta-rendszer origójából a $(4,2)$
pontba, ha $10$ lépést teszünk, minden lépésünk egységnyi hosszú
és párhuzamos a tengelyek valamelyikével? \emph{(OKTV 2012/2013; II.
kategória, 1. forduló)}
\end{extraproblem}

\begin{solution}
Jelöljük a lépéseket a B (balra), J (jobbra), F (fel) és L (le) betűkkel.
Ahhoz, hogy az origóból a $(4,2)$ pontba eljussunk legalább $4$-et
kell jobbra lépnünk és $2$-őt felfele. Ez összesen $6$ lépés, a
maradék négy lépést pedig úgy tehetjük csak meg, hogy a 4 lépés elvégzése
után helyben maradjunk: ha jobbra lépünk, kell szerepeljen egy balra
lépésnek is, valamint ha felfele lépünk akkor egyet kell lépnünk lefele
is. Így a maradék négy lépés a $(J,B)$, $(F,L)$ párokból tevődik
össze, melyek így az alábbiak lehetnek: 
\begin{itemize}
\item két $(J,B)$ pár, azaz $2$ db $J$, $2$ db $B$; 
\item két $(F,L)$ pár, azaz $2$ db $F$, $2$ db $L$; 
\item egy $(J,B)$ pár és egy $(F,L)$ pár, azaz $1$ db $J$, $1$ db $B$,
$1$ db $F$, $1$ db $L$. 
\end{itemize}
Ezeket összevetve az első hat lépéssel a következő lépésekből tevődhetnek
össze a megfelelő utak: 
\begin{itemize}
\item J, J, J, J, J, J, B, B, F, F; 
\item J, J, J, J, F, F, F, F, L, L; 
\item J, J, J, J, J, B, F, F, F, L. 
\end{itemize}
Az összes lehetséges út számát úgy kapjuk meg, ha megszámoljuk, hogy
hányféleképpen rendezhetőek sorrendbe a fenti sorok. Ehhez ismétléses
permutációt alkalmazunk, mert a sorozatokban ismétlődő elemek is megjelennek.
Tehát az összes lehetséges sorrendet az alábbiak szzerint kapjuk meg:
\begin{align*}
P_{10}^{6,2,2}+P_{10}^{4,4,2}+P_{10}^{5,1,3,1} & =\frac{10!}{6!\cdot2!\cdot2!}+\frac{10!}{4!\cdot4!\cdot2!}+\frac{10!}{5!\cdot1!\cdot3!\cdot1!}\\
 & =1260+3150+5040=9450.
\end{align*}
\end{solution}
\begin{extraproblem}[Péter Róbert]
Egy édességboltban 6 féle cukorka kapható: csokoládés, epres, vaníliás,
karamellás, mentolos és kókuszos. Egy vásárló 10 darab cukorkát szeretne
venni úgy, hogy bármelyik fajtából bármennyit választhat.
\begin{enumerate}
\item[a)] Hányféleképpen választhatja ki a 10 cukorkát? 
\item[b)] Hányféleképpen választhatja ki a 10 cukorkát, ha legalább egy csokoládésnak
lennie kell benne? 
\end{enumerate}
\end{extraproblem}

\begin{solution}
{a) Bármilyen elosztás lehetséges}

Az ismétléses kombináció képlete:

\[
C(n+k-1,k)=\binom{n+k-1}{k}
\]

ahol: 
\begin{itemize}
\item $n=6$ (cukorkafajták száma), 
\item $k=10$ (választott cukorkák száma). 
\end{itemize}
Tehát:

\[
C(6+10-1,10)=C(15,10)=\binom{15}{10}
\]

Számoljuk ki:

\[
\binom{15}{10}=\frac{15!}{10!(5!)}=\frac{15\times14\times13\times12\times11}{5\times4\times3\times2\times1}=3003
\]

\textbf{Válasz:} A vásárló \textbf{3003} különböző módon választhat
cukorkát.\\

{b) Legalább egy csokoládés cukorkának lennie kell}

Ha legalább egy csokoládés kell, akkor először választunk egy csokoládéset,
majd a maradék 9 cukorkát osztjuk el a 6 féle cukorka között:

\[
C(6+9-1,9)=C(14,9)=\binom{14}{9}
\]

Számoljuk ki:

\[
\binom{14}{9}=\frac{14!}{9!(5!)}=\frac{14\times13\times12\times11\times10}{5\times4\times3\times2\times1}=2002
\]

\textbf{Válasz:} A vásárló \textbf{2002} különböző módon választhat
cukorkát, ha legalább egy csokoládésnak lennie kell.
\end{solution}
\begin{extraproblem}[Seres Brigitta-Alexandra]
A nagypályás fiú focicsapat legutóbbi mérkőzésének első félidejében
leginkább védekezett. Mind a 16 szabadrúgást, amit a bíró nekik ítélt,
saját térfelükről végezhették el -- ezért a három csatár nem, de
a kapus rúghatott szabadot. A szabadrúgások elvégzőit a másodedző
jelölte ki. Hányféle lehetőség közül választhatott, ha 
\begin{itemize}
\item[a.)] a szabadrúgások időbeli sorrendjét is figyelembe vesszük; 
\item[b.)] csak azt vesszük figyelembe, hogy kik végezték el a szabadrúgásokat? 
\end{itemize}
A második félidőre az edző kicsit átszervezte a csapatot, így a játék
végig az ellenfél térfelén folyt. Most 13 szabadrúgást végeztek el,
de a kapus már nem rúgott labdába. Így hányféle lehetőség volt a szabadrúgások
elvégzésére, ha 
\begin{itemize}
\item[c.)] a szabadrúgások időbeli sorrendjét is figyelembe vesszük; 
\item[d.)] csak azt tekintjük, hogy kik végezték el a szabadrúgásokat? 
\end{itemize}
\begin{flushright}
\textit{(Magyarországi emelt szintű érettségi feladat)} 
\par\end{flushright}
\end{extraproblem}

\begin{solution}
Egy focimeccs alatt $11$ ember van a pályán egy csapatból, beleszámítva
a kapust is. Így ha a szabadrúgásokat a kapus is elvégezhette, de
a három csatár nem, $11-3=8$ ember végezhetett szabadrúgást. \\
 a.) $16$ szabadrúgást kellett leadjon $8$ ember. A fociban semmi
sem tiltja, hogy két vagy több szabadrúgást ne végezhessen el ugyanaz
a játékos. Így minden szabadrúgás esetén $8$ emberből lehet választani,
még akkor is ha az illető már többször is rúgott. Figyelembe véve
azt, hogy számít az időrendi sorrend, ismétléses variációval (értelmezés:
$8$-féle elemből (focistából) a sorrend figyelembevételével kiválasztunk
$16$ darabot (egyféle elemből többet is választhatunk)) tudjuk megadni
a kért eredményt: 
\[
\underbrace{8\cdot8\cdot\ldots\cdot8}_{\text{16-szor}}=8^{16}.
\]
b.) Mivel szintén az a kérdés, hogy $8$ emberből hányféleképpen tudunk
$16$-ot kiválasztani, úgy hogy egy embert többször is választhatunk,
de a kiválasztásuk sorrendje nem számít, ez értelmezés szerint $n=8$-nak
$k=16$-onkénti ismétléses kombinációja. Tehát 
\[
C_{8+16-1}^{16}=C_{23}^{16}=\frac{23!}{16!7!}=245157
\]
féleképpen választhatóak ki a fiúk, ha a sorrendet nem vesszük figyelembe.
\\
 c.) A csapat $11$ emberéből csak a kapus nem rúghatott, így $11-1=10$
emberből választhatták ki a szabadrúgást leadó embert. Mivel egy embert
akárhányszor is ki lehet választani, figyelembe véve az időbeli sorrendet,
a $13$ szabadrúgást a $10$ embernek: 
\[
\underbrace{10\cdot10\cdot\ldots\cdot10}_{\text{13-szor}}=10^{13}
\]
féle sorrendben lehetett leadni (ismétléses variáció).\\
 d.) Mivel szintén az a kérdés, hogy $10$ emberből hányféleképpen
tudunk $13$-at kiválasztani, úgy hogy egy embert többször is választhatunk,
de a kiválasztásuk sorrendje nem számít, ez értelmezés szerint $n=10$-nek
$k=13$-asonkénti ismétléses kombinációja. Tehát 
\[
C_{10+13-1}^{13}=C_{22}^{13}=\frac{22!}{13!9!}=497420
\]
féleképpen választhatóak ki a fiúk, ha a sorrendet nem vesszük figyelembe.
\end{solution}
\begin{extraproblem}[Seres Brigitta-Alexandra]
Tekintsük azokat a dominókat, amelyek mindkét felén a pontok száma
0-tól 8-ig terjed. Ezeket a pontok számának megfelelően így azonosíthatjuk:
$xy$, ahol $0\leq x\leq y\leq8$.
\begin{enumerate}
\item[a)] Hány ilyen dominó van? 
\item[b)] Hányféleképpen lehet a 45 ilyen dominó közül kettőt kiválasztani
úgy, hogy a két dominót egymás mellé lehessen tenni (azaz valamelyik
felükön a pontok száma azonos)?
\end{enumerate}
\end{extraproblem}
\begin{solution}
\begin{enumerate}
\item[a)] Ha $x=0$, akkor $y$ értékei $0,1,2,\dots,8$ közül lehetnek, ez
$9$ lehetőség. Ha $x=1$, akkor $y$ értékei $1,2,\dots,8$ közül
lehetnek (hiszen $x\leq y$, ezért nem lehet már $y=0$), ez $8$
lehetőség. Továbbá is ahogy növeljük $x$ számát $y$-ra eggyel kevesebb
választási lehetőség lesz $x\leq y$ feltétel miatt. Utolsó esetben,
ha $x=8$, akkor csak $y=8$ lehet, ez $1$ lehetőség.

Mivel a felsorolt esetek egymást kizárják és az összes lehetséges
esetet végigvettük, így az összegzési szabály ( „összes lehetőségek
száma $=$ az egymást kizáró eseteknek megfelelő lehetőségek számainak
összege”) szerint a dominók száma

\[
9+8+\dots+1=45.
\]

\item[b)] Válasszunk egy dominót.

Legyen az \textbf{1. eset}: "dupla" dominó, azon dominók, melyben
$x=y$, azaz $00,11,22,\dots,88$, ebből $9$ darab van.

A \textbf{2. eset}: olyan dominó, amelyre $x<y$ (az összes többi
ilyen, hiszen $x\leq y$ mindig igaz és $x=y$ előző esetben tárgyaltuk
le, így amik maradnak azok csak a $x<y$ típusú dominók), ezek száma
$45-9=36$.

Az \textbf{1. esetben} a második dominó 8-féleképpen választható (minden
dupla dominóhoz még $8$ darab olyan dominó tartozik, melyen szerepel
az adott szám, pl. $00$ mellé a $01$,$02$,...,$08$ dominók tehetőek
vagy pl. $11$ mellé $01,12,13,...,18$). Így minden dupla dominóhoz
$8$ társa tehető, összesen $9\cdot8=72$ eset van.

A \textbf{2. esetben} Egy olyan dominóhoz, melyben $x<y$ választhatunk
egy $x$ számjegyet tartalmazó dominót (ebből $8$ db van még az $xy$-on
kívül) vagy egy $y$ számjegyet tartalmazó dominót (ebből $8$ db
van még az $xy$-on kívül). Olyan, hogy mind a kettőt tartalmazza
$x,y$-t is, nem lehet, mert az maga az $xy$ és fordított sorrendben
nincsenek meg a dominók, hiszen $x<y$. Tehát a $x$-et tartalamzó
$8$ dominó és $y$-t tartalmazó $8$ dominó között nincsnenek közös
elemek, így $xy$ mellé $8+8=16$ dominót lehet választani.

Például $27$ lehetséges párjai: $02,07,12,17,22,23,24,25,26$, $28,37,47,57,67,77,78$.

Mivel a \textbf{2. esetben} 36 dominó van, mindegyik mellé $16$ pár
tehető le, az összes lehetőség: 
\[
36\cdot16=576.
\]
Az \textbf{1. eset} és \textbf{2. eset} dominói egymást kizáró eseteknek
számítanak, és egybevéve az összes lehetőséget kiadják, így az összegzési
szabály („összes lehetőségek száma $=$ az egymást kizáró eseteknek
megfelelő lehetőségek számainak összege”) szerint a lehetőségek száma
\[
72+576=648.
\]

Az előbbiekben különbséget tettünk aszerint, hogy a két dominót milyen
sorrendben húzzuk. Minden összeillő dominó-pár kétszer szerepelt,
így pl. először $27$ választása és hozzá $47$, majd először $47$
választása és hozzá $27$ párosítást két különbözőnek számítottuk.
Ha ezeket nem tekintjük különbözőknek, akkor a lehetőségek száma az
előbbi fele, azaz $324$. 
\end{enumerate}
\end{solution}

\begin{extraproblem}[Sógor Bence]
Hány megoldása van az $x+y+z+w=21$ egyenletnek a nem negatív egész
számok halmazán? Hány egész megoldása van az $x+y+z+w=21$ egyenletnek,
ha $x\geq3$, $y\geq1$, $z\geq-2$, $w\geq-4$?
\end{extraproblem}

\begin{solution}
Az első kérdést megválaszolva, tekinthetünk úgy a feladatra, mintha
21 1-est szeretnénk szétosztani $x$, $y$, $z$, $w$ között úgy,
hogy egy betűhöz több 1-est is rendelhetünk. Ezt $C_{4}^{21,i}$-féleképpen
tehetjük meg. A megoldások száma: 
\[
C_{4}^{21,i}=C_{24}^{21}=C_{24}^{3}.
\]

Próbáljuk visszavezetni ezt a kérdést az előzőre. Vegyük észre, hogy
ha az egyenletet átírjuk

\[
(x-3)+(y-1)+(z+2)+(w+4)=21-3-1+2+4=23
\]

alakba, akkor $x-3\geq0$, $y-1\geq0$, $z+2\geq0$ és $w+4\geq0$.
Tehát összesen:

\[
C_{4}^{23,i}=C_{26}^{23}=C_{26}^{3}
\]

egész megoldása van az egyenletnek.
\end{solution}
\begin{extraproblem}[Szabó Kinga]
Egy iskolában a tanulók $10$ fős csapatokat szerveztek. Egy diák
több csapatnak is tagja lehet, vagy akár egyiknek sem. A csapatok
száma $500$. Bizonyítsuk be, hogy a diákokat el lehet helyezni két
teremben úgy, hogy minden csapatnak mindkét teremben legyen tagja.
\emph{(Róka Sándor: 2000 feladat az elemi matematika köréből) }
\end{extraproblem}

\begin{solution}
A diákok száma $n$. Őket a két terembe $2^{n}$-féleképp lehet szétosztani.
A tanulók két teremben való szétosztásai között vannak "rossz" elhelyezések,
amelyekben valamelyik (vagy több) csapatnak minden tagja ugyanabba
a terembe kerül. Ha a rossz elhelyezések száma kisebb, mint $2^{n}$;
abból következik hogy van jó elhelyezés is. Tekintsünk egy tetszőleges
csapatot, és számítsuk ki, hogy hány olyan rossz elhelyezés van, amikor
ennek a csapatnak minden tagja ugyanabba a terembe kerül. Először
el kell döntenünk, hogy melyik ez a terem, ez $2$ lehetőség. Ezután
a többi $n-10$ embert kell szétosztanunk, amit $2^{n-10}$-féleképpen
tehetünk meg. Összesen tehát $2\cdot2^{n-10}=2^{n-9}$ olyan rossz
elhelyezés van, amikor egy kiszemelt csapat minden tagját ugyanabban
a teremben helyeztük el. $500$ csapat lévén az összes rossz elhelyezés
száma legfeljebb $500\cdot2^{n-9}.$ (Lehetnek esetek, amelyeket többször
is megszámoltunk.) Mivel $500\cdot2^{n-9}=\frac{500}{512}\cdot2^{n}<2^{n},$
ezért a feladat állítása igaz. 
\end{solution}
\begin{extraproblem}[Száfta Antal]
 Péter és Anna öt kiskutyát fogadott örökbe. A kiskutyáknak még nincsen
nevük, szóval elkezdtek tanakodni, hogy kereszteljék el őket. Végül
a követekező neveket jegyezték ki: 
\[
\text{Ali, Eli, Csillu, Tomi, Andi, Kamillus, Gáspa, Vivus, Jánus, Csibusz.}
\]
Eldöntötték, hogy a névadás során az alábbi szabályokat kell betartani: 
\begin{itemize}
\item Az első kutya neve nem kezdődhet „A” betűvel. 
\item Az öt kutyus neve közül pontosan két névnek kell magánhangzóval kezdődnie
(A, E, I, O, U). 
\item Egy adott név csak egyszer szerepelhet. 
\item A nevek sorrendje számít.

Hányféle sorrendben keresztelhetik el a kiskutyákat? 

\end{itemize}
\end{extraproblem}

\begin{solution}

\underline{ Az összes lehetséges sorrend ismétlés nélkül}\\

Ha nem lennének megkötések, akkor a nevek kiválasztása és sorrendje:
\[
V_{10}^{5}=\frac{10!}{(10-5)!}=\frac{10!}{5!}.
\]
Kiszámolva: 
\[
\frac{10!}{5!}=\frac{10\cdot9\cdot8\cdot7\cdot6}{1}=30240.
\]
Ez azonban nem veszi figyelembe a megadott feltételeket.

\underline{A feltételek figyelembevétele ismétléses permutációval}\\

\textbf{Az első név kiválasztása}

Az első név nem kezdődhet "A" betűvel, tehát Alo és Andi nem lehet
első. Ezért: 
\[
8\text{ lehetőség.}
\]

\textbf{Két magánhangzós név kiválasztása}\\

A magánhangzóval kezdődő nevek: \{\text{Ali, Eli, Andi}\}. Ebből
pontosan kettőt kell választani: 
\[
C_{3}^{2}=3.
\]

\textbf{Három mássalhangzós név kiválasztása}\\

A mássalhangzóval kezdődő nevek:

\{\text{Csillu, Tomi, Kamillus, Gáspa, Vivus, Jánus, Csabusz}\}.
Ebből pontosan hármat kell választani: 
\[
C_{7}^{3}=\frac{7!}{3!(7-3)!}=35.
\]

\underline{Az 5 kiválasztott név sorrendje}\\

A kiválasztott nevek permutációinak száma: 
\[
5!=120.
\]

\underline{A végső számítás}\\

A teljes lehetőségek száma: 
\[
8\cdot C_{3}^{2}\cdot C_{7}^{3}\cdot5!.
\]
Kiszámolva: 
\[
8\cdot3\cdot35\cdot120.
\]
\[
8\cdot3=24.
\]
\[
24\cdot35=840.
\]
\[
840\cdot120=100800.
\]

\[
\mathbf{100800}
\]
Tehát \textbf{100 800} különböző módon adhatják a neveket az öt kiskutyának.
\end{solution}
\begin{extraproblem}[Száfta Antal]
 Hány különböző módon lehet az összes karaktert átrendezni úgy, hogy
semmilyen két azonos karakter ne álljon egymás mellett, ha az eredeti
karakterlánc: 
\[
S=AABBBCCCC
\]
\end{extraproblem}

\begin{solution}

\underline{1. Az ismétléses permutációk teljes száma}

Először megszámoljuk az összes lehetséges átrendezést a karakterek
ismétléseinek figyelembevételével: 
\[
\frac{9!}{2!3!4!}=\frac{362880}{2\cdot6\cdot24}=\frac{362880}{288}=1260.
\]
Ez az összes lehetséges átrendezés, ha nem lenne megkötés.

\underline{2. A tiltott esetek meghatározása}

A feltétel szerint kizárjuk azokat az eseteket, amikor két azonos
karakter egymás mellett áll.

\textbf{Az A-k érintkezőek}\\

Ha az A-k mindig együtt állnak, akkor egyetlen "AA" blokkot kapunk:
\[
\frac{8!}{3!4!}=\frac{40320}{6\cdot24}=\frac{40320}{144}=280.
\]

\textbf{A B-k érintkezőek}\\

Ha a B-k mindig együtt állnak, akkor egyetlen "BBB" blokkot kapunk:
\[
\frac{7!}{2!4!}=\frac{5040}{2\cdot24}=\frac{5040}{48}=105.
\]

\textbf{A C-k érintkezőek}\\

Ha a C-k mindig együtt állnak, akkor egyetlen "CCCC" blokkot kapunk:
\[
\frac{6!}{2!3!}=\frac{720}{2\cdot6}=\frac{720}{12}=60.
\]

\underline{3. Az átfedések kezelése}

\textbf{Az A-k és B-k egyaránt érintkezőek}\\

\[
\frac{6!}{4!}=\frac{720}{24}=30.
\]

\textbf{Az A-k és C-k egyaránt érintkezőek}\\

\[
\frac{6!}{3!}=\frac{720}{6}=120.
\]

\textbf{A B-k és C-k egyaránt érintkezőek}\\

\[
\frac{5!}{2!}=\frac{120}{2}=60.
\]

\textbf{Az A-k, B-k és C-k mind érintkezőek}\\

\[
3!=6.
\]

\underline{4. Szitaformula alkalmazása}

Az érvényes permutációk száma: 
\[
1260-(280+105+60)+(30+120+60)-6
\]
\[
=1260-445+210-6=1019.
\]

Tehát \textbf{1019} különböző módon lehet az összes karaktert átrendezni
úgy, hogy semmilyen két azonos karakter ne álljon egymás mellett.
\end{solution}
\begin{extraproblem}[Csapó Hajnalka]
Egy matematikatáborban 3 nap alatt összesen 10 foglalkozásra kerül
sor az alábbi bontásban:
\begin{itemize}
\item 1. nap: 2 foglalkozás
\item 2. nap: 5 foglalkozás
\item 3. nap: 3 foglalkozás
\end{itemize}
A foglalkozások az alábbi témakörökből állnak:
\begin{itemize}
\item  Kombinatorika: 3 foglalkozás
 \item Geometria: 2 foglalkozás
 \item Algebra: 1 foglalkozás
 \item Analízis: 2 foglalkozás
 \item Számelmélet: 2 foglalkozás
\end{itemize}

a) Hányféleképpen lehet összeállítani az órarendet?

b) Hányféleképpen lehet összeállítani az órarendet, ha egy nap nem
lehet ugyanaz a téma kétszer? 
\end{extraproblem}

\begin{solution}
a) a lehetséges órarendek száma $\dfrac{10!}{3!\cdot2!\cdot1!\cdot2!\cdot2!}=75600$

b) Ha egy témakör nem szerepelhet egy nap kétszer, akkor a második
nap minden témakörből 1-1 foglalkozás lesz. Ezt $5!$ módon lehet az
órarendbe tenni. Az első és a harmadik napra kell tenni egy-egy kombinatorika
foglalkozást. Ezt 2, illetve 3 módon lehet. A 1. és a 3. nap a maradék
helyekre 1-1 Geometria, Analízis és Számelmélet foglalkozást lehet
tenni. Ezt első nap 3 féleképpen lehetséges, 3. nap pedig a maradék
két témakörből a maradék két helyre kétféleképpen lehet tenni a foglalkozásokat,
tehát az összes ilyen órarend: $2\cdot3\cdot5!\cdot3\cdot2=4320$.
\end{solution}
\begin{extraproblem}[Csapó Hajnalka]
 a) Hány olyan hatjegyű szám létezik, amelyben van legalább két azonos
számjegy? b) Hány olyan hatjegyű szám létezik, amelyben pontosan két
azonos számjegy van? 
\end{extraproblem}

\begin{solution}
a) Az összes hatjegyű természetes szám száma: $9\cdot10^{5}=900000$
Azon hatjegyű számok száma, amelyek számjegyei különböznek: $9\cdot9\cdot8\cdot7\cdot6\cdot5=136080$.
Tehát azok száma, amelyekben legalább két azonos számjegy van: $900000-136080=763920$.

b) Egy ilyen számhoz 5 számjegyet használunk. Ha nem tartalmaz 0-t,
akkor $5\cdot\dfrac{6!}{2!\cdot1!\cdot1!\cdot1!\cdot1!}=1800$ ilyen
szám képezhető ezzel az 5 számjeggyel. Az 5 számjegyet pedig $C_{9}^{5}=126$
módon választhatjuk ki, tehát $126\cdot18000=226800$ ilyen szám van.
Ha ez az 5 számjegy tartalmaz 0-t, akkor ha két 0-t tartalmaz, azokat
5 helyre kell letenni $\dfrac{4\cdot5}{2}=10$ módon, a többi helyre
pedig a megmaradt 4 számjegyet kell sorba rendezni $4!=24$ módon,
ezt a 4 számjegyet $C_{9}^{4}=126$ módon választatjuk ki, tehát összesen
$126\cdot10\cdot24=30240$ ilyen szám van. Ha nem a nulla ismétlődik,
akkor a maradék 4 számjegyet $4\cdot\dfrac{5!}{2!\cdot1!\cdot1!\cdot1!}=240$
féleképpen rendezhetjük. A nullát minden esetben 5 helyre tehetjük,
tehát $126\cdot240\cdot5=151200$

Tehát az összes ilyen szám 408240.
\end{solution}
\begin{extraproblem}[Szélyes Klaudia]
 Egy 5 házból álló házsort szeretnénk kifesteni, és 4-féle festékünk
van. A szomszédos házak nem lehetnek egyforma színűek. (Egy házhoz
csak egyféle festéket használunk, a festékeket nem lehet keverni.)
\end{extraproblem}

\begin{solution}
Az első házhoz 4-féle festékből választhatunk. A második házhoz már
csak a maradék 3 színből választhatunk, mivel a szomszédos ház nem
lehet ugyanaz a színű. Ugyanez igaz a harmadik, negyedik és ötödik
házra is, tehát mindegyikhez 3 szín közül választhatunk, kivéve az
előzőhöz használt színt.

Ezért az összes lehetséges kifestési lehetőség száma:

\[
4\cdot3\cdot3\cdot3\cdot3=4\cdot3^{4}=324
\]

Tehát összesen 324 féle kifestés létezik.
\end{solution}
\begin{extraproblem}[Szélyes Klaudia]
 Az ötös, vagy a hatos lottón lehet kevesebb szelvénnyel biztosan
telitalálatot elérni. (Az ötös lottó esetén 90 szám közül kell ötöt
találni, a hatoson 45-ből 6-ot.)
\end{extraproblem}

\begin{solution}
Ötös lottó esetén 90-ből 5 szelvényre, hatos lottónál pedig 45-ből
6 szelvényre van szükség. Vizsgáljuk a $\binom{90}{5}-\binom{45}{6}$
különbség előjelét:

\[
\binom{90}{5}-\binom{45}{6}=\frac{90\cdot89\cdot88\cdot87\cdot86}{5!}-\frac{45\cdot44\cdot43\cdot42\cdot41\cdot40}{6!}
\]

\[
=\frac{1}{6!}\left((90\cdot89\cdot88\cdot87\cdot86)-(45\cdot44\cdot43\cdot42\cdot41\cdot40)\right)
\]

\[
=\frac{1}{6!}\left((90\cdot89\cdot88\cdot87\cdot86)-(45\cdot44\cdot43\cdot42\cdot41\cdot40)\right)>0
\]

Azaz az ötös lottó esetén több szelvényt kell kitölteni.
\end{solution}
\begin{extraproblem}[Szélyes Klaudia]
Hányféle eredmény születhet akkor, ha egy csomag magyar kártyából
4 lapot egymás után kihúzunk, és a húzásnál
\begin{itemize}
\item[a)] a kihúzott lapokat mind megkülönböztetjük egymástól, 
\item[b)] a kihúzott lapokat csak a szín szerint különböztetjük meg, 
\item[c)] a kihúzott lapokat csak az értéke szerint különböztetjük meg? 
\end{itemize}
Oldjuk meg a feladatot úgy is, hogy az egyenkénti húzás után mindig
visszatesszük, illetve úgy is, hogy nem tesszük vissza a lapot (leosztjuk)!
\end{extraproblem}

\begin{solution}
\textbf{a)} visszatevés nélkül: Mivel minden lapot különbözőnek tekintünk,
minden húzás után csökken a lehetőségek száma. Az első lapnál 32 lehetőség
van, a másodiknál 31, és így tovább. Ezért a lehetséges eredmények
száma:

\[
32\cdot31\cdot30\cdot29=863040
\]

\textbf{a)} visszatevéssel: Ha minden lapot visszateszünk, akkor minden
húzásnál 32 lehetőség van, így az eredmények száma:

\[
32\cdot32\cdot32\cdot32=32^{4}=1048576
\]

\textbf{b)} minden esetben: Mivel csak a színeket különböztetjük meg,
minden húzásnál 4 szín közül választhatunk. Ezért a lehetséges eredmények
száma:

\[
4\cdot4\cdot4\cdot4=4^{4}=256
\]

\textbf{c)} minden esetben: Ha csak az értékeket különböztetjük meg,
akkor minden húzásnál 8 érték közül választhatunk. Az eredmények száma
tehát:

\[
8\cdot8\cdot8\cdot8=8^{4}=4096
\]
\end{solution}
\begin{extraproblem}[Lukács Andor]
Egy 10 fős társaság három csónakkal indul csónakázni: egy $5$, egy
$3$ és egy $2$ személyessel. Mennyi a valószínűsége annak, hogy
Anna és Béla (a társaság két tagja) egy csónakba kerülnek? 
\end{extraproblem}

\begin{solution}
Az összes lehetséges ültetések száma 
\[
N=\frac{10!}{5!\cdot3!\cdot2!}=2520.
\]
A kedvező esetek száma három részből tevődik össze: 
\begin{itemize}
\item Ha Anna és Béla az $5$ személyes csónakban ülnek, akkor a maradék
$8$ főből $3$ főt kell kiválasztani a csónakjukba, ezért a kedvező
esetek száma 
\[
K_{1}=\frac{8!}{3!\cdot3!\cdot2!}=560.
\]
\item Ha Anna és Béla a $3$ személyes csónakban ülnek, akkor a maradék
$8$ főből $1$ főt kell kiválasztani a csónakjukba, ezért a kedvező
esetek száma 
\[
K_{2}=\frac{8!}{5!\cdot1!\cdot2!}=168.
\]
\item Ha Anna és Béla a $2$ személyes csónakban ülnek, akkor a maradék
$8$ főből $0$ főt kell kiválasztani a csónakjukba, ezért a kedvező
esetek száma 
\[
K_{3}=\frac{8!}{5!\cdot3!\cdot0!}=56.
\]
\end{itemize}
Tehát a kedvező esetek száma 
\[
K=K_{1}+K_{2}+K_{3}=560+168+56=784.
\]
A keresett valószínűség 
\[
P=\frac{K}{N}=\frac{784}{2520}=\frac{14}{45}.
\]
\end{solution}

