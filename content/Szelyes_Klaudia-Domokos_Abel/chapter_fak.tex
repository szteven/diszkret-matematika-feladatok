
\chapter{Fák}\label{chap:fak}


\section{Alapfogalmak}\label{sec:fak_alapok}
\begin{description}
	{\large \item [{Szerző:}] Szélyes Klaudia (Didaktikai mesteri -- Matematika, I.
		év)}
\end{description}

\subsection*{Mi az a fa gráf?}

A fa egy olyan gráf, amely összefüggő, és nem tartalmaz kört. Ez azt
jelenti, hogy a gráf bármely két csúcsa között pontosan egy út vezet.
Mivel nincs benne kör, és minden csúcs elérhető minden másiktól, ezért
a fa struktúrája különösen hasznos sok gyakorlati területen.

Példaként említhetjük a víz- és csatornahálózatokat, valamint az elektromos
hálózatokat. Az emberi érrendszer -- különösen az artériák vagy a
vénák hálózata -- szintén fagráfként modellezhető. Ezek hátránya,
hogy ha egy él megsérül vagy elzáródik, a gráf már nem lesz összefüggő,
ami például az érrendszer esetén sejtpusztuláshoz vezethet.

A fa adatszerkezetek szempontjából is fontos, mivel homogén, dinamikus
és hierarchikus struktúrákat lehet vele ábrázolni.

\subsection*{Alapfogalmak:}
\begin{itemize}
	\item \textbf{Csúcs (csomópont)}: A fának, mint gráf csomóponthalmazának
	egy eleme.
	\begin{center}
		\includegraphics[width=0.5\linewidth]{\string"content/Szelyes_Klaudia-Domokos_Abel/image1\string".png}
		\par\end{center}
	\item \textbf{Levél}: Olyan csúcs, amelynek nincs utóda.
	\begin{center}
		\includegraphics[width=0.5\linewidth]{\string"content/Szelyes_Klaudia-Domokos_Abel/image3\string".png}
		\par\end{center}
	\item \textbf{Belső elem}: Olyan csúcs, amelynek legalább egy utóda van.
	\begin{center}
		\includegraphics[width=0.5\linewidth]{\string"content/Szelyes_Klaudia-Domokos_Abel/image4\string".png}
		\par\end{center}
	\item \textbf{Él}: Kapcsolat két csúcs között.
	\begin{center}
		\includegraphics[width=0.5\linewidth]{\string"content/Szelyes_Klaudia-Domokos_Abel/image5\string".png}
		\par\end{center}
	\item \textbf{Út}: Egy olyan csúcssorozat, amelyet élek kötnek össze.
	\begin{center}
		\includegraphics[width=0.5\linewidth]{\string"content/Szelyes_Klaudia-Domokos_Abel/image6\string".png}
		\par\end{center}
	\item \textbf{Részfa}: Egy csúcsból és annak összes utódjából álló fa.
	\begin{center}
		\includegraphics[width=0.5\linewidth]{\string"content/Szelyes_Klaudia-Domokos_Abel/image7\string".png}
		\par\end{center}
	\item \textbf{Szint}: A gyökértől való távolság alapján meghatározott csúcsok
	halmaza.
	\begin{center}
		\includegraphics[width=0.5\linewidth]{\string"content/Szelyes_Klaudia-Domokos_Abel/image8\string".png}
		\par\end{center}
	\item \textbf{Gyökér}: A fa legfelső szintjén lévő csúcs.
	\begin{center}
		\includegraphics[width=0.5\linewidth]{\string"content/Szelyes_Klaudia-Domokos_Abel/image2\string".png}
		\par\end{center}
	\item \textbf{Magasság}: A gyökértől egy levélig vezető leghosszabb út hossza.
	\begin{center}
		\includegraphics[width=0.5\linewidth]{\string"content/Szelyes_Klaudia-Domokos_Abel/image9\string".png}
		\par\end{center}
\end{itemize}

\subsection*{Rendezett és rendezetlen fák}

Egy \textbf{rendezetlen} fában nem számít az élek sorrendje, azaz
az, hogy egy csúcsból milyen sorrendben ágaznak ki az utódok. Ezzel
szemben egy \textbf{rendezett} fában az élek sorrendje fontos: például
az első, második stb. gyermek külön jelentéssel bír. 
\begin{center}
\includegraphics[width=0.5\linewidth]{\string"content/Szelyes_Klaudia-Domokos_Abel/rendezett rendezetlen\string".png}
\par\end{center}
Ekvivalensnek tekinthető a két fa, ha rendezetlen fákról van szó.
Ha viszont rendezett fákról beszélünk, akkor a sorrend számít, így
a két fa különbözőnek minősül.

\begin{theorem}{thm:fa-min2level}
Ha egy fának legalább két csúcsa van, akkor abban biztosan legalább
két levél található.
\end{theorem}
\begin{proof}[Bizonyítás (vázlat).]
Vegyünk egy leghosszabb utat a fában,
jelöljük a végpontjait $v_{1}$ és $v_{k}$-val. Ezek a csúcsok nem
lehetnek több él részei, mert akkor hosszabb út is létezne -- ez
ellentmondás. Ezért ezek biztosan levelek. (Kép: leghosszabb út, kör
és hosszabb út kizárása)
\end{proof}

\begin{theorem}{thm:fa-n-n-1}
Egy $n$ csúcsú fa pontosan $n-1$ élből áll.
\end{theorem}
\begin{proof}
Indukcióval bizonyítható. Alapeset: ha a fa
1 csúcsból áll, nincs benne él; ha 2 csúcsból, akkor pontosan egy
él kapcsolja őket össze. - Indukciós lépés: ha feltételezzük, hogy
$k$ csúcs esetén a fa $k-1$ élből áll, akkor $n$ csúcsú fa esetén
eltávolítva egy levelet és élét, $(n-1)$ csúcsú fát kapunk, amiben
indukciós feltevés szerint $n-2$ él van. Visszatéve a levelet, a
fa élszáma $n-1$ lesz.
\end{proof}

\begin{theorem}{thm:fak-jellemzese}
Egy gráf akkor és csak akkor fa, ha összefüggő és pontosan $n-1$
éle van. Ez a karakterizációs tétel nagyon hasznos algoritmusok tervezésénél
és gráfok elemzésénél.
\end{theorem}

\begin{definition}[Kiegyensúlyozott fák]{def:kiegyensulyozott}
Egy fa kiegyensúlyozottnak tekinthető, ha bármely csomópont bal és
jobb részfájának magasságkülönbsége legfeljebb 1. 
\end{definition}
Az ilyen struktúrák
különösen hatékonyak keresési műveletek szempontjából, mert minimalizálják
a fa magasságát, így gyorsabb hozzáférést biztosítanak az adatokhoz.

\begin{definition}[Erdő]{def:Erdo}
Az erdő fák különálló gyűjteménye. Ha egy fából eltávolítunk egy élt,
az erdővé válik, azaz több összefüggő komponensre bomlik. Az erdő
minden komponense egy-egy fa.
\end{definition}

\begin{definition}[Feszítőfa]{def:feszitofa}
Egy gráf \textbf{feszítőfája} olyan részgráf, amely tartalmazza az
eredeti gráf összes csúcsát, összefüggő és nem tartalmaz kört, tehát
fa.
\end{definition}

\begin{theorem}{thm:feszitofa}
Minden összefüggő gráfnak van feszítőfája.
\end{theorem}
\begin{proof}
Vegyük az összefüggő gráfot, és kezdjünk el eltávolítani
köröket úgy, hogy közben az összefüggőség ne szűnjön meg. Amikor már
nincs több kör, a kapott részgráf feszítőfa lesz.
\begin{center}
	\includegraphics[width=0.85\linewidth]{\string"content/Szelyes_Klaudia-Domokos_Abel/feszitofa_biz\string".png}
	\par\end{center}
\end{proof}



\section*{Házi feladatok}
\begin{problem}
Egy versenyen $n$ $(n=47)$ csapat játszik egyenes kieséses rendszerben:
\begin{itemize}
\item A csapatokat párokba állítják. 
\item A győztes továbbjut a következő fordulóba. 
\item A vesztes kiesik. 
\item Ha páratlan számú csapat van, valaki meccs nélkül jut tovább. 
\end{itemize}
\vspace{1em}

\textbf{Kérdések:} 
\begin{enumerate}
\item Hány mérkőzés szükséges, hogy kiderüljön, ki a bajnok? 
\item Hány fordulóra van szükség? 
\item Milyen gráffal szemléltethető ez a rendszer? 
\item Hány mérkőzés kell ahhoz, hogy biztosan megtudjuk, melyik csapat lett
a második? Indokolja válaszát! 
\end{enumerate}
\end{problem}

\begin{solution}
	\begin{enumerate}
\item Egy egyenes kieséses rendszerben minden mérkőzésen egy csapat kiesik,
míg egy marad bajnoknak. Ha $n$ csapat indul, akkor pontosan $n-1$
csapat esik ki, tehát: 46 mérkőzés szükséges.

\item A fordulók száma a legkisebb $k$, amelyre: 
\[
2^{k}\geq n.
\]
Mivel $n=47$, és 
\[
2^{5}=32<47<64=2^{6},
\]
ezért: 
\[
\boxed{6}
\]
fordulóra van szükség.
\item A verseny lefolyását egy \textbf{fa} gráffal lehet szemléltetni, ahol: 
\begin{itemize}
\item A csúcsok a csapatokat jelölik. 
\item Az élek a mérkőzéseket, amelyek az egyik csapat kieséséhez vezetnek. 
\item A fa gyökere a bajnok. 
\item A levélcsúcsok az induló csapatok. 
\end{itemize}
A fa összefüggő és körmentes, és az élek száma $n-1$.
\item A második helyezett kiléte nem biztos, hogy egyértelműen eldönthető
az egyenes kieséses rendszerben, mivel a vesztesek kiesnek, és nem
játszanak további mérkőzéseket.

Ez azt jelenti, hogy a döntőben vesztes csapat lesz a második, de
lehet, hogy nem a ténylegesen második legjobb csapat, mert a későbbi
bajnok már korán legyőzhette a valódi második legjobbat.

Így nem létezik biztosan meghatározható minimális mérkőzésszám a pontos második hely eldöntéséhez egyetlen egyenes kieséses versenyben.

Ha pontosabb rangsort akarunk (második, harmadik hely), más versenyrendszert,
például vigaszágat vagy körmérkőzést kell alkalmazni. 

\end{enumerate}
\end{solution}
\begin{problem}
Egy 7 csúcsú fa gráfban a csúcsok fokszáma a következő: 
\begin{itemize}
\item Egy csúcsnak a fokszáma 3, 
\item Három csúcsnak a fokszáma 2, 
\item Három csúcsnak a fokszáma 1. 
\end{itemize}
\begin{enumerate}
\item Igazolja, hogy ez a fokszámeloszlás lehetséges-e egy fán! 
\item Rajzoljon egy példát erre a fára! 
\end{enumerate}
\end{problem}

\begin{solution}
~
\begin{enumerate}
\item Ellenőrizzük, hogy a fokszámok összege megegyezik-e a fa éleinek kétszeresével.

Egy 7 csúcsú fában az élek száma: 
\[
|E|=7-1=6.
\]
A fokszámok összege legyen: 
\[
\sum\deg(v)=1\times3+3\times2+3\times1=3+6+3=12.
\]
Mivel a gráf éleinek kétszerese: 
\[
2\times|E|=2\times6=12,
\]
a feltételek teljesülnek, így a fokszámeloszlás lehetséges egy fán.
\item A fa rajzolása:

\medskip{}

\begin{itemize}
\item Vegyünk egy központi csúcsot, amelynek fokszáma 3. 
\item Ehhez csatlakozzanak három olyan csúcs, amelyeknek a fokszáma 2. 
\item Minden ilyen 2 fokú csúcshoz kapcsolódjon egy levélcsúcs (fokszám
1). 
\end{itemize}
Ez egy olyan fa, amelynek a fokszámeloszlása megfelel a feladatban
adottaknak. 

\end{enumerate}
\end{solution}
\begin{problem}
Rajzolj olyan erdőt, amely 4 komponensből (fából) és 10 csúcsból áll!
Mennyi éle van?
\end{problem}

\begin{solution}
Egy komponens egy összefüggő fa, amelyben az élek száma a csúcsok
számánál eggyel kevesebb. Ha az erdő 4 komponensből (4 fából) áll,
és összesen 10 csúcsa van, akkor:

\[
\text{Élek száma}=10-4=6
\]

\textbf{Rajz:}

\begin{tikzpicture}[
  vertex/.style=circle,draw,fill=black,inner sep=2pt,
  scale=1, every node/.style=transform shape
]

% Fa 1 (3 csúcs)
\node[vertex] (a1) at (0,0) {};
\node[vertex] (a2) at (1,0) {};
\node[vertex] (a3) at (0.5,-1) {};
\draw (a1) -- (a2);
\draw (a1) -- (a3);

% Fa 2 (3 csúcs)
\node[vertex] (b1) at (3,0) {};
\node[vertex] (b2) at (4,0) {};
\node[vertex] (b3) at (3.5,-1) {};
\draw (b1) -- (b2);
\draw (b2) -- (b3);

% Fa 3 (2 csúcs)
\node[vertex] (c1) at (0,-3) {};
\node[vertex] (c2) at (1,-3) {};
\draw (c1) -- (c2);

% Fa 4 (2 csúcs)
\node[vertex] (d1) at (3,-3) {};
\node[vertex] (d2) at (4,-3) {};
\draw (d1) -- (d2);

\end{tikzpicture}

\noindent\textbf{Válasz:} Az erdőnek 6 éle van.
\end{solution}
\begin{problem}
Egy fa egyetlen ággal indul. Ez az ág két évig nő, majd a második
év leteltekor tavasszal hajt egy új ágat. Ezután minden évben tavasszal
új ágat hajt. Az új hajtások az első évükben csak növekednek, és a
második év végétől ők is ágat hajtanak, majd minden évben újabbat.
Hány ág lesz a hatodik év végén?
\end{problem}

\begin{solution}
Rekurzívan számoljuk az ágak számát:
\begin{itemize}
\item 0. év: 1 ág (kezdeti) 
\item 1. év: 1 ág 
\item 2. év: 2 ág 
\item 3. év: 3 ág 
\item 4. év: 5 ág 
\item 5. év: 8 ág 
\item 6. év: 13 ág 
\end{itemize}
\noindent\textbf{Válasz:} A fa hatodik év végére 13 ágból fog állni.

\end{solution}
\begin{problem}
Egy 15 pontú fagráfban 12 levél van. Hány élen halad át a gráf leghosszabb
útja? 
\end{problem}

\begin{solution}
A maradék pontok száma: $15-12=3$, ezek a belső pontok. Ha a három
belső pont láncot alkot, és a két széléhez kapcsolódik egy-egy levél,
akkor a leghosszabb út így nézhet ki:

\[
\text{levél}\rightarrow\text{belső1}\rightarrow\text{belső2}\rightarrow\text{belső3}\rightarrow\text{levél}
\]

Ez az út 5 csúcsot és 4 élt tartalmaz.

\noindent\textbf{Válasz:} A leghosszabb út hossza 5 él.
\end{solution}
\begin{problem}
Egy társaság tagjai most találkoztak először, eddig 13 e-mailcím-csere
történt a megbeszélés alatt. Ez elegendő ahhoz, hogy pontosan egy
úton bárki üzenete bárkihez eljusson. Legfeljebb hány fős lehet ez
a társaság? 
\end{problem}

\begin{solution}
Ez azt jelenti, hogy a kapcsolati háló egy \emph{fa}, mert bármely
két személy között pontosan egy út van, és nincsenek körök. A fának
$n$ csúcsa és $n-1$ éle van.

\[
n-1=13\Rightarrow n=14
\]

\noindent\textbf{Válasz:} Legfeljebb 14 fős lehetett a társaság.
\end{solution}

\section*{Nehezebb feladatok}
\begin{extraproblem}[Gergely Verona]
Öt város közé négy egyenes szakaszból álló vasúthálózatot akarnak
építeni úgy, hogy bármelyik városból bármelyik másikba el lehessen
jutni. (A városok közül semelyik három nem esik egy egyenesbe. A szakaszok
keresztezhetik is egymást.) Hány ilyen vasúthálózat lehetséges? \emph{(KöMaL,
1992) }
\end{extraproblem}

\begin{solution}
A feladatban adott, hogy 5 város van, és 4 összekötő szakasz (él)
építhető ki közöttük. Az a feltétel, hogy bármelyik városból el lehessen
jutni bármelyik másikba, azt jelenti, hogy a hálózatnak összefüggőnek
kell lennie.

Ez tehát egy olyan gráfot ír le, amelynek:
\begin{itemize}
\item 5 csúcsa van (a városok), 
\item 4 éle van (a vasúti szakaszok), 
\item és a gráf összefüggő. 
\end{itemize}
Ezek a feltételek pontosan egy fát definiálnak. Egy fa olyan összefüggő,
körmentes gráf, amely $n$ csúccsal és $n-1$ éllel rendelkezik. Jelen
esetben $n=5$, tehát az összes 5 csúcsú fa számát keressük.

A Cayley-tétel szerint az $n$ címkézett csúcsú fa lehetséges száma:

\[
n^{n-2}
\]

Esetünkben:

\[
5^{5-2}=5^{3}=125
\]

Tehát $125$ különböző vasúthálózat lehetséges. 
\end{solution}
\begin{extraproblem}[Kiss Andrea-Tímea]
Egy kommunikációs hálózatot szeretnénk kiépíteni 21 számítógépből
úgy, hogy: 
\begin{itemize}
\item a hálózat minden gépe elérje a többit (azaz a hálózat összefüggő legyen), 
\item ne legyenek hurkok vagy körök (azaz a hálózat gráfja fa legyen), 
\item továbbá minden gép legfeljebb 4 másik géppel lehet közvetlen kapcsolatban. 
\end{itemize}
Mi a lehető legnagyobb fokszámú gépek száma a hálózatban? 
\end{extraproblem}

\begin{solution}
Mivel a hálózat gráfja egy \emph{fa}, azaz összefüggő és körmentes,
ezért 21 csúcs esetén pontosan $21-1=20$ él van.

A gráfelmélet egyik alapvető tétele szerint a fokszámok összege kétszerese
az élek számának: 
\[
\sum\text{fokszám}=2\cdot20=40.
\]

Tegyük fel, hogy $x$ darab csúcs (számítógép) fokszáma 4. Ezek együtt
$4x$ fokot adnak. A többi $21-x$ csúcs fokszáma legfeljebb 3, de
minden csúcs fokszáma legalább 1 (mivel a fa összefüggő).

Próbáljunk meg konkrét értékeket behelyettesíteni:
\begin{itemize}
\item Ha $x=8$, akkor a 8 gép összesen $8\cdot4=32$ fokot ad. Ekkor a
maradék $21-8=13$ gépnek $40-32=8$ fok marad, azaz ezeket úgy kell
elosztani, hogy a fokszámuk átlaga kevesebb mint 1. Ez csak úgy lehet,
ha pontosan 8 csúcsnak a fokszáma 1, a többinek 0 --- ez még lehetséges.
\item Ha $x=9$, akkor $9\cdot4=36$ fokot kapunk, a maradék 12 csúcsra
$40-36=4$ fok marad. Ez legfeljebb 4 darab 1 fokú csúcs lehet, és
8 darab 0 fokú. De egy fa nem lehet összefüggő, ha 8 csúcs teljesen
elszigetelt --- tehát ez nem megengedett.
\item Ha $x=10$, akkor $10\cdot4=40$ fok jön ki, a többi 11 csúcsnak nulla
fokszáma lenne, vagyis teljesen különálló csúcsok --- így a gráf
nem lenne összefüggő, tehát ez sem lehet. 
\end{itemize}
Tehát a maximális számú olyan csúcs, amely fokszáma 4 lehet, az 8.
\end{solution}
\begin{extraproblem}[Kovács Levente]
Egy összefüggő, irányítatlan, ciklusmentes gráf (azaz egy fa) 17
csúcsot tartalmaz, és a csúcsok fokszámösszege 32. Hány levél található
ebben a fában? 
\end{extraproblem}

\begin{solution}
\noindent Egy $n$-csúcsú fa $n-1$ élt tartalmaz, így: 
\[
\text{Élek száma}=17-1=16.
\]
A fokszámösszeg egy irányítatlan gráfban: 
\[
\sum\deg(v)=2\cdot\text{élek száma}=2\cdot16=32,
\]
ez megegyezik a feladatban szereplő értékkel.

Tegyük fel, hogy $k$ darab levél van (azaz $\deg(v)=1$). A többi
$17-k$ csúcs fokszáma legalább 2.

A fokszámösszeg tehát: 
\[
k\cdot1+\sum_{i=1}^{17-k}\deg(v_{i})=32.
\]

Minimumértékkel becsülve: 
\[
k+2(17-k)=32\Rightarrow k+34-2k=32\Rightarrow-k=-2\Rightarrow k=2.
\]

\textbf{Válasz: } $\boxed{2}$ levél található a fában. 
\end{solution}
\begin{extraproblem}[Kovács Levente]
Hány különböző címkézett fa létezik pontosan 6 csúccsal? 
\end{extraproblem}

\begin{solution}
\noindent A Cayley-tétel szerint egy $n$-csúcsú címkézett fára az
összes lehetséges fa száma: 
\[
n^{n-2}.
\]

Ebben az esetben $n=6$, tehát: 
\[
6^{6-2}=6^{4}=\boxed{1296}.
\]

\textbf{Válasz: } Összesen $\boxed{1296}$ különböző címkézett fa
létezik. 
\end{solution}
\begin{extraproblem}[Lukács Andor]
Bizonyítsd be, hogy egy fa automorfizmusának van fixpontja vagy fixéle! 
\end{extraproblem}

\begin{solution}
Legyen $L$ az $F$ fa elsőfokú pontjainak halmaza. Mivel $F$ egy
egyszerű gráf, ezért egy automorfizmusa leírható mint pontjainak egy
permutációja. Ha $\varphi:V(F)\to V(F)$ egy automorfizmusa $F$-nek,
akkor $\varphi(L)=L$, hiszen egy gráf automorfizmusa elsőfokú pontokat
elsőfokú pontokba visz. Így 
\[
\varphi_{1}:V(F)\setminus L\to V(F)\setminus L,
\]
ahol $\varphi_{1}$ a $\varphi$ automorfizmus leszűkítése, az $F_{1}=F-L$
fa egy automorfizmusa (feltéve, hogy $V(F)\setminus L\neq\emptyset$).
Az elsőfokú pontok elhagyásával egy $F,F_{1},F_{2},\ldots,F_{k}$
fasorozatot nyerünk úgy, hogy az eredeti automorfizmus leszűkítése
az $F_{i}$ fa ponthalmazára ($1\leq i\leq k$) az $F_{i}$ fa egy
automorfizmusát adja.

A sorozat utolsó eleme egy üres gráf vagy egy 1 pontú gráf. Ha üres
gráfhoz jutunk, akkor az előző lépésben egy 1 élű gráfunk volt. Ezt
az élt az automorfizmus fixen hagyja. Ha egy 1 pontú gráfhoz jutunk,
akkor ezt a pontot az automorfizmus fixen hagyja. 
\end{solution}
\begin{extraproblem}[Lukács Andor]
Bizonyítsd be, hogy egy $2n$ élű fa élhalmaza felbontható $n$ darab
diszjunkt párra, ahol egy párban szomszédos élek szerepelnek! 
\end{extraproblem}

\begin{solution}
Legyen $x$ egy elsőfokú pont szomszédja az adott $F$ fában. (A feltételek
miatt $F$-nek $2n+1$ pontja van.) Ekkor $(F-x)$ komponensekre esik
szét. Az $x$ választása miatt az egyik komponens 1 pontú lesz. Mivel
$(F-x)$-ben $2n$ pont van, ezért lesz másik páratlan pontú komponens.
Az $x$-ből ehhez a páratlan komponenshez és az elsőfokú ponthoz vezető
két él egy szomszédos élpár. Ezt a két élt elhagyva $F$-ből a maradék
gráfban az összes komponens páros pontú (azaz páros élű) fa lesz.
Ez alapján indukcióval igazolható az állítás. 
\end{solution}
\begin{extraproblem}[Lukács Andor]
Az $1,2,\ldots,n$ természetes számokat úgy akarjuk kiszínezni, hogy
a relatív prímszámok különböző színt kapjanak. Legalább hány színre
van szükségünk? 
\end{extraproblem}

\begin{solution}
Legyen $\pi(n)$ az $n$-nél nem nagyobb prímek száma. Ekkor $1$
és az első $\pi(n)$ darab prímszám közül bármely kettő relatív prím,
így legalább $\pi(n)+1$ szín szükséges. De $\pi(n)+1$ szín elég:
az $n>1$ szám színe legyen egyik prímosztója, $1$ pedig kapjon egy
külön színt. 
\end{solution}
\begin{extraproblem}[Lukács Andor]
Legalább hány csoportba kell beosztanunk az első száz pozitív egész
számot, ha azt akarjuk, hogy egyetlen csoportban se legyen két olyan
szám, amelyek egyike többszöröse a másiknak? 
\end{extraproblem}

\begin{solution}
Az $1,2,4,8,16,32,64$ számok egyike sem kerülhet ugyanabba az osztályba,
azaz legalább hét osztályra van szükségünk. Az 
\begin{gather*}
\{1\},\ \{2,3\},\ \{4,5,6,7\},\ \{8,9,\ldots,15\},\\
\{16,17,\ldots,31\},\ \{32,33,\ldots,63\},\ \{64,65,\ldots,100\}
\end{gather*}
osztályozás megfelelő. Így a szükséges osztályok minimális száma $7$. 
\end{solution}
\begin{extraproblem}[Péter Róbert]
Legyen $T$ egy fa $n$ csúccsal. Tudjuk, hogy minden levél fokszáma
1, minden belső csúcs fokszáma páros szám. Bizonyítsd be, hogy a levelek
száma páros. 
\end{extraproblem}

\begin{solution}
Jelölések: 
\begin{itemize}
\item $L$ --- levelek száma, 
\item $I$ --- belső csúcsok száma, 
\item $n=L+I$ --- összes csúcs. 
\end{itemize}
Mivel $T$ fa, élei száma $n-1=L+I-1$.

A fokszámok összege kétszerese az élek számának: 
\[
\sum_{v\in V}\deg(v)=2(n-1)=2(L+I-1)
\]

A levelek fokszáma 1, a belső csúcsoké páros: 
\[
\sum_{v\in V}\deg(v)=\sum_{i=1}^{I}d_{i}+L
\]
ahol $d_{i}$ mind páros szám.

Ebből: 
\[
\sum_{i=1}^{I}d_{i}+L=2(L+I-1)\Rightarrow\sum_{i=1}^{I}d_{i}=L+2I-2
\]

A bal oldal páros (mert minden $d_{i}$ páros), tehát a jobb oldal
is páros: 
\[
L+2I-2\equiv0\pmod 2\Rightarrow L\equiv0\pmod 2
\]

\bigskip{}

\textbf{Következtetés:} A levelek száma páros. 
\end{solution}
\begin{extraproblem}[Sógor Bence]
Egy $F$ fának 17 csúcsa van, és bármely csúcsának fokszáma 1 vagy
4. Határozzuk meg, legalább hány élet kell $F$-be behúzni ahhoz,
hogy a keletkező gráfnak legyen Euler-köre! 
\end{extraproblem}

\begin{solution}
Egy összefüggő gráfnak pontosan akkor van Euler-köre, ha minden csúcs
fokszáma páros. Mivel $F$ egy fa, ezért az összefüggőség teljesül.
$F$ fának 17 csúcsa van, tehát 16 él fut a csúcsok között. Az összfokszám
32. Jelölje $n$ a 4 fokszámú csúcsok számát és jelölje $m$ az 1
fokszámú csúcsok számát. Keressük az $n$ és $m$ pozitív számokat
úgy, hogy $n\cdot4+m\cdot1=32$. Próbálgatás után megkapjuk, hogy
$n=5$ és $m=12$. Két 1 fokszámú csúcs biztosan nincs összekötve
egy éllel, mert ez egy különálló összefüggő komponenst határozna meg.
Mivel tetszőleges kettőt összetudunk kötni, ezért 6 él felhasználásával
elérhető, hogy csak 4 és 2 fokszámú csúcsok legyenek. Legalább 6 élt
kell ahhoz behúzni, hogy legyen $F$-nek Euler-köre. 
\end{solution}
\begin{extraproblem}[Száfta Antal]
Feladat 1 Legyen $n\geq3$ egész szám. Hány olyan \textbf{címkézett}
fa van $n$ csúcsra, amelyben: 
\begin{itemize}
\item a $1$-es csúcs fokszáma legalább $3$, és 
\item a $1$ és $2$ csúcsok nem szomszédosak? 
\end{itemize}
\end{extraproblem}

\begin{solution}
Megoldás 1 Egy $n$-csúcsú fa Prüfer-kódja egy $n-2$ hosszúságú sorozat
$\{1,2,\dots,n\}$ elemeiből. Egy csúcs fokszáma: 
\[
\deg(v)=\#\text{előfordulás}(v)+1.
\]
\begin{itemize}
\item A $1$-es csúcs fokszáma legalább $3$ $\Rightarrow$ $1$ legalább
$2$-szer szerepel a kódban. 
\item A $1$ és $2$ nem szomszédos $\Rightarrow$ elég, ha a $2$ \emph{nem
szerepel} a Prüfer-kódban (ekkor fokszáma $1$, tehát levélként nem
kapcsolódhat máshoz, mint egyetlen másik csúcshoz -- nem lehet 1-hez). 
\end{itemize}
A kód hossza: $n-2$, a megengedett elemek halmaza: $\{1,3,4,\dots,n\}$,
azaz $n-2$ elem.

\paragraph{Összes kód, amelyben $2$ nem szerepel:}

\[
A=(n-2)^{n-2}.
\]

\paragraph{Levonjuk azokat, ahol $1$ legfeljebb egyszer szerepel:}
\begin{itemize}
\item $1$ egyáltalán nem szerepel: $(n-3)^{n-2}$, 
\item $1$ pontosan egyszer szerepel: $(n-2)\cdot(n-3)^{n-3}$. 
\end{itemize}

\paragraph{Megfelelő kódok száma:}

\[
B=A-\left[(n-3)^{n-2}+(n-2)(n-3)^{n-3}\right].
\]

\paragraph{Végső válasz:}

Azon $n$-csúcsú címkézett fák száma, amelyek megfelelnek a feltételeknek:

\[
\boxed{(n-2)^{n-2}-(n-3)^{n-2}-(n-2)(n-3)^{n-3}}
\]
\end{solution}

\section{Címkézett fák, Prüfer-kód}\label{sec:cimkezett_fak}

\begin{description}
	{\large \item [{Szerző:}] Domokos Ábel (Didaktikai mesteri -- Matematika, I.
		év)}
\end{description}

\begin{definition}[Címkézett gráfok]{def:cimkezett_graf} A címkézett gráf olyan gráf, amelyben a csúcsokhoz
	és/vagy élekhez címkék (azaz nevek, számok, szimbólumok vagy egyéb
	azonosítók) vannak rendelve.\par A következőkben olyan gráfokat fogunk
	érteni alatta, ahol a csúcsokhoz rendeltünk egyedi azonosítót. 
\end{definition}

\begin{definition}[Súlyozott gráf]{def:sulyozott_graf} Egy gráfot súlyozott gráfnak nevezünk, ha minden
	éléhez hozzá van rendelve egy valós szám. 
\end{definition}

\begin{definition}[Minimális feszítőfa]{def:minimalis_feszitofa} A gráf feszítőfáját minimális feszítőfának
	nevezzük nevezzük, ha az éleihez rendelt súlyok összege minimális. 
\end{definition}

\begin{problem}
	Alkalmazott feladat: Egy ország n városát úttal szeretnénk összekötni
	úgy, hogy minden városból minden másik elérhető legyen. A városok
	közti út elkészítésének költségeit ismerjük. A kormány hogyan találhatja
	meg a minimális költségű úthálózatot?
\end{problem}

Megjegyzés: úthálózat például jelenthet még áram-, telefon-, TV-,
víz- vagy internetösszeköttetést is.
\begin{solution}
	A város és úthálózat alkotta gráfban a minimális feszítőfa megtalálása
	a cél. A feladat optimálisan megoldható például Kruskal- vagy Prim-algoritmussal. 
\end{solution}
\begin{definition}[Bináris fa]{def:binaris_fa} A bináris fa egy olyan fa, melyben minden csomópontnak
	legtöbb két utóda van. 
\end{definition}

\begin{definition}[Szabályos bináris fa]{def:szabalyos_bin_fa} Egy bináris fát szabályos bináris fának nevezünk,
	ha minden csomópontnak vagy nincs gyereke, vagy pontosan két gyereke
	van. 
\end{definition}

Adatszerkezetek témakörében a (bináris) fa nagyon gyakran használatos,
íme néhány példa: 
\begin{itemize}
	\item keresőfák (teljes rendezés -- bármely elem gyors keresése) 
	\item min-max kupacok (legkisebb vagy legnagyobb elem gyors kiválasztása) 
	\item kifejezésfák (pl. aritmetikai kifejezések tárolása és kiértékelése) 
	\item mélységi és szélességi bejárás (pl. gráfban feszítőfa generálására
	alkalmas, inorder bejárás pedig teljesen rendezett sorozatot ad meg
	egy keresőfában) 
\end{itemize}
\begin{center}
	\begin{minipage}[c]{0.3\textwidth}%
		\centering \includegraphics[width=1\textwidth]{\string"content/Szelyes_Klaudia-Domokos_Abel/1.kereso\string".png}
		
		{\small\textbf{Keresőfa}} % manual caption text
		%
	\end{minipage}\hspace{0.05\textwidth} %
	\begin{minipage}[c]{0.3\textwidth}%
		\centering \includegraphics[width=1\textwidth]{\string"content/Szelyes_Klaudia-Domokos_Abel/1.minkupac\string".png}
		
		{\small\textbf{Min kupac}} %
	\end{minipage}
	\par\end{center}

\begin{center}
	\vspace{1em}
	\par\end{center}

\begin{center}
	% Second row: one image centered with manual caption
	\begin{minipage}[c]{0.35\textwidth}%
		\centering \includegraphics[width=1\textwidth]{\string"content/Szelyes_Klaudia-Domokos_Abel/1.kifejezesfak\string".png}
		
		{\small\textbf{4 + 2 {*} 10 + 3 {*} (5 + 1) kifejezésfája}} %
	\end{minipage}
	\par\end{center}

Egyéb fastruktúrákra épülő alkalmazási területek: 
\begin{itemize}
	\item számítógépes fájlrendszer tárolása 
	\item adatelemző módszerek: döntési fák, véletlen erdők 
	\item R-fa (regtangle tree): térbeli adatok hatékony lekérdezésére 
	\item Érdekesség: A szabályos bináris fák részfáinak száma megadható a Catalan-számokkal. 
	\item fraktálok (szépen visszaadja egy igazi fa szerkezetét) 
\end{itemize}
\begin{center}
	\includegraphics[width=0.4\textwidth]{\string"content/Szelyes_Klaudia-Domokos_Abel/tree\string".png} 
	\par\end{center}
\begin{theorem}[Fák jellemzési tételei]{thm:fak_jellemzese}
	
	Egy $n$ pontú $G$ gráfra az alábbi kijelentések ekvivalensek: 
	\begin{enumerate}
		\item $G$ fa (azaz összefüggő és körmentes). 
		\item $G$ körmentes, és $n-1$ éle van. 
		\item $G$ összefüggő, és $n-1$ éle van. 
		\item $G$ összefüggő, és bármely élt eltávolítva $G$-ből a maradék gráf
		már nem összefüggő. 
		\item $G$ bármely két csúcsa között pontosan egy út vezet. 
		\item $G$ körmentes, de egy tetszőleges éllel bővítve az $E(G)$ élhalmazt,
		már keletkezik kör. 
	\end{enumerate}
\end{theorem}

Megjegyzés: Az összefüggőség és a körmentesség tulajdonságai bizonyos
értelemben ellentétesek egymással. A $4.$ és $6.$ tételek alapján
a fákat akár ,,minimálisan összefüggő" vagy ,,maximálisan kör nélküli"
gráfoknak is nevezhetnénk.\\
\\

Ahogy az igazi fákon is újabb-újabb ágak nőnek, ugyanúgy gráf-fákat
is növeszthetünk. 
\begin{theorem}{thm:fa_novesztese}
	Az alábbi eljárással növesztett gráfok mindegyike fa és minden fa
	megkapható ilyen módon: 
	\begin{itemize}
		\item Kezdjük egyetlen pontból álló G gráffal 
		\item Ismételjük meg a következő lépést: a meglevő G gráf mellett vegyünk
		fel egy új pontot és kössük össze G egy tetszőleges pontjával.\\
		
	\end{itemize}
\end{theorem}
\begin{extraproblem}
\textbf{Fő feladat}: számoljuk össze az összes lehetséges n csúcsból
álló fát! 
\end{extraproblem}
\begin{solution}
	1. lépés: elkezdjük kicsi n értékekre megvizsgálni a feladatot. Hamar
	feltehetjük azt a kérdést, hogy két fa megegyezhet vagy sem? Matematikailag
	fogalmazva, a két fa mikor izomorf egymással?
	
	Ere a kérdésre adott válasz attól függ, hogy a csúcsok egyediek-e,
	vagyis címkézettek vagy sem.
	
	Egy másik tényező, ami befolyásolja két fa különbözőségét, hogy számít-e
	az egymás mellett lévő élek elhelyezkedésének sorrendje? A következőkben
	nem szeretnénk az élek sorrendjét figyelembe venni, ezért a következőképpen
	definiáljuk két fa megegyezését: 

\begin{itemize}
	\item \textbf{Két címkézett fa akkor egyezik meg}, ha bennük ugyanazok a
	pontok vannak éllel összekötve. 
	\item \textbf{Két címkézetlen fa akkor egyezik meg}, ha megadható olyan
	bijektív megfeleltetés a két fa címkézett pontjai között, hogy az
	egyik fa bármelyik két pontja akkor van éllel összekötve, ha a másik
	fában a hozzá rendelt pontok is össze vannak kötve, és vice versa. 
\end{itemize}
A címkézett fákból egyértelműen több van, a címkézetlenekből pedig
kevesebb. Tehát két részre oszlik a feladat: 
\begin{itemize}
	\item[a)] címkézett fák száma 
	\item[b)] címkézetlen fák száma 
\end{itemize}
\textbf{Spoiler}: A címkézett fák számát lehet pontosan kiszámolni,
a címkézetlenek számára eddig csak becslés létezik.

A következőkben csak a címkézett fákkal fogunk foglalkozni.\\

\textbf{Címkézett fák megszámlálása}

\textbf{Ötlet}: Próbáljunk minden fának megfeleltetni egy kódot, amiből
visszafejthető a fa! Másképp fogalmazva keressünk olyan módszereket,
ahogy a memóriában tudnánk tárolni a fákat, minél kevesebb tárhellyel!
Vajon tudunk találni egy bijektív megfeleltetést?

Az n csúcsot címkézzük 0-tól $n-1$-ig. A következő fát hogyan tárolnánk
a számítógépben?
\begin{center}
	\includegraphics[width=0.6\textwidth]{\string"content/Szelyes_Klaudia-Domokos_Abel/tarolando\string".png} 
	\par\end{center}

Tekintsünk át különböző tárolási módszereket.\\

\textbf{1. Szomszédossági mátrix} 
\begin{center}
	\includegraphics[width=0.5\textwidth]{\string"content/Szelyes_Klaudia-Domokos_Abel/szomsz\string".png} 
	\par\end{center}

\textbf{Megj}: a 0 címkéjú csomópontot az utolsó sornak és utolsó
oszlopnak vettük.

\textbf{Elemzés}: n csúcs esetén az $n^{2}$ elemű táblázat alsó háromszögmátrixa
elég a tárolásra, a főátlót sem kell figyelembe venni, mert az mindig
nulla. Tehát összesen $(n^{2}-n)/2$ darab, 0 vagy 1 értékű bit elégséges
a tárolásra. Ez összesen $2^{(n^{2}-n)/2}$ különböző kódot jelent.

\textbf{A leképezés nem lesz bijektív}: minden fa ábrázolható így,
de nem minden háromszögmátrix fa.\\

\textbf{2. Élek felsorolása} 
\begin{center}
	\includegraphics[width=0.5\textwidth]{\string"content/Szelyes_Klaudia-Domokos_Abel/elekfels\string".png} 
	\par\end{center}

\textbf{Elemzés}: A táblázatnak már csak két sora van. Azonban a számok
mostmár nem csak 0 és 1 értékeket, hanem 0-tól $n-1$-ig bármilyen
értéket felvehetnek. Ekkor Egy számot $\log_{2}n$ bit segítségével
ábrázolhatunk, ezért most $2n\log_{2}n$ bitre van szükség az egész
kódhoz. Ez sokkal kevesebb tárhely a szomszédossági mátrixhoz képest
nagy n-ek esetén. A kódok száma kevesebb, mint $(n)^{2n}$, ez a szám
pedig kisebb az előző módszer kódjainak számától nagy n-ek esetén.

\textbf{A leképezés nem lesz bijektív}: minden fa ábrázolható ilyen
kóddal, de nem minden kétsoros kód reprezentál fát.\\

\textbf{3. Apakód} A 0 címkéjű pont a továbbiakban kitüntetett jelentőséggel
bír: ezt tekintjük majd a fa gyökerének. Egy élt ezek után fiú - apa
sorrendben adunk meg. Az éleket pedig rendezzük az első végpontjuk
szerint növekvő sorrendbe.
\begin{center}
	\includegraphics[width=0.5\textwidth]{\string"content/Szelyes_Klaudia-Domokos_Abel/apakod\string".png} 
	\par\end{center}

\textbf{Elemzés}: A 0 címkéjű csúcsnak nincs apja, tehát neki nem
kell oszlop, minden más csúcsnak igen, tehát $n-1$ oszlop van. Az
első oszlop értékei 1-től $n-1$-ig tart, ezért ez igazából nem tartalmaz
lényeges információt, ezt elhagyhatjuk. Maradunk n-1 számmal, amely
$(n-1)log_{2}n$ bittel ábrázolható és $n^{n-1}$ féle kód létezik.
Úgy látszik mégjobb eredményhez jutottunk. Bijektív lesz a leképezés?

\textbf{A leképezés nem lesz bijektív}: Ellenpélda 4 csúcsú mátrix
esetén: (3 2 1). De már elég közel állunk a megoldáshoz.\\

\textbf{4. Prüfer-kód} A Prüfer-kód az apakód kifinomultabb változatának
is tekinthetjük.\\

Íme az új szabály: megkeressük a 0-tól különböző 1 fokszámú pontok
közül a legkisebb címkéjűt, beírjuk az első sorba (a megtalálási sorrendben)
és alá rögzítjük az apját. Ezután töröljük a szóban forgó pontot,
és az eljárást megismételjük a megmaradt fával. Az eljárást addig
folytatjuk, amíg minden csúcs sorra nem kerül.
\begin{center}
	\includegraphics[width=0.5\textwidth]{\string"content/Szelyes_Klaudia-Domokos_Abel/prufer\string".png} 
	\par\end{center}

Az így kapott táblázatot a fa \emph{bővített Prüfer-kódjának} nevezzük.
(Azért nevezzük „bővítettnek”, mert a „valódi” Prüfer-kód ennek a
táblázatnak csak egy részéből áll.)
\begin{theorem}{thm:Prufer-masodik}
	A bővitett Prüfer-kód második sora egyértelműen meghatározza az elsőt. 
\end{theorem}

\begin{proof}
	Alapgondolat: a bővített Prüfer-kód első sorának minden eleme a legkisebb
	egész szám, amely nem szerepel korábban az első sorban, vagy később
	a második sorban... 
\end{proof}
\textbf{Példák}: Határozd meg a következő Prüfer-kódok bővített Prüfer-kódját:\\
(3, 2, 2), (1, 1, 0, 5). A bővített kódok a következők lesznek: 
\[
\begin{bmatrix}1 & 3 & 4 & 2\\
	3 & 2 & 2 & 0
\end{bmatrix},\quad\begin{bmatrix}2 & 3 & 1 & 4 & 5\\
	1 & 1 & 0 & 5 & 0
\end{bmatrix}
\]

A bővített Prüfer-kód első sorát és a második sor utolsó elemét (ami
mindig 0) elhagyva kapjuk meg az $n-2$ számból álló \textbf{Prüfer-kódot}.

A kiindulási fához rendelt Prüfer-kód a következő lesz: 
\begin{center}
	(6, 0, 2, 6, 2, 9, 9, 2) 
	\par\end{center}

Tehát bármelyik fához hozzárendelhetünk egy Prüfer-kódot.

A visszafele irányra vagyunk már csak kíváncsiak. A kérdés: akármilyen
kód (n-2 tagú sorozat, amelynek tagjai 0 és n-1 közötti számok) egy
n pontú fa Prüfer-kódja? Mind az $n^{n-2}$ lehetséges kódból visszafejthető
egy fa?

Két részre osszuk most a feladatot:
\begin{itemize}
	\item[a)] tetszőleges kód kibővíthető a bővített-Prüfer kód mintájára? 
	\item[b)] Az így kapott kibővített kód valóban fa lesz? 
\end{itemize}
\begin{proof}
	Legyen egy tetszőleges n-2 elemű sor.\\
	a) Előbb a sort kiegészítjük 0-val, majd minden tag fölé a legkisebb
	olyan pozitív számot írjuk, amely sem az első sorban a szóban forgó
	tag előtt, sem a második sorban utána nem szerepel (jegyezzük meg,
	hogy ilyen egész szám mindig létezik: a feltétel $n$-ből legfeljebb
	$n-1$ értéket zárhat ki). Ekkor létrehoztuk a kibővített, kétsoros
	táblázatot.
	
	b) Minden csúcsnak van apja a 0-t kivéve, ezért a gráfnak n-1 éle
	van.
	
	Tegyük fel, hogy a kapott táblázatból szerkesztett gráfban van kör.
	Hogyha élek segítségével meghatározzuk a kört, akkor minden csúcs
	kétszer kell szerepeljen a éllistán. Válasszuk ki a kör éleit, mindegyik
	oszlop egy élt jelent. A kiválasztott legbaloldalibb oszlop felső
	elemére teljesül, hogy nem található meg saját sorában jobbra és az
	alatta levő sorban sem tőle jobbra. Tehát ez a csúcs nem szerepel
	kétszer az éllistán, de így ellentmondásba kerültünk. Tehát a gráf
	körmentes.
	
	Mivel a gráf körmentes és n-1 éle van, ezért tétel szerint egy fát
	kaptunk. 
\end{proof}
Összessítve tehát minden címkézett fának van Prüfer-kódja és bármilyen
n-2 tagból álló sorozathozz hozzárendelhetünk egy fát. Ezáltal a címkézett
fák száma megegyezik a lehetséges Prüfer-kódok számával, aminek száma
$n^{n-2}$.
\end{solution}
Erre az eredményre Cayley-tétele címmel szoktak hivatkozni. 
\begin{theorem}{thm:cimkezett_fak_szama}
	Az n pontú címkézett fák száma $n^{n-2}$. 
\end{theorem}


\section*{Házi feladatok}
\begin{problem}
	Keressük meg az összes 2, 3, 4 és 5 pontú címkézetlen fát. Hány címkézett
	fát készíthetünk ezekből? Határozzuk meg ennek alapján a 2, 3, 4 és
	5 pontú címkézett fák számát! 
\end{problem}

\begin{solution}
	\begin{center}
		\includegraphics[width=0.9\textwidth]{\string"content/Szelyes_Klaudia-Domokos_Abel/1\string".pdf} 
		\par\end{center}
\end{solution}
\begin{problem}
	A csilag egy olyan fagráf, aminek a gyökeréhez kapcsolódik minden
	elem. Hány $n$ pontú címkézett csillag létezik? 
\end{problem}

\begin{solution}
	A gyökér címkézése számít egyedül, a többi csúcs mind egyenrangú helyzetűek.
	Tehát n címke közül egyet kell kiválasztani a gyökérnek, a maradék
	címkék pedig a fiaknak jutnak. Tehát n lehetséges címkézés van. 
\end{solution}
\begin{problem}
	A következő apakódok közül melyik kódol fát? (0, 1, 3, 4, 5, 6, 7);
	(7, 6, 5, 4, 3, 2, 1, 0); (0, 0, 0, 0, 0, 0, 0, 0); (2, 3, 1, 2, 3,
	1, 2,3) 
\end{problem}

\begin{solution}
	Egészítsük ki a kódokat: 
	\[
	\begin{bmatrix}1 & 2 & 3 & 4 & 5 & 6 & 7\\
		0 & 1 & 3 & 4 & 5 & 6 & 7
	\end{bmatrix},\quad\begin{bmatrix}1 & 2 & 3 & 4 & 5 & 6 & 7 & 8\\
		7 & 6 & 5 & 4 & 3 & 2 & 1 & 0
	\end{bmatrix}
	\]
	\[
	\begin{bmatrix}1 & 2 & 3 & 4 & 5 & 6 & 7 & 8\\
		0 & 0 & 0 & 0 & 0 & 0 & 0 & 0
	\end{bmatrix},\quad\begin{bmatrix}1 & 2 & 3 & 4 & 5 & 6 & 7 & 8\\
		2 & 3 & 1 & 2 & 3 & 1 & 2 & 3
	\end{bmatrix}
	\]
	\\
	
	Az első két apakód nem helyes, mert mind a kettőben a 4 önmagának
	apja kellene legyen.
	
	A 3. apakód helyes és fát ábrázol.
	
	A 4. kód nem lehet fa a következő ellentmondás miatt: 1-nek 2, 2-nek
	3, és 3-nak 1 az apja, de 1-nek nem lehetne önmaga az őse. 
\end{solution}
\begin{problem}
	Legyen egy n csúcsú gráf, amelynek éleit egy $2n-2$ számból álló
	éllistában ábrázoltunk. Igazold, hogy ha ez a gráf összefüggő, akkor
	fa! 
\end{problem}

\begin{solution}
	$2\cdot(n-1)$ elemből áll az éllista, ez azt jelenti, hogy a gráfnak
	n-1 éle van. Ekkor azonban, ha az n csúcsú gráf összefüggő és n-1
	éle van, akkor a fák jellemzési tételei alapján ez a gráf fa. 
\end{solution}
\begin{problem}
	A következő Prüfer-kódokat egészítsd ki és rajzold meg a hozzá tartozó
	fát: (0, 0, 0, 0), (4, 4, 4), (0, 1, 2, 3, 4), (4, 3, 2, 1). 
\end{problem}

\begin{solution}
	\[
	\begin{bmatrix}1 & 2 & 3 & 4 & 5\\
		0 & 0 & 0 & 0 & 0
	\end{bmatrix},\quad\begin{bmatrix}1 & 2 & 3 & 4\\
		4 & 4 & 4 & 0
	\end{bmatrix},\quad\begin{bmatrix}5 & 6 & 1 & 2 & 3 & 4\\
		0 & 1 & 2 & 3 & 4 & 0
	\end{bmatrix},\quad\begin{bmatrix}5 & 4 & 3 & 2 & 1\\
		4 & 3 & 2 & 1 & 0
	\end{bmatrix}
	\]
	\\
	\begin{center}
		\includegraphics[width=0.8\textwidth]{\string"content/Szelyes_Klaudia-Domokos_Abel/5\string".jpg} 
		\par\end{center}
\end{solution}

\section*{Nehezebb feladatok}
\begin{extraproblem}[Kiss Andrea-Tímea]
	Egy hegyvidéki településen 8 falu között új úthálózatot szeretnének
	kiépíteni úgy, hogy:
	\begin{itemize}
		\item minden falu elérhető legyen a többi faluból (azaz a hálózat összefüggő
		legyen), 
		\item a lehető legkevesebb útköltséggel építsék ki a kapcsolatokat, 
		\item nem kell minden lehetséges út megépüljön, csak annyi, amennyi minimálisan
		szükséges az összefüggéshez. 
	\end{itemize}
	A falvak közötti lehetséges utak és azok építési költségei (kilométerben): 
	\begin{center}
		\begin{tabular}{|c|c|}
			\hline 
			\textbf{Út} & \textbf{Hossz (km)}\tabularnewline
			\hline 
			A--B & 3\tabularnewline
			A--C & 5\tabularnewline
			A--D & 4\tabularnewline
			B--C & 2\tabularnewline
			B--E & 7\tabularnewline
			C--D & 1\tabularnewline
			C--F & 6\tabularnewline
			D--F & 2\tabularnewline
			E--F & 3\tabularnewline
			E--G & 4\tabularnewline
			F--G & 2\tabularnewline
			G--H & 1\tabularnewline
			F--H & 3\tabularnewline
			\hline 
		\end{tabular}
		\par\end{center}
	Határozd meg a lehető legalacsonyabb összköltségű úthálózatot úgy,
	hogy minden falu elérhető legyen minden másiktól!
\end{extraproblem}

\begin{solution}
	A feladat egy \emph{minimális feszítőfa} (Minimum Spanning Tree, MST)
	keresését jelenti egy súlyozott gráfban. Ehhez a Kruskal-algoritmust
	használjuk: az éleket súly szerint növekvő sorrendbe rendezzük, majd
	mindig a legkisebb súlyút választjuk, ami nem hoz létre kört, amíg
	$n-1=7$ élt nem választottunk ki (mivel $n=8$ csúcs).
	
	\textbf{Súly szerint rendezett élek:}
	
	\begin{center}
		\begin{tabular}{|c|c|}
		\hline 
		\textbf{Él} & \textbf{Hossz}\tabularnewline
		\hline 
		C--D & 1\tabularnewline
		G--H & 1\tabularnewline
		D--F & 2\tabularnewline
		F--G & 2\tabularnewline
		B--C & 2\tabularnewline
		A--B & 3\tabularnewline
		E--F & 3\tabularnewline
		F--H & 3\tabularnewline
		A--D & 4\tabularnewline
		E--G & 4\tabularnewline
		A--C & 5\tabularnewline
		C--F & 6\tabularnewline
		B--E & 7\tabularnewline
		\hline 
	\end{tabular}
	\end{center}
	
	\vspace{0.5em}
	\textbf{Kiválasztott élek a Kruskal-algoritmus szerint:}
	\begin{itemize}
		\item C--D (1) 
		\item G--H (1) 
		\item D--F (2) 
		\item F--G (2) 
		\item B--C (2) 
		\item A--B (3) 
		\item E--F (3) 
	\end{itemize}
	Ezek összesen 7 él, lefedik mind a 8 falut, és nem tartalmaznak kört.
	
	\vspace{0.5em}
	\textbf{Összköltség:} 
	\[
	1+1+2+2+2+3+3=\boxed{14\text{ km}}
	\]
	
	\vspace{0.5em}
	\textbf{Válasz:} A minimális úthálózat hossza \textbf{14 km}, az
	utak: 
	\[
	\text{C--D, G--H, D--F, F--G, B--C, A--B, E--F}.
	\]
\end{solution}
\begin{extraproblem}[Kovács Levente]
	Legyen adott egy 10 csúcsból álló címkézett fa. Hány különböző módon
	választható ki 3 él úgy, hogy azok együtt egy fát alkossanak (azaz
	összefüggők legyenek és ne tartalmazzanak ciklust)? 
\end{extraproblem}

\begin{solution}
	\noindent Egy 3 élből álló fa 4 csúcsot tartalmaz (hiszen egy fa mindig
	$n-1$ élből áll, ha $n$ csúcsból áll). Tehát a kérdés: hány darab
	olyan 4 csúcsú részfát tartalmaz egy 10 csúcsú fa, amely pontosan
	3 élből áll?
	
	Nem tudjuk, hogyan néz ki a fa szerkezete, de a teljes gráf címkézett,
	és a kérdés a részfákra vonatkozik. Ezért az összes lehetséges 4 csúcsú
	részgráf közül kell kiválasztani azokat, amelyek önmagukban is fák
	és pontosan 3 élük van.
	
	A 10 csúcsból $\binom{10}{4}=210$ módon választhatunk ki 4 csúcsot.
	
	Egy 4 csúcsú gráfnak akkor lesz 3 éle és ciklusmentes szerkezete,
	ha az önmagában fa --- tehát összefüggő.
	
	Nem minden 3 élű 4 csúcsú részgráf lesz összefüggő (pl. 2 komponens).
	Az összes 4 csúcsú fa szerkezete azonban véges: - "lánc" típusú:
	$v_{1}-v_{2}-v_{3}-v_{4}$ - "Y" típusú: egy csúcsból 3 él indul
	- stb.
	
	Általánosan nem számolható ki pontosan, hány ilyen van, ha nem ismerjük
	a gráf pontos szerkezetét. Viszont ha az összes lehetséges {*}{*}4
	csúcsú részfa{*}{*} számát kérdezzük, akkor a válasz az, hogy: 
	\begin{center}
		\boxed{\text{A pontos válasz függ a fa szerkezetétől.}}
		\par\end{center}
	\begin{center}
		\boxed{\text{Nem határozható meg egyértelműen csupán a csúcsszámból.}} 
		\par\end{center}
\end{solution}
\begin{extraproblem}[Kovács Levente]
	Igazoljuk, hogy ha egy fa minden csúcsának fokszáma legfeljebb 3,
	akkor a levelek száma legalább $\frac{n+2}{2}$, ahol $n$ a csúcsok
	száma. 
\end{extraproblem}

\begin{solution}
	\noindent Tegyük fel, hogy a fa $n$ csúcsból áll, és minden csúcs
	fokszáma legfeljebb 3.
	
	Jelölje $L$ a levelek számát (azaz azokét a csúcsokét, ahol $\deg(v)=1$).
	A többi csúcs fokszáma 2 vagy 3 lehet. Legyen: 
	\[
	n_{1}=L,\quad n_{2}=\text{2 fokú csúcsok száma},\quad n_{3}=\text{3 fokú csúcsok száma}.
	\]
	
	A csúcsok száma: 
	\[
	n=n_{1}+n_{2}+n_{3}.
	\]
	
	A fokszámösszeg: 
	\[
	\sum\deg(v)=1\cdot n_{1}+2\cdot n_{2}+3\cdot n_{3}=2(n-1),
	\]
	mivel a fa $n-1$ élt tartalmaz.
	
	Ezért: 
	\[
	n_{1}+2n_{2}+3n_{3}=2(n_{1}+n_{2}+n_{3}-1)
	\]
	
	Fejtsük ki a jobb oldalt: 
	\[
	n_{1}+2n_{2}+3n_{3}=2n_{1}+2n_{2}+2n_{3}-2
	\]
	
	Vonjuk ki mindkét oldalból a bal oldalt: 
	\[
	0=(2n_{1}-n_{1})+(2n_{2}-2n_{2})+(2n_{3}-3n_{3})-2=n_{1}-n_{3}-2\Rightarrow n_{1}=n_{3}+2
	\]
	
	Tehát: 
	\[
	L=n_{1}=n_{3}+2
	\]
	
	Most vegyük észre, hogy: 
	\[
	n=n_{1}+n_{2}+n_{3}=(n_{3}+2)+n_{2}+n_{3}=2n_{3}+n_{2}+2\Rightarrow n-2=2n_{3}+n_{2}
	\]
	\[
	\Rightarrow2(n-2)\ge4n_{3}\Rightarrow n-2\ge2n_{3}\Rightarrow n_{3}\le\frac{n-2}{2}
	\]
	
	Ez alapján: 
	\[
	L=n_{3}+2\ge\frac{n-2}{2}+2=\frac{n+2}{2}
	\]
	
	\textbf{Q.E.D.} A levélcsúcsok száma legalább $\boxed{\frac{n+2}{2}}$. 
\end{solution}
\begin{extraproblem}[Sógor Bence]
	Igazoljuk a következőket: 
	\begin{enumerate}
		\item Mutasd meg, hogy véges összefüggő egyszerű gráfban bármely két leghosszabb
		útnak van közös csúcsa. 
		\item Mutasd meg, hogy egy fagráfban az összes leghosszabb út egy csúcson
		megy át. 
	\end{enumerate}
\end{extraproblem}

\begin{solution}
	~
	\begin{enumerate}
		\item Feltételezzük, hogy van két olyan leghosszabb út, amelyiknek nincsen
		közös csúcsa. Legyenek ezek $l$ hosszúságú $a:A_{1}A_{2}\dots A_{l+1}$
		és $b:B_{1}B_{2}\dots B_{l+1}$ utak. Mivel a gráf összefüggő, ezért
		léteznek $A_{i}$ és $B_{j}$ pontok úgy, hogy a köztük lévő út mellőzi
		az $a$ és $b$ utak csúcsait (ha lenne köztes $A_{k}$ vagy $B_{k}$
		csúcs akkor csak az eredeti cseréljük le ezekre). Vegyük észre, hogy
		az $a$ mentén az $A_{1}-A_{i}$ és $A_{l+1}-A_{i}$ utak közül valamelyik
		legalább $\frac{l}{2}$ hosszú. Hasonlóan, $b$ mentén a $B_{1}-B_{j}$
		és $B_{l+1}-B_{j}$ utak közül valamelyik legalább $\frac{l}{2}$
		hosszú. Az általánosság megsértése nélkül feltehetjük, hogy az $A_{1}-A_{i}$
		és $B_{1}-B_{j}$ utak legalább $\frac{l}{2}$ hosszúak (különben
		átindexelünk). Ekkor viszont az $A_{1}-A_{i}-B_{J}-B_{1}$ t legalább
		$\frac{l}{2}+\frac{l}{2}+1=l+1$ hosszú, ami ellentmond annak, hogy
		$l$ hosszúságú a leghosszabb út. Tehát egy véges összefüggő egyszerű
		gráfban két leghosszabb útnak van közös csúcsa. 
		\item Feltételezzük, hogy nem egy csúcson mennek át a leghosszabb utak.
		Az előző alpont szerint minden leghosszabb útnak van közös csúcsa.
		Ekkor létezik három $l$ hosszúságú út $a$, $b$, $c$ úgy, hogy
		például az $a$ és $b$ utak közös csúcsa $C$, $b$ és $c$ közös
		csúcsa $A$, és $c$ és $a$ közös csúcsa $B$. Az egyszerűség kedvéért
		tegyük fel, hogy amikor $C$ csúcsból az $a$ út mentén megyünk a
		$B$ felé, akkor a $B$ az első csúcs, ami közös $a$-ban és $c$-ben.
		Ezt megtehetjük minden út mentén. Ekkor az $A$, $B$, $C$ csúcsok
		a $c$, $a$ és $b$ utak mentén egy kört határoznának meg, ami ellentmondana
		annak, hogy egy fagráfon vagyunk. Tehát, egy fagráfban az összes leghosszabb
		út egy csúcson megy át. 
	\end{enumerate}
\end{solution}
\begin{extraproblem}[Száfta Antal]
	Legyen $n\geq3$ egész szám. Egy $n$-csúcsú címkézett fa minden
	élének súlya a két végpont címkéinek összege. (Például: az $\{2,5\}$
	él súlya $7$.)
	
	Hány olyan címkézett fa van, amely:
	\begin{itemize}
		\item a teljes súlya \textbf{minimális}, és 
		\item a fa Prüfer-kódjában \textbf{csak a $\{1,2,3\}$} címkék szerepelnek? 
	\end{itemize}
\end{extraproblem}

\begin{solution}
	~
	
	\paragraph{1. Prüfer-kód és levélcsúcsok}
	
	Egy $n$-csúcsú fa Prüfer-kódja egy $n-2$ hosszú sorozat $\{1,\dots,n\}$
	elemeiből.
	
	A kódban szereplő címkék azok a csúcsok, amelyek \textbf{nem levelek},
	tehát ha a kódban csak $\{1,2,3\}$ szerepel, akkor: 
	\[
	\{4,5,\dots,n\}\quad\text{levél lesz.}
	\]
	
	Ez pontosan $n-3$ darab levél.
	
	\paragraph{2. A teljes súly minimalizálása}
	
	Minden él súlya: $\text{összeg a végpontok címkéiből}$. A teljes
	súly akkor a legkisebb, ha a nagy címkék (pl. $\geq4$) csak levelek,
	és őket kis címkéjű csúcsokhoz kapcsoljuk.
	
	Ez pontosan akkor teljesül, ha a belső csúcsok $1,2,3$, tehát a Prüfer-kódban
	\emph{csak} ezek szerepelnek.
	
	\paragraph{3. Kódok száma}
	
	A kód hossza: $n-2$, minden pozícióba 3 lehetőség ($1,2,3$):
	
	\[
	\text{Kódok száma}=3^{n-2}
	\]
	
	\paragraph{4. Végső válasz}
	
	Az összes olyan fa száma, amely megfelel a feltételeknek:
	
	\[
	\boxed{3^{n-2}}
	\]
\end{solution}
\begin{extraproblem}[Szélyes Klaudia]
	Adott az alábbi élhalmaz, amely egy címkézett fát reprezentál 7 csúccsal:
	\[
	(1,4),\ (4,5),\ (4,2),\ (2,0),\ (2,3),\ (3,6)
	\]
	\begin{enumerate}
		\item[a)] Rajzold meg a fát! 
		\item[b)] Írd fel a fa \textbf{Prüfer-kódját}! 
		\item[c)] Hány levél (fokszám 1-es csúcs) van ebben a fában? 
	\end{enumerate}
\end{extraproblem}
\begin{solution}
	\begin{itemize}
		\item \textbf{a)} A fa gráfjának vizuális ábrája:
		\begin{center}
			\begin{tikzpicture}[every node/.style=circle,draw, node distance=1.5cm]
				\node (0) {0};
				\node (2) [above right of=0] {2};
				\node (4) [above left of=2] {4};
				\node (1) [left of=4] {1};
				\node (5) [above of=4] {5};
				\node (3) [right of=2] {3};
				\node (6) [right of=3] {6};
				
				\draw (0) -- (2);
				\draw (2) -- (4);
				\draw (4) -- (1);
				\draw (4) -- (5);
				\draw (2) -- (3);
				\draw (3) -- (6);
			\end{tikzpicture} 
			\par\end{center}
		\item \textbf{b)} Prüfer-kód meghatározása:
		
		A lépések: 
		\begin{itemize}
			\item Levél: 0 → (0, 2) \ensuremath{\Rightarrow} kód: 2 
			\item Levél: 1 → (1, 4) \ensuremath{\Rightarrow} kód: 4 
			\item Levél: 5 → (5, 4) \ensuremath{\Rightarrow} kód: 4 
			\item Levél: 4 → (4, 2) \ensuremath{\Rightarrow} kód: 2 
			\item Levél: 6 → (6, 3) \ensuremath{\Rightarrow} kód: 3 
			\item Levél: 3 → (3, 2) \ensuremath{\Rightarrow} kód: 2 
		\end{itemize}
		Tehát a Prüfer-kód: 
		\[
		(2,4,4,2,3,2)
		\]
		
		\item \textbf{c)} A levelek a következők: 0, 1, 5, 6 \ensuremath{\Rightarrow}
		összesen \textbf{4 darab}. 
	\end{itemize}
\end{solution}
\begin{extraproblem}[Szélyes Klaudia]
	Legyen adott a következő Prüfer-kód: 
	\[
	(3,3,5,5,5,3)
	\]
	\begin{enumerate}
		\item[a)] Hány csúcsú fa tartozik ehhez a kódhoz? 
		\item[b)] Építsd meg a fát és sorold fel az éleit! 
		\item[c)] Mely csúcs(ok) a levelek? 
		\item[d)] Mely csúcs(ok) fokszáma legalább 4? 
	\end{enumerate}
\end{extraproblem}
\begin{solution}
	\begin{itemize}
		\item \textbf{a)} A Prüfer-kód hossza 6 → a fa csúcsainak száma: 
		\[
		n=6+2=8
		\]
		Azaz a fa a \{0, 1, 2, 3, 4, 5, 6, 7\} címkéjű csúcsokból áll.
		\item \textbf{b)} Felépítés:
		
		Kezdő kód: (3, 3, 5, 5, 5, 3)
		
		Gyakoriság: 
		\begin{itemize}
			\item 3: 3-szor 
			\item 5: 3-szor 
		\end{itemize}
		Kezdeti fokszám: mindenki 1, kivéve 3 és 5 (4-es fokszám lesz)
		
		Lépések: 
		\begin{enumerate}
			\item Legkisebb levél: 0 → (3, 0) 
			\item Legkisebb új levél: 1 → (3, 1) 
			\item 2 → (5, 2) 
			\item 4 → (5, 4) 
			\item 6 → (5, 6) 
			\item 7 → (3, 7) 
			\item Marad: 3, 5 → (3, 5) 
		\end{enumerate}
		\textbf{Élek}: 
		\[
		(3,0),(3,1),(5,2),(5,4),(5,6),(3,7),(3,5)
		\]
		
		\item \textbf{c)} Levelek: 0, 1, 2, 4, 6, 7 \ensuremath{\Rightarrow} \textbf{6
			darab} levél
		\item \textbf{d)} Fokszám szerint: 
		\begin{itemize}
			\item 3 szerepel 3-szor a kódban + 1 \ensuremath{\Rightarrow} fokszám: 4 
			\item 5 szerepel 3-szor + 1 \ensuremath{\Rightarrow} fokszám: 4 
		\end{itemize}
		Tehát a következő csúcsok fokszáma $\geq4$: \textbf{3 és 5} 
	\end{itemize}
\end{solution}