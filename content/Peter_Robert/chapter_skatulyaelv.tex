\chapter{A skatulyaelv}\label{chap:skatulya}
\begin{description}
	{\large 
	\item [{Szerző:}] Péter Róbert (Didaktikai mesteri -- Matematika, I. év)
}
\end{description}
\textbf{Skatulyaelv általánosan: }Ha $m$ testet szétosztunk $n$ dobozba
és $m>n$, akkor legalább két test azonos dobozba fog kerülni. 
%Könyvészet: \cite{diszkretmatek,elte}

\section*{Házi feladatok}
\begin{problem}
	Egy vonaton 400 ember utazik. Bizonyítsuk be, hogy utazik rajta két
	olyan utas, akiknek ugyanazon a napon van a születésnapja! Mi történik
	akkor, ha még felszáll 333 ember a vonatra? \\
	
\end{problem}

\begin{solution}
	Egy évben 365 nap van, ezért 365 skatulyát kell használnunk. De mivel
	szökőév is van ezért legyen 366 skatulyánk. Sorra vesszük a születési
	dátumokat és helyezzük be őket a skatulyákba, míg el nem érünk a 367.
	emberig. Ebben az esetben mind a 366 skatulya megtelt, így a 367.
	ember valamelyik skatulyába kell kerüljön, ami azt jelenti, hogy biztosan
	lesz két ember, aki ugyanazon a napon született.
	
	Miután felszállt még 333 ember, így 733 ember lett a vonaton. Ha mindegyiket
	skatulyákba helyeznénk, akkor biztosan lenne 3 olyan ember aki ugyanazon
	a napon született, mivel $2\cdot366=732$, így a 733. ember egy olyan
	skatulyába kerül, ahol van már 2 ember, aki ugyanazon a napon született. 
\end{solution}
\begin{problem}
	Van 80 golyónk, közülük 35 piros, 25 zöld, 15 sárga, 5 fekete. Legkevesebb
	hány darabot kell kivenni, hogy biztosan legyen köztük 
	
	\item a) piros; 
	
	\item b) piros vagy fekete; 
	
	\item c) piros és fekete; 
	
	\item d) két különböző színű; 
	
	\item e) valamelyik színből legalább három? \\
	
\end{problem}

\begin{solution}
	a) 46-ot. 45-öt ugyanis még ki lehet húzni piros nélkül $(25+15+5=45)$,
	de ha 46-ot húzunk ki, akkor csak 34 golyó marad meg a 80-ból, nem
	maradhat meg mindegyik piros.
	
	b) 41-et. 40-et ugyanis még ki lehet húzni piros nélkül $(25+15=40)$,
	de ha 41-et húzunk ki, akkor csak 39 golyó marad meg a 80-ból, nem
	maradhat meg az összes piros is és az összes fekete is (35 + 5 = 40).
	
	c) 76-ot. 75-öt ugyanis még ki lehet húzni fekete nélkül (35 + 25
	+ 15 = 75), de ha 76-ot húzunk ki, akkor csak 4 golyó marad meg a
	80-ból, nem maradhat meg mindegyik fekete. Ugyanezen okból 76 golyó
	kihúzása esetén az összes piros se maradhat meg (lásd az a) alpontot),
	tehát a kihúzottak között piros és fekete is lesz.
	
	d) 36-ot. Ha még nincs a kihúzottak között két különböző színű, akkor
	a kihúzottak mind azonos színűek. A legtöbb azonos színű pirosból
	van, 35. Tehát 35 golyót még ki lehet húzni úgy, hogy ne legyenek
	köztük különböző színűek, de 36 golyó kihúzása esetén már biztosan
	lesznek köztük különbözőek is.
	
	e) 9-et. Ha nem teljesül a feltétel, akkor mindegyik színből legfeljebb
	két golyó van. Négyféle szín van, tehát a kihúzott golyók száma legfeljebb
	$4\cdot2=8$. Ez lehetséges is, mindegyik színből van legalább 2 golyó.
	Tehát 8 kihúzott golyó esetén még előfordulhat, hogy nem teljesül
	a feltétel, 9-nél már biztosan teljesül. 
\end{solution}
\begin{problem}
	Van G darab golyónk, közülük P piros, Z zöld, S sárga és F fekete. 
	
	\item a) Tudjuk, hogy legkevesebb 5 darabot kell kivenni, hogy biztosan
	legyen köztük piros. Határozzuk meg F és G értékét, ha ismert, hogy
	Z = 1, \\
	S = 2, P = 3. 
	
	\item b) Tudjuk, hogy legkevesebb 10 darabot kell kivenni, hogy biztosan
	legyen köztük piros és fekete. Határozzuk meg S és G értékét, ha ismert,
	hogy \\
	P = 2, F = 3, Z= 4! \\
	
\end{problem}

\begin{solution}
	a) Legkevesebb $F+S+Z+1$ húzás esetén lesz biztosan piros a kihúzott
	golyók között. Most tehát $F+S+Z+1=5$, amiből $F=1$ és $G=F+S+Z+P=7$.
	
	b) Legkevesebb $Z+S+max(P,F)+1$ húzás esetén lesz biztosan piros
	is és fekete is a kihúzott golyók között. Ezek szerint $4+S+3+1=10$,
	amiből $S=2,G=F+S+Z+P=11$. 
\end{solution}
\begin{problem}
	Adott 21 különböző pozitív egész szám, mindegyik kisebb 70-nél. Mutassuk
	meg, hogy páronkénti különbségeik közt van négy egyenlő! \\
	
\end{problem}

\begin{solution}
	A 21 számból $\frac{21\cdot20}{2}=210$ számpár készíthető. Ezeknek
	210 lehetséges pozitív különbsége van: a lehetséges legkisebb az 1,
	a legnagyobb a 69, tehát a különbségek lehetséges értékei csak 69-félék
	lehetnek. $3\cdot69<210$, így valóban lesz olyan különbség érték,
	amelyik négyszer is előfordul a párokban képzett különbségek között. 
\end{solution}
\begin{problem}
	Legyen a, b és c három egész szám! Bizonyítsuk be, hogy az $abc(b-a)(c-a)(c-b)$
	szorzat osztható 6-tal! \\
	
\end{problem}

\begin{solution}
	Ha a három szám közül valamelyik páros, akkor a szorzat biztosan osztható
	2-vel. Viszont ha mindhárom számunk páratlan, akkor a szorzatban megjelenő
	különbség lesz páros, így a szorzat szintén osztható lesz 2-vel.
	
	Most azt kell belátnunk, hogy a szorzat osztható 3-mal. A 3-mal való
	osztás során a maradékok 0,1,2 lehetnek. Ha az egyik szám osztási
	maradéka 0, akkor a szorzat osztható 3-mal. Viszont ha két szám osztási
	maradéka megegyezik, akkor a különbségük lesz osztható 3-mal.
	
	Tehát ha a szorzat osztható 2-vel és 3-mal is, akkor osztható lesz
	6-tal is. 
\end{solution}
\begin{problem}
	Adott 4 természetes szám. Bizonyítsuk be, hogy közülük legalább kettő
	azonos maradékot ad 3-mal osztva! \\
	
\end{problem}

\begin{solution}
	A 3-mal való osztási maradékok 0,1,2 lesznek. Így 3 skatulyánk lesz
	amelybe tesszük a számokat az osztási maradékok alapján. Ha az első
	3 szám esetében a maradékok különbözőek, de a 4. szám maradéka biztosan
	meg fog egyezni, egy már skatulyában lévő számmal. Tehát lesz legalább
	két szám, amelynek az osztási maradéka megegyezik. 
\end{solution}
\begin{problem}
	Adott egy 3\texttimes 3-as, 9 egység-négyzetből álló négyzet. Minden
	egység-négyzetbe a következő számok egyikét írjuk: -1, 0, vagy 1.
	Bizonyítsuk be, hogy a sorok, oszlopok és átlók összegei között van
	két egyenlő! \\
	
\end{problem}

\begin{solution}
	Mivel 3x3-as négyzetrácsunk van, ezért lesz 3 sorösszegünk, 3 oszlopösszegünk
	és 2 átlóösszegünk, ami összesen 8 összeget alkot. A -1, 0, 1 számokból
	-3, -2, -1, 0, 1, 2 ,3 összegeket alkothatjuk, ami 7 féle összeget
	jelent. A skatulyaelv alapján, lesz a sorok, oszlopok és átlók összegei
	közül lesz két egyforma összeg. 
\end{solution}
\begin{problem}
	Egy diáknak 9 feladatot kell megoldania egy hét alatt. Magyarázzuk
	meg, miért igaz, hogy a diáknak legalább az egyik napon nem kevesebb
	mint 2 feladatot kell megoldania! 
\end{problem}

\begin{solution}
	Ha a diák minden nap megold 1 feladatot, akkor marad még 2 feladata,
	így lesz olyan nap, amikor nem kevesebb mint két feladatot kell megoldjon. 
\end{solution}
\begin{problem}
	El lehet-e szállítani 7 kéttonnás teherautóval 50 kőtömböt, melyek
	súlya $250, 251, 252, \dots, 299$ kg? (A kövek nem darabolhatók, a
	teherautók csak egyszer vehetők igénybe, és mindegyikre legfeljebb
	2 tonna teher rakható.) \\
	
\end{problem}

\begin{solution}
	Ha 50 kőtömböt kell hét teherautóra rakni, akkor lesz olyan teherautó,
	amelyre legalább nyolc kőtömb kerül, hiszen $7\cdot7=49<50$. A nyolc
	legkisebb kőtömb össztömege azonban nagyobb, mint $8\cdot250=2000$
	kg, azaz több mint két tonna. A leírt szabályokkal tehát az elszállítás
	nem lehetséges. 
\end{solution}
\begin{problem}
	Egy falánk ember 10 édességet evett meg egy cukros dobozból, amiben
	3-féle édesség van. Találjuk meg n legnagyobb lehetséges értékét,
	amiről biztosan állítható, hogy a falánk ember legalább n édességet
	evett meg az azonos fajtából. \\
	
\end{problem}

\begin{solution}
	3 féle édesség van, ezért 3 skatulyánk amelyekbe különválasszuk az
	édességeket fajtájuk szerint. Mivel 10 édességet evett meg, ezért
	minden édességből legalább hármat megevett,vagyis $3\cdot3=9$ darabot,
	így még maradt 1 édesség a 3 fajta közül. Tehát a falánk ember legalább
	4 édességet evett meg az egyik fajtából. 
\end{solution}
\begin{problem}
	Hat osztálytárs részt vesz a “találjuk el a célt” versenyben. Mutassuk
	meg, hogy közülük legalább kettőnek azonos számú találata van, ha
	a találatok száma összesen 14. \\
	
\end{problem}

\begin{solution}
	Tegyük fel, hogy minden gyereknek különböző találata volt, vagyis
	0,1,2,3,4,5 lehet a találatok száma. Ha a találatok számát összegezzük
	$0+1+2+3+4+5=15$ kapunk eredményül, de csak 14 lehet a találatok
	száma, ezért egyik gyerek pontszámát 1-gyel kell csökkentsünk, ami
	azt eredményezi, hogy lesz két gyerek, akinek azonos számú találata
	lesz. 
\end{solution}
\begin{problem}
	Tizenhat csapat játszik egy futball versenyen, ahol mindenki játszik
	mindenki ellen. Bizonyítsuk be, hogy minden meccs után van legalább
	két csapat, akik ugyanannyi meccset játszottak. \\
	
\end{problem}

\begin{solution}
	A 16 csapat $\frac{16\cdot15}{2}=120$ meccset játszik. Kezdetben
	minden csapatnak 0 meccse van. 1 meccs után két csapatnak lesz 1 meccse,
	2 meccs után 4 csapatnak lesz valahány meccs stb. Ezt a gondolatmenetet
	követve, minden meccs után valamelyik két csapatnak nő a meccsszáma
	1-gyel. Mivel 16 csapat van, ezért minden csapat legfeljebb 15 meccset
	játszhat. Viszont nem lehet, hogy egy csapat egyetlen meccset se játsszon.
	
	A következtetéseket levonva mindig lesz két olyan csapat, akinek ugyanannyi
	meccse van. 
\end{solution}
\begin{problem}
	Egy 70 cm oldalhosszúságú négyzet alakú céltáblára 50 lövést adunk
	le. Jól célzunk, egyetlen lövésünk sem kerüli el a táblát. Igazoljuk,
	hogy a találati pontok között van két olyan, amelynek távolsága kisebb,
	mint 15 cm! \\
\end{problem}

\begin{solution}
	Tekintsük a céltáblát egy sakktáblának, ami 7x7-es, vagyis 49 mezőnk
	lesz. Mivel 50 lövés van, ezért lesz olyan mező, ahova 2 lövés érkezik.
	Be kell látnunk,hogy két találati pont távolsága kisebb, mint 15 cm.
	
	A sakktábla minden mezője 10 cm oldalú négyzet, aminek az átlója Pitagorasz
	tétel alapján $\sqrt{200}\approx14,1cm<15cm$.
	
	Tehát lesz két olyan pontunk amelynek a távolsága kisebb mint 15 cm.
\end{solution}

\section*{Nehezebb feladatok}
\begin{extraproblem}[Csurka-Molnár Hanna]
	Legyen adott 51 egész szám, melyek mindegyike a {[}1, 100{]} intervallumba
	esik. Bizonyítsd be, hogy mindig létezik két olyan szám a kiválasztottak
	között, melyek különbsége legfeljebb 1. 
\end{extraproblem}

\begin{solution}
	Osszuk az {[}1, 100{]} intervallumot 50 darab két elemű intervallumra,
	például így:
	
	\[
	(1,2),(3,4),(5,6),\ldots,(99,100)
	\]
	
	Így $50$ db „skatulya” jön létre, minden skatulyába két egymást követő
	szám tartozik, vagyis olyan számok, melyek különbsége legfeljebb $1$.
	Vegyük észre ha $51$ számot választunk az $[1,100]$ tartományból,
	akkor a $50$ skatulyából legalább egybe kettő szám is kerülni fog.
	Mivel minden skatulyán belüli számok különbsége legfeljebb 1, ezért
	ez a két szám is legfeljebb 1-gyel tér el egymástól. 
\end{solution}
\begin{extraproblem}[Czofa Vivien]
	Bizonyítsuk be, hogy bármely 8 darab egész szám között van legalább
	kettő, amelyeknek a különbsége osztható 7-tel.
\end{extraproblem}

\begin{solution}
	A 7-tel történő osztásnál hétféle maradék lehet, azaz a 7-tel való
	osztás szempontjából a számok: 
	\[
	7k,\ 7k+1,\ 7k+2,\ 7k+3,\ 7k+4,\ 7k+5,\ 7k+6
	\]
	alakban írhatók. Legyen az ilyen alakú számoknak egy-egy ``skatulyája''.
	Ez összesen hét darab doboz. Tegyük be a 8 darab szám mindegyikét
	a megfelelő dobozba. Legalább egy dobozba legalább két számnak kell
	kerülnie.
	
	Abban a dobozban, amelyben egynél több szám van, vegyük két számnak
	a különbségét. Ez a különbség osztható 7-tel.
\end{solution}
\begin{extraproblem}[Czofa Vivien]
	Bizonyítsuk be, hogy bármely $a,b,c,d$ egész számokból képzett 
	\[
	(a-b)(a-c)(a-d)(b-c)(b-d)(c-d)
	\]
	szorzat osztható 12-vel.
\end{extraproblem}

\begin{solution}
	Mivel $12=2\cdot2\cdot3$, a 12-vel való oszthatóságot úgy bizonyíthatjuk,
	hogy megmutatjuk egy tényezőnek a 3-mal való oszthatóságát és két
	tényezőnek a 2-vel való oszthatóságát.
	
	A 3-mal való oszthatóság szempontjából a számoknak háromféle alakjuk
	lehet: $3k,\ 3k+1,\ 3k+2$ \ ($k\in\mathbb{Z}$). Válasszunk 3 ilyen
	``skatulyát''\ és ezekbe tegyük bele az $a,b,c,d$ számokat. Legalább
	az egyikbe, legalább két szám kerül. Ebből a kettőnek a különbsége
	osztható 3-mal. A szorzat osztható 3-mal, mert ez a különbség (vagy
	az ellentettje) ott van a hat tényező között.
	
	\textit{Ennek belátásához} meg kell vizsgálnunk, hogy négy szám közül
	hányféle módon választhatunk ki kettőt. Válasszunk ki elsőnek az $a$-t.
	Mellé másodiknak választhatjuk $b$-t, $c$-t, $d$-t. Ez eddig 3
	lehetőség. Az $a$-val több lehetőség nem lehet. Tekintsük a $b$
	választását. Az $a$-val már választottunk, így $b$ mellé $c$-t
	vagy $d$-t választhatjuk. Ez két újabb lehetőség, az előzőekkel együtt
	ez 5 lehetőség. A $c$ mellé csak a $d$-t választhatjuk. Ez egy újabb
	lehetőség. Az eddigiekkel együtt ez 6 lehetőség, több már nincs.
	
	Tehát négy számból kettőt, 6 féle módon választhatunk ki. A 6 tényező
	az összes lehetőség.
	
	A 2-vel való oszthatóság vizsgálatánál gondoljunk arra, hogy az $a,b,c,d$
	számok között párosak is, páratlanok is lehetnek. Ezek vagy úgy oszlanak
	meg, hogy két páros és két páratlan van közöttük, vagy az egyik fajtából
	legalább 3 darab van.
	
	Ha két páros van és két páratlan, akkor a két páros különbsége is
	páros, a két páratlan különbsége is páros, tehát 6 tényező közül kettő
	osztható 2-vel.
	
	Ha akár 3 páratlan, akár 3 páros közül választunk ki kettőt, akkor
	abból a három számból már biztosan lehet 2-t találni, melyek különbsége
	páros. Most is a 6 tényező közül legalább kettő osztható 2-vel.
	
	Ezzel be van bizonyítva a $3\cdot2\cdot2=12$-vel való oszthatóság.
\end{solution}
\begin{extraproblem}[Fábián Nóra]
	Legyenek az $x_{1},x_{2},\dots,x_{11}$ tetszőleges egész számok.
	Az $a_{1},a_{2},\dots,a_{11}$ számok a $\{-1,0,1\}$ halmazbeli értékeket
	vehetik fel, de úgy, hogy nem mind egyenlők 0-val. Igaz-e, hogy van
	olyan értéke az $a_{1}x_{1}+a_{2}x_{2}+\dots+a_{11}x_{11}$ kifejezésnek,
	ami $2002$-vel osztható?\textit{ (XI. Nemzetközi Magyar Matematikaverseny,
		2002, XII. osztály)} 
\end{extraproblem}

\begin{solution}
	Tekintsük az összes olyan összeget, amely $b_{1}x_{1}+b_{2}x_{2}+\dots+b_{11}x_{11}$
	alakú, ahol minden $b_{i}$ $0$ vagy $1$ lehet. Ez összesen $2^{11}=2048$
	különböző összeg. A \textbf{skatulyaelv} szerint ezek között van kettő,
	amelyek $2002$-vel vett osztási maradéka megegyezik.
	
	Legyenek ezek $c_{1}x_{1}+c_{2}x_{2}+\dots+c_{11}x_{11}$ és $d_{1}x_{1}+d_{2}x_{2}+\dots+d_{11}x_{11}$.
	A két kifejezés különbsége formálisan 
	\[
	(c_{1}-d_{1})x_{1}+(c_{2}-d_{2})x_{2}+\dots+(c_{11}-d_{11})x_{11}.
	\]
	Ez az előbbiek miatt osztható lesz $2002$-vel és minden $c_{i}-d_{i}$
	érték az $a_{i}$-k értékkészletébe tartozik.
	
	Tehát lesznek megfelelő $a_{i}$-k, amelyekre az összeg osztható lesz
	$2002$-vel.
\end{solution}
\begin{extraproblem}[Fábián Nóra]
	Hány olyan hatjegyű szám létezik, amelyben van két azonos számjegy?
	És hány ilyen 15-jegyű szám létezik?
\end{extraproblem}

\begin{solution}
	Az összes hatjegyű számok száma $10^{6}-10^{5}$, elegendő a különböző
	számjegyekből álló számokat megszámolni.
	
	Egy ilyen szám első számjegye $9$-féle lehet (a $0$ nem megengedett),
	a második számjegy szintén $9$-féle (ez már lehet $0$, de nem lehet
	azonos az első számjeggyel), a harmadik számjegy $8$-féle stb. A
	jó számok száma mindezek alapján 
	\[
	10^{6}-10^{5}-9\cdot8\cdot7\cdot6\cdot5=763920.
	\]
	
	A 15-jegyű számoknak a \textbf{skatulyaelv} alapján mindig van két
	azonos számjegye. Ezek száma: 
	\[
	10^{15}-10^{14}.
	\]
\end{solution}
\begin{extraproblem}[Gál Tamara]
	Legfeljebb hány pozitív egész számot adhatunk meg úgy, hogy semelyik
	kettő összege és különbsége se legyen osztható 2015-tel? 
\end{extraproblem}

\begin{solution}
	Két szám különbsége pontosan akkor osztható 2015-tel, ha ugyanannyi
	a 2015-ös maradékuk. Két szám összege pedig pontosan akkor osztható
	2015-tel, ha vagy mindkét szám osztható 2015-tel, vagy az egyik 2015-ös
	maradéka $n$, a másiké 2015-n (ahol $1\leq n\leq2014$ ).\\
	Készítsünk 1008 db skatulyát a 2015-ös maradékok alapján a következőképpen:
	egy skatulyába kerüljenek a 0 maradékot adó számok, a következőbe
	az 1 vagy 2014 maradékot adók, a következőbe a 2 vagy 2013 maradékot
	adók, és így tovább - általában az $n$ és a 2015 - $n$ maradékot
	adó számok kerülnek azonos skatulyába -, végül az utolsóba az 1007
	vagy 1008 maradékot adók. Ekkor pontosan abban az esetben lesz két
	szám összege vagy különbsége 2015-tel osztható, ha ezek ugyanabba
	a skatulyába kerülnek.
	
	Az eddigiek alapján 1008-nál több számot biztosan nem tudunk megadni,
	hiszen ekkor a skatulyaelv miatt valamelyik skatulyába egynél több
	szám kerülne. 1008 számot viszont meg tudunk adni úgy, hogy mindegyik
	különböző skatulyába kerüljön, például az $1,2,3,\ldots,1007,2015$
	számok teljesítik ezt a feltételt.\\
	Tehát legfeljebb 1008 számot adhatunk meg a kívánt módon.
\end{solution}
\begin{extraproblem}[Gál Tamara]
	a.)Igazoljuk, hogy 16 egész szám között mindig van néhány, amelyek
	összege 16-tal osztható. (Egytagú összeget is megengedünk.)\\
	b) Igazoljuk, hogy a 10-es számrendszerben felírt 16-jegyű pozitív
	egész számnak van néhány egymást követő számjegye, melyek szorzata
	négyzetszám. (Egytényezős szorzatot is megengedünk.\\
	
\end{extraproblem}

\begin{solution}
	a) Legyenek a számok $a_{1},a_{2},...,a_{16}$ . Tekintsük az 
	\[
	a_{1},a_{1}+a_{2},a_{1}+a_{2}+a_{3},...,a_{1}+a_{2}+...+a_{1}6
	\]
	összegeket! Ezek száma éppen 16. Ha valamelyik osztható 16-tal, akkor
	készen vagyunk. Ellenkező esetben a skatulya-elv miatt lesz kettő,
	melynek a 16-os maradéka megegyezik. A több tagból állóból kivonva
	a kevesebb tagból állót, a kapott összeg 16-tal osztható lesz. b)
	Legyen $n=a_{1}a_{2}...a_{16}$ a kiválasztott szám. Ha az számjegyek
	valamelyike 0, 1, 4 vagy 9, akkor készen vagyunk. Feltehetjük, hogy
	$a_{i}\in\{2,3,5,6,7,8\}$ . Ekkor bármelyik szorzat, amely egymást
	követő számjegyekből képezhető 
	\[
	2^{a}3^{b}5^{c}7^{d}
	\]
	alakú lesz, ahol $a,b,c,d$ természetes számok. Elegendő belátnunk,
	hogy van olyan szorzat, amelyben minden kitevő páros. Tekintsük az
	\[
	a_{1},a_{1}a_{2},a_{1}a_{2}a_{3},...,a_{1}a_{2}...a_{16}
	\]
	szorzatokat! Paritás alapján az ( a,b ,c ,d ) kitevő 4-esek 16 különböző
	osztályába sorolhatók: (ptl, ptl, ptl, ptl), … , (ps, ps, ps, ps).
	Ha az utolsó osztályba kerül kitevő 4-es, akkor készen vagyunk. Ellenkező
	esetben lesz két olyan kitevő 4-es, amelyik azonos osztályban van.
	Tegyük fel, hogy ezek az $a_{1}...a_{k}$ és $a_{1}...a_{l}$ szorzatokhoz
	tartoznak, ahol $k>l.$ Ekkor 
	\[
	\frac{a_{1}...a_{k}}{a_{1}...a_{l}}=a_{l+1}...a_{k}
	\]
	olyan szorzat, melyhez tartozó $a,b,c,d$ kitevő 4-es minden kitevője
	páros. Ezt akartuk belátni. 
\end{solution}
\begin{extraproblem}[Gergely Verona]
	Egy nemzetközi konferencián 200 tudós vesz részt. Mindegyikük legfeljebb
	4 nyelven beszél, továbbá bármely három között van kettő, akik beszélnek
	közös nyelven. Bizonyítsuk be, hogy van olyan nyelv, amit közülük
	legalább 26-an beszélnek. \emph{(KöMaL, 2006) }
\end{extraproblem}

\begin{solution}
	Tegyük fel, hogy van két olyan tudós, akik nem beszélnek közös nyelvet.
	Mivel bármely három ember között van kettő, akik megértik egymást,
	így ha ehhez a két emberhez kiválasztunk még egy harmadikat, akkor
	ő beszél közös nyelvet a két ember valamelyikével. Ezen a két emberen
	kívül még 198 van, így a két ember valamelyike legalább 99 emberrel
	beszél közös nyelvet; nevezzük ezek egyikét A-nak. Mivel egy ember
	legfeljebb 4 nyelvet beszél, a skatulya-elv alapján ebből a 99 emberből
	legalább 25 beszél egy közös nyelvet. Az A-t is beleszámítva tehát
	van 26 ember, aki egy nyelvet beszél.
	
	Ha nincs két olyan ember, aki nem érti meg egymást, akkor mindenki
	mindenkivel tud beszélgetni. Válasszunk ki egy B embert; ekkor ő 199
	emberrel tud beszélni. Mivel egy ember legfeljebb 4 nyelvet beszél,
	ismét a skatulya-elv alapján B legalább 50 emberrel beszél egy bizonyos
	nyelven.
	
	Mindkét esetben beláttuk, hogy van olyan nyelv, amit legalább 26-an
	beszélnek. 
\end{solution}
\begin{extraproblem}[Gergely Verona]
	Adott 20 különböző pozitív egész szám, mindegyik kisebb 70-nél. Mutassuk
	meg, hogy páronkénti különbségeik közt van négy egyenlő. \emph{(1974,
		Arany Dániel verseny II. forduló) }
\end{extraproblem}

\begin{solution}
	Megoldás 2 Képzeljük el a 20 számot a számegyenesen, és vágjuk fel
	az egyenest a legkisebb és a legnagyobb számnál. Így egy olyan szakaszt
	kapunk, ami kisebb 70 egységnél, és ezen 18 pontot helyezhetünk el.
	Ha most az e szakasz mentén levő 18 pontot is feldaraboljuk, akkor
	19 kisebb szeletet kapunk. Most állítjuk, hogy ezek között a 19 szelet
	között biztosan lesz négy egyenlő hosszúságú. Ha minden hosszúságú
	szeletből legfeljebb három darab lenne, akkor a szeletek összhossza
	legalább:
	
	\[
	1+1+1+2+2+2+3+3+3+\dots+6+6+6+7=70
	\]
	
	Ez azonban meghaladja a 70-et, tehát biztos, hogy van olyan szelet,
	ami legalább négyszer előfordul. 
\end{solution}
\begin{extraproblem}[Kis Aranka-Enikő]
	Egy $5\times5$ kiterjedésű négyzet 25 egységnégyzetre van osztva,
	amik mind pirosra vagy kékre vannak festve. Bizonyítsuk be, hogy létezik
	4 olyan egyszínű négyzet, amik 2 sor és 2 oszlop kereszteződésénél
	fekszenek az eredeti négyzetben.
\end{extraproblem}

\begin{solution}
	Először tekintsük az egyik oszlop egységnégyzeteit a fiókoknak. A
	skatulya-elv alapján következik, hogy a kiválasztott oszlopban dominálni
	fog az egyik szín. Hasonló módon dominálni fog az egyik szín a többi
	oszlopban is.
	
	Ezután most az 5 oszlop fogja a fiókokat jelenteni, míg a szétosztandó
	tulajdonságok az egyes oszlopok domináló színei. A skatulya-elv értelmében
	következik, hogy a domináló szín legalább 3 oszlopban ugyanaz. Így
	3 ilyen oszlop van, és mindegyikben legalább 3 ugyanolyan színű egységnégyzet
	van.
	
	Tegyük fel, hogy ez a szín a kék! Továbbá, számozzuk meg az eredeti
	négyzet sorait 1-től 5-ig és tekintsünk 5 fiókot ugyanezekkel a számokkal
	megszámozva!
	
	Ezután rendeljük a három tekintett oszlopban található kék egységnégyzetekhez
	azoknak a soroknak számát, amelyekhez tartoznak. Így a feladat 9 (vagy
	több) egész szám 5 fiókba, az 1, 2, 3, 4 és 5 számokhoz való hozzárendelésére
	egyszerűsödik. Minden egész számot a megfelelő számú fiókba kell helyezni.
	
	Most be kell bizonyítanunk, hogy létezik 2 olyan fiók, hogy mindegyikben
	legalább 2 szám van. Első ránézésre látható, hogy létezik olyan fiók,
	amiben legalább 2 szám van. Két eset lehetséges.
	
	Az elsőben feltesszük, hogy létezik egy olyan fiók, amiben legalább
	3 szám van. Ekkor a visszamaradó 6 számot a 4 visszamaradó fiókba
	kell szétosztani. A skatulya-elv értelmében következik, hogy az egyik
	fiók legalább 2 számot tartalmaz. Ez a fiók a 3 számot tartalmazó
	fiókkal együtt megoldja a feladatot.
	
	A második esetben tekintsünk egy olyan fiókot, amiben 2 szám van.
	A visszamaradó 7 számot ekkor a visszamaradó 4 fiókba kell szétosztani.
	A skatulya-elv értelmében az egyik fiók legalább 2 számot tartalmaz,
	és így készen is van a feladat.
\end{solution}
\begin{extraproblem}[Kis Aranka-Enikő]
	Az 1-től $n^{2}$-ig terjedő egész számokat tetszőlegesen helyezzük
	el az $n\times n$-es sakktábla mezőin. Tekintsük a következő kijelentést:
	Létezik két olyan szomszédos oldallal rendelkező mező, amelyeken a
	rajtuk lévő két szám különbsége nagyobb, mint 5. Igazolja, hogy
	\begin{itemize}
		\item[\foreignlanguage{english}{a)}] Az állítás nem mindig teljesül az $n=5$ esetben; 
		\item[\foreignlanguage{english}{b)}] Az állítás mindig igaz az $n>5$ esetben. 
	\end{itemize}
\end{extraproblem}

\begin{solution}
	\textbf{a)} Vegyük az $5\times5$-ös sakktáblát és írjuk sorba a számokat,
	1-től 5-ig az első sorba, 6-tól 10-ig a második sorba és így tovább,
	végül a 21-től a 25-ig az utolsó sorba. A maximális különbség így
	5-tel egyenlő.
	
	\textbf{b)} Az előző feladathoz hasonló módon az a mező, amelyik az
	$n^{2}$ egészet tartalmazza, az 1-et tartalmazó mezőről legfeljebb
	$2(n-1)$ mezőn keresztül érhető el. Mivel az 1-et most $n^{2}-1$-el
	növeljük, a skatulya-elvből következik, hogy lesz egy olyan lépés,
	ahol az emelés nem kevesebb mint 
	\[
	\frac{2(n^{2}-1)}{n-1}.
	\]
	Ha $n\geq10$, akkor az 
	\[
	\frac{2n+1}{n}>5,
	\]
	és következésképpen az állítás mindig teljesül.
	
	Amikor $n=9$, a növekedés 1 és 81 között $81-1=80$ és legfeljebb
	16 lépésben elérhető (8 vízszintes és 8 függőleges lépés). Ha a lépések
	száma legfeljebb 15, akkor a skatulya-elv szerint van egy olyan lépés,
	ahol az emelés nem kevesebb, mint 
	\[
	\frac{15}{80}.
	\]
	Ez a szám nyilván nagyobb mint 5. Tegyük fel, hogy pontosan 16 lépés
	van. Ekkor 
	\[
	\frac{80}{16}=5,
	\]
	és ha létezik egy olyan lépés, ahol a növekedés kevesebb mint 5, akkor
	a teljes növekedés 76, legalábbis a fennmaradó 15 lépésben. A skatulya-elv
	alapján, van legalább egy lépés, ahol az emelkedés nem kevesebb, mint
	\[
	\frac{76}{15},
	\]
	azaz az emelkedés több mint 5.
	
	Végül tekintsük azt az esetet, amikor nincs sem olyan lépés, amelyben
	a növekedés több mint 5, sem olyan, amikor kevesebb. Tehát minden
	lépésben 5 az emelkedés, vagyis az 1-ből kiindulva a szomszédos mezőkön
	a számok a következők lesznek: 6, 11, 16, 21 és így tovább. Másrészt
	nem csak ez a lehetséges út 1 és 81 között, hanem több ilyen út létezik.
	Ez azt jelenti, hogy a megfelelő mezőkön az egész számok nem mind
	ugyanazok, vagyis a növekedés nem 5, és minden lépésben meg lehet
	ismételni a gondolatmenetet. Következik, hogy az állítás igaz $n=9$
	esetén is. Hasonló úton bebizonyítható, hogy az állítás az $n=6$,
	$n=7$ és $n=8$ esetekben is igaz.
\end{solution}
\begin{extraproblem}[Kis Brigitta]
	Egy cipőboltban 12 pár cipőt helyeznek el 5 különböző polcon. Bizonyítsuk
	be, hogy legalább egy polcon legalább 3 pár cipő van. 
\end{extraproblem}

\begin{solution}
	A skatulyaelv kimondja, hogy ha $n$ tárgyat $k$ skatulyába helyezünk,
	és $n>k$, akkor legalább egy skatulyába legalább
	
	\[
	\lceil n/k\rceil
	\]
	
	tárgy kerül.
	
	Ebben az esetben: 
	\begin{itemize}
		\item A cipők (12 pár) a tárgyak. 
		\item A polcok (5 darab) a skatulyák. 
	\end{itemize}
	Mivel a cipők száma egész, a legjobb esetben minden polcon legfeljebb
	2 pár cipő lenne. De mivel $12$ nem osztható pontosan $5$-tel, legalább
	egy polcon 3 pár cipőnek kell lennie.
	
	Következtetés: Bármilyen elosztás esetén legalább egy polcon legalább
	3 pár cipő lesz. 
\end{solution}
\begin{extraproblem}[Kis Brigitta]
	Egy halmazban 20 különböző egész szám található. Bizonyítsuk be,
	hogy mindig létezik két nem üres, különböző részhalmaz, amelyek elemeinek
	összege megegyezik. 
\end{extraproblem}

\begin{solution}
	Vegyük észre, hogy egy $S$ halmaz összes nem üres részhalmazának
	összegeit vizsgáljuk. Egy $n$ elemű halmaznak
	
	\[
	2^{n}-1
	\]
	
	nem üres részhalmaza van.
	
	Ebben az esetben: 
	\begin{itemize}
		\item A halmaz 20 különböző egész számot tartalmaz. 
		\item A halmaz nem üres részhalmazainak száma:
		
		\[
		2^{20}-1=1,048,575.
		\]
		
	\end{itemize}
	Mivel minden részhalmazhoz tartozik egy részösszeg, és a számok maximumösszege
	véges, az összes lehetséges különböző részösszeg száma nem lehet nagyobb,
	mint a legnagyobb lehetséges összeg.
	
	A lehetséges maximális részhalmaz-összegek száma akkor érhető el,
	ha a számok a lehető legkisebbek és pozitívak. Ha például az elemek
	az első 20 természetes szám lennének ($\{1,2,3,\dots,20\}$), akkor
	a legnagyobb lehetséges összeg:
	
	\[
	1+2+\dots+20=\frac{20\cdot21}{2}=210.
	\]
	
	Tehát az összes részhalmaz összege legfeljebb 210 különböző érték
	lehet, de a részhalmazok száma több mint egymillió. A skatulyaelv
	alapján, mivel több részhalmaz van, mint lehetséges összegérték, biztosan
	léteznie kell két különböző részhalmaznak, amelyek összege megegyezik.
	
	Következtetés: Mindig léteznek különböző, nem üres részhalmazok, amelyek
	összege azonos. 
\end{solution}
\begin{extraproblem}[Kis Andrea-Tímea]
	Igazoljuk, hogy létezik a $7$-nek két olyan természetes kitevőjű,
	különböző hatványa, amelyeknek különbsége osztható $2025$-tel. 
\end{extraproblem}

\begin{solution}
	A $7$-nek pontosan akkor létezik két olyan természetes kitevőjű hatványa,
	amelyeknek különbsége osztható $2025$-tel, ha a $7$-nek létezik
	két olyan hatványa, amelyeknek a $2025$-tel való osztási maradékuk
	megegyező.
	
	\textit{Red. ad abs.:} Feltételezzük, hogy a $7$-nek nem létezik
	két olyan természetes kitevőjű hatványa, amelyeknek a $2025$-tel
	való osztási maradékuk megegyező. Ez azt jelenti, hogy a $7$ összes
	természetes kitevőjű hatványa különböző maradékot ad $2025$-tel osztva.
	
	A $2025$-tel való osztási maradékok $2025$-félék lehetnek, ezek
	a $0,1,\ldots,2024$.
	
	Ha például vennénk a $7,7^{2},7^{3},\ldots,7^{2025},7^{2026}$ hatványokat,
	akkor $2025$ skatulyában $2026$ számot kellene elrendezzek a legrosszabb
	esetben a $2025$-tel való osztási maradékuk alapján. Így a skatulyaelv
	alapján biztos kerül egy skatulyába két hatvány, ami alapján ellentmondáshoz
	jutottunk.
	
	Tehát létezik a $7$-nek két olyan természetes kitevőjű, különböző
	hatványa, amelyeknek különbsége osztható $2025$-tel. 
\end{solution}
\begin{extraproblem}[Kis Andrea-Tímea]
	Igazoljuk, hogy $n\geq2$ focicsapatból álló bajnokság esetén, ahol
	bármely két focicsapat legfentebb egyszer játszik egymással, mindig
	lesz két olyan csapat, amelyek lejátszott mérkőzéseinek száma egyenlő
	lesz. 
\end{extraproblem}

\begin{solution}
	Az $n$ focicsapatból ha egyet kiválasztunk, akkor ez az egy csapat
	lejátszhat $1,2,\ldots,n-1$ mérkőzést.
	
	Ez az állítás igaz bármelyik csapatra, vagyis mind az $n$ csapat
	összesen $n-1$ lejátszott mérkőzésszámmal rendelkezhet.
	
	Ha az $1,2,\ldots,n-1$ mérkőzésszámokat skatulyáknak tekintem, akkor
	az $n$ csapatot $n-1$ skatulyába kell elhelyeznem. Ez a skatulyaelv
	alapján azt jelenti, hogy lesz két olyan csapat, amelyek ugyanannyi
	mérkőzést játszottak le. 
\end{solution}
\begin{extraproblem}[Lukács Andor]
	Tekintsük a Gauss-féle egész számok $\mathbb{Z}[i]=\{a+ib|a,b\in\mathbb{Z}\}$
	halmazát. 
	\begin{enumerate}
		\item[(a)] Igazold, hogy bárhogyan is választunk ki öt elemet a halmazból, van
		köztük két olyan $z_{1}$ és $z_{2}$ elem, amelyre $\frac{z_{1}+z_{2}}{2}\in\mathbb{Z}[i]$.
		\item[(b)] Igazold, hogy bárhogyan is választunk ki $13$ elemet a halmazból,
		van köztük három olyan $z_{1},z_{2}$ és $z_{3}$ elem, amelyre $\frac{z_{1}+z_{2}+z_{3}}{3}\in\mathbb{Z}[i]$. 
	\end{enumerate}
	\begin{flushright}
		(OMMO 2025, megyei szakasz, XII. osztály) 
		\par\end{flushright}
\end{extraproblem}

\begin{solution}
	~
	\begin{enumerate}
		\item[(a)] Vizsgáljuk az öt elem valós és képzetes részének a paritását. Az
		öt elem közt biztosan van három olyan, amelyek valós részének ugyanaz
		a paritása. Ezen három elem közül két elem képzetes részének a paritása
		megegyezik. Erre a két számra igaz, hogy $\frac{z_{1}+z_{2}}{2}\in\mathbb{Z}[i]$.
		\item[(b)] A $\mathbb{Z}[i]$ halmaz minden eleméhez rendeljük hozzá a 
		\[
		(0;0),(0;1),(0;2),(1;0),(1;1),(1;2),(2;0),(2;1),(2;2)
		\]
		számpárok egyikét aszerint, hogy a valós és a képzetes résznek mennyi
		a 3-mal való osztási maradéka. Például a 
		\[
		z=13-2i
		\]
		-nek az $(1;1)$ számpár felel meg. Rendezzük ezeket a számpárokat
		a következő táblázatba:
		
		\[
		\begin{tabular}{|c|c|c|}
			\hline  (0;0)  &  (0;1)  &  (0;2)\\
			\hline  (1;0)  &  (1;1)  &  (1;2)\\
			\hline  (2;0)  &  (2;1)  &  (2;2)\\
			\hline   &   &  
		\end{tabular}
		\]
		
		Tegyük fel, hogy a $13$ elemet elhelyeztük a nekik megfelelő cellákba.
		Mivel három sor van, a skatulya elv alapján az egyik sorba legalább
		$5$ elem kerül. Két esetet különböztetünk meg:
		
		\emph{I. eset.} Ha az adott sorban, amelyben az öt elem található,
		van olyan cella, amelyikbe három elem kerül, akkor ezeknek az elemeknek
		az átlaga teljesíti a megadott feltételt.
		
		\emph{II. eset.} Ha az adott sorban, amelyben az öt elem található,
		mindegyik cellában legfeljebb két elem van, akkor az adott sor mindegyik
		cellájában van elem, és ezek átlaga fogja teljesíteni a megadott feltételt. 
		
	\end{enumerate}
	\textbf{Megjegyzés.} Igazolható, hogy a (b) alpont állítása teljesül
	már $9$ tetszőlegesen választott elem esetén is. Eredetileg ezt akarták
	feladni a versenyen, hiányos megoldással, ezt leszavaztuk hogy túl
	nehéz, és ekkor ötöltük ki Sanyival a $13$-mast. Ha lesz időm, beírom
	a $9$-es verziót megoldását is valamikor. Egyelőre csak egy nagyon
	tárgyalásos megoldásunk van, emiatt elég hosszú. 
\end{solution}
\begin{extraproblem}[Miklós Dóra]
	Bizonyítsuk be, hogy minden $n$ pozitív egész szám esetén léteznek
	az egymástól különböző $r$ és $s$ pozitív egész számok úgy, hogy
	$3^{r}-3^{s}$ szám osztható legyen az $n$ számmal. \emph{(Felvidéki
		Magyar Matematika Verseny, XXII. 3. évfolyam) }
\end{extraproblem}

\begin{solution}
	Használjuk a skatulya elvet. Tekintsük a következő $n+1$ darab természetes
	számot: $3^{1},3^{2},3^{3},\dots,3^{n},3^{n+1}$. Közülük $n$-nel
	való osztás után legalább kettőnek ugyanazt a maradékot kell adnia,
	mert összesen $n$ lehetséges maradék van: $0,1,2,\dots,n-1$. A kérdéses
	két szám legyen $3^{r}$ és $3^{s}$. Így teljesül, hogy $3^{r}=kn+z$
	és $3^{s}=mn+z$, ahol $k$ és $m$ megfelelő nem negatív egész számok,
	a $z$ pedig a lehetséges maradékok egyike. Így $3^{r}-3^{s}=(k-m)n$,
	vagyis az állítást bizonyítottuk. 
\end{solution}
\begin{extraproblem}[Miklós Dóra]
	Bizonyítsuk be, hogy egy $nk+1$ tagú sorozatban vagy van $n+1$
	szám, amelyek a sorozatbeli sorrendjükben szigorúan növekednek, vagy
	van $k+1$ szám, amelyek a sorozatbeli sorrendjükben monotonan csökkennek.
	\emph{(Egy nevezetes Erdős-Szekeres feladat) }
\end{extraproblem}

\begin{solution}
	A sorozat minden tagja mellé képzeletben odaírjuk, hogy mekkora a
	leghosszabb, vele kezdődő monoton növekvő sorozat. Ha a sorozat két
	tagja mellé ugyanaz a szám kerül, akkor közülük az amelyik előrébb
	van nem kisebb az utóbbinál. Másrészt az is nyilvánvaló, hogy ha valamelyik
	tag mellé az $n+1$ szám kerül akkor kész vagyunk (vagy $n+1$-nél
	nagyobb): van $n+1$ tagú szigorúan növekvő sorozat. Ha nincs ilyen
	szám, akkor minden szám alá $1,2,3,...,n$ értékek közül kerülhet
	valamelyik. Mivel összesen $nk+1$ szám van, ezért biztosan van $k+1$
	olyan tagja a sorozatnak, amelyek alá ugyanaz a szám kerül. Ezek viszont
	a korábbi megállapítás alapján egy monoton csökkenő sorozatot fognak
	alkotni, így teljesül a másik feltétel. 
\end{solution}
\begin{extraproblem}[Seres Brigitta-Alexandra]
	Igazoljuk, hogy bárhogyan is választunk ki a $2000$-nél nem nagyobb
	pozitív egész számok közül $1001$-et, biztosan lesz a kiválasztottak
	között két olyan szám, amelyek különbsége $4$. \textit{(NMMV, Dunaszerdahely,
		2000, 9. évfolyam, 3. feladat)} 
\end{extraproblem}

\begin{solution}
	$2000$-nél nem nagyobb pozitív egész számok halmaza: \\
	$\{1,2,3,\ldots,2000\}$, a halmaz $2000$ elemet tartalmaz. Rendezzük
	el ezeket a számokat egy 4x500-as táblázatba, úgy, hogy egy adott
	sorban négyesével növeljük a számok értékét, és az első oszlopba $1,2,3,4$
	számokat írjuk be! Ezzel lényegében egy sorba rendezzük a $4$-el
	való osztási maradékjaik szerint megegyező számokat. Mivel $2000=4\cdot500$
	minden számunk benne lesz a táblázatban. 
	\[
	\begin{array}{cccccc}
		1 & 5 & 9 & \ldots & 1993 & 1997\\
		2 & 6 & 10 & \ldots & 1994 & 1998\\
		3 & 7 & 11 & \ldots & 1995 & 1999\\
		4 & 8 & 12 & \ldots & 1996 & 2000
	\end{array}
	\]
	
	Minden sorban $500$ szám található. Tekintsük a $4$ sort egy-egy
	doboznak. Tekintve, hogy $1001=4\cdot250+1$, "legrosszabb" esetben
	minden sorból kiválasztunk $250$ elemet, majd a skatulyaelv alapján
	($1001>1000$) lesz egy olyan sor, amelyből legalább $251$ elemet
	választunk ki.
	
	Mivel azonban az 500 páros szám, azért csak 250 számot választhatunk
	ki úgy, hogy ne legyen köztük két szomszédos tag az adott sorból.
	Az előbb meghatározottak alapján lesz egy sorunk ahol legalább $251$
	elemet választunk ki. Ezen két információ együttesen azt jelenti,
	hogy valamelyik sorból kiválasztunk két szomszédos elemet, ezek különbsége
	pedig 4 lesz (hiszen a sorokat úgy szerkesztettük hogy a négyessével
	mentünk felfele, így két szomszédos tag különbsége éppen $4$ lesz).
	Ezzel az igazolandó állítást bebizonyítottuk.
\end{solution}
\begin{extraproblem}[Sógor Bence]
	Legfeljebb hányat választhatunk ki az első ezer pozitív egész közül
	úgy, hogy egyik se ossza a másikat? 
\end{extraproblem}

\begin{solution}
	Először is vegyük észre, hogy ha az utolsó 500 számot kiválasztjuk,
	akkor egyik sem lesz osztója egy másiknak. 501, 502, $\dots$, 1000
	számok közül a legkisebb az 501. Ennel a legkisebb tőle különböző
	egész számú többszöröse az 1002. Tehát a többi számnak is bármely
	egész számú többszöröse nagyobb lesz, mint 1000.
	
	Igazoljuk, hogy bárhogy választanánk ki 501 számot lesz 2 ami osztja
	egymást. Vegyük észre, hogy minden pozitív egész szám felírható $2^{m}\cdot p$
	alakban, ahol p egy páratlan szám. Mivel 500 páratlan szám van 1-től
	1000-ig ezért, ha 501 számot választunk, akkor lesz kettő, amelyiknek
	a $2^{m}\cdot p$ alakú felbontásában a $p$ megegyezik. Két ilyen
	szám közül a kisebb osztja a nagyobbat, hiszen ha $m_{2}>m_{1}$ akkor
	$2^{m_{2}}\cdot p:(2^{m_{1}}\cdot p)=2^{m_{2}-m_{1}}$.
	
	Tehát legfeljebb 500 számot tudunk kiválasztani a feladatnak megfelelően. 
\end{solution}
\begin{extraproblem}[Szabó Kinga]
	Mutassuk meg, hogy ha az $1,2,3,\dots,2n$ számok közül kiválasztunk
	$n+1$ darab számot, azok között mindig lesz kettő olyan, hogy az
	egyik a másiknak osztója. \emph{(Róka Sándor: Válogatás Erdős Pál
		kedvenc feladataiból) }
\end{extraproblem}

\begin{solution}
	A $2n$ darab számot beosztjuk $n$ darab kupacba, amely kupacoknak
	megvan az a tulajdonsága, hogy bármelyik kupacból két számot véve
	az egyik osztja a másikat. Ha ezt meg tudjuk tenni, akkor bizonyítottuk
	az állítást. A kupacok a következők lesznek:
	
	\[
	[1,2,4,8,\dots],\quad[3,6,12,24,\dots],\quad[5,10,20,40,\dots],\dots
	\]
	
	$2n$-ig $n$ darab páratlan szám van, tehát $n$ darab kupacunk van,
	és a kupacokban $2n$-ig mindegyik számnak van helye.
\end{solution}
\begin{extraproblem}[Szabó Kinga]
	Felírtam néhány $24$-nél nem nagyobb pozitív egész számot a táblára.
	Ezekből bárhogyan választok néhányat (lehet egyet, vagy az összeset
	is), azok összegei különbözőek. Legfeljebb hány szám lehet a táblán?
	\emph{(Róka Sándor: Válogatás Erdős Pál kedvenc feladataiból)}
\end{extraproblem}

\begin{solution}
	$5$ számot könnyű találni, például: $1,2,4,8,16$.
	
	$6$ szám is választható: $11,17,20,22,23,24.$
	
	$8$ szám nem lehet a táblán, mert $8$ számból $2^{8}-1=255$ összeget
	képezhetünk, ám az összegek értéke legfeljebb $24+23+\dots+17=164$
	lehet. Így a skatulyaelv miatt vannak azonos értékű összegek.
	
	$7$ szám sem lehet a táblán. Tegyük fel, hogy mégis. Ha csak a kéttagú
	összegek különbözőségét nézzük, akkor két-két különböző szám különbsége
	nem lehet azonos. Emiatt a lehető legnagyobb számok $x,x-1,x-2,x-4,x-7,x-12,x-17$.
	Ezek összege, a legnagyobb összeg még $x=24$-re is kisebb $127$-nél,
	azonban $7$ számból $2^{7}-1=127$ összeget képezhetünk, így a skatulyaelv
	miatt vannak azonos értékű összegek.
\end{solution}
\begin{extraproblem}[Száfta Antal]
	17 tudós mindegyike levelezést folytat az összes többivel. Összesen
	háromféle témáról leveleznek, de bármelyik pár mindig csak ugyanarról
	az egy témáról. Bizonyítsuk be, hogy van közöttük legalább három olyan
	tudós, akik közül bármely kettő azonos témáról levelez egymással.
\end{extraproblem}

\begin{solution}
	\textit{Kiválasztva egy $T_{1}$ tudóst, ez legalább az egyik témáról
		legalább 6 tudóssal levelez}, hiszen különben legfeljebb 15 tudóssal
	levelezhetne. Jelöljük ezt a témát $t_{1}$-gyel, és csak azokkal
	a tudósokkal foglalkozzunk, akik $T_{1}$-gyel $t_{1}$-ről leveleznek.
	
	Ha ezek közt van kettő, akik egymással is $t_{1}$-ről leveleznek,
	akkor ezek és $T_{1}$ egy kívánt hármast alkotnak. Ha viszont ezek
	mind csak a másik két témáról leveleznek, akkor --- \textit{a skatulyaelv
		szerint} --- kiválaszthatunk közülük egy $T_{2}$ tudóst úgy, hogy
	ez legalább az egyik témáról --- mondjuk $t_{2}$-ről --- legalább
	3-mal levelez.
	
	Ha a három közül valamelyik kettő szintén $t_{2}$-ről levelez, akkor
	ez a kettő és $T_{2}$ csak a $t_{2}$ témáról leveleznek egymás között;
	ha pedig mindnyájan a harmadik, $t_{3}$ témáról leveleznek, akkor
	ők azok, akik hárman egymással ugyanarról a témáról, $t_{3}$-ról
	leveleznek, s így a bizonyítást befejeztük.
\end{solution}
\begin{extraproblem}[Szélyes Klaudia]
	Egy gulyában két falu 65 tehene legel, amelyek lehetnek vörösek,
	fehérek, feketék vagy tarkák. Igazoljuk, hogy ha nincs öt különböző
	korú, azonos színű tehén, akkor van három azonos színű, egyidős tehén
	ugyanabból a faluból.
\end{extraproblem}

\begin{solution}
	A bizonyításhoz a \textbf{skatulya-elvet} alkalmazzuk.
	
	\textbf{1. Skatulya meghatározása:}
	
	- A gulyában \textbf{65 tehén} van, amelyek \textbf{két faluból} származnak.
	
	- A tehenek \textbf{négy különböző színűek} lehetnek: \textit{vörös,
		fehér, fekete, tarka}.
	
	- Azt is tudjuk, hogy nincs \textbf{öt azonos korú és azonos színű
		tehén}.
	
	- Bizonyítani kell, hogy létezik \textbf{három azonos színű és egyidős
		tehén ugyanabból a faluból}.
	
	\textbf{Maximális különbözőség kiszámítása:}
	
	- Ha egy adott színt vizsgálunk, akkor az azonos színű tehenek legfeljebb
	\textbf{4 különböző korúak} lehetnek, hiszen 5 azonos korú tehén egy
	színből nem létezhet.
	
	- Ha $k$ a különböző életkorok száma egy adott szín esetén, akkor
	a négy szín összesen legfeljebb:
	
	\[
	4k\times4=16k
	\]
	
	tehenet tartalmazhat.
	
	- Mivel a gulya összesen 65 tehenet tartalmaz, ezért:
	
	\[
	16k\geq65
	\]
	
	- Innen következik, hogy:
	
	\[
	k\geq5
	\]
	
	tehát legalább \textbf{5 különböző életkor} létezik.
	
	\textbf{Három azonos színű és egyidős tehén ugyanabból a faluból:}
	
	- Mivel egy adott szín és adott korcsoport esetén legfeljebb \textbf{4
		tehén} lehet, és ezek két faluból származnak, ezért vagy \textbf{(2,2)
		arányban oszlanak meg}, vagy az egyik faluban legalább \textbf{3}
	tehén található.
	
	- Ha létezik olyan korcsoport és szín, ahol az egyik faluban legalább
	három tehén van, akkor megtaláltuk a kívánt \textbf{három azonos színű,
		egyidős tehén ugyanabból a faluból}.
	
	\textbf{Következtetés:} A feladat állítása igaz, tehát mindig létezik
	\textbf{három azonos színű és egyidős tehén ugyanabból a faluból}.
\end{solution}
\begin{extraproblem}[Szélyes Klaudia]
	100 kavicsot 50 kupacba rendezünk úgy, hogy egyik kupac sem üres.
	Igazoljuk, hogy a kupacok két csoportba oszthatók úgy, hogy a két
	csoportban egyenlő számú, azaz 50-50 kavics legyen.
\end{extraproblem}

\begin{solution}
	A feladat egy \textbf{kombinatorikus skatulya-elv} alkalmazására épülő
	bizonyítás.
	
	\textbf{1. Két csoportba osztás lehetősége}
	
	Mivel az összes kavics száma \textbf{100}, és a célunk az, hogy ezt
	két egyenlő részre, azaz \textbf{50-50 kavicsra} osszuk, olyan megfelelő
	kupacválogatást kell találnunk, amely ezt biztosítja.
	
	\textbf{2. Az elrendezések vizsgálata}
	
	Tekintsük a kupacok bármely lehetséges elosztását két tetszőleges
	csoportba. Ekkor az egyes csoportokban lévő kavicsszámokat egy \textbf{egész
		számok halmazaként} értelmezhetjük, amelyeket különböző módokon választhatunk
	ki a kupacok közül.
	
	Az összes kavics számának fele \textbf{50}, így olyan részhalmazt
	keresünk a kupacok kavicsszámaiból, amelynek az összege \textbf{50}.
	
	\textbf{3. Skatulya-elv alkalmazása:}
	
	A \textbf{Szűkített összeghalmaz-elv} szerint ha van egy halmazunk
	($S$) pozitív egész számokból, amelyek összege 100, akkor mindig
	létezik egy részhalmaz, amelynek összege pontosan \textbf{50}.
	
	Ez abból következik, hogy az egyes kupacok kavicsszámait növekvő sorrendben
	összeadva egyszer el kell érnünk az 50-et, vagy ha túllépnénk azt,
	akkor előtte léteznie kell egy olyan részösszegnek, amely pontosan
	\textbf{50}.
	
	Mivel a kupacokban lévő kavicsok száma pozitív egész szám, és összegük
	pontosan 100, mindig találunk egy olyan megfelelő kiválasztást, amely
	kielégíti a feladat feltételeit.
	
	\textbf{Következtetés:} Mindig létezik egy olyan két csoportra bontás,
	amely pontosan 50-50 kavicsot tartalmaz.
\end{solution}