
\chapter{Variációk, permutációk, kombinációk}\label{chap:perm}
\begin{description}
	{\large \item [{Szerző:}] Kiss Andrea-Tímea (Korszerű módszerek a matematikatanításban, Didaktikai mesteri -- Matematika, 
		II. év)}
	
\end{description}
%Könyvészet: \cite{diszkretmatek,andreescu2013102,ASzi}
\begin{problem}[\cite{ASzi}]
\label{fel1} Egy polcon $5$ különböző trófeát kell elhelyeznünk.
Hányféleképpen tehetjük meg ezt? 
\end{problem}

\begin{problem}
\label{fel2} Hány különböző számjegyekből álló négyjegyű természetes
szám képezhető az $1,5,7,9$ számjegyekből? 
\end{problem}

\begin{problem}
\label{fel3} Móniék a moziban a harmadik sor 10., 11., 12., 13.,
14., 15. helyeit foglalták le. Hányféleképpen foglalhatja el a helyét
Móni és öt barátja? 
\end{problem}

\textit{Oldjuk meg a fenti feladatokat, majd fogalmazzuk meg a feladványokban
rejlő közös tulajdonságot!}
\begin{solution}
{\bf A \ref{fel1}. gyakorlat megoldása:}

Feltételezzük, hogy a polcon balról jobbra a trófeák pozíciója 1.,
2., 3., 4. és 5. Először vizsgáljuk meg az egyszerűbb eseteket.
\begin{itemize}
\item Ha 1 trófeánk van, akkor ezt egyféleképpen tehetjük fel a polcra:
T1. 
\item Ha 2 trófeánk van, akkor ezeket két különböző sorrendben helyezhetjük
el: T1, T2 vagy T2, T1. 
\item Ha 3 trófea van, akkor az előző két trófeához hozzá kell rakni valahova
a T3-at, amit háromféleképpen tehetünk meg: az előző két trófea elé,
mögé, vagy közéjük. Így az előbbi 2 sorrend mindegyikéből 3 különböző
három trófeát tartalmazó sor készíthető: $2\cdot3=6$ lehetséges sorrend. 
\item Ha 4 trófea van, akkor a T4 trófeát be kell illeszteni az előző három
trófea által alkotott sorba, amit négyféleképpen tehetünk meg: a sor
elejére, az 1. és 2. trófea közé, a 2. és 3. közé, vagy a sor végére.
Ezért az előbbi 6 sorrend mindegyikéből 4 különböző négy trófeát tartalmazó
sor készíthető: $6\cdot4=24$ lehetséges sorrend. 
\item Ha 5 trófeánk van, akkor a T5 trófeát az előző négy trófeát tartalmazó
sorba öt különböző helyre tehetjük. Így az előző 24 sorrend mindegyikéből
5 különböző sorrend alakítható: $24\cdot5=120$ lehetséges sorrend. 
\end{itemize}
Tehát öt trófea 120-féleképpen rendezhető sorrendbe.

\textit{Megjegyzés:} Ha $t_{n}$-nel jelöljük $n$ különböző trófea
összes lehetséges sorba rendezésének számát, akkor fennáll az 
\begin{equation}
t_{n+1}=(n+1)\cdot t_{n},\forall n\in\mathbb{N}^{*}\label{permutacio_rekurzio}
\end{equation}
rekurzív összefüggés, mert minden $n$ trófeát tartalmazó sor esetén
az $(n+1).$ trófeát $n+1$ lehetséges helyre szúrhatjuk be, így $t_{n}$
sorrend mindegyikéből $n+1$ különböző $n+1$ trófeát tartalmazó sor
alakítható ki.

{\bf A \ref{fel2}. gyakorlat megoldása:}

Azokat az $\overline{abcd}$ négyjegyű természetes számokat keressük,
amelyekre az $a,b,c,d$ számjegyek páronként különbözőek és $a,b,c,d\in\{a,b,c,d\}$.

Ekkor az $\overline{abcd}$ négyjegyű szám esetén az $a$ számjegy
$4$ féle lehet. Ezen a négy eset mindegyikében a $b$ számjegynek
már csak 3 félét tudunk választani, ezért az első két számjegyet $4\cdot3=12$-féleképpen
lehet lerögzíteni.

Amikor a $c$ számjegyet szeretnénk lerögzíteni, akkor figyelembe
kell vegyük, hogy az $1,5,7,9$ számjegyekből már kettőt felhasználtunk
az első két számjegy rögzítésénél. Az előbbi $12$ darab $\overline{ab}$
szám mindegyikéhez így a $c$ számjegyet már csak kétféleképpen tudjuk
megválasztani, ezért összesen $12\cdot2=24$ darab $\overline{abc}$
számot tudok képezni.

A négy számjegy közül a rögzítéshez még csak egyet nem használtunk
fel, ezért ez egyértelműen az utolsó pozícióra kerül, így az így képezhető
$\overline{abcd}$ számok számossága $24$ lesz.

\textit{Megjegyzés}: Győződjünk meg arról, hogy a számjegyek rögzítésének
sorrendjétől eltekinthetünk. Akármilyen sorrendbe is rögzítsük a számjegyeket,
mindig ugyanaz a $24$ négyjegyű számot képezzük.

{\bf A \ref{fel3}. gyakorlat megoldása:}

Feltételezzük, hogy a hat barát a helyüket balról jobbra egyesével
foglalják el.

A 10. helyre kezdetben bárki leülhet a hat barát közül, viszont a
11. helyre már csak a maradék öt állva maradt ember közül valaki.
A 12. helyre már csak a 10. és 11. helyeket nem elfoglaló négy barát
közül valaki ülhet, a 13. helyre hasonló gondolatmenettel három, a
14. helyre kettő, valamint a 15. helyre már csak az egyedül állva
maradt személy ülhet.

Bárki is kezdje a sort és folytassa tovább, a lehetőségek száma mindig
a fent leírtak alapján alakul. Így $6\cdot5\cdot4\cdot3\cdot2\cdot1=720$-féleképpen
tudja elfoglalni a helyét a moziban a hat barát. 
\end{solution}
\textit{Az előző három feladat mindegyikében $n$ különböző elem összes
lehetséges sorba rendezéseinek számát vizsgáltuk.}

\begin{definition}{permutacio}
Valamely $n$ elemű $A$ halmaz összes eleméből szerkesztett rendezett
halmazt az $A$ halmaz egy \textbf{permutáció}jának nevezzük.
\end{definition}

Jelöljük $n$ különböző elem összes sorba rendezéseinek számát, vagyis
valamely $n$ elemű $A$ halmaz összes permutációinak számát \textbf{$P_{n}$}-nel.
(\textit{n-ed rendű permutáció})
\begin{theorem}{perm}
\[
P_{n}=n!,n\in\mathbb{N},\text{ ahol }n!=1\cdot2\cdot\ldots\cdot n
\]
\end{theorem}

\textit{Megjegyzés}: 
\begin{itemize}
\item $P_{0}=1$ 
\end{itemize}
\begin{proof}
\Aref{permutacio_rekurzio}-es összefüggés alapján itt is érvényes
lesz a 
\[
P_{k+1}=(k+1)\cdot P_{k},\forall k\in\mathbb{N}^{*}
\]
rekurzív összefüggés. 
\[
\begin{array}{llcl}
k=n-1\text{ -re:} & P_{n} & = & n\cdot P_{n-1}\\
k=n-2\text{ -re:} & P_{n-1} & = & (n-1)\cdot P_{n-2}\\
k=n-3\text{ -re:} & P_{n-2} & = & (n-2)\cdot P_{n-3}\\
 & \vdots\\
k=2\text{ -re:} & P_{3} & = & 3\cdot P_{2}\\
k=1\text{ -re:} & P_{2} & = & 2\cdot P_{1}
\end{array}
\]
egyenlőségek megfelelő oldalait összeszorozva kapjuk, hogy 
\[
\begin{array}{c}
P_{n}\cdot P_{n-1}\cdot P_{n-2}\cdot\ldots\cdot P_{2}=n\cdot(n-1)\cdot(n-2)\cdot\ldots\cdot3\cdot2\cdot P_{1}\cdot P_{n-1}\cdot P_{n-2}\cdot\ldots\cdot P_{2}\\
\Leftrightarrow P_{n}=1\cdot2\cdot3\cdot\ldots\cdot(n-1)\cdot n=n!
\end{array}
\]
\end{proof}
\begin{problem}[\cite{ASzi}]
\label{fel4} Egy hattagú társaság tagjai egy titkárt és egy írnokot
kell kiválasszanak. Hány különböző módon tehetjük meg? Ha az írnok
és titkár mellé egy küldöncöt is kell választanunk, akkor ez hányszorosára
növeli a lehetséges választások számát? 
\end{problem}

\begin{problem}[\cite{diszkretmatek}]
\label{fel5} Egy versenyen 100-an vesznek részt, de csupán az első
tíz sorrendjét rögzítik. Hányféle (a krónikákban nyilvántartott) eredménye
lehet a versenynek? 
\end{problem}

\textit{Oldjuk meg a fenti feladatokat, majd fogalmazzuk meg a feladványokban
rejlő közös tulajdonságot!}
\begin{solution}
{\bf\ref{fel4}. gyakorlat megoldása:}

A titkárt hat ember közül kell kiválasztanunk, tehát erre 6 lehetőségünk
van. Minden kiválasztott titkárhoz a megmaradt öt tag közül bárkit
választhatunk írnoknak, tehát összesen $6\cdot5=30$ lehetőségünk
van.

Ha a társaság tagjait 1-től 6-ig megszámozzuk, és a számpárok első
tagja a titkárt, míg a második az írnokot azonosítja, akkor a következő
felsorolás az összes lehetséges esetet mutatja be: 
\[
\begin{array}{cccccc}
 & (2,1) & (3,1) & (4,1) & (5,1) & (6,1)\\
(1,2) &  & (3,2) & (4,2) & (5,2) & (6,2)\\
(1,3) & (2,3) &  & (4,3) & (5,3) & (6,3)\\
(1,4) & (2,4) & (3,4) &  & (5,4) & (6,4)\\
(1,5) & (2,5) & (3,5) & (4,5) &  & (6,5)\\
(1,6) & (2,6) & (3,6) & (4,6) & (5,6)
\end{array},
\]
amely számpárok megadják az $\{1,2,3,4,5,6\}$ halmaz összes kételemű
rendezett részhalmazát. Vegyük észre, hogy nem számít az, hogy melyik
pozícióra melyik szerepet tesszük, mert a táblázat szimmetrikus. Tehát
ha az $(a,b)$ számpár megjelenik, akkor a $(b,a)$ is.

A küldöncöt a titkár és az írnok minden lehetséges választása esetén
a megmaradt négy tag közül kell kiválasztanunk, ez a négyszeresére
növeli a lehetséges választások számát.

{\bf\ref{fel5}. gyakorlat megoldása:}

A $100$ versenyző közül 10-et kell kiválasztanunk úgy, hogy a kiválasztási
sorrend számít.

A válasz a már ismert gondolatmenetet követi. A 100 versenyző közül
bármelyik lehet első. A második helyre már csak 99-en pályázhatnak,
az első két helyre tehát $100\cdot99$-féleképpen futhatnak be. Ha
az első kettő már megvan, a harmadik helyet $98$-an szerezhetik meg,
és így tovább. Ezt a gondolatmenetet folytatva, az első tíz helyen
$100\cdot99\cdot\ldots\cdot91$-féleképpen végezhettek a versenyzők.

Ebben az esetben, ha a versenyzőket 1-től 100-ig megszámozzuk, és
a pozíciók a helyezések számát jelölik, akkor tulajdonképpen az $\{1,2,\ldots,100\}$
halmaz tízelemű rendezett részhalmazainak számát keressük. 
\end{solution}
\textit{Az előző két feladatban $n$ különböző elemből kiválasztott
$k$ elem összes lehetséges sorba-rendezéseinek számát vizsgáltuk.}

\begin{definition}{def:var}
Valamely $n$ elemű $A$ halmaz egy $k$ elemű rendezett részhalmazát
az $A$ halmaz egy \textbf{$k$-ad osztályú variáció}jának nevezzük.
\end{definition}

Jelöljük egy $n$ elemű halmaz $k$-ad osztályú variációinak számát
$V_{n}^{k}$-val. (\textit{variáció $n$ alapon $k$})
\begin{theorem}{thm:var}
\[
V_{n}^{k}=n\cdot(n-1)\cdot(n-2)\cdot\ldots\cdot(n-k+1)=\frac{n!}{(n-k)!},\forall k,n\in\mathbb{N},0\leq k\leq n.
\]
\end{theorem}

\textit{Megjegyzés:} 
\begin{itemize}
\item Ha $k>n$, akkor $V_{n}^{k}=0$. 
\item $V_{n}^{n}=P_{n},\forall n\in\mathbb{N}$. 
\item $V_{n}^{0}=1$, $\forall n\in\mathbb{N}$. 
\end{itemize}
\begin{proof}
Jelöljük $A$-val az $n$ elemű halmazunkat. Minden $k$ elemű $X$
rendezett részhalmazból pontosan $(n-k)$ darab $(k+1)$ elemű rendezett
részhalmazt szerkeszthetünk, ha az $X$-hez $(k+1)$-edik elemként
hozzáadjuk az $A\setminus X$ valamelyik elemét. Így minden $(k+1)$
elemű rendezett részhalmazt megkapunk, tehát 
\[
V_{n}^{k+1}=V_{n}^{k}\cdot(n-k),\forall n\in\mathbb{N}^{*},k\in\mathbb{N},0\leq k\leq n.
\]
A fenti rekurzív összefüggést írjuk fel $k=k,k-1,\ldots,2,1$-re:
\[
\begin{array}{llcl}
k=k-1\text{ -re:} & V_{n}^{k} & = & (n-k+1)\cdot V_{n}^{k-1}\\
k=k-2\text{ -re:} & V_{n}^{k-1} & = & (n-k+2)\cdot V_{n}^{k-2}\\
k=k-3\text{ -re:} & V_{n}^{k-2} & = & (n-k+3)\cdot V_{n}^{k-3}\\
 & \vdots\\
k=2\text{ -re:} & V_{n}^{3} & = & (n-2)\cdot V_{n}^{2}\\
k=1\text{ -re:} & V_{n}^{1} & = & (n-1)\cdot V_{n}^{1}
\end{array}.
\]
A fenti egyenlőségek megfelelő oldalait összeszorozva kapjuk, hogy
\[
\begin{array}{c}
V_{n}^{k}\cdot V_{n}^{k-1}\cdot\ldots\cdot V_{n}^{2}=V_{n}^{1}\cdot(n-1)\cdot(n-2)\cdot\ldots\cdot(n-k+1)\cdot V_{n}^{k-1}\cdot V_{n}^{k-2}\cdot\ldots\cdot V_{n}^{2}\\
\Leftrightarrow V_{n}^{k}=n\cdot(n-1)\cdot(n-2)\cdot\ldots\cdot(n-k+1)=\dfrac{n!}{(n-k)!}.
\end{array}
\]
\end{proof}
\begin{problem}[\cite{ASzi}]
\label{fel6} A terroristák hat túsz közül hármat szabadon akarnak
engedni. Hány különböző módon lehetséges ez? 
\end{problem}

\begin{problem}
\label{fel7} Egy vendégségbe hét vendég érkezett. A vendégek egymást
egyszeri kézfogással üdvözölték. Hány kézfogás történt? 
\end{problem}

\begin{problem}[\cite{ASzi}]
\label{fel8} Egy ötöslottón a 90 számból húznak ki ötöt. Hányféleképpen
tölthető ki a lottószelvény? 
\end{problem}

\textit{Oldjuk meg a fenti feladatokat, majd fogalmazzuk meg a feladványokban
rejlő közös tulajdonságot!}
\begin{solution}
{\bf\ref{fel6}. gyakorlat megoldása:}

Számozzuk meg a túszokat 1-től 6-ig. Mivel a túszokat egyszerre engedik
szabadon, itt nem a rendezett számhármasokat kell megszámolnunk, hanem
az $\{1,2,3,4,5,6\}$ halmaz háromelemű részhalmazait.

Mivel az $\{1,2,3,4,5,6\}$ halmaznak $V_{6}^{3}=\dfrac{6!}{(6-3)!}=6\cdot5\cdot4=120$
rendezett háromelemű részhalmaza van, és három különböző elemet $P_{3}=3!=6$-féleképpen
lehet sorba rendezni. Ez azt jelenti, hogy minden $\{a,b,c\}$ háromelemű
részhalmazt a rendezett részhalmazok között pontosan hatszor számolunk
meg: $(a,b,c),(a,c,b),(b,a,c),(b,c,a),(c,a,b),(c,b,a)$.

Ezért $\dfrac{V_{6}^{3}}{P_{3}}=\dfrac{120}{6}=20$ háromelemű részhalmaza
van az $\{1,2,3,4,5,6\}$ halmaznak, ami alapján a túszokat is 20-féleképpen
engedhetik szabadon a terroristák.

{\bf\ref{fel7}. gyakorlat megoldása:}

A hét vendég közül mindig egyszerre csak kettő fog kezet egymással,
és páronként csak egyszer fognak kezet. Ezért a fenti kérdésnek megfelel
a következő feladvány: A hét vendég közül hányféleképpen választható
ki egyszerre két vendég?

Ha számítana a vendégek kiválasztásának sorrendje, akkor $V_{7}^{2}=\dfrac{7!}{(7-2)!}=7\cdot6=42$-féleképpen
lehetne összesen kiválasztani 7 vendégből 2-t. Viszont bármely $A$
és $B$ vendég kézfogása az $(A,B)$ és $(B,A)$ rendezett részhalmazzal
is leírható, ezért a fenti összeszámlálásban minden kézfogást kétszer
számoltunk.

Ezek alapján a kézfogások száma $\dfrac{42}{2}=21$ lesz.

{\bf\ref{fel8}. gyakorlat megoldása:}

Ha az $\{1,2,3,\ldots,90\}$ halmaz rendezett ötelemű részhalmazaira
lennénk kíváncsiak, akkor a válasz $V_{90}^{5}$ lenne.

Viszont a számok kiválasztásának sorrendjét nem kell figyelembe vegyük,
így a lottószelvényen bejelölt öt szám $P_{5}$ lehetséges sorbarendezései
mind ugyanazon kitöltési lehetőségnek bizonyulnak.

Ezért $\dfrac{V_{90}^{5}}{P_{5}}=\dfrac{\dfrac{90!}{(90-5)!}}{5!}=\dfrac{90!}{85!\cdot5!}$-féleképpen
tölthető ki a lottószelvény. 
\end{solution}
\textit{Az előző három feladatban $n$ különböző elemből $k$ elem
összes lehetséges kiválasztásának számát vizsgáltuk úgy, hogy a kiválasztási
sorrendet nem vettük figyelembe.}

\begin{definition}{kombinacio}
Valamely $n$ elemű $A$ halmaz egy $k$ elemű részhalmazát az $A$
halmaz egy \textbf{$k$-ad osztályú kombináció}jának nevezzük.
\end{definition}

Jelöljük egy $n$ elemű halmaz $k$-ad osztályú kombinációinak számát
$C_{n}^{k}$-val. (\textit{kombináció $n$ alapon $k$})
\begin{theorem}{komb}
\[
C_{n}^{k}=\dfrac{V_{n}^{k}}{P_{k}}=\dfrac{n!}{(n-k)!k!},\forall n,k\in\mathbb{N},0\leq k\leq n.
\]
\end{theorem}

\textit{Megjegyzés:} 
\begin{itemize}
\item Ha $k>n$, akkor $C_{n}^{k}=0$. 
\item $C_{n}^{0}=1$, $\forall n\in\mathbb{N}$ esetén. 
\item $C_{n}^{k}=C_{n}^{n-k}$, $\forall n,k\in\mathbb{N},0\leq k\leq n.$ 
\end{itemize}
\begin{proof}
Egy $n$ elemű halmaz rendezett $k$ elemű részhalmazainak számát
megadja $V_{n}^{k}$. Ebben minden ,,rendezetlen'' részhalmaz többszörösen
is szerepel, méghozzá annyiszor, ahányféleképpen $k$ elem sorba rendezhető
($P_{k}$).

Így ha a rendezett $k$ elemű részhalmazok számát elosztjuk $P_{k}$-val,
akkor éppen a $k$ elemű részhalmazok számát kapjuk, vagyis $C_{n}^{k}=\dfrac{V_{n}^{k}}{P_{k}}$. 
\end{proof}

\section*{Házi feladatok}
\begin{problem}
Határozzuk meg az $\{a_{1},a_{2},\ldots,a_{100}\}$ halmaz azon részhalmazainak
számát, amelynek az $\{a_{1},a_{5}\}$ halmaz részhalmaza, de az $a_{7}$
nem eleme! 
\end{problem}

\begin{solution}
Keressük azokat az $X\subseteq\{a_{1},a_{2},\ldots,a_{100}\}$ részhalmazokat,
ahol $\{a_{1},a_{5}\}\subseteq X$ és $a_{7}\notin X$. Ekkor tulajdonképpen
az $X$ halmaz felírható, mint $X=Y\cup\{a_{1},a_{5}\}$, ahol $Y\subseteq\{a_{2},a_{3},a_{4},a_{6},a_{8},a_{9},\ldots a_{100}\}$
tetszőleges részhalmaz.

Mivel $|\{a_{2},a_{3},a_{4},a_{6},a_{8},a_{9},\ldots,a_{100}\}|=|\{a_{1},a_{2},\ldots,a_{100}\}\setminus\{a_{1},a_{5},_{7}\}|=100-3=97$,
ezért $0\leq|Y|\leq97$. 
\begin{itemize}
\item ha $|Y|=0$, akkor az $Y$ halmaz $C_{97}^{0}$-féle lehet 
\item ha $|Y|=1$, akkor az $Y$ halmaz $C_{97}^{1}$-féle lehet 
\item ha $|Y|=2$, akkor az $Y$ halmaz $C_{97}^{2}$-féle lehet 
\item $\vdots$ 
\item ha $|Y|=97$, akkor az $Y$ halmaz $C_{97}^{97}$-féle lehet 
\end{itemize}
Így az $Y$ halmaz összesen $C_{97}^{0}+C_{97}^{1}+\ldots+C_{97}^{97}$
-féle lehet. Hozzuk ezt az összeget zárt formára. Az $Y$ halmaz az
$\{a_{2},a_{3},a_{4},a_{6},a_{8},a_{9},\ldots,a_{100}\}$ $97$ elemű
halmaz tetszőleges részhalmaza, amiknek száma $2^{97}$. Ezért 
\[
C_{97}^{0}+C_{97}^{1}+\ldots+C_{97}^{97}=2^{97},
\]

ami alapján az $Y$, és így az $X$ halmaz is $2^{97}$-féle lehet.

\textit{Megjegyzés:} A fenti összefüggés általánosítható: 
\[
C_{n}^{0}+C_{n}^{1}+\ldots+C_{n}^{n}=2^{n},\forall n\in\mathbb{N}.
\]
\end{solution}
\begin{problem}[\cite{ASzi}]
Egy osztályban $13$ lány és $12$ fiú tanul. Találomra kiválasztunk
két lányt és egy fiút versmondásra, két lányt és egy fiút színházjegyvásárlásra,
valamint két lányt és egy fiút tűzoltósági felkészítőre. Hányféleképpen
állíthatunk össze ilyen kilenctagú csapatot? 
\end{problem}

\begin{solution}
A versmondásra a két lányt $C_{13}^{2}$, az egy fiút $C_{12}^{1}$
módon választhatjuk ki, tehát összesen $C_{13}^{2}\cdot C_{12}^{1}$
különböző versmondó csapatot tudunk összeállítani.

A megmaradt $13-2=11$ lány és $12-1=11$ fiú közül színházjegyvásárlásra
a két lányt $C_{11}^{2}$, az egy fiút pedig $C_{11}^{1}$ módon választhatjuk
ki, így összesen $C_{11}^{2}\cdot C_{11}^{1}$ különböző jegyvásárló
csapat állítható össze.

A többiekből, vagyis a $11-2=9$ lány és $11-1=10$ fiú közül a tűzoltósági
felkészítőre a két lányt $C_{9}^{2}$, az egy fiút pedig $C_{10}^{1}$
módon választhatjuk ki, így összesen $C_{9}^{2}\cdot C_{10}^{1}$
különböző felkészítőre járó csapatot állíthatunk össze.

Tehát a kilencfős csapat $C_{13}^{2}\cdot C_{12}^{1}\cdot C_{11}^{2}\cdot C_{11}^{1}\cdot C_{9}^{2}\cdot C_{10}^{1}=\dfrac{13\cdot12}{2}\cdot12\cdot\dfrac{11\cdot10}{2}\cdot11\cdot\dfrac{9\cdot8}{2}\cdot10=13\cdot12^{2}\cdot11^{2}\cdot10^{2}\cdot9$
különböző összetétellel rendelkezhet. 
\end{solution}
\begin{problem}[\cite{ASzi}]
\label{fel11} Az $1,2,3,4,5,6,7,8,9$ számjegyekből hány darab olyan
kilencjegyű szám alkotható, amelynek számjegyei egymástól különbözőek,
és 
\begin{enumerate}
\item[a)] az $1$-es közvetlenül a $2$-es előtt áll, 
\item[b)] az $1$-es előbb áll, mint a $2$-es? 
\item[a)] Mivel az $1$-es közvetlenül a $2$-es előtt áll, ezért a $12$-t
tekintsük egy ,,számjegynek''.

Így azokat a ,,nyolcjegyű'', különböző ,,számjegyekből'' felépülő
természetes számokat számát keressük, amelyek az $12,3,4,5,6,7,8,9$
,,számjegyekből'' alakíthatóak.

Minden ilyen számot megkapunk az $12,3,4,5,6,7,8,9$ számok összes
permutációjaként, ezért $P_{8}=8!$ ilyen szám létezik.
\item[b)] \textit{1. megoldás:} A keresett kilencjegyű szám számjegyei közül
két pozícióra rakjuk be az $1$ és $2$ számjegyeket. Ezt a két pozíciót
a kilencből $C_{9}^{2}=\dfrac{9\cdot8}{2}$-féleképpen választhatjuk
ki. Viszont ekkor ezekben az esetekben a két pozíció közül az elsőre
rakjuk az $1$-est, míg a másodikra a $2$-est, a feltételeknek megfelelően.

A további $3,4,5,6,7,8,9$ hét számjegyet a maradék hét pozícióra
$P_{7}$-féleképpen helyezhetem el.

Ezért a keresett számok száma $C_{9}^{2}\cdot P_{7}=\dfrac{9!}{2}$.

\textit{2. megoldás:} Minden $\varphi$ permutációhoz hozzárendeljük
azt, amelyiket a $\varphi$-ből úgy kapunk, hogy az $1$-est és a
$2$-est felcseréljük. Így az $1,2,3,4,5,6,7,8,9$ számjegyek permutációit
kettesével csoportosíthatjuk. Mivel minden csoportból pontosan egy
származtat nekünk megfelelő számot, ezért a kívánt számok száma $\dfrac{P_{9}}{2}=\dfrac{9!}{2}$. 

\end{enumerate}
Hány különböző módon lehet $10$ embert egy kerek asztal köré leültetni? 
\end{problem}

\begin{solution}
Ha a helyek $1$-től $10$-ig meg lennének számozva, akkor erre a
tíz helyre a tíz embert $P_{10}$-féleképpen lehetne leültetni.

Jelöljük az ülésrendet $(a_{1},a_{2},\ldots,a_{10})$-zel, ahol a
képzeletbeli helyszámozásoknak megfelelően az $i$-edik helyen az
$a_{i}$ személy ül. Mivel a székek nincsenek megszámozva, ezért a
kerek asztalnál az 
\[
\begin{array}{c}
(a_{1},a_{2},\ldots,a_{10})\\
(a_{10},a_{1},a_{2},\ldots,a_{9})\\
(a_{9},a_{10},a_{1},a_{2},\ldots,a_{8})\\
\vdots\\
(a_{2},a_{3},\ldots,a_{10},a_{1})
\end{array}
\]
rendezett halmazok ugyanazt az ültetést származtatják, mert az asztal
körül csak az számít, hogy ki kinek a szomszédja.

Ezért minden lehetséges ülésrend $10$ permutációt származtat, ezért
az ülésrendek száma $\dfrac{P_{10}}{10}=9!$. 
\end{solution}
\begin{problem}[\cite{andreescu2013102}]
Egy pók mind a nyolc lábára egy-egy zoknit és cipőt húz. Hány különböző
sorrendben veheti fel a zoknijait és cipőit, ha minden lábán először
a zoknit kell felhúznia a cipő előtt? 
\end{problem}

\begin{solution}
Számozzuk meg a pók lábait $1$-től $8$-ig, és jelöljük $a_{k}$-val
a zoknit és $b_{k}$-val a cipőt, amely a $k$-adik lábára kerül.

A zoknik és cipők egy lehetséges sorrendje a $16$ szimbólum ($a_{1},b_{1},a_{2},b_{2},\ldots,a_{8},b_{8}$)
egy lehetséges permutációja, amelyben minden $a_{k}$ megelőzi a hozzátartozó
$b_{k}$-t, minden $1\leq k\leq8$ esetén.

A $16$ szimbólum összes lehetséges permutációjának száma $P_{16}=16!$.
\Aref{fel11}-es gyakorlat megoldásában használt gondolatmenet alapján,
az összes permutáció felében az $a_{1}$ megelőzi $b_{1}$-et, azaz
$\dfrac{16!}{2}$ permutációban.

Hasonlóan, az $a_{2}$ megelőzi a $b_{2}$-t ezeknek a permutációknak
pontosan felében, azaz $\dfrac{16!}{2^{2}}$ permutációban.

Ezt folytatva arra a következtetésre jutunk, hogy minden $a_{k}$
megelőzi a hozzátartozó $b_{k}$-t pontosan $\dfrac{16!}{2^{8}}$
permutációban. Ezért a pók összesen $\dfrac{16!}{2^{8}}$ módon tud
felöltözni. 
\end{solution}
\begin{problem}[\cite{andreescu2013102}]
Hányféleképpen lehet kiválasztani az első $18$ pozitív egész szám
közül ötöt úgy, hogy ezek közül bármely két szám közötti különbség
(abszolút értéke) legalább $2$ legyen? 
\end{problem}

\begin{solution}
Legyenek az $a_{1}<a_{2}<a_{3}<a_{4}<a_{5}$ a választott számok.

Tudjuk, hogy $a_{1},a_{2},a_{3},a_{4},a_{5}\in\{1,2,3,\ldots,18\}$
és az $a_{1},a_{2},a_{3},a_{4},a_{5}$ bármely két szám közötti különbség
(abszolút értéke) legalább $2$.

Az $a_{1}<a_{2}<a_{3}<a_{4}<a_{5}$ növekvő sorrend és a fent leírtak
alapján az $a_{1}<a_{2}-1<a_{3}-2<a_{4}-3<a_{5}-4$ növekvő sorrend
szintén fennáll, valamint ekkor a $(b_{1},b_{2},b_{3},b_{4},b_{5})=(a_{1},a_{2}-1,a_{3}-2,a_{4}-3,a_{5}-4)$
számokra teljesülnek a $b_{1}<b_{2}<b_{3}<b_{4}<b_{5}$ és $b_{1},b_{2},b_{3},b_{4},b_{5}\in\{1,2,3,\ldots,14\}$
tulajdonságok.

A $(b_{1},b_{2},b_{3},b_{4},b_{5})$ rendezett halmaznak, ahol $b_{1}<b_{2}<b_{3}<b_{4}<b_{5}$,
tulajdonképpen megfelel az $\{1,2,3,\ldots,14\}$ halmaz egy ötelemű
részhalmaza, mivel a halmaz elemeinek sorrendje egyértelmű.

Mivel az $\{1,2,3,\ldots,14\}$ halmaznak $C_{14}^{5}=\dfrac{14!}{(14-5)!5!}=2002$
ötelemű részhalmaza van összesen, ezért a fenti feltételeknek megfelelően
a $(b_{1},b_{2},b_{3},b_{4},b_{5})$ számötös is $2002$-féle lehet
összesen.

A 
\[
\begin{array}{c}
(b_{1},b_{2},b_{3},b_{4},b_{5})=(a_{1},a_{2}-1,a_{3}-2,a_{4}-3,a_{5}-4)\\
\Leftrightarrow(a_{1},a_{2},a_{3},a_{4},a_{5})=(b_{1},b_{2}+1,b_{3}+2,b_{4}+3,b_{5}+4)
\end{array}
\]
kölcsönosen egyértelmű megfeleltetés alapján az első $18$ pozitív
egész szám közül $2002$-féle módon választhatjuk ki az $a_{1}<a_{2}<a_{3}<a_{4}<a_{5}$
számokat úgy, hogy bármely két kiválasztott szám különbsége legalább
$2$ legyen. 
\end{solution}

\section*{Nehezebb feladatok}
\begin{extraproblem}[Domokos Ábel]
Legyen $n\in\mathbb{N}^{*}$ és $A=\{1,2,\dots,n\}$. Határozd meg
azon $f:A\to A$ növekvő függvények számát, amelyekre $|f(x)-f(y)|\leq|x-y|$,
$\forall x,y\in A$. \emph{(Olimpiada Națională de Matematică, Etapa
finală, Problema 3, 2014)}
\end{extraproblem}

\begin{solution}
Vegyük észre, hogy $f(k+1)-f(k)\leq|k+1-k|=1$. Vezessük be a következő
jelölést: $a_{k}=f(k+1)-f(k)$, ekkor $a_{k}\in\{0,1\},\quad k=1,2,\dots,n-1$.

Rögzítsük le az $f(0)=a$, $f(n)=b$ értékeket. Ekkor kapjuk, hogy\\
 $b-a=(f(n)-f(n-1))+(f(n-1)-f(n-2))+\dots+(f(1)-f(0))=a_{n-1}+a_{n-2}+\dots+a_{0}$.
A kívánt egyenlőség eléréséhez $b-a$ tagjának 1-nek kell lennie az
összegből, a többi pedig 0. Mindezt $C_{n-1}^{b-a}$ módon valósíthatjuk
meg.

Az $f$ függvény megszerkesztéseinek száma adott $a$, $b$ értékek
esetén tehát $C_{n-1}^{b-a}$, de az $a$, $b$ számokat sokféleképpen
választhatjuk. Ezek kiválasztásait is, ha figyelembe vesszük, akkor
megkapjuk az összes lehetséges $f$ függvényt. A számolások a következőképpen
alakulnak: 
\begin{align*}
\sum_{0\leq a\leq b\leq n}C_{n-1}^{b-a} & =\sum_{k=0}^{n-1}(n-k)C_{n-1}^{k}=\sum_{k=0}^{n-1}nC_{n-1}^{k}-\sum_{k=0}^{n-1}kC_{n-1}^{k}\\
 & =n\sum_{k=0}^{n-1}C_{n-1}^{k}-\sum_{k=1}^{n-1}kC_{n-1}^{k}=n2^{n-1}-\sum_{k=1}^{n-1}kC_{n-1}^{k}\\
 & =n2^{n-1}-\sum_{k=1}^{n-1}(n-1)C_{n-2}^{k-1}\\
 & =n2^{n-1}-(n-1)2^{n-2}\\
 & =(n+1)2^{n-2}
\end{align*}
\end{solution}

\begin{extraproblem}[Csapó Hajnalka]
Hány olyan téglalap található az alábbi ábrákon, amelyek oldalai
rácsvonalak? 
\begin{center}
\includegraphics[width=0.6\textwidth]{\string"content/Kiss_Andrea_Timea/HF02_1\string".png} 
\par\end{center}
\end{extraproblem}

\begin{solution}
A téglalapot egy koordináta-rendszerbe helyezzük úgy, hogy két egymásra
merőleges oldaltartó egyenese párhuzamos legyen a koordinátatengelyekkel.
Ekkor egy téglalapot két számpár - $(a,b)$ és $(c,d)$ - határoz
meg, ahol $a,b\in\{0,1,\dots,n\}$ és $c,d\in\{0,1,\dots,m\}$, $a\neq b$,
$c\neq d$. Ezeket a számpárokat összesen $C_{n+1}^{2}\cdot C_{m+1}^{2}$
féle módon választhatjuk ki, tehát $\frac{nm(m+1)(n+1)}{4}$ téglalap
látható az ábrán. Ez a két ábrán 60 ill. 945 téglalapot jelent.
\end{solution}
\begin{extraproblem}[Csapó Hajnalka]
~
\begin{itemize}
\item[a)] Hány olyan négyzet található az alábbi ábrákon, amelyek oldalai rácsvonalak?
Általánosítsunk! 
\begin{center}
\includegraphics[width=0.6\textwidth]{\string"content/Kiss_Andrea_Timea/HF02_2.png\string"} 
\par\end{center}
\item[b)] Hány olyan négyzet van a fenti ábrákon, amelynek csúcsa az adott
rácsnak rácspontja? 
\end{itemize}
\end{extraproblem}

\begin{solution}
a) Az első ábra esetén 16 darab 1 egység oldalú, 9 db 2 egység oldalú,
4 db 3 egység oldalú és 1 db 4 egység oldalú négyzet van. Tehát $1^{2}+2^{2}+3^{2}+4^{2}=30$
négyzet látható az első ábrán.\\
 A második ábrán $6\cdot9$ db egység oldalú, $5\cdot8$ db 2 egység
oldalú, $4\cdot7$ db 3 egység oldalú, $3\cdot6$ db 4 egység oldalú,
$2\cdot5$ db 5 egység oldalú és $1\cdot4$ db 6 egység oldalú négyzet
látható, azaz összesen $154$.

Általánosan: Ha a rács $m\times n$-es, ahol $m\le n$, akkor összesen
${\displaystyle {\sum_{i=1}^{m}{i\cdot(n-m+i)}}}$ darab négyzet látható.

b) Az első ábrán az előzőek mellett még 4 típusú négyzet szerkeszthető:
az 1-es típusúból 9 db van, a 2-es típusúból 1 db, a 3-as típusúból
$2\cdot4$ db, a 4-es típusúból pedig 2 darab. Ez még 20 négyzetet
jelenet, tehát összesen 50 négyzet van. 
\begin{center}
\includegraphics[width=0.3\textwidth]{\string"content/Kiss_Andrea_Timea/HF02_21.jpg\string"} 
\par\end{center}
A második ábrán $1\cdot4+3\cdot6+5\cdot8+2\cdot(7\cdot4+1\cdot4+6\cdot3+5\cdot2+5\cdot2+4\cdot1)=210$
négyzet van még, tehát összesen $364$. 
\begin{center}
\includegraphics[width=0.6\textwidth]{\string"content/Kiss_Andrea_Timea/HF02_22.jpg\string"}
\par\end{center}
\end{solution}
\begin{extraproblem}[Csurka-Molnár Hanna]
Hányféleképpen festhetjük ki 6 különböző színűre egy kocka lapjait,
ha egymásba forgatható példányokból csak egyet-egyet akarunk? 
\end{extraproblem}

\begin{solution}
A lehetséges színeket jelöljük $c_{1},c_{2},c_{3},c_{4},c_{5},c_{6}$-al.
Tegyük fel, hogy az összes lehetséges módon kifestettük a kockákat.
Ahhoz, hogy ezeket megszámoljuk, eslő lépésben alulra forgatjuk minden
kocka $c_{1}$ színű lapját. Ekkor a $c_{2}$ színú lap bizonyos kockáknál
felül, másoknál a kocka valamely oldalán helyezkedi el.

Tekintsük azokat a kockákat, amelyeknél a $c_{2}$ színű lap felül
van. Ezeknek az oldalai a $c_{3},c_{4},c_{5},c_{6}$ színekre vannak
festve. Forgassuk mindenik kocka $c_{3}$ színű lapját magunk fele.
A $c_{1}$ és $c_{3}$ színű oldalakat rögzítve már nem tudjuk a kockákat
forgatni, ezért a kockák jobb, bal és hátsó lapját a maradék $3$
színnel bármilyen módon színezve különböző kockákat kapunk. Ezt a
színezést $3!$-féleképpen tudjuk megtenni.

Most tekintsük azokat a kockákat, amelyeknek a jobb, bal, első, vagy
hátsó lapja van $c_{2}$ színűre festve. Ezt a $c_{2}$ színű lapot
fordítsuk magunk felé. Ekkor a kockák alsó $c_{1}$ színű és első
$c_{2}$ színű lapja rögzítve van, így nem tudjuk tovább forgatni.
A fennmaradó $4$ oldalon bármilyen módon permutálható a megmaradt
$4$ szín, vagyis $4!$ ilyen kocka van. Összesen $3!+4!=(1+4)\cdot3!=5\cdot3!=\frac{5!}{4}$
féle színezése lehet a kockáknak. 
\end{solution}
\begin{extraproblem}[Csurka-Molnár Hanna]
Egy zsákban 10 különböző pár cipő van. Mi a valószínűsége, hogy ha
kiveszünk belőle 6 darabot, lesz köztük legalább egy pár? 
\end{extraproblem}

\begin{solution}
Összesen $2\cdot10=20$ cipő van. Mivel ezek 10 különböző pár jobb-
és bal cipői, $20$ különböző cipő van a zsákban. A zsákból $6$ cipőt
kihúzva összesen $C_{20}^{6}=\frac{20!}{(20-6)!\cdot6!}=\frac{20!}{14!\cdot6!}$
féle eredményhez juthatunk. Ahhoz, hogy megszámoljuk hány olyan eset
van, amikor van legalább egy pár cipő, megszámolhatjuk, hogy hány
eset lehetséges amikor egy pár cipő sincs és ezt az összes eset számából
kivonjuk. Ahhoz, hogy egy pár cipő se legyen a kihúzott cipők között,
minden kihúzott cipő különböző pár jobb-, vagy bal cipője közül húztak
ki a pár egyik felét $C_{10}^{6}$ féle képpen tudjuk kiválasztani
és mivel ezek mindegyike lehet jobb, vagy bal egyenként, $2^{6}$
eset van egy pár-kombináció esetén, hogy azoknak konkrétan melyik
felét húztuk ki. Így tehát $C_{10}^{6}\cdot2^{6}=\frac{10!}{(10-6)!\cdot6!}\cdot2^{6}=\frac{10!}{4!\cdot6!}$
eset lehetséges amikor egy pár sincs a kihúzott cipők között.

\[
p=\frac{\frac{20!}{14!\cdot6!}-\frac{10!}{4!\cdot6!}\cdot2^{6}}{\frac{20!}{14!\cdot6!}}=\frac{20\cdot19\cdot...\cdot15-10\cdot9\cdot...\cdot5\cdot2^{6}}{20\cdot19\cdot...\cdot15}=
\]
\[
=1-\frac{20}{20}\cdot\frac{18}{19}\cdot\frac{16}{18}\cdot\frac{14}{17}\cdot\frac{12}{16}\cdot\frac{10}{15}=\frac{323-112}{323}=0.65325
\]
\end{solution}
\begin{extraproblem}[Czofa Vivien]
Adott $n$ különböző tárgy és $r$ doboz ($n,r\geq1$). Helyezzük
a tárgyakat a dobozokba úgy, hogy az 1. dobozba $k_{1}$ tárgy kerüljön,
a 2. dobozba $k_{2}$ tárgy, ..., az $r$-edik dobozba $k_{r}$ tárgy,
ahol $k_{1}+k_{2}+\cdots+k_{r}=n$. Hányféleképpen lehetséges ez?
(Az egyes dobozokban számít a tárgyak sorrendje.)
\end{extraproblem}
Adjatok példát: $n=10$, $k_{1}=2$, $k_{2}=4$, $k_{3}=3$, $k_{4}=1$
esetén.\emph{ (Pécsi Tudományegyetem: ``Kombinatorika jegyzet és
feladatgyűjtemény'', 2011)}
\begin{solution}
Az 1. dobozba $\binom{n}{k_{1}}$-féleképpen választhatók meg a $k_{1}$
tárgy. Ezután a 2. dobozba a megmaradt $n-k_{1}$ közül választhatjuk
ki a $k_{2}$-t: $\binom{n-k_{1}}{k_{2}}$. A 3. dobozba a megmaradt
$n-k_{1}-k_{2}$ közül választhatjuk a $k_{3}$-at: $\binom{n-k_{1}-k_{2}}{k_{3}}$,
és így tovább.

Lehetőségek száma:

\[
\binom{n}{k_{1}}\binom{n-k_{1}}{k_{2}}\binom{n-k_{1}-k_{2}}{k_{3}}\cdots\binom{n-k_{1}-\cdots-k_{r-1}}{k_{r}}
\]

Egyszerűsítve:

\[
\frac{n!}{k_{1}!\cdot k_{2}!\cdots k_{r}!}
\]

Ez megegyezik annak a számával, hogy az $n$ elemet hányféleképpen
tudjuk \emph{ismétléses permutációval} elosztani $r$ csoportba $k_{1},k_{2},\dots,k_{r}$
elemszámmal:

\[
P_{n}^{(k_{1},k_{2},\dots,k_{r})}=\frac{n!}{k_{1}!\cdot k_{2}!\cdots k_{r}!}
\]

Példa:

$n=10$, $k_{1}=2$, $k_{2}=4$, $k_{3}=3$, $k_{4}=1$

\[
\begin{array}{cccccccccc}
1 & 2 & 3 & 4 & 5 & 6 & 7 & 8 & 9 & 10\\
\hline 1 & 1 & 2 & 2 & 2 & 2 & 3 & 3 & 3 & 4
\end{array}\quad\text{vagy}\quad\begin{array}{cccccccccc}
1 & 2 & 3 & 4 & 5 & 6 & 7 & 8 & 9 & 10\\
\hline 2 & 3 & 1 & 2 & 2 & 2 & 3 & 3 & 4 & 1
\end{array}\quad\text{stb.}
\]
\end{solution}
\begin{extraproblem}[Czofa Vivien]
Az $1,1,1,2,2,3,3$ számjegyekből
\end{extraproblem}

\begin{enumerate}
\item hány hétjegyű számot lehet készíteni? 
\item hány 13-mal kezdődő szám képezhető? 
\end{enumerate}
\emph{(Pécsi Tudományegyetem: ``Kombinatorika jegyzet és feladatgyűjtemény'',
2011)}
\begin{solution}
~
\begin{itemize}
\item[a)] Különböztessük meg egymástól az azonos számjegyeket (színezzük őket).
Ekkor $7!=5040$ sorrend létezik. Viszont az azonos számok sorrendjeit
többször számoltuk:

A 2-esek és 3-asok felcserélései nem számítanak különbözőnek, tehát:
\[
\frac{7!}{3!}=840
\]

Ha a 2-esek és 3-asok színezésétől is eltekintünk: 
\[
\frac{7!}{3!\cdot2!\cdot2!}=210
\]

Ez a megoldás a kérdés eredeti feltételének megfelelő.
\item[b)] A szám kezdődjön 13-mal. Fixáljuk az első két számjegyet: $1,3$.
Marad $5$ hely a $5$ számjegynek: két $1$-es, két $2$-es és egy
$3$-as.

A lehetséges elrendezések száma: 
\[
\frac{5!}{2!\cdot2!\cdot1!}=30
\]

\end{itemize}
\end{solution}
\begin{extraproblem}[Fábián Nóra]
Hány olyan $(x_{1},x_{2},x_{3},x_{4})$ természetes páratlan számokból
álló számnégyes van, amelyre teljesül, hogy: $x_{1}+x_{2}+x_{3}+x_{4}=98$? 
\end{extraproblem}

\begin{solution}
Mivel $x_{1},x_{2},x_{3},x_{4}$ páratlan számok, így ezeket átírhatjuk
a következőképpen: $x_{i}=2k_{i}-1$, $\forall i\in\{1,2,3,4\}$,
ahol $k_{i}\in\mathbb{N}^{*}$ \\
 Ekkor 
\[
2k_{1}-1+2k_{2}-1+2k_{3}-1+2k_{4}-1=98
\]
\[
2(k_{1}+k_{2}+k_{3}+k_{4})=102
\]
\[
k_{2}+k_{2}+k_{3}+k_{4}=51
\]
\\
Le tudjuk modellezni a feladatot. Veszünk 51 darab 1-est és ezeket
valamilyen módon szétválasztjuk 3 darab $|$ választóval (azért, hogy
4 részre bontsuk az 1-es sort). Pl: \\
 11111111 | 1111111111111111111 | 1111111111111111 | 11111111 \\
 Ez a szétválasztás megfelelne a $(8,19,16,8)$ számnégyesnek. Ha
a választóvonalat a sor elejére vagy végére rakjuk vagy esetleg két
választó vonalat rakunk egymás mellé, akkor megjelennek a $0$ értékek
is, ami nekünk nem lesz jó, hiszen $k_{i}\in\mathbb{N}^{*}$. Tehát
csak az egyesek közé rakhatunk választékot, emiatt csak 50 lehetséges
helyre tehetjük a 3 választékot valamilyen sorrendben. Következtetésképpen
${\displaystyle {C_{50}^{3}}=19600}$ olyan számnégyes van, ami teljesíti
a feladat követelményeit. 
\end{solution}
\begin{extraproblem}[Fábián Nóra]
A $\{1000,1001,1002,\ldots,200\}$ halmazban hány olyan egymásutáni
egész számpár van, amelyek összegezve egyik helyi értéken sem lépünk
túl a 9-es számjegyen, tehát nincs szükség átvitelre a következő helyi
értékre?
\end{extraproblem}

\begin{solution}
Válasszunk ki egy $\overline{abcd}$ számot, ekkor az utána lévő szám
egyértelmű, innen arra következtethetünk, hogy $a=1$, vagyis a keresett
szám $1\overline{bcd}$ alakú. \\
-Ha $b,c$ vagy $d$ számjegyek közül valamelyik $5,6,7$ vagy $8$,
akkor ebben az esetben biztos, hogy átvisszük a helyi értéket. \\
-Ha $c=9$, de $d\neq9$, akkor megint biztos, hogy átvitel lesz (pl.
1492+1493) \\
-Ha $b=9$, de $c\neq9$ vagy $d\neq9$, akkor is átvitel lesz.

Vegyük észre, hogy csak a következő alakú számok lehetnek megfelelőek:
$1\overline{bcd}$ , $1\overline{bc}9$ , $1\overline{b}99$ , $1999$
\\
Mivel $b,c,d\in\{0,1,2,3,4\}$, így összesen $5^{3}+5^{2}+5+1=156$
ilyen számpárunk lehet.
\end{solution}
\begin{extraproblem}[Gál Tamara]
 Hány olyan legfeljebb ötjegyű szám van, amelynek minden jegye különböző,
és a számjegyei balról jobbra csökkenő sorrendben követik egymást? 
\end{extraproblem}

\begin{solution}
Ha meghatározzuk, hogy mely számjegyek szerepeljenek egy ilyen számban,
azokat már csak egyféle sorrendben (csökkenő sorrendben) írhatjuk
egymás mellé. Elég tehát azt megnéznünk, hányféleképpen választhatjuk
ki a keresett számokban szereplő számjegyeket.

Ha a keresett szám ötjegyű, akkor a 10 számjegyből $C_{10}^{5}$-féleképpen
választhatunk ki 5-öt, így ennyi ötjegyű szám felel meg a feltételnek
( 0 -val nem kezdődhet szám, de a kiválasztott 5 számjegy legnagyobbika
biztosan nem 0). Hasonlóképpen a keresett négyjegyű számokból $C_{10}^{4}$,
a háromjegyűekből $C_{10}^{3}$, a kétjegyűekből pedig $C_{10}^{2}$
darabot találunk. Az egyjegyű számok esetében már figyelnünk kell
arra, hogy a 0 -t nem választhatjuk, így ott $C_{9}^{1}$ lehetőséget
kapunk.
\end{solution}
\begin{extraproblem}[Gál Tamara]
Egy zsákban 10 különböző pár cipő található. Hányféleképpen vihetünk
magunkkal a zsákból 6 darab cipőt úgy, hogy legyen közöttük összetartozó
pár? 
\end{extraproblem}

\begin{solution}
A jó esetek számát megkaphatjuk úgy, ha az összes eset számából kivonjuk
a rossz esetek számát . A zsákban lévő 20 cipő közül 6-ot összesen
$C_{20}^{6}$ féleképpen vehetünk ki. Ezek közül rossz esetnek számít,
ha nincsen összetartozó pár a kivett cipők között, vagyis mind a 6
cipő különböző párból való. Erre $C_{10}^{6}\cdot2^{6}$ lehetőségünk
van, hiszen ki kell választanunk a 10 párból 6-ot, majd minden ilyen
párból a bal vagy a jobb cipőt.

Tehát a kedvező esetek száma $C_{20}^{6}-C_{10}^{6}\cdot2^{6}=38760-13440=25320$.\\
 Megjegyzés: a rossz esetek számát a $\frac{20\cdot18\cdot16\cdot14\cdot12\cdot10}{6!}$
képlettel is megkaphatjuk (az első kiválasztott cipő 20 -féle lehet,
a második már csak a fennmaradó 9 párból valamelyik, azaz 18 -féle
stb., majd leosztunk a lehetséges sorrendek számával, hiszen a magunkkal
vitel során nem számít a 6 cipő kiválasztásának sorrendje).
\end{solution}
\begin{extraproblem}[Gergely Verona]
Hány olyan legfeljebb négyjegyű természetes szám van, amelynek számjegyei
között több 2-es szerepel, mint 1-es? \emph{(Arany Dániel Matematikai
Tanulóverseny 2014/2015; haladók, I. kategória, 1. forduló) }
\end{extraproblem}

\begin{solution}
Megoldás 1 A számok maximum négyjegyűek, tehát a vizsgált számokat
tekinthetjük négyjegyűeknek úgy, hogy a szám elejére nullákat írunk
(például a 21-t 0021-ként kezeljük).

Ha egy $2$-es és $0$ $1$-es szerepel akkor a $2$-es négy helyen
állhat, a többi három helyen pedig $8$-féle számot irhatunk, összesen
$4\cdot8^{3}=2048$ szám.

Ha két $2$-es és nulla $1$-es szerepel akkor a négy helyből kettön
$2$-es áll, a többi két helyen pedig $8$-féle számot irhatunk, összesen
$\binom{4}{2}\cdot8^{2}=384$ szám.

Ha két $2$-es és egy $1$-es szerepel akkor, összesen $\binom{4}{2}\cdot\binom{2}{1}\cdot8=96$
szám.

Ha három $2$-es és nulla $1$-es szerepel akkor, összesen $\binom{4}{3}\cdot8=32$
szám.

Ha három $2$-es és egy $1$-es szerepel akkor, összesen $\binom{4}{3}\cdot1=4$
szám.

Ha négy $2$-es és nulla $1$-es szerepel akkor, összesen $1$ szám.

Tehát $2048+384+96+32+4+1=2565$ megfelelő szám van. 
\end{solution}
\begin{extraproblem}[Gergely Verona]
 Hány olyan ötjegyű tízes számrendszerbeli pozitív egész szám van,
amelyben a számjegyek szorzata 50-re végződik? \emph{(OKTV 2013/2014;
II. kategória, 1. forduló) }
\end{extraproblem}

\begin{solution}
Megoldás 2 A feladat szerint olyan ötjegyű számokat kell találni,
amelyek számjegyeinek szorzata 50-re végződik. Ha egy szám $50$-re
végződik, akkor osztható $50=2\cdot5^{2}$-nel, de nem osztható $100$-al.
A szorzatban tehát pontosan egy $2$-es és legalább két $5$-ös szerepel.

A számjegyek között pontosan egy páros szám szerepelhet, amely lehet
2 vagy 6,(4-es nem lehet mert akkor két kettes lenne a szorzatban)
és legalább két 5-ösnek is szerepelnie kell.

1. Ha a számban pontosan két 5-ös szerepel:

A két 5-ös helyét 5 hely közül kell kiválasztani, tehát ennek a számítása:
$\binom{5}{2}=10\text{ féle mód van.}$ Az egyetlen páros számjegy
lehet 2 vagy 6, tehát 2 választásunk van. A páros számjegy helyét
3 hely közül választhatjuk ki: $\binom{3}{1}=3\text{ féle mód van.}$
A maradék két helyre pedig 4-4 számjegyet választhatunk (1, 3, 7,
9), tehát a lehetőségek száma: $4\cdot4=16\text{ féle mód van.}$
Tehát, ha pontosan két 5-ös szerepel:
\[
10\cdot2\cdot3\cdot16=960\text{ féle szám van.}
\]

2. Ha a számban pontosan három 5-ös szerepel:

A három 5-ös helyét 5 hely közül kell kiválasztani, tehát ennek a
számítása: $\binom{5}{3}=10\text{ féle mód van.}$Az egyetlen páros
számjegy lehet 2 vagy 6, tehát 2 választásunk van. A páros számjegy
helyét 2 hely közül választhatjuk ki: $\binom{2}{1}=2\text{ féle mód van.}$
A maradék egy helyre pedig 4 féle számjegy választható (1, 3, 7, 9):
$4$ féle mód van. Tehát, ha pontosan három 5-ös szerepel:
\[
10\cdot2\cdot2\cdot4=160\text{ féle szám van.}
\]

3. Ha a számban pontosan négy 5-ös szerepel:

A négy 5-ös helyét 5 hely közül kell kiválasztani, tehát ennek a számítása:
$\binom{5}{4}=5\text{ féle mód van.}$Az egyetlen páros számjegy
lehet 2 vagy 6, tehát 2 választásunk van.

Tehát, ha pontosan négy 5-ös szerepel: 
\[
5\times2=10\text{ féle szám van.}
\]

Összeadva az összes lehetőséget, a végső eredmény:
\[
960+160+10=1130
\]

Tehát összesen $1130$ olyan ötjegyű szám van, amelynek számjegyeinek
szorzata 50-re végződik.
\end{solution}
\begin{extraproblem}[Kis Aranka-Enikő]
Tíz diák vízitúrára menne, kapnak is túravezetőt, de csak egy három-
és egy ötszemélyes csónak áll rendelkezésükre. A kisebb csónakban
az egyik utasnak eveznie, a másiknak kormányoznia kell, a harmadik
utas csak pihen. A nagyobbik csónakban ül a túravezető, ő kormányoz,
mellette négy túrázó fér el, ők mind evezni fognak. Hányféleképpen
állítható össze a tíz diákból a túrán részt vevők 7 fős csoportja?
\end{extraproblem}

\begin{solution}
Először kiválasztunk a tíz diák közül hármat, ők majd a kisebb csónakba
ülnek. Mindhármukra más-más feladat hárul, tehát itt számít a kiválasztottak
sorrendje. Ezért a kiválasztási lehetőségek száma: 
\[
10\cdot9\cdot8=720.
\]

Most a többi hét diák közül választunk négyet, ők ülnek a túravezetővel
a nagyobbik csónakba. Egyforma a beosztásuk, tehát itt nem számít
a kiválasztottak sorrendje. Ezért a kiválasztási lehetőségek száma:
\[
\binom{7}{4}=\frac{7!}{4!\cdot3!}=35.
\]

Bárhogyan választjuk is ki az első csónakba a 3 diákot, hozzájuk 35-féleképpen
választhatók a második csónak utasai, tehát összesen: 
\[
10\cdot9\cdot8\cdot\binom{7}{4}=720\cdot35=25200
\]
módon állítható össze a tíz diákból a túrán részt vevők csoport.
\end{solution}
\begin{extraproblem}[Kis Aranka-Enikő]
Hányféle (nem feltétlenül értelmes) anagramma alkotható a PARALELOGRAMMA
szó betűiből?
\end{extraproblem}

\begin{solution}
Ha a 14 betű mindegyike különböző lenne, az $14!$-féle lehetséges
sorrendet jelentene. Gondolatban színezzük be az azonos betűket különböző
színűre. Mivel két ``L'' betűt használtunk fel, minden esetet kétszer
számoltunk: akkor, amikor a ``piros L'' betű van előrébb, és akkor
is, amikor a ``kék'' van előrébb, de a többi betű sorrendje változatlan.
Így a $14!$-féle lehetőségnek csak a fele lesz az, amelyben nem különböztetjük
meg az ``L'' betűket.

De még így is kétszer számoltunk minden különböző esetet, a két azonos
``M'' betű miatt, valamint ismét kétszer az ``R'' betűk miatt.
Így további $2\cdot2$-vel kell még osztanunk. ``A'' betűből négy
darab is van, amelyek egymáshoz képest vett sorrendje szintén nem
számít, így minden megmaradt esetet $4!$-szor számoltunk.

A 14 betű, amelyek közt rendre 2, 2, 2, 4 azonos betű van, tehát az
alábbi módon rakható ki:

\[
\frac{14!}{2!\cdot2!\cdot2!\cdot4!}
\]

Megjegyzés: A tört nevezőjében a 2-es tényezők helyett írhattunk volna
$2!$-t is.
\end{solution}
\begin{extraproblem}[Kis Brigitta]
Legyen $A=\{1,2,3,\dots,20\}$ az első 20 pozitív egész szám halmaza.
Hány olyan részhalmaza van az $A$ halmaznak, amelyben pontosan 5
páratlan szám szerepel? 
\end{extraproblem}

\begin{solution}
Az $A$ halmaz páratlan elemei: $\{1,3,5,7,9,11,13,15,17,19\}$. A
feladatban pontosan 5 páratlan számot kell választanunk ezen 10 páratlan
szám közül. Ennek a kiválasztásnak a száma: 
\[
\binom{10}{5}=\frac{10\cdot9\cdot8\cdot7\cdot6}{5\cdot4\cdot3\cdot2\cdot1}=252
\]
Az $A$ halmaz páros elemei: $\{2,4,6,8,10,12,14,16,18,20\}$. A páros
számok közül bármilyen számú elemet választhatunk (0-tól 10-ig), így
a választás lehetőségeinek száma: 
\[
\sum_{k=0}^{10}\binom{10}{k}=2^{10}=1024
\]
Mivel a páratlan számok és a páros számok kiválasztása független,
az összes lehetséges részhalmazok száma a két szám szorzataként adódik:

\[
\binom{10}{5}\cdot2^{10}=252\cdot1024=258048
\]

Az $A$ halmaznak 258,048 olyan részhalmaza van, amely pontosan 5
páratlan számot tartalmaz. 
\end{solution}
\begin{extraproblem}[Lukács Andor]
Egy csomag francia kártyát megkevertünk, majd egyesével elkezdjük
kihúzni a lapokat. Hányadik helyen a legvalószínűbb a második ász
kihúzása? 
\end{extraproblem}

\begin{solution}
A francia kártya paklija $52$ lapos, $4$ ász található benne. Legyen
\[
\mathcal{E}(k)=\text{a második ász a \ensuremath{k}-adik helyen kerül kihúzásra},\quad k=2,3,4,\ldots,50.
\]
Az $\mathcal{E}(k)$ esemény bekövetkezéséhez az alábbiaknak kell
teljesülniük: 
\begin{itemize}
\item az első $(k-1)$ helyen pontosan $1$ ász kerül kihúzásra, ez $C_{k-1}^{1}$
féleképpen tehető meg; 
\item a $k$-adik helyen kihúzzuk a második ászt, ezt egyféleképpen tehetjük
meg; 
\item a $(k+1)$-edik kártya húzásától az 52. kártya húzásáig akárhogyan
húzhatjuk a pakliban maradó két ászt, erre $C_{52-k}^{2}$ féleképpen
van lehetőség. 
\end{itemize}
Ezért az $\mathcal{E}(k)$ kedvező eseteinek a száma 
\[
K(k)=C_{k-1}^{1}\cdot C_{52-k}^{2}=(k-1)\cdot\frac{(52-k)\cdot(51-k)}{2},
\]
az összes esetek száma pedig 
\[
N(k)=C_{52}^{4}=\frac{52\cdot51\cdot50\cdot49}{24}.
\]
Az előbbiek alapján az $\mathcal{E}(k)$ esemény valószínűsége 
\[
P(\mathcal{E}(k))=\frac{K(k)}{N(k)}=(k-1)\cdot(52-k)\cdot(51-k)\frac{12}{52\cdot51\cdot50\cdot49},
\]
ami akkor maximális, ha $(k-1)\cdot(52-k)\cdot(51-k)$ maximális (miközben
$2\leq k\leq50$). Az $f\colon[2,50]\to\mathbb{R}$, $f(x)=(x-1)\cdot(52-x)\cdot(51-x)$
függvény maximumhelyét keressük. Mivel 
\[
f'(x)=3x^{2}-208x+2755,
\]
ezért a maximumhely meghatározásához az $f'(x)=0$ egyenletet oldjuk
meg. Az egyenletnek két gyöke van, ezek közül az $x_{0}\approx17,5$
a keresett maximumhely. Mivel $f(18)>f(17)$, ezért a második ász
kihúzása a $18$-adik helyen a legvalószínűbb (és a valószínűsége
$\frac{17\cdot561}{270725}\approx0,03522$).

\textbf{Megjegyzés.} A megoldás befejezhető derivált használata nélkül
is, ha észrevesszük, hogy 
\begin{itemize}
\item $\frac{K(k+1)}{K(k)}>1$, ha $k<17$; 
\item $\frac{K(k+1)}{K(k)}<1$, ha $k>18$, 
\end{itemize}
tehát elégséges a $K(k)$ értékeket összehasonlítani $k=17$ és $k=18$
esetén. 
\end{solution}
\begin{extraproblem}[Lukács Andor]
Egy $n$ elemű halmaznak hány olyan részhalmaza van, amelynek elemeinek
száma osztható $3$-mal? 
\end{extraproblem}

\begin{solution}
A feladatunk az, hogy számoljuk ki az 
\[
S_{n}=\sum_{\substack{k=0\\
3\mid k
}
}^{n}C_{n}^{k}
\]
összeget. Legyen $\omega=\cos\frac{2\pi}{3}+i\sin\frac{2\pi}{3}$,
tehát $1+\omega+\omega^{2}=0$ és ezért 
\[
S_{n}=\frac{1}{3}\Bigl[(1+1)^{n}+(1+\omega)^{n}+(1+\omega^{2})^{n}\Bigr].
\]
Mivel $1+\omega=-\omega^{2}=\cos\frac{\pi}{3}+i\sin\frac{\pi}{3}=\omega$
és $1+\omega^{2}=\cos\frac{\pi}{3}-i\sin\frac{\pi}{3}$, az előző
összefüggésből következik, hogy

\[
S_{n}=\frac{1}{3}\left(2^{n}+2\cos\frac{n\pi}{3}\right).
\]
\end{solution}
\begin{extraproblem}[Miklós Dóra]
Hány téglalap látható a $7\times7$-es sakktáblán? \emph{(Felvidéki
Magyar Matematika Verseny, XXIX. 1. évfolyam)}
\end{extraproblem}

\begin{solution}
Egy $7\times7$-es sakktáblán nyolc vízszintes és nyolc függőleges
sor van. Egy téglalap előállítható két ilyen vízszintes és két függőleges
sor metszéspontjai által. Ebből a nyolc vízszintes vagy függőleges
sorból kettőt $C_{8}^{2}$ féleképpen tudunk kiválasztani, de ahhoz,
hogy ebből téglalapunk legyen két ilyen párra van szükségünk: egy
pár vízszintes és egy pár függőleges egyenesre. Minden pár, minden
másik párral egy új téglalapot alkot, ezért a létrejövő téglalapok
száma: 
\[
\left(C_{8}^{2}\right)^{2}=\left(\frac{8!}{2!\cdot6!}\right)^{2}=\left(\frac{8\cdot7}{2}\right)^{2}=28^{2}=784.
\]
\end{solution}
\begin{extraproblem}[Miklós Dóra]
Igazoljuk az alábbi azonosságot: 
\[
1\cdot C_{n}^{1}+2\cdot C_{n}^{2}+3\cdot C_{n}^{3}+\dots+n\cdot C_{n}^{n}=n\cdot2^{n-1}.
\]
\end{extraproblem}

\begin{solution}
A feladat megoldható a kombináció képletének és a binomiális tételnek
a felhasználásával is, de ebben a megoldásban az a célunk, hogy a
kettős összeszámlálás módszerére adjunk egy példát. Ennek nyomán megfogalmazunk
egy olyan feladatot, melyet kétféle összeszámlálással megoldva az
egyenlet két oldalán szereplő kifejezéseket kapjuk. Ehhez a feladathoz
a következő kérdésre keressük a választ: egy n tagú társaságból hányféleképpen
tudunk k tagú bizottságot kiválasztani úgy, hogy a bizottság elnöke
bárki lehet?

Az első gondolatmenetben abból indulunk ki, hogy a bizottságok $k\in\{1,2,\dots,n\}$
tagból állhatnak. Ha egy $k$ tagú bizottságot szeretnénk, akkor azt
$C_{n}^{k}$ féleképpen választhatjuk ki, melynek minden tagja lehet
ugyanúgy elnök is, így a különböző $k$ tagú bizottságok száma $k\cdot C_{n}^{k}$.
Ezt felhasználva a lehetséges bizottságok számát pedig az alábbi összegzés
adja meg: 
\[
\sum\limits_{k=1}^{n}k\cdot C_{n}^{k}=1\cdot C_{n}^{1}+2\cdot C_{n}^{2}+\dots+n\cdot C_{n}^{n}.
\]

A második gondolatmenetben abból indulunk ki, hogy minden tag lehet
elnök, ami $n$ lehetőség. Továbbá megvizsgáljuk, hogy melléjük milyen
bizottságok állíthatóak elő. Ha egy valakit megválasztunk elnöknek,
akkor a maradék $n-1$ tag mindegyikéről elmondható, hogy vagy benne
van a bizottságban vagy nem, ezért összesen $2^{n-1}$ lehetőség van
erre. Így az összes lehetséges bizottság száma $n\cdot2^{n-1}$.

Mivel mindkét eredmény ugyanazt a feladatot oldja meg, ezért az azonosság
valóban teljesül. 
\end{solution}
\begin{extraproblem}[Seres Brigitta-Alexandra]
Egy dobozból, amelyben $8$ piros és bizonyos számú fehér, számozott
golyó van, egymás után, visszatevés nélkül $1280$- féleképpen húzható
ki $3$ golyó úgy, hogy két piros, vagy két fehér golyó ne következzen
egymás után. Hány fehér golyó van a dobozban? \emph{(Pécsi Tudományegyetem:
}\textit{Kombinatorika jegyzet és feladatgyűjtemény},\emph{ 2011) }
\end{extraproblem}

\begin{solution}
Jelöljük $x$-el a dobozban lévő fehér golyók számát. Ha csak egyszínű
golyókat húznánk ki, akkor lenne közte két piros, vagy két fehér golyó
úgy hogy egymás után következnek, szóval ez az eset nincs benne a
megszámolt 1280-ban.

Ha két fehér golyót húzunk pontosan, akkor a harmadik piros lesz.
Ahhoz, hogy ne legyen két fehér egymás után húzva a sorrend: fehér-piros-fehér
lesz. Mivel a golyók számozottak, más-más húzásnak számít ha két fehér
golyót felcserélve húzunk ki, vagyis számít a sorrend. Ezt $V_{x}^{2}\cdot V_{8}^{1}$
féleképpen tehetjük meg, figyelembe véve azt, hogy a két fehér golyó
ne kövesse egymást.

Ha két piros golyót húzunk pontosan, akkor a harmadik fehér lesz.
Ahhoz, hogy ne legyen két piros egymás után húzva a sorrend: piros-fehér-piros
lesz. Mivel a golyók számozottak, más-más húzásnak számít ha két piros
golyót felcserélve húzunk ki, vagyis számít a sorrend. Ezt $V_{x}^{1}\cdot V_{8}^{2}$
féleképpen tehetjük meg, figyelembe véve azt, hogy a két piros golyó
ne kövesse egymást.

Mivel máshogy nem húzható ki a $3$ golyó a feltételeknek megfelelően,
ekkor a feladat adatai szerint: 
\[
V_{x}^{2}\cdot V_{8}^{1}+V_{x}^{1}\cdot V_{8}^{2}=1280\iff\frac{x!}{(x-2)!}\cdot8+\frac{x!}{(x-1)!}\cdot56=1280
\]
\[
\iff8x(x-1)+56x=1280.
\]
Ekkor $8x^{2}+48x-1280=0$ másodfokú egyenlet gyökei $10$ és $-16$.
Mivel a golyók száma pozitív mennyiség a feladat megoldása $x=10$
lesz, vagyis $10$ fehér golyó volt a dobozban. 
\end{solution}
\begin{extraproblem}[Seres Brigitta-Alexandra]
$2n$ különböző magasságú ember hányféleképpen tud két $n$-hosszúságú
sorba állni úgy, hogy az első sorban mindenki alacsonyabb legyen,
a hátsó sorban a megfelelő helyen állónál? 
\begin{flushright}
(Pécsi Tudományegyetem: \textit{"Kombinatorika jegyzet és feladatgyűjtemény"},
2011) 
\par\end{flushright}
\end{extraproblem}

\begin{solution}
Minden helyes sorba rendezés kölcsönösen egyértelműen megfeleltethető
a $2n$ személy egy párosításának, hiszen ha a párokat kialakítjuk,
akkor azzal egyértelműen meghatároztuk azt is, hogy egy-egy párból
ki áll az első és ki a hátsó sorba, mert különböző magasságú emberek.
A párosítások száma abból adódik, hogy hányféleképpen tudunk kiválasztani
2 embert a megmaradt csoportból, amit kombinációval tudunk kiszámolni
(hiszen nem számít a kiválasztás sorrendje, mert csak a kiválasztás
után hasonlítják össze őket magasság szerint, ami egyértelmű): 
\begin{align*}
\binom{2n}{2}\cdot\binom{2n-2}{2}\cdot\ldots\binom{2}{2} & =\frac{2n\cdot(2n-1)}{2}\cdot\frac{(2n-2)\cdot(2n-3)}{2}\cdot\ldots\cdot\frac{2\cdot1}{2}\\
 & =\frac{(2n)!}{2^{n}}
\end{align*}
Tehát $\frac{(2n)!}{2^{n}}$ féleképpen lehet $2n$ különböző magasságú
embert két $n$-hosszúságú sorba állítani úgy, hogy az első sorban
mindenki alacsonyabb legyen, a hátsó sorban a megfelelő helyen állónál. 
\end{solution}
\begin{extraproblem}[Sógor Bence]
A törpék és óriások közös vacsorán vesznek részt. A rendezvényszervező
9 óriást és 4 törpét egy kerekasztal köré kell leültessen. Mivel a
törpék közismerten mogorvák, ezért az est sikeressége érdekében nem
ültetheti őket egymás mellé. Hányféle ültetés lehetséges, ha a forgatással
egymásba vihető ültetéseket \textbf{különbözőnek} tekintjük? Mi van
akkor, hogy \textbf{nem} tekintjük különbözőnek?
\end{extraproblem}

\begin{solution}
Számozzuk meg az ülőhelyeket 1-től 13-ig. Előbb oldjunk meg egy könnyebb
feladatot. Tekintsük el attól, hogy az 1 és 13 helyeken ülők egymás
mellett ülnek. Ültessük le a törpéket az első 9 hely valamelyikére
tetszőlegesen. Ez $C_{9}^{4}$ lehetőség, majd kérjük meg a legmagasabb
sorszámú helyen ülő törpét, hogy üljön a hárommal nagyobb sorszámú
helyre. Kérjük meg a második legnagyobb sorszámú helyen ülő törpét,
hogy üljön a kettővel nagyobb sorszámú helyre, és így tovább, míg
a legutolsó törpe marad a helyén. A törpék az 1, 13 helyeket nem nézve
biztosan nem ülnek egymás mellett és minden a könnyített feladatnak
megfelelő helyes ültetést megkaphatunk így, ezért összesen $C_{9}^{4}$-féleképpen
ültethetünk.

Most térjünk vissza az eredeti feladatra. A $C_{9}^{4}=126$-ban beleszámoltuk
azokat az eseteket is, amikor az 1. és 13. helyeken törpe ül, de azon
kívül nincsenek szomszédos törpék. Ezekből összesen $\frac{7\cdot8}{2}=28$
van (Két törpe ül az 1., 13. helyeken. A maradék kettőből 7 esetben
ülhet a kisebb sorszámú a 3. helyen. 6 esetben ülhet a 4. helyen és
így tovább). Összesen tehát $126-28=98$-féleképpen szervezheti meg
a rendezvényszervező az ültetést.\\

Ha nem tekintjük különbözőnek a forgatással egymásba vihető eseteket,
akkor közelítsük meg másképp a feladatot. Vegyünk egy tetszőleges
ültetést. Körben haladva mindegyik törpe esetén írjuk le, hogy a tőle
balra ülő legközelebbi törpe hány helyre ül. Négy számot kell leírjunk,
amelyek mind 1-nél nagyobbak és összegük 13. Ennek a négy számnak
a sorrendje nem egyértelműen, de meghatároz egy ültetést. A forgatás
miatt például a 2 3 3 5 és 5 2 3 3 ugyanannak az ültetésnek felelnek
meg. Számoljuk össze, hogy a számnégyesek hány különböző ültetésben
fordulhatnak elő.

A lehetséges eseteket foglaljuk táblázatba.

\begin{table}[h!]
\centering %
\begin{tabular}{|c|ccccc|}
\hline 
2, 2, 2, 7 & 2 2 2 7 &  &  & $\leftarrow$ & 1 eset\tabularnewline
2, 2, 3, 6 & 3 6 2 2 & 3 2 6 2 & 3 2 2 6 & $\leftarrow$ & 3 eset\tabularnewline
2, 2, 4, 5 & 4 5 2 2 & 4 2 5 2 & 4 2 2 5 & $\leftarrow$ & 3 eset\tabularnewline
2, 3, 3, 5 & 2 5 3 3 & 2 3 5 3 & 2 3 3 5 & $\leftarrow$ & 3 eset\tabularnewline
2, 3, 4, 4 & 2 3 4 4 & 2 4 3 4 & 2 4 4 3 & $\leftarrow$ & 3 eset\tabularnewline
3, 3, 3, 4 & 3 3 3 4 &  &  & $\leftarrow$ & 1 eset\tabularnewline
\hline 
\end{tabular}\caption{Lehetséges kombinációk}
\label{tab:my_label}
\end{table}

Látható, hogy összesen 14 ültetési lehetőség van.
\end{solution}
\begin{extraproblem}[Szabó Kinga]
Mutassuk meg, hogy ha $0<m\leq k<n$, akkor az $\binom{n}{k}$
és $\binom{n}{m}$ számoknak van 1-nél nagyobb közös osztója. \emph{(Róka
Sándor: Válogatás Erdős Pál kedvenc feladataiból) }
\end{extraproblem}

\begin{solution}

Az állítás igaz, ha $n=p$ prím, hiszen ekkor $\binom{p}{k}$ és $\binom{p}{m}$
is osztható $p$-vel.

Nézzünk egy példát. Miért van a $\binom{28}{5}$ és $\binom{28}{3}$
számoknak 1-nél nagyobb közös osztója?

Ha a $\binom{28}{5}$ és $\binom{28}{3}$ számok relatív prímek, akkor
az egyenlőség miatt $\binom{28}{3}$ osztója $\binom{5}{3}$-nak,
ami a két szám nagyságára figyelve lehetetlen. Tehát hibás a feltevés,
hogy $\binom{28}{5}$ és $\binom{28}{3}$ relatív prímek.

Lássuk a bizonyítást!

\[
\binom{n}{k}\cdot\binom{k}{m}=\binom{n}{m}\cdot\binom{n-m}{k-m}
\]

hiszen

\[
\frac{n!}{k!(n-k)!}\cdot\frac{k!}{m!(k-m)!}=\frac{n!}{m!(n-m)!}\cdot\frac{(n-m)!}{(k-m)!(n-k)!}
\]

Ha az $\binom{n}{m}$ és $\binom{n}{k}$ számok relatív prímek, akkor
az egyenlőség miatt $\binom{n}{m}$ osztója $\binom{k}{m}$-nek.

Az oszthatóság miatt $\binom{n}{m}\leq\binom{k}{m}$ is teljesül (ugyanis
$\binom{k}{m}\neq0$), de ez a $0<m\leq k<n$ feltételek esetén nem
áll fenn.

Feltevésünk hamis volt, azaz az $\binom{n}{m}$ és $\binom{n}{k}$
számoknak van 1-nél nagyobb közös osztója.
\end{solution}
\begin{extraproblem}[Száfta Antal]
 Bizonyítsuk be, hogy bármely társaságban található két olyan ember,
akinek abban a társaságban ugyanannyi ismerőse van. \emph{(OTV 1966,
II. 1.)}
\end{extraproblem}

\begin{solution}
Legyen a társaság tagjainak száma $n$. Társaságról van szó, tehát
$n\geq2$. Ekkor a társaság egy tagjának $0$ vagy $1$, vagy $\ldots$,
vagy $n-1$ ismerőse lehet jelen. Nem lehet azonban olyan is, akinek
nincs ismerőse a társaságban, meg olyan is, akinek a társaság minden
tagja ismerőse, mert akkor az utóbbinak ismernie kellene az előbbit
is, és így az előbbinek is lenne ismerőse. Ha mindenki megmondja,
hogy hány ismerőse van jelen, akkor az $n$ ember legfeljebb $n-1$
különböző számot mondhat (az $1,2,3,...,n-1$ vagy $0,1,2,..,n-2$
számok között válogathatnak), tehát legalább ketten ugyanazt a számot
mondják, vagyis ugyanannyi ismerősük van jelen.
\end{solution}


