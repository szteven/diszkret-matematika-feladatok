
\chapter{Fibonacci-számok}\label{chap:Fibonacci}
\begin{description}
{\large \item [{Szerző:}] Kovács Levente (Didaktikai mesteri -- Matematika, II.
év)}
\end{description}

\section*{Rövid történelmi áttekintő}

Leonardo Fibonacci, más néven Leonardo Pisano, az egyik leghíresebb
középkori olasz matematikus volt, aki körülbelül 1170-ben született
Pisában. Fiatalkorát Észak-Afrikában töltötte, ahol apja kereskedelmi
megbízottként dolgozott, és itt ismerkedett meg az arab matematikával.
1202-ben írta meg híres művét, a \textit{Liber Abaci}-t, amelyben
bemutatta az arab-indiai számrendszert Európának \cite{fibonacci2002}.
Ez a könyv forradalmasította a számolást a római számokkal szemben,
és elősegítette a modern számjegyek elterjedését. A műben szerepel
egy híres nyúlszaporodási probléma, amelyből származik a róla elnevezett
Fibonacci-sorozat. Bár maga Fibonacci nem tulajdonított különösebb
jelentőséget a sorozatnak, később a matematika és a természettudomány
számos területén felfedezték annak jelentőségét \cite{devlin2011}.
Munkássága hatással volt a kereskedelemre, a könyvelésre és az algebra
fejlődésére is. Fibonacci az elsők között volt, akik bevezették Európában
a zéró fogalmát \cite{boyer2011}. Művei hozzájárultak a reneszánsz
kori tudományos gondolkodás megalapozásához. Halálának pontos ideje
nem ismert, de valószínűleg 1250 körül hunyt el Pisában \cite{oconnor}.

\section*{Rövid elméleti összefoglaló}
\begin{definition}{def:Fib}
Az $(f_{n})_{n\in\mathbb{N}};\ f_{1}=f_{2}=1;\ f_{n+2}=f_{n+1}+f_{n}$
rekurzív sorozatot \textit{Fibonacci-sorozatnak} nevezzük.
\end{definition}
\begin{problem}
Határozzuk meg azokat az $(f_{n})_{n\in\mathbb{N}}$ sorozatokat,
amelyekre az 
\begin{equation}
f_{n+2}+af_{n+1}+bfx_{n}=0\label{thm1}
\end{equation}
egyenlet teljesül, ahol $a,b\in\mathbb{R}^{*}$! 
\end{problem}

\begin{solution}
Ha a megoldásokat $f_{n}=r^{n}$ mértani haladvány alakban keressük,
bármely $n\in\mathbb{N}$ és $r\in\mathbb{R}^{*}$ esetén, akkor a
(\ref{thm1}) egyenletből következik, hogy 
\[
r^{n+2}+ar^{n+1}+br^{n}=0.
\]
Mivel az $f_{n}=0,\forall n\in\mathbb{N}$ triviális megoldástól különböző
sorozatokat keresünk, ezért a (\ref{thm1}) egyenletet elosztva $r^{n}\not=0$-val,
az 
\[
r^{2}+ar+b=0
\]
$r$-ben másodfokú egyenletet kapjuk. Ha ennek két különböző valós
gyöke van és ezek $r_{1}\not=r_{2}$, akkor az $(Y_{n})_{n\in\mathbb{N^{*}}}={r_{1}}^{n}$
és $(Z_{n})_{n\in\mathbb{N^{*}}}={r_{2}}^{n}$ sorozatok megoldásai
a (\ref{thm1}) egyenletnek.

A továbbiakban egy segédtételre lesz szükségünk.
\begin{theorem}{thm:fib}
A $(\ref{thm1})$ egyenlet összes megoldása 
\[
f_{n}=c_{1}y_{n}+c_{2}z_{n},\forall n\in\mathbb{N^{*}},
\]
vagyis 
\[
f_{n}=c_{1}r_{1}^{n}+c_{2}r_{2}^{n},\forall n\in\mathbb{N^{*}}
\]
alakú, ahol $c_{1},c_{2}$ tetszőleges, valós konstansok. 
\end{theorem}

A Fibonacci-sorozat esetén a $(\ref{thm1})$ általános egyenletben
$a=b=-1$, így a karakterisztikus egyenlete 
\[
r^{2}-r-1=0
\]
alakú, ahonnan kapjuk, hogy a gyökök $r_{1,2}=\frac{1\pm\sqrt{5}}{2}$.
Az $x_{1}=x_{2}=1$ kezdőértékek segítségével kiszámíthatjuk a $c_{1},c_{2}$
konstansokat, így végül az 
\[
f_{n}=\frac{1}{2}\left[\left(1+\frac{1}{\sqrt{5}}\right)\left(\frac{1+\sqrt{5}}{2}\right)^{n}+\left(1-\frac{1}{\sqrt{5}}\right)\left(\frac{1-\sqrt{5}}{2}\right)^{n}\right]
\]
eredményhez jutunk. Így a fenti összefüggés megadja a Fibonacci-sorozat
$n.$ tagjának az értékét. 
\end{solution}
\vspace{1em}

\begin{definition}
A Fibonacci-sorozat egymást követő tagjai arányát vizsgálva, amikor
$n\rightarrow\infty$, az 
\[
\frac{f_{n+1}}{f_{n}}\approx\frac{\sqrt{5}-1}{2}\approx1,618=:\phi
\]
értéket kapjuk, amit \textit{aranymetszésnek} nevezünk.
\end{definition}
Ha egy kört két részre osztunk $\frac{1}{\phi}$ arányban, akkor a
keletkező kisebb rész által bezárt szöget \textit{aranyszögnek} nevezzük,
ami megközelítőleg $137,5^{\circ}$.

\begin{figure}[h]
\centering \includegraphics[width=2.915cm,height=2.9cm]{\string"content/Kovacs_Levente/GoldenAngle\string".png}
\caption{A piros színnel jelölt szög az aranyszög.}
\label{fig1} 
\end{figure}


\section*{Házi feladatok}
\begin{problem}
Egy nyári tábor résztvevői a kiránduláson a rajzon látható hegyet
másszák meg. A hegyre két meredek ösvény vezet, valamint egy szerpentin,
ami több helyen találkozik az ösvényekkel. Itt át lehet térni a szerpentinről
a meredek ösvényekre, vagy fordítva (ezeket az \ref{fig2} ábrán betűkkel
jelöltünk). A tábor résztvevői az A pontból mentek fel a hegyen az
I pontnál lévő zászlóhoz úgy, hogy mindig vagy a szerpentinen, vagy
valamelyik meredek ösvényen haladtak. Legfeljebb hányan lehettek,
ha végül megállapították, hogy mindenki különböző úton jutott fel
a hegy tetejére? (Két út különböző, ha legalább egy szakaszuk nem
ugyanaz) \cite{Moz2}.
\end{problem}
\begin{figure}
\centering \includegraphics[width=\linewidth]{\string"content/Kovacs_Levente/hegy\string".png}
\caption{}
\label{fig2} 
\end{figure}

\begin{solution}
A (\ref{fig3}) ábra alapján az $A$ pontból indulva vizsgáljuk meg,
hogy hányféleképpen juthatunk a betűkkel jelzett pontokba. Az ábrán
fel vannak tüntetve minden pont mellé az odavezető utak száma. Nézzük,
hogy ezeket hogyan is kapjuk.

\begin{figure}[h]
\centering \includegraphics[width=0.3\linewidth]{\string"content/Kovacs_Levente/ut\string".png}
\caption{}
\label{fig3} 
\end{figure}

Az \textbf{A} kiindulási pontba csak \textbf{1-féleképpen} juthatunk.

Úgy szintén a \textbf{B pontba} csak a szerpentinen mehetünk, azaz
\textbf{1-féleképpen} juthatunk.

A \textbf{C pontba} mehetünk az $A$-ból a meredek ösvényen vagy $B$-ből
a szerpentinen, ez \textbf{2 lehetőség}.

\textbf{A D pontba} mehetünk C-ből a szerpentinen vagy B-ből a meredeken.
Mivel C-be 2-féleképpen juthattunk, így ehhez a 2-féle C-be vezető
úthoz a CD szakaszt hozzávéve kétféle D-be vezető utat kapunk. B-be
csak 1 út vezetett, így B-n keresztül csak 1 út vezet D-be, vagyis
D-be összesen $2+1=\textbf{3-féleképpen}$ juthatunk.

\textbf{Az E pontba} juthatunk D-ből vagy C-ből. A D-be vezető 3 úthoz
a DE szakaszt hozzávéve 3 különböző, E-be vezető utat kapunk. Ugyanígy
a C-be vezető 2 úthoz a CE szakaszt hozzávéve 2 különböző, E-be vezető
utat kapunk. Nyilvánvalóan más utat kapunk E-be, ha C-n át megyünk,
és mást, ha D-n át, így ez $3+2=\textbf{5}$ út mind különböző.

\textbf{Az F pontba} vezető utak száma: mivel F-be E-ből vagy D-ból
juthatunk, az előzőekhez hasonlóan F-be annyiféle különböző úton lehet
menni, amennyi az E-be és a D-be vezető utak számának az összege,
azaz $5+3=\textbf{8-féleképpen}$.

\textbf{A G pontba} ugyanígy annyiféleképpen juthatunk, ahányféleképpen
F-be meg E-be, így a G-be vezető különböző utak száma: $8+5=\textbf{13}$.

\textbf{A H pontba} ez alapján $13+8=\textbf{21}$ különböző út vezet,
az \textbf{I pontban} levő zászlóhoz pedig $21+13=\textbf{34}$ különböző
út.

Tehát ha mindannyian különböző úton mentek fel, akkor nem lehettek
többen 34-nél. Ha éppen 34-en voltak, akkor minden útvonalon végighaladt
egy kiránduló a táborozók közül.

Vegyük észre, hogy mindegyik pontba annyiféleképpen juthatunk, ahányféleképpen
az előző két pontba összesen, így sorra véve a pontokba vezető utak
számát (\textbf{A, B, C, D, E, F, G, H, I}), a következő sorozatot
kapjuk: $1,1,2,3,5,8,13,21,34$. Ez a sorozat a Fibonacci-sorozat
egy részsorozata. 
\end{solution}
\begin{problem}
Előttünk van egy lépcső, melyen úgy akarunk felmenni, hogy egyszerre
mindig egy vagy két lépcsőfokot lépünk. A lépcső aljáról (első lépcsőfoktól)
indulva hányféleképpen mehetünk fel az $n$-edik lépcsőfokra? 
\end{problem}

\begin{solution}
Legyen $f(n)$ az $n$-edik lépcsőfokra való feljutás módjainak száma.
Mivel minden lépésben egy vagy két lépcsőfokot léphetünk, az $n$-edik
lépcsőfokra csak kétféleképpen juthatunk el: 
\begin{itemize}
\item az $(n-1)$-edik lépcsőfokról, ha egyet lépünk, 
\item az $(n-2)$-edik lépcsőfokról, ha kettőt lépünk. 
\end{itemize}
Példa: legyen $n=3$: A következő lehetőségek vannak: 
\begin{itemize}
\item 1 + 1 + 1 
\item 1 + 2 
\item 2 + 1 
\end{itemize}
\[
f(3)=f(2)+f(1)=2+1=3
\]

A fentiek alapján az alábbi rekurzió adódik: 
\[
f(n)=f(n-1)+f(n-2),
\]
ahol $f(1)=1,\quad f(2)=2$. 
\end{solution}
\begin{problem}
Egy emeletes házat úgy akarunk kifesteni, hogy minden szintet vagy
csupa fehérre, vagy csupa kékre festünk, de két szomszédos szintet
nem festünk fehérre. Hányféleképpen festhetünk ki egy házat, ha a
szintek száma $n$? 
\end{problem}

\begin{solution}
Legyen $F_{n}$ azon a festéseknek a száma, amelyekben az $n$-edik
szint \emph{fehér} és $K_{n}$ azon a festéseknek a száma, amelyekben
az $n$-edik szint \emph{kék}. A megkötés miatt: 
\begin{itemize}
\item Ha az $n$-edik szint fehér, akkor az $(n-1)$-edik csak kék lehet,
vagyis $F_{n}=K_{n-1}$. 
\item Ha az $n$-edik szint kék, akkor az $(n-1)$-edik lehet bármi, vagyis
$K_{n}=F_{n-1}+K_{n-1}$. 
\end{itemize}
Jelölje $a_{n}$ az összes festési lehetőséget. Számoljunk néhány
értéket: 
\begin{itemize}
\item $n=2$: \quad{}$f\ k$, $k\ f$, $k\ k$, vagyis $F_{2}=K_{1}=1$,
\quad{}$K_{2}=F_{1}+K_{1}=1+1=2$, \quad{}$a_{2}=1+2=3$ 
\item $n=3$: \quad{}$f\ k\ f$, $k\ f\ k$, $f\ k\ k$, $k\ f\ f$, $k\ k\ k$,
vagyis $F_{3}=K_{2}=2$, \quad{}$K_{3}=F_{2}+K_{2}=1+2=3$, \quad{}$a_{3}=2+3=5$ 
\item $n=4$: \quad{}$f\ k\ f\ k$, $k\ f\ k\ f$, $k\ k\ f\ k$, $k\ f\ k\ k$,
$k\ k\ k\ f$, $f\ k\ k\ k$, $f\ k\ k\ f$, $k\ k\ k\ k$, vagyis
$F_{4}=K_{3}=3$, \quad{}$K_{4}=F_{3}+K_{3}=2+3=5$, \quad{}$a_{4}=3+5=8$. 
\end{itemize}
Indukcióval bizonyítható, hogy a rekurzió a következő:

\[
F_{n}=K_{n-1},\quad K_{n}=F_{n-1}+K_{n-1},
\]
\[
a_{n}=F_{n}+K_{n},
\]

ahol a kezdőértékek: 
\[
F_{1}=1,\quad K_{1}=1,\quad a_{1}=F_{1}+K_{1}=2.
\]
\end{solution}
\begin{problem}
Határozzuk meg, hogy hány darab olyan $n$ hosszúságú bináris számsorozat
létezik, amelyben nincs két egymást követő nulla? Például $n=2$ esetén
$01;10;11\rightarrow3$ db. 
\end{problem}

\begin{solution}
Jelöljük $a_{n}$-nel azon $n$ hosszúságú bináris számsorozatok számát,
amelyekben nem szerepel két egymást követő nulla. Vizsgáljuk meg:
\begin{itemize}
\item $n=1$: A lehetséges sorozatok: $\{0,1\}$, tehát $a_{1}=2$. 
\item $n=2$: Az engedélyezett sorozatok: $\{01,10,11\}$, tehát $a_{2}=3$. 
\end{itemize}
Egy érvényes $n$ hosszúságú sorozat felépítéséhez nézzük meg az utolsó
számjegyét:
\begin{itemize}
\item Ha 1-gyel végződik, akkor az előző $n-1$ bit bármelyik érvényes $(n-1)$-hosszúságú
sorozat lehet. Ez $a_{n-1}$ lehetőséget jelent. 
\item Ha 0-val végződik, akkor az $(n-1)$. bitnek mindenképp 1-nek kell
lennie (hogy elkerüljük a „00”-t), és az első $n-2$ bit bármilyen
érvényes $(n-2)$-hosszúságú sorozat lehet. Ez $a_{n-2}$ lehetőséget
jelent. 
\end{itemize}
Ezért $n\ge3$ esetén: 
\[
a_{n}=a_{n-1}+a_{n-2},
\]
ahol $a_{1}=2$ és $a_{2}=3$. 
\end{solution}
\begin{problem}
Igazoljuk, hogy 
\[
f_{2}+f_{4}+f_{6}+...+f_{2n}=f_{2n+1}-1,\forall\ n\in\mathbb{N}^{*}
\]
esetén, ahol $f_{n}$ az $n$-edik Fibonacci-szám! 
\end{problem}

\begin{solution}
Legyen $A(n)$ a fenti állítás a Fibonacci-számokra vonatkozóan.

$A(1)$ esetén 
\[
f_{2\cdot1}=f_{2\cdot1+1}-1
\]
\[
f_{2}=f_{3}-1
\]
\[
1=2-1
\]

\noindent Tegyük fel, hogy

\[
A(k):\quad f_{2}+f_{4}+f_{6}+f_{8}+\cdots+f_{2k}=f_{2k+1}-1.
\]

\noindent Ekkor az indukciós feltevés és a Fibonacci-képlet $f_{n+2}=f_{n+1}+f_{n}$
alapján:

\[
\begin{aligned}f_{2}+f_{4}+f_{6}+f_{8}+\cdots+f_{2(k+1)} & =(f_{2}+f_{4}+f_{6}+\cdots+f_{2k})+f_{2(k+1)}\\
 & =f_{2k+1}-1+f_{2(k+1)}\\
 & =f_{2k+1}-1+f_{2k+2}\\
 & =f_{2k+1}+f_{2k+2}-1\\
 & =f_{2k+3}-1\\
 & =f_{2(k+1)+1}-1,
\end{aligned}
\]

tehát következik $A(k+1)$ is. Ezért $A(k)\Rightarrow A(k+1)$.

Az indukció elve alapján tehát $A(n)$ igaz minden $n\in\mathbb{N}^{*}$
esetén.
\end{solution}
\begin{problem}
Igazoljuk, hogy 
\[
{f_{1}}^{2}+{f_{2}}^{2}+...+{f_{n}}^{2}=f_{n}\cdot f_{n+1},\forall\ n\in\mathbb{N}^{*}
\]
esetén, ahol $f_{n}$ az $n$-edik Fibonacci-szám! 
\end{problem}

\begin{solution}
Legyen $A(n)$ a fenti állítás a Fibonacci-számokra vonatkozóan. Matematikai
indukcióval bizonyítjuk az állítást.

$A(1)$ esetén: 
\[
f_{1}^{2}=1^{2}=1=f_{1}f_{2},
\]
hiszen $f_{1}=f_{2}=1$. Tehát az állítás igaz $n=1$-re.

\noindent Tegyük fel, hogy az állítás igaz $n=k$-ra, azaz: 
\[
f_{1}^{2}+f_{2}^{2}+\cdots+f_{k}^{2}=f_{k}f_{k+1}.
\]

\noindent Bizonyítsuk be, hogy az állítás igaz $n=k+1$-re is: 
\[
f_{1}^{2}+f_{2}^{2}+\cdots+f_{k}^{2}+f_{k+1}^{2}=f_{k+1}f_{k+2}.
\]

\noindent Az indukciós feltevés alapján: 
\[
f_{1}^{2}+f_{2}^{2}+\cdots+f_{k}^{2}=f_{k}f_{k+1}.
\]

\noindent Adjuk mindkét oldalhoz $f_{k+1}^{2}$-et: 
\[
f_{1}^{2}+f_{2}^{2}+\cdots+f_{k}^{2}+f_{k+1}^{2}=f_{k}f_{k+1}+f_{k+1}^{2}.
\]

\noindent A jobb oldalon kiemelhető $f_{k+1}$: 
\[
f_{k}f_{k+1}+f_{k+1}^{2}=f_{k+1}(f_{k}+f_{k+1})=f_{k+1}f_{k+2},
\]
hiszen a Fibonacci-számok definíciója szerint $f_{k+2}=f_{k}+f_{k+1}$.

Mivel az állítás igaz $n=1$-re, és ha igaz $n=k$-ra, akkor igaz
$n=k+1$-re is, az indukció elve alapján az $A(n)$ igaz minden $n\geq1$
természetes számra.

\[
\boxed{\sum_{i=1}^{n}f_{i}^{2}=f_{n}f_{n+1}}
\]
\end{solution}

\section*{Nehezebb feladatok}
\begin{extraproblem}[Csapó Hajnalka, Domokos Ábel]
Igazoljuk, hogy minden nem nulla természetes szám esetén:

\[
F_{n-1}\cdot F_{n+1}-F_{n}^{2}=(-1)^{n}.
\]
\end{extraproblem}

\begin{solution}
A matematikai indukció módszerével bizonyítjuk.

\textbf{Ellenőrzés:}

\[
n=1
\]

Ekkor: 
\[
F_{0}=0,\quad F_{1}=1,\quad F_{2}=1
\]
\[
F_{0}\cdot F_{2}-F_{1}^{2}=0\cdot1-1^{2}=-1=(-1)^{1}
\]
Teljesül. 
\[
n=2
\]

Ekkor: 
\[
F_{1}=1,\quad F_{2}=1,\quad F_{3}=2
\]
\[
F_{1}\cdot F_{3}-F_{2}^{2}=1\cdot2-1^{2}=-1=(-1)^{2}
\]
Teljesül.

\textbf{Indukciós lépés:} Feltételezzük, hogy az állítás igaz egy
adott $n$-re: 
\[
F_{n-1}\cdot F_{n+1}-F_{n}^{2}=(-1)^{n}
\]

és bizonyítsuk be, hogy igaz $n+1$-re is: 
\[
F_{n}\cdot F_{n+2}-F_{n+1}^{2}=(-1)^{n+1}
\]

Felhasználjuk a Fibonacci definíciót: $F_{n+2}=F_{n+1}+F_{n}$

Ekkor: 
\[
F_{n}\cdot F_{n+2}-F_{n+1}^{2}=F_{n}(F_{n+1}+F_{n})-F_{n+1}^{2}=F_{n}F_{n+1}+F_{n}^{2}-F_{n+1}^{2}
\]
\[
=(F_{n}F_{n+1}-F_{n+1}^{2})+F_{n}^{2}=-F_{n+1}(F_{n+1}-F_{n})+F_{n}^{2}
\]

Mivel $F_{n+1}-F_{n}=F_{n-1}$, ez: 
\[
=-F_{n+1}F_{n-1}+F_{n}^{2}=-(F_{n-1}F_{n+1}-F_{n}^{2})
\]

Az indukciós feltevés szerint: 
\[
F_{n-1}F_{n+1}-F_{n}^{2}=(-1)^{n}\Rightarrow-(F_{n-1}F_{n+1}-F_{n}^{2})=-(-1)^{n}=(-1)^{n+1}
\]

Tehát az állítás igaz minden $n\geq1$-re. 
\end{solution}
\begin{extraproblem}[Csapó Hajnalka]
Hány 2-vel, 3-mal, 4-gyel, 5-tel osztható Fibonacci-szám van az első
2025 Fibonacci-szám között? 
\end{extraproblem}

\begin{solution}
A 2-vel való osztási maradékok sorozata a rekurzió alapján:

$0,1,1,0,1,1,0,1,1\dots$, azaz azonnal megállapítható, hogy $F_{3k}\equiv0\mod 3$,
$F_{3k+1}\equiv F_{3k+2}\equiv0\mod 3$, tehát a páros számok $F_{0},F_{3},F_{6},\dots F_{2025}$,
azaz összesen 676 szám.

A 3-mal való osztási maradékok sorozata a rekurzió alapján: $0,1,1,2,0,2,2,1,0,1,1,2\dots$,
azaz a maradékok sorozatának periódusa 8, viszont minden negyedik
osztható 3-mal, azaz a 3 többszörösei $F_{0},F_{4},F_{8},\dots F_{2024}$,
azaz összesen 507 szám.

A 4-gyel való osztási maradékok sorozata a rekurzió alapján: $0,1,1,2,3,1,0,1,1,2,3,1\dots$,
azaz a maradékok sorozatának periódusa 6 és minden hatodik szám osztható
4-gyel, azaz a 4 többszörösei $F_{0},F_{6},F_{1}2,\dots F_{2022}$,
azaz összesen 375 szám.

Az 5-tel való osztási maradékok sorozata a rekurzió alapján: 
\[
0,1,1,2,3,0,3,3,1,4,0,4,4,3,2,0,2,2,4,1,0,1,1,2,3,0,\dots
\]
 azaz a maradékok sorozatának periódusa 20, viszont minden ötödik
osztható 5-tel, azaz az 5 többszörösei $F_{0},F_{5},F_{1}0,\dots F_{2025}$,
azaz összesen 406 szám. 
\end{solution}
\begin{extraproblem}[Czofa Vivien]
\textit{\emph{Egy fiú a földszintről indul és felfelé megy a lépcsőn,
lépésenként találomra egy vagy két lépcsőfokot haladva. Mindegyik
emeletre 17 lépcsőfok vezet. Hányféleképpen érhet fel az $n$-edik
emeletre? (Csak a lépcsőkön történő, felfelé irányuló lépéseket vesszük
számításba.)}}\textit{ (Matlap 2006)}
\end{extraproblem}

\begin{solution}
\textbf{(1. megközelítés)} Legyen $x_{k}$ annak a lehetőségeknek
a száma, ahányféleképpen eljuthatunk a $k$-adik lépcsőfokra. A legelső
lépcsőfokra ($k=1$) nyilvánvalóan csak egyféleképpen léphetünk fel:
$x_{1}=1$. A második lépcsőfokra két különböző módon juthatunk: vagy
kétszer egyet lépünk, vagy egyből kettőt, tehát $x_{2}=2$.

Általánosan, a $k$-adik fokra két úton érkezhetünk:

vagy az $(k-1)$-edik fokra értünk előzőleg, és onnan léptünk egyet,

vagy a $(k-2)$-edik fokon álltunk, és onnan ugrottunk kettőt.

Ez alapján:

\[
x_{k}=x_{k-1}+x_{k-2},
\]

azaz a $x_{k}$ sorozat a Fibonacci-sorozat szabályait követi. Mivel
17 lépcsőfok vezet az emeletre, a kérdés az $x_{17}$ meghatározása.
A Fibonacci-sorozat első néhány tagja:

\[
x_{1}=1,\quad x_{2}=2,\quad x_{3}=3,\quad x_{4}=5,\quad x_{5}=8,\quad x_{6}=13,\quad\ldots
\]

A 17-edik tag értéke: $x_{17}=2584$.

Tehát a fiú \textbf{2584-féleképpen} juthat fel egy emeletre. Következésképpen
az $n$-edik emelet elérésének módjainak száma is: $2584^{n}$.

\textbf{(2. megközelítés)} Alternatívaként a különböző feljutási lehetőségek
számát kombinatorikus módszerrel is kiszámíthatjuk. Tegyük fel, hogy
a fiú $k$ darab kétlépéses és $(17-2k)$ darab egy lépéses lépést
tesz meg. Ekkor a teljes lépésszám: $k+(17-2k)=17-k$.

A lépések sorrendjének megválasztására $\binom{17-k}{k}$ mód van.
Összesen tehát:

\[
S=\sum_{k=0}^{8}\binom{17-k}{k}=\binom{17}{0}+\binom{16}{1}+\binom{15}{2}+\cdots+\binom{9}{8}=2584.
\]

Így tehát ismét azt kapjuk, hogy $x_{17}=2584$, vagyis a fiú pontosan
ennyi különböző módon érhet fel egy emeletre. 
\end{solution}
\begin{extraproblem}[Czofa Vivien]
\textit{\emph{Adott $n$ ház egy sorban. A házakat le szeretnénk
festeni a zöld és barna színek egyikével. Hányféle olyan festés létezik,
amelyben nincs két egymás melletti barna ház?}}
\end{extraproblem}

\begin{solution}
Jelöljük $a_{n}$-nal azoknak a festési módoknak a számát, amelyek
kielégítik a feltételt $n$ darab ház esetén. Mivel minden ház lehet
barna (B) vagy zöld (Z), egyetlen ház esetén nyilvánvalóan két lehetőségünk
van ($a_{1}=2$). Két ház esetén a lehetséges kombinációk: BZ, ZB,
ZZ (a BB tilos), tehát $a_{2}=3$.

Most tekintsük az általános $n$-edik esetet. A sorban első ház kétféleképpen
lehet befestve:

\textbf{Ha zöld:} akkor a maradék $n-1$ ház bármilyen, a feltételnek
megfelelő módon színezhető, azaz $a_{n-1}$ különböző lehetőség áll
rendelkezésre.

\textbf{Ha barna:} akkor a második ház csak zöld lehet (a BB kombináció
nem engedélyezett). A fennmaradó $n-2$ ház szabadon festhető, a feltétel
szerint, így ezek száma $a_{n-2}$.

Ezek alapján az alábbi rekurzió adódik:

\[
a_{n}=a_{n-1}+a_{n-2},\quad\text{ahol }a_{1}=2,\quad a_{2}=3.
\]

Ez pontosan a Fibonacci-sorozat definícióját követi, eltolva indexekkel:
$a_{n}=F_{n+1}$, ahol $(F_{n})$ a Fibonacci-számok sorozata. Így:

\[
a_{n}=F_{n+1}.
\]

Zárt alakban, Binet-formulával kifejezve:

\[
a_{n}=\frac{1}{\sqrt{5}}\left[\left(\frac{1+\sqrt{5}}{2}\right)^{n+1}-\left(\frac{1-\sqrt{5}}{2}\right)^{n+1}\right].
\]
\end{solution}
\begin{extraproblem}[Fábián Nóra]
Hány olyan részhalmaza van az $f\{1,2,\ldots,n\}$ halmaznak, amelyek
elemei között nincs két egymásutáni szám? 
\end{extraproblem}

\begin{solution}
Jelölje $x_{n}$ ezt a számot. Itt $x_{1}=2$ (mert jó az $\emptyset$
és az $\{1\}$), $x_{2}=3$ (jó: $\emptyset$, $\{1\}$, $\{2\}$,
nem jó: $\{1,2\}$), $x_{3}=5$ (jó: $\emptyset$, $\{1\}$, $\{2\}$,
$\{3\}$, $\{1,3\}$, nem jó: $\{1,2\}$, $\{2,3\}$, $\{1,2,3\}$).

Legyen $n\geq3$ tetszőleges, és tekintsük az $f\{1,2,\ldots,n\}$
halmaz olyan $H$ részhalmazait, amelyek elemei között nincs két egymásutáni
szám. Csoportosítsuk ezeket így:
\begin{itemize}
\item[I.] Ha $n\in H$, akkor $n-1\notin H$ és annyi $H$ részhalmaz van,
amennyi az $x_{n-2}$ érték. 
\item[II.] Ha $n\notin H$, akkor annyi $H$ részhalmaz van, amennyi az $x_{n-1}$
érték. 
\end{itemize}
Ezért $x_{n}=x_{n-2}+x_{n-1}$, és kapjuk, hogy $x_{n}=F_{n+2}$,
ahol $n\geq1$ és $F_{1}=F_{2}=1$.
\end{solution}
\begin{extraproblem}[Gál Tamara]
Egy fiú a földszintről indul és felfelé megy a lépcsőn, lépésenként
találomra egy vagy két lépcsőfokot haladva. Mindegyik emeletre 17
lépcsőfok vezet. Hányféleképpen érhet fel az n-edik emeletre? (Csak
a lépcsőkön történő, felfelé irányuló lépéseket vesszük számításba.)
(Matlap 2006) 
\end{extraproblem}

\begin{solution}
Jelöljük $x_{k}$-val a $k$-adik lépcsőfokra jutás lehetőségeit.
Az első lépcsőfokra egyféleképpen lehet jutni: $x_{1}=1$. A második
lépcsőfokra kétféleképpen juthat: kétszer lép egy lépcsőfokot, vagy
egyszer lép két lépcsőfokot: $x_{2}=2$. A $k$-adik lépcsőfokra úgy
juthat, hogy a $(k-2)$-edik lépcsőfokról lép a $k$-adikra, vagy
a $(k-1)$-edikről a $k$-adikra, tehát $x_{k}=x_{k-2}+x_{k-1}$.
Ez azt jelenti, hogy $x_{k}$ egy Fibonacci-féle sorozat $k$-adik
tagja. Tehát $x_{17}$-et kell meghatározni, például a tagok felsorolása
által: $1,\ 2,\,3,\,5,\ 8,\ 13,\ 21,\ 34,\ 55,\ 89,\ 144,\ 233,\ 377,\ 610,\ 987,\ 2584,\dots$.\\
 Tehát $x_{17}=2584$-féleképpen juthat fel egy emeletre, következik,
hogy az $n$-edik emeletre való feljutási lehetőségek száma: $2584^{n}$. 
\end{solution}
\begin{extraproblem}[Gál Tamara]
Egy macska egy 10 szintes lépcső legfelső szintjéről szeretne lejutni
a talajra néhány ugrással. Hányféleképpen teheti ezt meg, ha egyszerre
mindig csak 1 vagy 2 szintet ugorhat lefelé, de nem ugorhat rá az
alulról 4. lépcsőre? 
\end{extraproblem}

\begin{solution}
Ha a 4. lépcsőfok kihagyására vonatkozó feltételtől eltekintenénk,
akkor éppen a Fibonacci-sorozat tagjait kapnánk: 1, 2, 3, 5, 8, 13,
21, 34, 55, 89. Így összesen 89 útvonal lenne, amelyekből ki kell
vonnunk a rossz -- tehát a 4. lépcsőfokot érintő -- utak számát.
A 10. lépcsőfokról a 4. lépcsőfokig 6 lépcsőt kell megtenni, tehát
itt éppen annyi út vezet, mint amennyi a 6. lépcsőfokról a talajra
vezetne, vagyis az előbbi sorozat 6. tagja, ami a 13. A 4. lépcsőfokról
a talajig 5 út vezet, így a 4. lépcsőfokot érintő útvonalak száma
$13\cdot5=65$. Vagyis a jó útvonalak száma $89-65=24$. 
\end{solution}
\begin{extraproblem}[Gergely Verona]
Fibonacci-sorozatnak nevezzük az $F_{1}=1$, $F_{2}=1$, $F_{n}=F_{n-1}+F_{n-2}$
($n\ge3$) kikötésekkel értelmezett sorozatot. Bizonyítsuk be, hogy
ha $0<p<n$, akkor az $F_{2p+1}+F_{2p+3}+\ldots+F_{2n+1}$ és $F_{2p}+F_{2p+2}+\ldots+F_{2n}$
számok nem lehetnek a Fibonacci-sorozat tagjai.\emph{ (IV. Nemzetközi
Magyar Matematika Verseny) }
\end{extraproblem}

\begin{solution}
Jelöljük az első számot $A$-val, a másodikat $B$-vel! A Fibonacci-sorozat
definícióját felhasználva 
\[
A+B=F_{2p+2}+F_{2p+4}+\ldots+F_{2n+2}
\]
\[
A-B=F_{2p-1}+F_{2p+1}+\ldots+F_{2n-1}
\]

Továbbá $A=(A+B)-B$ miatt $A=F_{2n+2}-F_{2p}$, és $B=A-(A-B)$ miatt
$B=F_{2n+1}-F_{2p-1}$. A sorozat nyilvánvalóan növekvő és tagjai
pozitívak, továbbá $F_{2n+2}=F_{2n+1}+F_{2n}$, így hát $A=F_{2n+2}-F_{2p}>F_{2n+2}-F_{2n}=F_{2n+1}$,
de nyilván $A<F_{2n+2}$. Teljesen hasonló módon $F_{2n}<B<F_{2n+1}$,
mivel azonban a sorozat monoton növekszik, ezért két szomszédos tagja
között nem lehet újabb tag, tehát $A$ és $B$ valóban nem a Fibonacci-sorozat
tagjai. 
\end{solution}
\begin{figure}
\centering \includegraphics[width=0.5\linewidth]{\string"content/Kovacs_Levente/Fibo\string".png}
\label{Fibo} \caption{}
\end{figure}

\begin{extraproblem}[Kiss Andrea-Tímea]
A \aref{Fibo}.ábrán látható 26 mezőből álló ,,tábla'' hányféleképpen
fedhető 13 ,,dominóval''? Egy-egy dominó két szomszédos mezőt fed
le. (Az egymásba forgatható megoldásokat különbözőnek tekintjük.) 
\end{extraproblem}

\begin{solution}
Ha a már dominóval lefedett ábrát az egyetlen függőleges 2 hosszú
szakasza mentén (azaz "12 óránál") elvágjuk, akkor két eset lehetséges:
vagy nem vágunk el dominót, vagy pontosan két dominót vágunk el. Ugyanis,
ha egyetlen dominót vágnánk ketté, akkor az kikényszerítené, hogy
a 26 mező ,,külső'' 13 mezőjét és a ,,belső'' 13 mezőjét is úgy
kellene lefednünk dominókkal, hogy minden dominó vagy két külső, vagy
két belső mezőt fedne le, ami (mivel 13 nem páros) lehetetlen.

\medskip{}

\textbf{a)} Ha nem vágunk el dominót. Az ábrát ,,kiterítve'' $13\times2$
méretű téglalapot kell lefednünk dominókkal. (Ez egy "klasszikus"
Fibonacci feladat.)\\
 Megmutatjuk, hogy tetszőleges $n\times2$ méretű téglalapot $f_{n+1}$
féleképpen fedhetünk le (ahol $f_{n}$ az $f_{1}=f_{2}=1;\quad f_{n+2}=f_{n}+f_{n+1}$
rekurzióval definiált Fibonacci-sorozat.)

\medskip{}

Jelölje $l_{n}$ az $n\times2$ méretű téglalap megfelelő lefedéseinek
a számát. Az $1\times2$ és a $2\times2$ méretű téglalapokat nyilván
$l_{1}=1$, illetve $l_{2}=2$ féleképpen fedhetjük le, míg, ha $n>2$,
akkor az $n\times2$ méretű téglalap esetén a bal felső sarokban lévő
mezőt vagy egy álló dominóval fedjük le, és ekkor a maradék $(n-1)\times2$
méretű téglalapot $l_{n-1}$-féleképpen fedhetjük le; vagy egy fekvő
dominóval fedjük le, ami kikényszeríti, hogy ezen dominó alatt lévő
két mezőt szintén egy fekvő dominóval fedjünk le, és ekkor a maradék
$(n-2)\times2$ méretű téglalap $l_{n-2}$-féleképpen fedhető le.
Azaz ekkor $l_{n}=l_{n-1}+l_{n-2}$, ami (a fenti $l_{1},l_{2}$ értékek
mellett) azt jelenti, hogy a lefedések száma valóban $l_{n}=f_{n+1}$.
Mivel most $n=13$, ezen az ágon $f_{14}=377$ eset van.

\medskip{}

\textbf{b)} Ha két dominót vágunk ketté. Ekkor a két szétvágott dominót
az ábrából elhagyva és a maradék ábrát ,,kiterítve'': $11\times2$
méretű téglalapot kell lefednünk dominókkal.

\medskip{}

Ez az az előző \textbf{a)} pont 2-vel kisebb méretben; azaz ekkor
$f_{12}=144$ eset van.

\medskip{}

Tehát összesen $377+144=\boxed{521}$ eset van.
\end{solution}
\begin{extraproblem}[Kovács Levente]
Legyen adott az alábbi sorozat: 

\[
a_{n}=F_{n+1}+2F_{n},\quad\text{ahol }F_{n}\text{ a Fibonacci-számok sorozata.}
\]
\end{extraproblem}

\begin{solution}
~
\begin{enumerate}
\item Határozzuk meg az $a_{n}$ sorozat generátorfüggvényét! 
\item Fejezzük ki az $a_{n}$ sorozatot zárt alakban a Binet-formula segítségével! 
\item A Fibonacci-sorozat generátorfüggvénye: 
\[
F(x)=\sum_{n=0}^{\infty}F_{n}x^{n}=\frac{x}{1-x-x^{2}}
\]

A megadott sorozat: 
\[
a_{n}=F_{n+1}+2F_{n}
\]
Ez alapján a generátorfüggvény: 
\[
A(x)=\sum_{n=0}^{\infty}a_{n}x^{n}=\sum_{n=0}^{\infty}(F_{n+1}+2F_{n})x^{n}=\sum_{n=0}^{\infty}F_{n+1}x^{n}+2\sum_{n=0}^{\infty}F_{n}x^{n}
\]

Tudjuk, hogy: 
\[
\sum_{n=0}^{\infty}F_{n+1}x^{n}=\frac{F(x)}{x}\quad\text{és}\quad\sum_{n=0}^{\infty}F_{n}x^{n}=F(x)
\]

Ezért: 
\[
A(x)=\frac{F(x)}{x}+2F(x)=\left(\frac{1}{x}+2\right)\cdot\frac{x}{1-x-x^{2}}=\frac{1+2x}{1-x-x^{2}}
\]

\item A Fibonacci-számok zárt alakja: 
\[
F_{n}=\frac{1}{\sqrt{5}}\left(\varphi^{n}-\bar{\varphi}^{n}\right),\quad\text{ahol }\varphi=\frac{1+\sqrt{5}}{2},\;\bar{\varphi}=\frac{1-\sqrt{5}}{2}
\]

Ez alapján: 
\[
\begin{aligned}a_{n} & =F_{n+1}+2F_{n}=\frac{1}{\sqrt{5}}\left(\varphi^{n+1}-\bar{\varphi}^{n+1}+2\varphi^{n}-2\bar{\varphi}^{n}\right)\\
 & =\frac{1}{\sqrt{5}}\left(\varphi^{n}(\varphi+2)-\bar{\varphi}^{n}(\bar{\varphi}+2)\right)
\end{aligned}
\]

Mivel: 
\[
\varphi+2=\frac{5+\sqrt{5}}{2},\quad\bar{\varphi}+2=\frac{5-\sqrt{5}}{2}
\]

A végső zárt alak: 
\[
a_{n}=\frac{1}{\sqrt{5}}\left[\left(\frac{5+\sqrt{5}}{2}\right)\varphi^{n}-\left(\frac{5-\sqrt{5}}{2}\right)\bar{\varphi}^{n}\right].
\]

\end{enumerate}
\end{solution}
\begin{extraproblem}[Lukács Andor]
Határozd meg az $m^{2}+n^{2}$ maximális értékét, ha az $m,n$ olyan
1981-nél kisebb pozitív egész számok, amelyek teljesítik az 
\begin{equation}
(n^{2}-mn-m^{2})^{2}=1\label{eq:fib_ta1}
\end{equation}
összefüggést! 
\begin{flushright}
(IMO, 1981) 
\par\end{flushright}
\end{extraproblem}

\begin{solution}
Ha az $(m,n)$ pár teljesíti \aref({eq:fib_ta1}) összefüggést és $0<m<n$,
akkor $m<n\le2m$ és 
\begin{align*}
(n^{2}-mn-m^{2})^{2} & =\left((n-m)^{2}+mn-2m^{2}\right)^{2}\\
 & =\left((n-m)^{2}+m(n-m)-m^{2}\right)^{2}\\
 & =\left(m^{2}-m(n-m)-(n-m)^{2}\right),
\end{align*}
tehát az $(n-m,m)$ számpár is teljesíti \aref({eq:fib_ta1}) összefüggést
és $n-m\le m$. Következésképpen, ha egy kiinduló $(m,n)$, $0<m<n$
számpárt lépésenként helyettesítünk az $(n-m,m)$ számpárral, akkor
az új számpár is teljesíti az összefüggést és $0<n-m\le m$. Ezért
véges sok lépés után le kell álljon a folyamat. Viszont a leállás
csak egy $(k,k)$ számpárban történhet ($k\in\mathbb{N}^{*}$) és
az ilyen számpárok közül egyedül az $(1,1)$ teljesíti \aref({eq:fib_ta1})
összefüggést.

Az előbb vázolt gondolatmenetből viszont az következik, hogy az összes
olyan $(m,n)$ számpár, amely teljesíti \aref({eq:fib_ta1}) összefüggést
elérhető az $(1,1)$ számpárból a $(k,l)\mapsto(l,k+l)$ transzformációval
(esetleg a végén még végre kell hajtani egy $(k,l)\mapsto(l,k)$ transzformációt
is az utolsó lépésben, ha $m>n$).

Az $(1,1)$ számpárból a következők érhetők el a transzformációval:
\[
(1,1)\mapsto(1,2)\mapsto(2,3)\mapsto(3,5)\mapsto(5,8)\mapsto\dots\mapsto(F_{k},F_{k+1})\mapsto\dots,
\]
ahol $F_{k}$-val jelöltük a $k$-adik Fibonacci-számot.

Tehát \aref({eq:fib_ta1}) összefüggést teljesítő számpárok egymásutáni
Fibonacci-számokból állnak. Az utolsó, 1981-nél kisebb Fibonacci-szám
$F_{17}=1597,$ tehát a feladatban feltett kérdésre a válasz 
\[
F_{16}^{2}+F_{17}^{2}=987^{2}+1597^{2}=3524578.
\]
\end{solution}
\begin{extraproblem}[Lukács Andor]
Előbb igazold, hogy ha létezik olyan $(x,y,z)$ pozitív egészekből
álló számhármas, amely teljesíti az 
\begin{equation}
x^{2}+y^{2}+1=xyz\label{eq:ta5}
\end{equation}
egyenletet, akkor $z=3$, majd határozd meg az összes megoldását az
egyenletnek! 
\end{extraproblem}

\begin{solution}
Tételezzük fel, hogy $(x,y,z)$ olyan megoldás, amelyre $z\ne3$.
Ebben az esetben $x\ne y,$ mivel ellenkező esetben teljesülne az
$x^{2}(z-2)=1$ összefüggés, ami lehetetlen. \Aref({eq:ta5}) egyenletet
hozzuk az 
\[
(yz-x)^{2}+y^{2}+1=(yz-x)yz
\]
ekvivalens alakra. Ebből az alakból következik, hogy ha $(x,y,z)$
pozitív egészekből álló megoldása az egyenletnek, akkor $(yz-x,y,z)$
is az, mivel $x(yz-x)=xyz-x^{2}=y^{2}+1>0$. Az előző tárgyalás és
a szimmetria alapján ugyanakkor feltételezhetjük, hogy a kiinduló
$(x,y,z)$ olyan megoldás, amelyre teljesül, hogy $x>y$ és $x+y$
minimális. Viszont ekkor $x^{2}>y^{2}+1=x(yz-x)$, tehát $x>yz-x$.
Tehát az új $(yz-x,y,z)$ vagy $(y,yz-x,z)$ megoldások közül az amelyiknek
az első tagja nagyobb, mint a második ellentmond annak, hogy a kiinduló
megoldás első két tagjának összege minimális volt. Összefoglalva,
csak akkor van megoldása az egyenletnek, ha $z=3.$

Vizsgáljuk meg a $z=3$ esetet, vagyis oldjuk meg az 
\begin{equation}
x^{2}+y^{2}+1=3xy\label{eq:ta6}
\end{equation}
egyenletet a pozitív egészeken! Észrevesszük, hogy $(1,1)$ az egyetlen
olyan megoldás, amelyre $x=y$. Ha $(x,y)$ egy másik pozitív egészekből
álló megoldás, akkor feltételezhetjük a szimmetria miatt, hogy $x>y\ge1$.
\Aref({eq:ta6}) egyenlet ekvivalens az 
\[
y^{2}+(3y-x)^{2}+1=3y(3y-x)
\]
egyenlettel, ezért $(y,3y-x)$ is pozitív egészekből álló megoldás,
mert $x(3y-x)=3xy-x^{2}=y^{2}+1>0$. Ugyanakkor $y\ge3y-x$ is teljesül,
mivel ez az egyenlőtlenség ekvivalens az $xy\ge y^{2}+1$ igaz egyenlőtlenséggel
(mert $x>y$). Ha a szerkesztett új megoldásban egyenlőség áll fent,
vagyis $y=3y-x$, akkor láttuk, hogy ez a megoldás csak az $(1,1)$
lehet, tehát $x=2$ és $y=1$. Az összes többi esetben viszont a kiinduló
$(x,y),$ $x>y$ megoldásból szerkesztettünk egy \emph{kisebb} $(y,3y-x)$,
$y>3y-x$ megoldást.

Az előbbi eredmények alapján, ha iteráljuk a szerkesztést, véges sok
lépés után el kell jutnunk a tetszőleges $(x,y),$ $x>y$ megoldásból
a $(k,l)\mapsto(l,3l-k)$ transzformációval az $(1,1)$ megoldásba.
Az előző feladat megoldásához hasonlóan ez azt jelenti, hogy az egyenlet
összes olyan $(x,y)$ megoldása, amelyre $x>y$ az $(x_{1},y_{1})=(1,1)$
megoldásból generálható az 
\[
(x_{n},y_{n})=(y_{n+1},3y_{n+1}-x_{n+1})
\]
\emph{fordított rekurzió} segítségével véges sok lépésben. A rekurzió
alapján $x_{n}=y_{n+1}$ és $x_{n-1}=3x_{n}-x_{n+1}$, valamint $x_{1}=y_{1}=1.$
Viszont ez az $(x_{n})$ sorozat pontosan a páratlan indexű tagokból
álló Fibonacci-sorozat, tehát \aref({eq:ta6}) egyenlet megoldásai
az $(1,1)$, $(F_{2k+1},F_{2k-1})$ és $(F_{2k-1},F_{2k+1})$ alakú
számpárok. 
\end{solution}
\begin{extraproblem}
Andrea azt állítja, hogy ismer a Fibonacci-számokra egy másik képletet:
$F_{n}=\left\lceil e^{n/2-1}\right\rceil ,\quad n=1,2,$\ldots ,

ahol e = 2{,}718281828\ldots a természetes logaritmus alapja.

Igaza van? 
\end{extraproblem}

\begin{solution}
A képlet $n=1,2,\ldots,10$ esetén működik, $n=11$-re viszont már
nem jó: ekkor ugyanis 91-et ad. A hiba $n$-nel együtt növekszik.

\[
F_{n}\sim\frac{1}{\sqrt{5}}\left(\frac{1+\sqrt{5}}{2}\right)^{n}=(0{,}447\ldots)\cdot(1{,}618\ldots)^{n}
\]

Andrea képletében a kerekítés jelentősége folyamatosan elhalványul:

\[
\left\lceil e^{n/2-1}\right\rceil \sim e^{n/2-1}=(0{,}367\ldots)\cdot(1{,}648\ldots)^{n},
\]

az Andrea-féle számok és a Fibonacci-számok hányadosa így:

\[
\frac{\left\lceil e^{n/2-1}\right\rceil }{F_{n}}\approx\frac{(0{,}367\ldots)\cdot(1{,}648\ldots)^{n}}{0{,}447\ldots}=(0{,}822\ldots)\cdot(1{,}018\ldots)^{n}.
\]

Mivel a hatvány alapja nagyobb, mint 1, az eltérés „minden határon
túl” növekszik, amint $n$ egyre nagyobb. 
\end{solution}
\begin{extraproblem}[Seres Brigitta-Alexandra]
Fibonacci-sorozatnak nevezzük az $F_{1}=1$, $F_{2}=1$, $F_{n}=F_{n-1}+F_{n-2}$
($n\ge3$) kikötésekkel értelmezett sorozatot. Bizonyítsuk be, hogy
ha $0<p<n$, akkor az $F_{2p+1}+F_{2p+3}+\ldots+F_{2n+1}$ és $F_{2p}+F_{2p+2}+\ldots+F_{2n}$
számok nem lehetnek a Fibonacci-sorozat tagjai. 
\begin{flushright}
\textit{(NMMV, 1995, XII. osztály, 6. feladat)} 
\par\end{flushright}
\end{extraproblem}

\begin{solution}
Jelöljük el 
\[
\begin{aligned}A= & F_{2p+1}+F_{2p+3}+\ldots+F_{2n+1}\\
B= & F_{2p}+F_{2p+2}+\ldots+F_{2n}.
\end{aligned}
\]
A rekurzív összefüggés alapján 
\[
\begin{aligned}A+B= & (F_{2p+1}+F_{2p})+(F_{2p+3}+F_{2p+2})+\ldots+(F_{2n+1}+F_{2n})\\
= & F_{2p+2}+F_{2p+4}+\ldots+F_{2n+2},
\end{aligned}
\]
valamint 
\[
\begin{aligned}A-B= & (F_{2p+1}-F_{2p})+(F_{2p+3}-F_{2p+2})+\ldots+(F_{2n+1}-F_{2n})\\
= & F_{2p-1}+F_{2p+1}+\ldots+F_{2n-1}.
\end{aligned}
\]
Ekkor igaz, hogy 
\[
A=(A+B)-B=F_{2n+2}-F_{2p},
\]
valamint 
\[
B=A-(A-B)=F_{2n+1}-F_{2p-1}.
\]
A Fibonacci-sorozat nyilvánvalóan szigorúan növekvő sorozat (első
két tagjától eltekintve, melyek egyenlőek), valamint tagjai pozitívak.
Mivel $0<p<n$, így $F_{2p}<F_{2n}$ így 
\[
F_{2n+1}=F_{2n+2}-F_{2n}<F_{2n+2}-F_{2p}=A.
\]
De $A=F_{2n+2}-F_{2p}<F_{2n+2}$ is teljesül, így beláttuk, hogy $F_{2n+1}<A<F_{2n+2}$.\\
 Hasonlóan mivel $F_{2p-1}<F_{2n-1}$ 
\[
F_{2n}=F_{2n+1}-F_{2n-1}<F_{2n+1}-F_{2p-1}=B,
\]
és $B=F_{2n+1}-F_{2p-1}<F_{2n+1}$. Beláttuk, hogy $F_{2n}<B<F_{2n+1}$.

Tehát $F_{2n+1}<A<F_{2n+2}$ és $F_{2n}<B<F_{2n+1}$ teljesül, de
a Fibonacci-sorozat szigorúan monoton növekvő (első két tagjától eltekintve),
így két szomszédos tagja között nem lehet újabb tag, ezért kijelenthető
hogy $A$ és $B$ valóban nem a Fibonacci-sorozat tagjai.
\end{solution}
\subsection*{Érdekességek az aranymetszésről, melyekkel a diákok figyelme
is felkelthető:} 
\begin{itemize}
\item A természetben (például az emberi testen vagy csigák mészvázán) és
művészetben is gyakran megjelenik, természetes egyensúlyt teremtve
a szimmetria és az aszimmetria között.(lásd \ref{1}., \ref{2}. ábra) 
\item Gyakori megjelenése miatt a geometriában már ókori matematikusok is
tanulmányozták az aranymetszést. Bizonyíthatóan az ókori Egyiptomban
is értették és használták ezt a törvényszerűséget. Az i. e. 2600 körül
épült gízai nagy piramis arányaiban is felfedezhető az aranymetszés
aránya. (lásd \ref{6}. ábra) 
\item Az aranymetszés jelölése, a $\phi$ (görög nagy fí betű) Pheidiász
görög szobrász nevéből származik. 
\item A művészetekben való alkalmazásának elsődleges oka a triviális szimmetriától
különböző kellemes hatást kiváltó szabályos "aszimmetria" keresése
volt. 
\item Leonardo da Vinci tudatosan alkalmazta az aranymetszést az alkotásaiban,
valamint ő volt az, aki először állította, hogy az ember csontjait
tekintve $\phi$ arányú építőkövekből áll. (lásd \ref{3}. ábra) 
\item A gyakorlati igazolást Adolph Zeising \textit{A kísérleti esztétikából}
című műben publikálja nagyszámú embercsoporton végzett mérésekből. 
\item Az embereken végzett kísérlet is azt mutatta, hogy minél kevésbé tér
el az egyed az aranymetszés arányától, annál vonzóbbnak, szebbnek
tűnik. (lásd \ref{4}., \ref{5}. ábra) 
\end{itemize}
\begin{figure}[h!]
\centering \includegraphics[scale=0.8]{\string"content/Kovacs_Levente/1\string".jpg}
\vspace{2mm}
 \caption{Fibonacci-sorozat tagjai reprezentálva}
\label{1}
\end{figure}

\begin{figure}[h!]
\centering \includegraphics[scale=0.4]{\string"content/Kovacs_Levente/3\string".jpg}
\vspace{2mm}
 \caption{Leonardo da Vinci, \textit{Mona Lisa}}
\label{2}
\end{figure}

\begin{figure}[h!]
\centering \includegraphics[scale=1.7]{\string"content/Kovacs_Levente/8\string".jpg}
\vspace{2mm}
 \caption{A Nagy Piramis, a piramis alapélének a fele (átlag 115,18 m) és oldallapjainak
a magassága (kb. 186,42 m) az aranymetszés szerint aránylik egymáshoz}
\label{6}
\end{figure}

\begin{figure}[h!]
\centering \includegraphics[scale=0.3]{\string"content/Kovacs_Levente/2\string".jpg}
\vspace{2mm}
 \caption{Leonardo da Vinci, az ún. \textbf{Vitruvius-tanulmány}, \textit{„Az
ember kinyújtott karjainak hossza megegyezik a magasságával. A haja
tövétől az álla hegyéig terjedő szakasz egytizede a magasságnak; az
álla hegyétől a feje tetejéig terjedő szakasz egynyolcada a magasságának;
a mellkasa tetejétől a haja tövéig egyhetede az egész embernek.”}}
\label{3}
\end{figure}

\newpage{}
\begin{figure}[h!]
\centering \includegraphics[scale=0.7]{\string"content/Kovacs_Levente/6\string".jpg}
\vspace{2mm}
 \caption{Balról jobbra: Marilyn Monroe filmsztár, Dammak Jázmin, Miss Universe
Hungary 2008, Angelina Jolie filmsztár. A homlokvonal, szem és száj
az aranymetszés arányában helyezkedik el a szépségek arcain}
\label{4}
\end{figure}

\begin{figure}[h!]
\centering \includegraphics[scale=0.7]{\string"content/Kovacs_Levente/7\string".jpg}
\vspace{2mm}
 \caption{Albert Einstein, Fizikai Nobel-díj, 1921. Látható, hogy a hosszabb
orr-része, és lefelé görbülő szemek miatt nem követi az arc az aránymetszés
arányait (baloldalon). A jobb oldali fotón az aranymetszésre korrigált
arc látható.}
\label{5}
\end{figure}

\begin{extraproblem}[Sógor Bence]
Határozd meg az 
\[
\sum_{k\geq0}\frac{1}{F_{2k+1}+1}
\]
összeg értékét, ahol $F_{2k+1}$ a $2k+1$-edik Fibonacci-számot jelöli. 
\end{extraproblem}

\begin{solution}
Próbáljuk meg kifejezni a Fibonacci-számokat csak $\varphi=\frac{1+\sqrt{5}}{2}$
segítségével. A Fibonacci-számokra tudjuk, hogy $F_{n}=\frac{\varphi^{n}-\overline{\varphi}^{n}}{\sqrt{5}}$,
ahol $\varphi=\frac{1+\sqrt{5}}{2}$ és $\overline{\varphi}=\frac{1-\sqrt{5}}{2}$.
Mivel $\overline{\varphi}=-\frac{1}{\varphi}$, ezért az összegben
levő törtet az alábbi átalakításokat használva írhatjuk fel. 
\[
F_{2k+1}=\frac{\varphi^{2k+1}-\frac{-1}{\varphi}^{2k+1}}{\sqrt{5}}=\frac{\varphi^{4k+2}+1}{\sqrt{5}\varphi^{2k+1}}
\]
Innen 
\[
\frac{1}{F_{2k+1}+1}==\frac{\sqrt{5}\varphi^{2k+1}}{\varphi^{4k+2}+\sqrt{5}\varphi^{2k+1}+1}
\]
Az $u=\varphi^{2k+1}$ jelölést bevezetve feltudjuk bontani a törtet
két $\frac{A}{u+a}$, $\frac{B}{u+b}$ alakú kifejezés összegére.
Az alábbi észrevételeket és átalakításokat használva is eljuthatunk
a felbontáshoz. Tujduk, hogy $\varphi^{2}=\varphi+1$. 
\[
\frac{\varphi^{2k+1}}{\varphi^{4k+2}+\sqrt{5}\varphi^{2k+1}+1}=\frac{\varphi^{2k}(\varphi^{2}-1)}{\varphi^{2k+2}\varphi^{2k}+(2\varphi-1)\varphi\varphi^{2k}+1}
\]

\[
\frac{\varphi^{2k}(\varphi^{2}-1)}{\varphi^{2k+2}\varphi^{2k}+(2\varphi^{2}-\varphi)\varphi^{2k}+1}=\frac{\varphi^{2k}(\varphi^{2}-1)}{\varphi^{2k+2}\varphi^{2k}+(\varphi^{2}+1)\varphi^{2k}+1}
\]
\[
\frac{\varphi^{2k+2}-\varphi^{2k}}{\varphi^{2k+2}\varphi^{2k}+\varphi^{2k+2}+\varphi^{2k}+1}=\frac{1}{\varphi^{2k}+1}-\frac{1}{\varphi^{2k+2}+1}.
\]

Ezek alapján, 
\[
\sum_{k\geq0}\frac{1}{F_{2k+1}+1}=\sqrt{5}\sum_{k\geq0}\frac{1}{\varphi^{2k}+1}-\frac{1}{\varphi^{2k+2}+1}=\frac{\sqrt{5}}{2}.
\]
\end{solution}
\begin{extraproblem}[Száfta Antal]
Legyen $a$ és $b$ két egész szám. Határozzuk meg $a$ értékét,
ha ismert, hogy az $x^{2}-x-1$ polinom osztója az 
\[
ax^{17}+bx^{16}+1
\]
kifejezésnek. \emph{(1988 AIME, 13-as számú probléma) }
\end{extraproblem}

\begin{solution}
Tegyük fel, hogy $x^{2}-x-1=0$. Ez a Fibonacci-rekurzióval azonos:
\[
x^{n}=F_{n}(x),\text{ ahol }x^{2}=x+1\text{ és }F_{n+2}=F_{n+1}+F_{n}
\]

Ebből következik: 
\[
x^{n+2}=F_{n+1}x+F_{n}\quad\text{(rekurzió alapján)}
\]

Írjuk át az eredeti kifejezést Fibonacci-formában: 
\begin{align*}
ax^{17}+bx^{16}+1 & =a(F_{17}x+F_{16})+b(F_{16}x+F_{15})+1\\
 & =(aF_{17}+bF_{16})x+(aF_{16}+bF_{15}+1)
\end{align*}

Mivel $x^{2}-x-1=0$ irracionális gyökei miatt $x\notin\mathbb{Q}$,
a fenti kifejezés akkor nulla, ha mindkét együttható külön-külön nulla:
\[
\begin{cases}
aF_{17}+bF_{16}=0\\
aF_{16}+bF_{15}+1=0
\end{cases}
\]

Oldjuk meg a rendszert. Használjuk a Fibonacci-számokat: 
\begin{align*}
F_{15} & =610\\
F_{16} & =987\\
F_{17} & =1597
\end{align*}

Első egyenletből: 
\[
a=-\frac{bF_{16}}{F_{17}}=-\frac{987b}{1597}
\]
Behelyettesítve a második egyenletbe: 
\[
-\frac{987^{2}b}{1597}+610b+1=0\Rightarrow\text{megoldva: }a=987
\]

\textbf{Válasz: } $\boxed{987}$
\end{solution}


