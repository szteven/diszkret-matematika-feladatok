
\chapter{Szimmetrikus kulcsú kriptográfia}\label{chap:szimmetrikus-kripto}
\begin{description}
{\large \item [{Szerző:}] Sógor Bence (Didaktikai mesteri -- Matematika, I. év)}
\end{description}
A kriptográfia szó a görög Kryptós=„rejtett” és gráphein=„írni” szavakból
ered. Üzenetek titkosítását jelenti és már az ókori civilizációk idején
is jelen volt. Elég korán megmutatkozott az igény arra, hogy két fél
bizalmasan tudjon egymás között információt cserélni anélkül, hogy
egy harmadik külső fél az üzenet elfogásával könnyen rájöjjön, hogy
mit tartalmaz az. Üzenetként gondolhatunk például háborús stratégiák
megbeszélésére két tábornok között, ami elég kellemetlen lenne, ha
az ellenfél kezébe kerülne. Önmagában egy üzenet kódolása még nem
jelent teljes biztonságot, mert ha az túl könnyen feltörhető, akkor
a kódolás semmit sem ért. Emiatt a kódolási, dekódolási stratégiák
és az üzenetek feltörésére szolgáló módszerek párhuzamosan fejlődtek.

A titkosítási módszereket két nagyobb kategóriába szokás sorolni.
Beszélünk szimmetrikus kulcsú és aszimmetrikus kulcsú kriptográfiáról.
A szimmetrikus kulcsú kriptográfia esetén a kódoló algoritmus és a
dekódoló algoritmus is ugyanazt a kulcsot használja. Ezek az algoritmusok
minden lehetséges kulcs esetén egy bijektív függvényt határoznak meg
a kódolandó üzenetek halmaza és a kódolt üzenetek halmaza között.
Például, ha a rejtjelezés függvényét $E$-vel (encryption) jelöljük,
akkor 
\[
E:K\times M\rightarrow C
\]
függvény, ahol bármely $k\in K$ esetén 
\[
E_{k}:M\rightarrow C,m\mapsto E(k,m)
\]
egy invertálható függvény. K a kulcsok halmazát, M üzenetek halmazát,
C titkosított üzenetek halmazát jelöli.

Hasonlóan a dekódolás függvényét $D$-vel (decryption) jelölve, 
\[
D:K\times C\rightarrow M
\]
függvény, ahol bármely $k\in K$ esetén 
\[
D_{k}:C\rightarrow M,c\mapsto D(k,c)
\]
egy invertálható függvény.

Az alábbi ábra grafikusan mutatja a módszer lényegét. A kódoló és
dekódoló algoritmusok nem feltétlen egyeznek meg, de ugyan azt a titkos
kulcsot használják. A két kommunikáló fél erről a titkos kulcsról
korábban meg kell egyezzen. Ezt vagy egy személyes találkozón vagy
egy biztonságos csatornán keresztül teszik meg. A biztonságos csatorna
lehet egy közös ismerős, akiben mind a ketten megbíznak és eljuttatja
az közös kulcsot egyik féltől a másikig vagy történhet online is például
aszimmetrikus kulcsú kriptográfiai módszereket használva.

\begin{figure}[h]
\includegraphics[width=1\textwidth]{\string"content/Sogor_Bence/kripto\string".jpg}
\caption{Szimmetrikus kulcsú kriptográfia}
\end{figure}

A legegyszerűbb példák és történelmileg a legkorábbi módszerek szimmetrikus
kulcsú kriptográfiára a helyettesítő kódok. A szakirodalom megkülönböztet
monoalfabetikus, polialfabetikus és poligrafikus módszereket, amikre
a következőkben egy-egy példa is.

Monoalfabetikus rejtjelezéskor egy kód-ábécét használunk. Legegyszerűbb
példa a Ceasar-kód, amikor az üzenet minden betűjét kicseréljük a
tőle jobbra x. pozícióra levő betűre. Például ha a magyar ábécét használjuk
kettős és hármas betűk nélkül, 9-es eltolással, akkor a \emph{Ceasar}
szöveg a \emph{Jlhyhx} szövegnek felelne meg: 
\[
\text{Ceasar}\rightarrow\text{Jlhyhx}.
\]
Matematikailag erre a módszerre tekinthetünk úgy, mint moduláris összeadással
való titkosításra. Az ábécé méretével megegyező maradékosztályon a
kulcsot hozzáadjuk a betűnek megfelelő pozíciónak. Ez a módszer gyakoriságelemzéssel
viszonylag könnyen feltörhető. A magyar nyelvű szavakban az $e$ betű
fordul elő leggyakrabban. Ha megfelelően hosszú a titkosítandó szöveg,
akkor a titkosított szövegben az $e$-nek megfelelő betű fog a leggyakrabban
megjelenni.

Ezt a gyengeséget próbálják a polialfabetikus helyettesítő módszerek,
például a Vigenère-kód több kód-ábécé használatával kiküszöbölni.
Például, ha a kulcs az „Orsi” szó és a \texttt{KOLOZSVÁRON NAGYON
HIDEG VAN} szöveget szeretnénk titkosítani, akkor Ceasar-kódoláshoz
hasonlóan járunk el csak egy ábécé helyett négyet használunk. 
{\setlength{\tabcolsep}{1.5pt}\bfseries
	\renewcommand*{\arraystretch}{1.4}
\begin{center}
	\begin{tabular}{ccccccccccccccccccccccccccccccccccc}
		A & Á & B & C & D & E & É & F & G & H & I & Í & J & K & L & M & N & O & Ó & Ö & Ő & P & Q & R & S & T & U & Ú & Ü & Ű & V & W & X & Y & Z\tabularnewline
		\hline
		O & Ó & Ö & Ő & P & Q & R & S & T & U & Ú & Ü & Ű & V & W & X & Y & Z & A & Á & B & C & D & E & É & F & G & H & I & Í & J & K & L & M & N\tabularnewline
		R & S & T & U & Ú & Ü & Ű & V & W & X & Y & Z & A & Á & B & C & D & E & É & F & G & H & I & Í & J & K & L & M & N & O & Ó & Ö & Ő & P & Q\tabularnewline
		S & T & U & Ú & Ü & Ű & V & W & X & Y & Z & A & Á & B & C & D & E & É & F & G & H & I & Í & J & K & L & M & N & O & Ó & Ö & Ő & P & Q & R\tabularnewline
		I & Í & J & K & L & M & N & O & Ó & Ö & Ő & P & Q & R & S & T & U & Ú & Ü & Ű & V & W & X & Y & Z & A & Á & B & C & D & E & É & F & G & H
	\end{tabular}
\end{center}
}

A kódolás: 

{\setlength{\tabcolsep}{1.5pt}\bfseries
	\renewcommand*{\arraystretch}{1.4}
	\begin{center}
		\begin{tabular}{cccccccccccccccccccccccccccc}
			K & O & L & O & Z & S & V & Á & R & O & N & \ & N & A & G & Y & O & N & \ & H & I & D & E & G & \  &  V & A & N\tabularnewline
			O & R & S & I & O & R & S & I & O & R & S & \ & I & O & R & S & I & O & \ & R & S & I & O & R & \  &  S & I & O\tabularnewline
			\hline
			V & E & C & Ú & N & J & Ö & Í & E & E & E & \  & U & O & W & Q & Ú & Y & \  & X & Z & L & Q & W & \  &  Ö & I & Y
		\end{tabular}
	\end{center}
}
Mindegyik kódolandó betű alatt a kulcs megfelelő része jelzi, hogy
pont melyik ábécét kell használni. Megjegyezendő, hogy ha a kulcs
teljesen random és hosszúsága megegyezik az üzenet hosszúságával,
akkor tökéletes lenne a kódolás. Gondoljunk arra, hogy mi van akkor
ha egy egység hosszúságú az üzenet és a kulcs is. Ha az $I$ jelentené
az igent és az $N$ a nemet, akkor a kulcs ismerete nélkül esélyünk
sem lenne megkülönböztetni, hogy az elfogott titkosított üzenetben
a $C$ betű mit jelent, mert egyenlő valószínűséggel lehetne bármelyik.
Mivel nem praktikus, hogy az üzenet és a kulcs ugyanolyan hosszúak
legyenek, ezért napjainkban a módszerek arra törekednek, hogy a lehető
legrövidebb kulcs használatával hosszú üzenetek feltörése is, szuperszámítógépekkel
legalább napokba, hetekbe teljen.

Poligrafikus helyettesítésre jó példa a Playfair-kód. Alapjáraton
az angol ábécét használja a Q betű nélkül. Mivel pontosan 25 betű
$5\times5$-ös elrendezésben kell a kulcshoz, ezért nehéz átalakítani
a módszert a magyar ábécére és már kérdéses lenne, hogy az új módszer
nevezhető-e egyáltalán Playfair-kódolásnak. A módszer a következő
szabályokat használva betűpárokat helyettesít betűpárokra: 
\begin{itemize}
\item Ha egy pár betűi megegyeznek, vagy csak egy betű maradt, akkor a második
betűt cseréljük le egy X-re és ezt kódoljuk. 
\item Ha egy pár betűi egy sorban vannak, akkor a betűket kicseréljük a
jobb oldali szomszédjukkal (jobb szélső elemet a sor első elemével). 
\item Ugyan abban az oszlopban levő betűpárokat az alattuk levővel cseréljük. 
\item Különben vegyük azt a téglalapot, amiben a betűpár elemei szemközti
csúcsokban vannak. A betűket cseréljük a saját soruk csúcsában levő
betűkre. 
\end{itemize}
Talán a legegyszerűbb ezt egy példán keresztül bemutatni:\\
 
\begin{center}
Használjuk az alábbi kulcstáblát:\\
 %
\begin{tabular}{ccccc}
\hline 
D  & I  & S  & Z  & K\tabularnewline
R  & E  & T  & M  & A\tabularnewline
B  & C  & F  & G  & H\tabularnewline
J  & L  & N  & O  & P\tabularnewline
U  & V  & W  & X  & Y\tabularnewline
\end{tabular}\\
 Kódoljuk a „Párosan szép az élet” szöveget.\\
 Először bontsuk fel kettes csoportokra az üzenetet:\\
 Eredeti szöveg: \texttt{PA RO SA NS ZE PA ZE LE T}\\
 Kód: \hspace{1.63cm} \texttt{YH MJ KT WT IM YH IM VC MW}\\
 
\par\end{center}

Sajnos a fenti módszereket ebben a formában elektromos kommunikáció
esetén nem lehet hatékonyan használni. Az 1970-es évek elején a United
States National Bureau of Standards (NBS) pályázatot hirdetett egy
kriptográfiailag biztonságos algoritmusra. A fő érv az volt, hogy
pénzügyi intézményeknek biztonságos kommunikációs csatornákat kellett
biztosítani. Ennek az eredménye lett a DES (Data Encryption Standard)
a maga 56 bites kulcsával.

A DES függvények egy 56-bites kulcsot felhasználva egy 64-bites szöveget
átalakít egy 64-bites kriptogrammá. 
\[
DES:\{0,1\}^{56}\times\{0,1\}^{64}\rightarrow\{0,1\}^{64}
\]

A 15 év védelemre tervezett algoritmus be is váltotta a hozzá fűzött
reményeket. 1977-től 22 éven keresztül számított standardnak egészen
az 1999-es feltöréséig. Az Electronic Frontier Foundation valóban
feltörte a DES-t 22 óra 15 perc alatt elosztott számítással.

A 15 éve lejártához közeledve a DES sebezhetőségétől tartva sokan
elkezdtek „triple DES”-t használni. Egyszeri kódolás helyett háromszor
kódolva az üzenetet. Az 1990-es évek közepére egyértelművé vált, hogy
szükséges egy új standard bevezetése. Az AES-t (Advance Encryption
Standard) tervezték a DES utódjának. Az AES-128 függvények egy 128-bites
kulcsot felhasználva egy 128-bites szöveget átalakít egy 128-bites
kriptogrammá. Egyéb változatai az AES-192 és AES-256 192 illetve 256-bites
kulcsokat használnak. 
\[
AES-128:\{0,1\}^{128}\times\{0,1\}^{128}\rightarrow\{0,1\}^{128}
\]
Ez a példa is mutatja, hogy mennyire dinamikusan fejlődő tudomány
a kriptográfia és mennyire fontos lenne az iskolában legalább említés
szintjén bemutatni. 

\section*{Házi feladatok}
\begin{problem}[Caeser-kód]
Milyen értelmes szöveget rejthet a „\texttt{LOF NLTMQQLR TLS RLÓLZ
QMZ RUDHZ SLOBRTÖ}” szöveg, ha Ceaser-rejtjelezést használtunk? Írdd
le, hogy hogyan törted fel a kódot! 
\end{problem}

\begin{solution}
Vizsgáljuk a betűk szövegben való előfordulásának gyakoriságát. A
leggyakoribb az „l” ami 7-szer fordul elő, utána jóval elmaradva az
„r” betű szerepel 4 előfordulással. Mivel tudjuk, hogy a magyar nyelvben
a leggyakrabban használt betű az „e”, ezért először nézzük meg azt
a Ceaser-rejtjelezést, amelyik az „e” betűt az „l” betűbe viszi. Ez
9 egység eltolásnak felel meg a magyar abc-ben. Ha visszafejtjük a
szöveget, akkor szerencsére értelmes:

„Egy fenékkel nem lehet két lovat megülni!” 
\end{solution}
\begin{problem}[Vigenère-kód]
Kódold a „Jobb adni, mint kapni” üzenetet Vigenère-rejtjelzéssel
a „házi” kulcsszót felhasználva. 
\end{problem}

\begin{solution}
Ha kézzel kódolunk Vigenère-rejtjelezéssel a legegyszerűbb, ha egymás
alá leírjuk a megfelelő kód-ábécéket és a kódolandó szöveget.

\[
\textbf{JOBB ADNI MINT KAPNI}
\]

{\setlength{\tabcolsep}{1.5pt}\bfseries
	\renewcommand*{\arraystretch}{1.4}
	\begin{center}
		\begin{tabular}{ccccccccccccccccccccccccccccccccccc}
			A & Á & B & C & D & E & É & F & G & H & I & Í & J & K & L & M & N & O & Ó & Ö & Ő & P & Q & R & S & T & U & Ú & Ü & Ű & V & W & X & Y & Z\tabularnewline
			\hline
			H & I & Í & J & K & L & M & N & O & Ó & Ö & Ő & P & Q & R & S & T & U & Ú & Ü & Ű & V & W & X & Y & Z & A & Á & B & C & D & E & É & F & G\tabularnewline
			Á & B & C & D & E & É & F & G & H & I & Í & J & K & L & M & N & O & Ó & Ö & Ő & P & Q & R & S & T & U & Ú & Ü & Ű & V & W & X & Y & Z & A\tabularnewline
			Z & A & Á & B & C & D & E & É & F & G & H & I & Í & J & K & L & M & N & O & Ó & Ö & Ő & P & Q & R & S & T & U & Ú & Ü & Ű & V & W & X & Y\tabularnewline
			I & Í & J & K & L & M & N & O & Ó & Ö & Ő & P & Q & R & S & T & U & Ú & Ü & Ű & V & W & X & Y & Z & A & Á & B & C & D & E & É & F & G & H
		\end{tabular}
	\end{center}
}

Ha kódoljuk a szöveget azt kapjuk, hogy: 
\[
\textbf{PÓÁJ HEMŐ SÍMA QÁŐUÖ}
\]
\end{solution}
\begin{problem}
A Playfair-kódolás mintájára az alábbi kulcstáblát használva fejtsd
vissza a „\texttt{HASLTAZJOCXMSDLTSZMCVAZY}” üzenetet! 

\begin{center}
	\begin{tabular}{ccccc}
		D  & I  & S  & Z  & K\tabularnewline
		R  & E  & T  & M  & A\tabularnewline
		B  & C  & F  & G  & H\tabularnewline
		J  & L  & N  & O  & P\tabularnewline
		U  & V  & W  & X  & Y\tabularnewline
	\end{tabular}
\end{center}

\end{problem}

\begin{solution}
A karaktereket párba állítva a következőképpen fejthető vissza az
üzenet: 
\[
\texttt{HA}\to\texttt{AK};\texttt{SL}\to\texttt{IN};\texttt{TA}\to\texttt{EM};\texttt{ZJ}\to\texttt{DO};\texttt{OC}\to\texttt{LG};\texttt{XM}\to\texttt{OZ};\texttt{SD}\to\texttt{IK}
\]
\[
\texttt{LT}\to\texttt{NE};\texttt{SZ}\to\texttt{IS};\texttt{MC}\to\texttt{EG};\texttt{VA}\to\texttt{YE};\texttt{ZY}\to\texttt{KX}
\]
Mivel ez a kódolás csak az angol ábécével működik, ezért a dekódolt
üzenetet kiegészítve ékezetekkel és az X-et elhagyva a vlglről azt
kapjuk, hogy: 
\[
\texttt{Aki nem dolgozik ne is egyék!}
\]
\end{solution}
\begin{problem}
Egy 10 billentyűs(10 számjegy) varázslatos írógép képes fényévekre
elküldeni egy számot, de ez nagyon sok energiát vesz igénybe, úgyhogy
egy használat után évezredeket kell várni az újratöltődésére. Az eredeti
feladat, hogy egy üzenetben szeretnénk egy természetes számot elküldeni
a fényévekre levő barátunknak (ezt könnyen megtehetjük, csak leírjuk
a számot és elküldjük). Megoldható-e, hogy egy üzenetben több természetes
számot küldjünk el (a barátunk a megkapott számjegyek sorozatából
meg kell tudja állapítani, hogy melyik természetes számokat küldtük
neki)? 
\end{problem}

\begin{solution}
Egy lehetséges megoldás a számjegyduplázás. Minden számjegyet, ami
az elküldendő üzenet egy része kétszer írunk le egymás után. Két különböző
számjegy (mondjuk 37) leírásával el tudjuk választani a különböző
számokat. Például: 
\[
\texttt{1122379933372277}
\]
üzenet láttán a számjegyeket kettes csoportokba osztva tudjuk, hogy
az üzenet a 12, 93, 27 számok. 
\end{solution}
\begin{problem}
Az előző feladat nehezítése. Adjál módszert az előző feladatra a lehető
legkevesebb billentyű használatával! 
\end{problem}

\begin{solution}
Egy billentyű felhasználásával eltudunk küldeni egy tetszőlegesen
hosszú természetes számot. A természetes számok prímtényezős felbontásának
egyértelműsége miatt több számot is el tudunk küldeni egy billentyű
használatával. Legyenek az elküldendő számok $n_{1},n_{2},\dots,n_{k}$.
Ekkor a 
\[
p_{1}^{n_{1}}\cdot p_{2}^{n_{2}}\dots p_{k}^{n_{k}}
\]
szám elküldésével a fogadó fél egyértelműen vissza tudja fejteni az
üzenetet, ahol $p_{1},\dots,p_{k}$ az első $k$ darab prímszám. Csak
a kapott természetes számot fel kell bontsa prímtényezőire. 
\end{solution}

\section*{Nehezebb feladatok}
\begin{extraproblem}[Gál Tamara]
A következő, a teljes (diakritikus) magyar ábécé szerint (melyben
nem hagyunk ki egyetlen betűt sem): A, Á, B, C, D, E, É, F, G, H,
I, Í, J, K, L, M, N, O, Ó, Ö, Ő, P, Q, R, S, T, U, Ú, Ü, Ű, V, W,
X, Y, Z --- összesen 34 karakter --- eltolásos (Caesar-) rejtjelezéssel
titkosított üzenetet kaptad: 
\[
\texttt{RESPI XOÉJRNGVSQ NYWÉSRNFX}
\]
\begin{enumerate}
\item Dekódold a szöveget: ismeretlen eltolással toltak jobbra (balra kell
visszatolni, $k=1,2,\dots,33$). 
\item A kapott három szóban kiírt számot írd le számjeggyel is! 
\item Számold ki a három szám összegét! 
\end{enumerate}
\end{extraproblem}

\begin{solution}
1. Az ábécé és a dekódolás menete

A dekódolás képlete: 
\[
P_{i}=\bigl(C_{i}-k\bigr)\bmod34,\quad k\in\{1,2,\dots,33\}.
\]
Bruteforce-szal próbáljuk végig az összes $k$-t, és a helyes $k$
mellett értelmes magyar szavakat kapunk.

A vizsgálat során $k=7$-tel (balra) tolva kapjuk: 
\[
\texttt{RESPI XOÉJRNGVSQ NYWÉSRNFX}\;\xrightarrow{k=7}\;\texttt{NYOLC TIZENHÁROM HUSZONHAT}.
\]

3. A számok és összegük

A dekódolt szavak: 
\[
\text{NYOLC}\to8,\quad\text{TIZENHÁROM}\to13,\quad\text{HUSZONHAT}\to26.
\]
Az összeg: 
\[
8+13+26=47.
\]

\bigskip{}
\textbf{Válasz:} 1.) A helyes eltolás: $k=7$. 

2.) Dekódolt üzenet: \text{NYOLC TIZENHÁROM HUSZONHAT}. 3.) A számok
összege: ${\displaystyle 47}$. 
\end{solution}
\begin{extraproblem}[Gál Tamara]
Egy régi papirosra a következő sor van Vigenère--kódolással titkosítva,
a teljes, diakritikus magyar ábécé használatával (35 betű):
\begin{itemize}
\item \textbf{Titkos üzenet:} 
\[
\text{OŐOTZMÓHWŰQN}
\]
\item \textbf{Kulcsszó:} \texttt{TITKOS} 
\item A szóközöket és írásjeleket nem kódolták, csak a betűket. 
\item A dekódolás képlete: 
\[
P_{i}=(C_{i}-K_{i})\bmod35,
\]
ahol az ábécé hossza 35. 
\end{itemize}
\end{extraproblem}

\begin{solution}
A titkos szöveg hossza 12 betű. A kulcsszó: \texttt{TITKOS}. Kulcsindexek:
T(25), I(10), T(25), K(13), O(17), S(25). Ismételve 12 hosszúságban:
\[
\underbrace{25,10,25,13,17,25,25,10,25,13,17,25}_{\text{12 elem}}
\]

\textbf{Betűnkénti dekódolás}\\

Az egyes betűkhöz tartozó számokból a kulcsindexet kivonva és mod
35-tel:
\begin{center}
\begin{tabular}{ccccccc}
$i$ & 1 & 2 & 3 & $\cdots$ & 12 & \tabularnewline
$C_{i}$ & O & Ő & O & $\cdots$ & N & \tabularnewline
${\rm idx}(C_{i})$ & 17 & 20 & 17 &  & 16 & \tabularnewline
$K_{i}$ & T & I & T &  & S & \tabularnewline
${\rm idx}(K_{i})$ & 25 & 10 & 25 &  & 25 & \tabularnewline
$P_{i}=(C_{i}-K_{i})\bmod35$ & 27 & 10 & 27 &  & 10 & \tabularnewline
$P_{i}$ & U & I & U &  & I & \tabularnewline
\end{tabular}
\par\end{center}
A teljes dekódolt szöveg: 
\[
\texttt{TITKOSUZENET}
\]

\textbf{Megfejtés:}\\

A dekódolt magyar kifejezés (szóközökkel elválasztva): 
\[
\textbf{TITKOS ÜZENET}
\]
\end{solution}
\begin{extraproblem}[Gergely Verona]
A következő idézet Steve Backley-től Caesar-kóddal lett titkosítva:
KPXGUV AQWT GPGTIA KP VJG VJKPIU AQW ECP EQPVTQN. Mit mondott? \emph{(Science
Olympiad CodeBusters --- 2023-2024 North Carolina Division A) }
\end{extraproblem}

\begin{solution}
A titkosított szövegben nincsenek egybetűs szavak (mint az "A" vagy
"I" az angolban), viszont van egy kétbetűs szó: "KP". Ez gyakran
segít, mert az angolban sok gyakori kétbetűs szó van. A Caesar-kód
gyors visszafejtéséhez használhatunk egy egyszerű trükköt, amely mindössze
8 karakter vizsgálatát igényli: hat betűt a szavak elejéről (A, B,
I, M, O, U) 
\begin{itemize}
\item A → As, At, An, Am 
\item B → Be, By 
\item I → In, It, Is, If 
\item M → Me, My 
\item O → Of, Or, On 
\item U → Up, Us 
\end{itemize}
és kettőt a végükről (O, E) 
\begin{itemize}
\item O → dO, gO, nO, sO, tO 
\item E → hE, wE. 
\end{itemize}
Ezek segítségével könnyen megállapítható, hogy mennyivel lett eltolva
a titkosított szöveg. Vizsgáljuk a kezdőbetűk szerint a KP szót: 
\begin{itemize}
\item Ha A=K, akkor ez egy 10-es eltolás visszafelé, vagyis P=F tehát az
eredeti szó AF. 
\item Ha B=K, akkor ez egy 9-es eltolás visszafelé, vagyis P=G tehát az
eredeti szó BG. 
\item Ha I=K, akkor ez egy 2-es eltolás visszafelé, vagyis P=N tehát az
eredeti szó IN ami egy értelmes szó. 
\item Ha M=K, akkor ez egy 2-es eltolás előre, vagyis P=R tehát az eredeti
szó MR. 
\item Ha O=K, akkor ez egy 4-es eltolás előre, vagyis P=T tehát az eredeti
szó OT. 
\item Ha U=K, akkor ez egy 10-es eltolás előre, vagyis P=Z tehát az eredeti
szó UZ. 
\end{itemize}
Vizsgáljuk a végzőbetűk szerint a KP szót: 
\begin{itemize}
\item Ha P=O, akkor ez egy 1-es eltolás visszafelé, vagyis J=K tehát az
eredeti szó JO. 
\item Ha P=E, akkor ez egy 11-es eltolás visszafelé, vagyis E=K tehát az
eredeti szó ZE. 
\end{itemize}
Mivel a "KP" = "IN", akkor ez egy 2-es eltolás visszafelé, így
az eredeti szöveg: INVEST YOUR ENERGY IN THE THINGS YOU CAN CONTROL.
\end{solution}
\begin{extraproblem}[Kis Brigitta]
Egy üzenetet Vigenère-kóddal titkosítottak ismeretlen kulccsal. A
titkosított szöveg a következő:

\[
\texttt{LXFOPVEFRNHR}
\]

Tudjuk, hogy az eredeti üzenet értelmes angol szöveg, amely várhatóan
az ATTACK szóval kezdődik. Mi volt az eredeti kulcs, és mi volt a
teljes eredeti üzenet?
\end{extraproblem}

\begin{solution}
A Vigenère-kód szerint minden betűhöz a kulcs megfelelő betűje adódik
hozzá modulo 26 szerint. Visszafejtéshez:

\[
\text{Kulcsbetű}=(\text{Titkosított betű}-\text{Eredeti betű})\bmod26
\]

Az első hat betű alapján:
\begin{center}
\begin{tabular}{|c|c|c|}
\hline 
Eredeti betű & Titkosított betű & Kulcsbetű\tabularnewline
\hline 
A (0) & L (11) & L (11)\tabularnewline
T (19) & X (23) & E (4)\tabularnewline
T (19) & F (5) & M (12)\tabularnewline
A (0) & O (14) & O (14)\tabularnewline
C (2) & P (15) & N (13)\tabularnewline
K (10) & V (21) & L (11)\tabularnewline
\hline 
\end{tabular}
\par\end{center}
Így a kulcs: LEMON.

A kulcs ismétlődik az üzenet teljes hosszán, így a visszafejtés:\\
ATTACKATDAWN 
\end{solution}
\begin{extraproblem}[Kis Brigitta]
Egy üzenetet Playfair-kóddal titkosítottak a következő kulcsmátrix
alapján:

\[
\begin{array}{ccccc}
P & L & A & Y & F\\
I & R & E & X & M\\
B & C & D & G & H\\
K & N & O & Q & S\\
T & U & V & W & Z
\end{array}
\]

Az alábbi titkosított üzenetet kaptuk:

\[
\texttt{BM\ OD\ ZB\ XD\ NA\ BE\ KU\ DM\ UI\ XM\ MO\ UV\ IF}
\]

Mi volt az eredeti szöveg? 
\end{extraproblem}

\begin{solution}
Például az első páros:

\[
\texttt{BM}\rightarrow\texttt{HI}
\]
Továbbá: 
\[
\texttt{OD}\rightarrow\texttt{DE}
\]
\[
\texttt{ZB}\rightarrow\texttt{TH}
\]
\[
\texttt{XD}\rightarrow\texttt{ET}
\]
\[
\texttt{NA}\rightarrow\texttt{RE}
\]
\[
\texttt{BE}\rightarrow\texttt{AS}
\]
\[
\texttt{KU}\rightarrow\texttt{UR}
\]
\[
\texttt{DM}\rightarrow\texttt{EI}
\]
\[
\texttt{UI}\rightarrow\texttt{NT}
\]
\[
\texttt{XM}\rightarrow\texttt{HE}
\]
\[
\texttt{MO}\rightarrow\texttt{DE}
\]
\[
\texttt{UV}\rightarrow\texttt{SE}
\]
\[
\texttt{IF}\rightarrow\texttt{RT}
\]
A teljes visszafejtés:

\[
\texttt{HIDE\ THE\ TREASURE\ IN\ THE\ DESERT}
\]
\end{solution}
\begin{extraproblem}[Kiss Andrea-Tímea]
Egy kávézó Wifi-hálózatához csak akkor csatlakozhatsz, ha megfejted
a jelszót. A Wifi-jelszót egy szó alkotja (csupa nagybetű), de titkosított
formában kaptad meg egy számsorozatként, amely az $A=1,B=2,...,Z=26$
betűkódolás szerint készült, de a számokat ezután egy különös művelettel
is átalakították.

A titkosító algoritmus így működik: 
\begin{enumerate}
\item A betű sorszámát vették $(A=1,B=2,...,Z=26)$. 
\item Hozzáadták a betű pozícióját a szóban (azaz 1-től kezdve sorszámozták
a betűket). 
\item Az így kapott számot $3$-mal megszorozták. 
\item Végül a $20$-szal való osztás maradékjának feleltették meg a betűket
az angol abc-ben levő sorszámuk szerint. 
\end{enumerate}
Ha a kódolt jelszó KIFIDCOP, mi lehet az eredeti jelszó? 
\end{extraproblem}

\begin{solution}
\begin{table}[ht]
\centering %
\begin{tabular}{c|c|c|c|c|c|c|c|c|c|c|c|c}
1 & 2 & 3 & 4 & 5 & 6 & 7 & 8 & 9 & 10 & 11 & 12 & 13\tabularnewline
\hline 
A & B & C & D & E & F & G & H & I & J & K & L & M\tabularnewline
\multicolumn{1}{c}{} & \multicolumn{1}{c}{} & \multicolumn{1}{c}{} & \multicolumn{1}{c}{} & \multicolumn{1}{c}{} & \multicolumn{1}{c}{} & \multicolumn{1}{c}{} & \multicolumn{1}{c}{} & \multicolumn{1}{c}{} & \multicolumn{1}{c}{} & \multicolumn{1}{c}{} & \multicolumn{1}{c}{} & \tabularnewline
14 & 15 & 16 & 17 & 18 & 19 & 20 & 21 & 22 & 23 & 24 & 25 & 26\tabularnewline
\hline 
N & O & P & Q & R & S & T & U & V & W & X & Y & Z\tabularnewline
\end{tabular}\caption{A betűk sorszáma az abc-ben}
\end{table}

Mivel 8 betűből áll a jelszó, ezért az 20-szal való osztás előtt a
legnagyobb lehetséges érték a $(8+26)\cdot3=102$.

A K sorszáma 11, ami nem osztható 3-mal, így a 11-hez olyan többszörösét
kell hozzáadni a 20-nak, hogy az így keletkezett összeg osztható legyen
3-mal és legfennebb 102 lehet. 
\[
3\nmid11,\phantom{xx}3\nmid31=11+20,\phantom{xx}3\phantom{l}|\phantom{l}51=11+2\cdot20,\phantom{xx}3\nmid71=11+3\cdot20,\phantom{xx}3\nmid91=11+4\cdot20,
\]
így az eredeti betű sorszámát megkapjuk, mint 
\[
(x+1)\cdot3=51\Leftrightarrow x+1=17\Leftrightarrow x=16,
\]
ami alapján a jelszó első betűje az abc 16. betűje, vagyis a P. Teljesen
hasonlóan dolgozunk tovább.

Az I sorszáma 9, ami osztható 3-mal, viszont a $9+3\cdot20=69$ is
osztható 3-mal. Ezért kódolt jelszóban megjelenő első I betűt rejtheti
az abc következő sorszámú betűjeit: 
\[
(x+2)\cdot3=9\Leftrightarrow x=1\text{ vagy }(x+2)\cdot3=69\Leftrightarrow x=21.
\]
Azaz az első I betű kódolhatja az A vagy U betűket is akár. A második
I betű a jelszóban rejtheti az abc következő sorszámú betűjeit: 
\[
(x+4)\cdot3=9\Leftrightarrow x=-1\text{(ellentmondás)}\text{ vagy }(x+4)\cdot3=69\Leftrightarrow x=19.
\]
Ezek alapján a második I betű az abc 19. karakterét rejti, vagyis
az S-et.

Az F sorszáma 6, ami osztható 3-mal, viszont a $6+3\cdot20=66$ is
osztható 3-mal. 
\[
(x+3)\cdot3=6\Leftrightarrow x=-1\text{(ellentmondás)}\text{ vagy }(x+3)\cdot3=66\Leftrightarrow x=19,
\]
így az F betű szintén az S-et rejti.

A D sorszáma 4, ami nem osztható 3-mal, viszont a $4+20=24$ és $4+4\cdot20=84$
osztható 3-mal, ezért a D betű a következő sorszámú karaktereket kódolhatja:
\[
(x+5)\cdot3=24\Leftrightarrow x=3\text{ vagy }(x+5)\cdot3=84\Leftrightarrow x=23,
\]
vagyis a D betű kódolhatja a C vagy akár a W karaktereket is.

A C sorszáma 3, ami osztható 3-mal, viszont a $3+3\cdot20=63$ is,
ami miatt a C kódolhatja a 
\[
(x+6)\cdot3=3\Leftrightarrow x=-5\text{(ellentmondás)}\text{ vagy }(x+6)\cdot3=63\Leftrightarrow x=15
\]
sorszámú elemet, ami az O betű.

Az O sorszáma 15, ami osztható 3-mal, viszont a $15+3\cdot20=75$
is, ami miatt a O kódolhatja a 
\[
(x+7)\cdot3=15\Leftrightarrow x=-2\text{(ellentmondás)}\text{ vagy }(x+7)\cdot3=75\Leftrightarrow x=18
\]
sorszámú elemet, ami az R betű.

Végül, a P sorszáma 16, ami nem osztható 3-mal, viszont a $16+20=36$
és $16+4\cdot20=96$ igen. Ezek alapján a P kódolhatja a 
\[
(x+8)\cdot3=36\Leftrightarrow x=4\text{ vagy }(x+8)\cdot3=96\Leftrightarrow x=24
\]
sorszámmal rendelkező karaktereket, amelyek a D és az X.

Így tehát a keresett szó a következőképpen nézhet ki: 
\[
\begin{array}{c|c|c|c|c|c|c|c}
K & I & F & I & D & C & O & P\\
\hline P & A/U & S & S & C/W & O & R & D/X.
\end{array}
\]

A lehetséges jelszók így: PASSCORD, PASSCORX, PASSWORD, PASSWORX,
PUSSCORD, PUSSCORX, PUSSWORD, PUSSWORX, viszont minden bizonyára a
PASSWORD jelszó fog beválni. 
\end{solution}
\begin{extraproblem}[Kovács Levente]
Egy üzenet első blokkját (128 bit) szeretnénk titkosítani AES-128
algoritmussal, CBC módban. A következő adatok adottak:

Titkosító kulcs (hexadecimálisan): \\
 \texttt{2b7e1516 28aed2a6 abf71588 09cf4f3c}

Inicializációs vektor (IV): \\
 \texttt{00010203 04050607 08090a0b 0c0d0e0f}

Üzenet első blokkja (plaintext): \\
 \texttt{6bc1bee2 2e409f96 e93d7e11 7393172a}

\textbf{Kérdés:} Mi lesz az első titkosított blokk CBC módban, hexadecimálisan?
\end{extraproblem}

\begin{solution}
A CBC (Cipher Block Chaining) módban a titkosítás lépései:

Az üzenetblokk XOR-olása az IV-vel: 
\begin{align*}
\text{Plaintext} & =\texttt{6b c1 be e2 2e 40 9f 96 e9 3d 7e 11 73 93 17 2a}\\
\text{IV} & =\texttt{00 01 02 03 04 05 06 07 08 09 0a 0b 0c 0d 0e 0f}\\
\text{XOR eredmény} & =\texttt{6b c0 bc e1 2a 45 99 91 e1 34 74 1a 7f 9e 19 25}
\end{align*}

AES-128 titkosítás a XOR eredményen, a megadott kulccsal: 
\[
\text{Kulcs}=\texttt{2b7e1516 28aed2a6 abf71588 09cf4f3c}
\]

Titkosított blokk eredménye (hex): 
\[
\boxed{\texttt{7649abac 8119b246 ceca2aee e7cd7f1f}}
\]
\end{solution}
\begin{extraproblem}[Kovács Levente]
Egy nagyon leegyszerűsített szimmetrikus titkosító algoritmus 16
bites kulcsot használ. Egy támadó rendelkezik az alábbi adatokkal:

Ciphertext: \texttt{0x9A23}

Ismert nyílt szöveg: \texttt{0x3C7D}

Titkosítási algoritmus: 
\[
\text{cipher}=(\text{plaintext}\oplus\text{key})<<<3
\]
ahol $<<<3$ jelentése: 3 bites balra forgatás 16 biten.

\textbf{Kérdés:} Mi volt a titkosítási kulcs?
\end{extraproblem}

\begin{solution}
A titkosított szöveget visszaforgatjuk 3 bittel jobbra: 
\[
\text{cipher}=0x9A23=1001101000100011_{2}
\]
\[
\text{jobbra forgatva (ROR 3)}=0111001101000100_{2}=0x7351
\]

Oldjuk meg: 
\[
\text{plaintext}\oplus\text{key}=0x7351\Rightarrow\text{key}=0x3C7D\oplus0x7351=0x4F2C
\]

\textbf{Válasz:} 
\[
\boxed{\text{A kulcs: }0x4F2C}
\]
\end{solution}
\begin{extraproblem}[Seres Brigitta-Alexandra]
Egy királylány kegyeiért két királyfi is harcol, közülük egybe szerelmes
a királylány. Hogy a királyságok között ne törjön ki háború, úgy kell
visszautasítania az általa nem szeretett kérőt, hogy annak ez ne tűnjön
fel. Ezért a királylány két kódolt szöveget ad a legényeknek. Aki
előbb megfejti az üzeneteket, az nyeri el a kezét. Mindkettejük feladata
egy-egy Caesar-kód visszafejtése. A megfejtéshez a magyar ábécét használják
(a két- és háromjegyű betűk nélkül).

Akibe szerelmes a királylány, annak ezt a szöveget adta: \textbf{"Íöáácé
ofiafm hfnfugivn".} Akiben nem szerelmes a királylány, ezt a szöveget
adta: \textbf{"Vdppru búvőúa űúáúíüvmá".}

Hogy segítette elő a királylány a szívének kedves királyfi biztosabb
győzelmét? 
\end{extraproblem}

\begin{solution}
A magyar ábécé dupla betűk nélkül: A\,Á\,B\,C\,D\,E\,É\,F\,G\,H\,\\
 I\,Í\,J\,K\,L\,M\,N\,O\,Ó\,Ö\,Ő\,P\,Q\,R\,S\,T\,U\,Ú\,Ü\,Ű\,V\,W\,X\,Y\,Z.

Próbálgatás után rá lehet jönni, hogy mindkét megadott szöveg a \textbf{"HOZZÁD
MEGYEK FELESÉGÜL"} kódolt változata, azzal a különbséggel, hogy akibe
szerelmes a királylány, annak $2$-es eltolással rejtjelezte az üzenetet,
még a számára nem kedves kérőjének $21$-es eltolással (ami fordított
irányban való számolással is $-13$ egység eltolás, így azt is jóval
hosszabb folyamat dekódolni, mint a $2$ egységnyi eltolással valót).

Feltételezhető, hogy mindkét királyfi azonos képességekkel rendelkezik,
és így akinek $21$-el volt eltolva az üzenete nehezebben dekódolta
azt, mint társa. Ezzel a királylány látszólag ugyanolyan nehézségű
fejtörőt adott fel a legényeknek, hiszen a két szöveg hossza megegyezett,
és ugyanazt az elvet kellett alkalmazniuk a visszafejtéshez. Mégis
elő tudta segíteni a számára kedves fiú győzelmét azzal, hogy az ő
szövegében az eltolás jóval rövidebb hosszúságú volt.
\end{solution}
\begin{extraproblem}[Seres Brigitta-Alexandra]
A Playfair-kódolás mintájára az alábbi kulcstáblát használva fejtsd
vissza a következő üzenetet (az eredményben, ha szerepelnek, a hosszú
magánhangzók is röviden fognak megjelenni, de olvasáskor, értelemszerűen
eldönthető lesz melyik hosszú és melyik rövid): \textbf{"TKRSTIRSMBRKTUSKTIIRTJTIITOSESKEOXCOHIRSUHLTTNSOGSHIUR"}. 
\begin{center}
\begin{tabular}{|c|c|c|c|c|}
\hline 
B & I & M & Z & K\tabularnewline
\hline 
R & E & T & S & A\tabularnewline
\hline 
D & O & X & G & H\tabularnewline
\hline 
J & L & N & C & U\tabularnewline
\hline 
P & V & W & F & Y\tabularnewline
\hline 
\end{tabular}
\par\end{center}
\end{extraproblem}

\begin{solution}
Először a szöveget kettesével csoportosítjuk, majd a Playfair-kódolás
szabályainak ismeretében visszafejtjük a kódot (pont fordítva járva
el, mint a kódolás esetén, vagyis pl. ha lefele cseréltük ki kódoláskor
a betűket, dekódoláskor felfele megyünk, stb.). Visszafejtés után
kapjuk (az első két sor az eredeti szöveg, a harmadik-negyedik sor
pedig a visszafejtés, mindegyik szótag az 1-2 sor megfelelő pozícióján
levő elemnek felel meg):

\vspace{0.2cm}
\begin{center}
	
	\ttfamily
	\begin{tabular}{llllllllllllllllllllllll}
		TK  & RS   &TI    &RS    &MB    &RK    &TU    &SK    &TI  &IR &TJ  &TI    &IT    &OS    &ES    &  KE    \\
		OX &CO    &HI    &RS    &UH    &LT    &TN    & SO   &   GS & HI   &  UR    \\
		\\[6pt] % térköz a két sor között
		
		AM & AT & EM & AT & IK & ÁB & AN & AZ & EM & BE & RN & EM & ME & GÉ & RT & IA \\
		DO & LG & OK & AT & HA & NE & MX & EG & SZ & OK & JA    \\
	\end{tabular}
\end{center}
\vspace{0.2cm}
Összeolvasva a szöveget és az "X" betűt kicserélve "M"-re (mivel
szövegen belül volt, vagyis dupla betűt jelölt), a következőt kapjuk:\texttt{ }\texttt{\textbf{"A
matematikában az ember nem megérti a dolgokat, hanem megszokja."}}
Eme idézet Neumann Jánostól származik.
\end{solution}
\begin{extraproblem}[Száfta Antal]
Egy természetes szám minden számjegyét külön-külön transzformáljuk
a következő módon: 
\[
f(d)=(7\cdot d+3)\mod 10,\quad d\in\{0,1,\dots,9\}
\]
\begin{itemize}
\item[a)] Határozd meg az $f$ függvény inverzét, azaz $f^{-1}$-et, amely
visszafejti a kódolt számjegyeket. 
\item[b)] Dekódold az üzenetet: \quad{}\texttt{8 2 3 1 9 6} 
\item[c)] Új kódolási függvény modulo 13 szerint: 
\[
f(d)=(5d+7)\mod{13},\quad d\in\{0,1,\dots,12\}
\]
Bizonyítsd, hogy $f$ permutáció az $\mathbb{Z}_{13}$ halmazon, és
határozd meg az inverzét. 
\end{itemize}
\end{extraproblem}

\begin{solution}
~
\begin{itemize}
\item[a)] Oldjuk meg az egyenletet: 
\[
y=7d+3\mod 10\Rightarrow7d\equiv y-3\pmod{10}
\]
A 7 inverze modulo 10 alatt: $7^{-1}\equiv3$, mert $7\cdot3=21\equiv1\mod 10$.
\[
d\equiv3(y-3)\mod{10}\Rightarrow f^{-1}(y)=3(y-3)\mod{10}
\]
\item[b)] Alkalmazzuk az inverz függvényt minden számjegyre: 
\begin{itemize}
\item $f^{-1}(8)=3(8-3)=3\cdot5=15\mod 10=5$ 
\item $f^{-1}(2)=3(2-3)=-3\mod 10=7$ 
\item $f^{-1}(3)=3(3-3)=0$ 
\item $f^{-1}(1)=3(1-3)=-6\mod 10=4$ 
\item $f^{-1}(9)=3(9-3)=18\mod 10=8$ 
\item $f^{-1}(6)=3(6-3)=9$ 
\end{itemize}
Eredeti üzenet: \boxed{5\ 7\ 0\ 4\ 8\ 9} 
\item[c)] 
\begin{itemize}
\item Mivel $lnko(5,13)=1$, a függvény permutáció. 
\item Megoldandó: $5d+7\equiv y\mod{13}\Rightarrow5d\equiv y-7\mod{13}$ 
\item $5^{-1}\mod{13}=8$, mivel $5\cdot8=40\equiv1\mod{13}$ 
\[
d\equiv8(y-7)\mod{13}\Rightarrow f^{-1}(y)=8(y-7)\mod{13}
\]
\end{itemize}
\end{itemize}
\end{solution}
\begin{extraproblem}[Száfta Antal]
Az alábbi titkosított szöveg Caesar-kód segítségével készült, ahol
a betűket az angol ábécében 1-től 25-ig terjedő számértékkel eltolják
modulo 26 szerint. A kódolt üzenet:
\begin{center}
\texttt{WKH TXLFN EURZQ IRA MXPSV RYHU WKH ODCB GRJ} 
\par\end{center}
\textbf{Feladat:} Fejtsd vissza a szöveget!
\end{extraproblem}

\begin{solution}
Vegyük észre, hogy a "WKH" szó akkor válik "THE"-vé, ha minden
betűt hárommal visszafelé tolunk az ábécében:

\[
\texttt{W}\rightarrow\texttt{T},\ \texttt{K}\rightarrow\texttt{H},\ \texttt{H}\rightarrow\texttt{E}\Rightarrow k=3
\]

Alkalmazzuk az eltolást az egész szövegre:
\begin{center}
\texttt{THE QUICK BROWN FOX JUMPS OVER THE LAZY DOG} 
\par\end{center}
Ez egy ismert pangram (olyan mondat, amely az ábécé minden betűjét
tartalmazza legalább egyszer).

\textbf{Megoldás:} $\boxed{\texttt{THE QUICK BROWN FOX JUMPS OVER THE LAZY DOG}}$ 
\end{solution}
\begin{extraproblem}[Szélyes Klaudia]
 Adott egy varázsírógép, amely egy természetes számot tud elküldeni
egy üzenetben. Megoldható-e, hogy több természetes számot küldjünk
el egyetlen üzenetben úgy, hogy a címzett a kapott számjegyek sorozatából
egyértelműen vissza tudja fejteni az eredeti számokat? 
\end{extraproblem}

\begin{solution}
A kérdés lényege: létezik-e olyan mód, hogy több természetes számot
(pl. $n_{1},n_{2},\dots,n_{k}$) egyetlen természetes számmá $N$
kódolunk, úgy hogy az $N$ számjegysorozatából egyértelműen visszanyerhetők
az eredeti számok?


\textbf{Megoldás ötlete: Konkatenálás önmagában nem elég}\\

Ha egyszerűen egymás után fűzzük a számokat (pl. $12$, $345$, $78$
→ $1234578$), akkor az eredmény nem dekódolható egyértelműen: nem
tudjuk, hol végződnek az egyes számok.

\textbf{1. módszer: Hosszúság-információ hozzáfűzése}\\

Kódoljuk a számokat a következőképpen:
\begin{enumerate}
\item Minden számhoz adjuk meg a hosszát: $l_{i}=\text{számjegyek száma}(n_{i})$ 
\item Az üzenet tartalmazza először a hosszakat egy rögzített módon (pl.
minden hossz legyen kétjegyű, vagy egy prefix szakasz hosszinformációval),
majd sorban a számjegyeket. 
\end{enumerate}
\textbf{Példa:} $n_{1}=12$, $n_{2}=345$, $n_{3}=78$

\[
\text{Hosszúságok: }2,3,2\Rightarrow\texttt{020302}
\]
\[
\text{Számjegyek: }\texttt{1234578}
\]

A teljes kódolt üzenet: \texttt{0203021234578}

Így a címzett tudja: 
\begin{itemize}
\item Első szám 2 jegyű: 12 
\item Második szám 3 jegyű: 345 
\item Harmadik szám 2 jegyű: 78 
\end{itemize}
Ez egyértelmű dekódolást tesz lehetővé.

\textbf{2. módszer: Gödel-féle kódolás (matematikailag elegáns megközelítés)}\\

Legyenek a természetes számok: $n_{1},n_{2},\dots,n_{k}$.

Legyenek $p_{1},p_{2},\dots,p_{k}$ a legkisebb $k$ darab prímszám
(pl. $2,3,5,7,\dots$). Kódoljuk az üzenetet:

\[
N=p_{1}^{n_{1}}\cdot p_{2}^{n_{2}}\cdot\dots\cdot p_{k}^{n_{k}}
\]

Ez a szám egyértelműen visszafejthető prímtényezős felbontás segítségével.
Mivel minden természetes számnak egyértelmű prímtényezős felbontása
van, a $p_{i}$ kitevője pontosan $n_{i}$.

\textbf{Példa:} $n_{1}=2$, $n_{2}=3$, $n_{3}=1$

\[
N=2^{2}\cdot3^{3}\cdot5^{1}=4\cdot27\cdot5=540
\]

A címzett a kapott $540$-et faktorizálva: 
\[
540=2^{2}\cdot3^{3}\cdot5^{1}\Rightarrow(2,3,1)
\]

\textbf{Hátrány:} A kapott szám $N$ gyorsan nő, különösen nagy számok
esetén, mivel a hatványozás exponenciálisan növeli az értéket.

\textbf{Következtetés:}\\
\begin{itemize}
\item IGEN, megoldható több természetes szám elküldése egyetlen üzenetben. 
\item Léteznek különböző kódolási módszerek, amelyek biztosítják az egyértelmű
visszafejtést. 
\item Az egyik módszer a hossz-információ előzetes továbbítása, a másik
egy matematikailag elegáns, de számértékben nagy Gödel-kódolás. 
\end{itemize}
\end{solution}
\begin{extraproblem}[Szélyes Klaudia]
 Az előző feladat nehezítése. Adjál módszert az előző feladatra a
lehető legkevesebb billentyű használatával! 
\end{extraproblem}

\begin{solution}
Több természetes számot szeretnénk elküldeni \textbf{egyetlen üzenetben},
miközben a felhasznált számjegyek (billentyűleütések) száma \textbf{minimális}.


\textbf{Előző módszerek ismétlése (összehasonlítás)}\\
\begin{enumerate}
\item \textbf{Konkatenálás + hossz-információ} --- egyszerű, de plusz karaktereket
igényel a hosszak miatt. 
\item \textbf{Gödel-kódolás (prímtényezős)} --- matematikailag szép, egyetlen
számot küld, de nagy számok keletkeznek, és sok számjegyet eredményezhet. 
\end{enumerate}

\textbf{Optimalizált megközelítés: Egyértelmű kódolás kevés számjeggyel}\\

Célunk olyan módszer, amely: 
\begin{itemize}
\item egyértelműen dekódolható, 
\item a lehető legkevesebb számjegyet (karaktert) tartalmazza. 
\end{itemize}

\textbf{Módszer: Elias-gamma kód vagy univerzális kódolás}\\

Az \textbf{Elias-kódok} vagy egyéb univerzális tömörítő kódolások
(mint pl. Huffman-kód) jól alkalmazhatók. Az Elias-gamma kód lényege,
hogy a szám hosszát önmagában kódolja előtagként, így nincs szükség
külön hossz-információra.

\textbf{Alternatíva: Egy darab számként küldött pozicionált bináris kódolás}\\

Mivel a gép csak \textbf{egy számot} küldhet, a következő trükk alkalmazható:

\[
N=n_{1}\cdot B^{k-1}+n_{2}\cdot B^{k-2}+\dots+n_{k}
\]

Ahol: 
\begin{itemize}
\item $B$: egy olyan szám, ami \textbf{nagyobb, mint bármelyik $n_{i}$},
így nem lesz átfedés. 
\item $k$: számok száma. 
\end{itemize}
\textbf{Példa:} $n_{1}=3$, $n_{2}=12$, $n_{3}=5$

Legyen $B=100$ (mert egyik szám sem nagyobb ennél), akkor:

\[
N=3\cdot100^{2}+12\cdot100^{1}+5=30000+1200+5=31205
\]

A címzett tudja, hogy $B=100$ és $k=3$, így dekódol:

\[
n_{1}=\left\lfloor \frac{N}{100^{2}}\right\rfloor =3,\quad n_{2}=\left\lfloor \frac{N\bmod100^{2}}{100}\right\rfloor =12,\quad n_{3}=N\bmod100=5
\]


\textbf{Miért hatékony?}\\
\begin{itemize}
\item Csak \textbf{egy darab számjegysorozatot} (természetes számot) küldünk. 
\item Nem szükséges hossz-információ külön karakterként. 
\item A $B$ értéket a feladó és fogadó előre egyeztetheti vagy megegyezhetnek
abban, hogy mindig pl. $B=1000$. 
\end{itemize}

\textbf{Következtetés}\\
\begin{itemize}
\item IGEN, az előző feladat megoldható \textbf{kevesebb karakterrel}, ha
okos kódolási technikát használunk. 
\item A \textbf{pozicionált hatványalapú kódolás} vagy az \textbf{Elias-féle
univerzális kódolás} jelentősen csökkentheti a küldött szám hosszát. 
\item A legkevesebb billentyűleütést elérhetjük, ha a több számot \textbf{egy
darab számként kódoljuk} a fentiek szerint. 
\end{itemize}
\end{solution}


