\documentclass[magyar]{package/fancy-book}

%%%%%%%%%% Default Package %%%%%%%%%%%%%
\usepackage{package/color-env}
%%%%%%%%%% %%%%%%%%%%%%%%% %%%%%%%%%%%%%

%%%%%%%%%% Required Packages %%%%%%%%%%%%%
\usepackage{calligra} %%% (optional) to make the Title text beautiful 
\usepackage[magyar]{babel}

%%%%%%%%%%%%%%%%%%%%%%%%

\usepackage{amssymb,amsmath,amsfonts}  %%% for maths
\usepackage{braket}
\usepackage{graphicx}
\usepackage{extarrows}
\usepackage{mathtools}
\usepackage{adjustbox}
\usepackage{tikz}
\usetikzlibrary{calc,intersections,arrows}
%%%%%%%%%%%%%%%%%%%%%%%%%%%%%%%%%%%%%

\usepackage{multicol}
\usepackage{enumitem}
\usepackage{cancel}
%\usepackage{cite}


\usepackage[nobottomtitles]{titlesec}
\usepackage[section]{placeins}
\usepackage{indentfirst}
\usepackage{pdfpages}

\usepackage{url}
\def\UrlBreaks{\do\/\do-}
\usepackage{breakurl}


\definecolor{ccqqqq}{rgb}{0.8,0,0}

\addbibresource{resource/references.bib}
\DeclareNameAlias{default}{family-given}

\sloppy

\fancypagestyle{plain}{%
	\fancyhf{}% Clear header/footer
	\renewcommand{\headrulewidth}{0pt}%
}

\usepackage{datetime}
\newdateformat{version}{%
	\THEYEAR. \twodigit{\THEMONTH}. \twodigit{\THEDAY}.}


\begin{document}
	
\includepdf[fitpaper=true, pages=-]{resource/cover_a4.pdf}

\newpage
\thispagestyle{plain}
~
\vfill
\noindent \emph{Diszkrét matematika feladatok} \copyright{} 2025 by Authors\footnote{Authors: Bartalis Dorottya, Boga Zsombor, Csapó Hajnalka, Csurka-Molnár Hanna, Czofa Vivien, Domokos Ábel, Fábián Nóra, Gergely Verona, Kis Aranka-Enikő, Kis Brigitta, Kiss Andrea-Tímea, Kovács Levente, Lőrincz Attila, Lukács Andor, Miklós Dóra, Péter Róbert, Seres Brigitta, Sógor Bence, Szabó Kinga, Száfta Antal, Szélyes Klaudia} and Editor\footnote{Editor: Szöllősi István} is licensed under CC BY-NC 4.0. To view a copy of this license, visit \url{https://creativecommons.org/licenses/by-nc/4.0/}

\vspace*{1em}

\noindent Borítóterv, háttérkép: Szöllősi István

\vspace*{1em}

\noindent A könyv mindenkori legfrissebb verziója innen tölthető le: \url{https://github.com/szteven/diszkret-matematika-feladatok/releases/latest/download/diszkret-matematika-feladatok.pdf}

\vspace*{1em}

\noindent A könyv a következő \LaTeX{} sablon alapján készült: \url{https://github.com/dev-aditya/LaTeX-template}

\vspace*{3em}

{\large\noindent\textbf{Utolsó módosítás dátuma (verzió): \version\today}}

\vspace*{1em}

\begin{titlepage}

\centering % Centre everything on the title page
		
\scshape % Use small caps for all text on the title page

\vspace*{\baselineskip} % White space at the top of the page

%------------------------------------------------
%	Title
%------------------------------------------------

\rule{\textwidth}{1.6pt}\vspace*{-\baselineskip}\vspace*{2pt} % Thick horizontal rule
\rule{\textwidth}{0.4pt} % Thin horizontal rule

\vspace{0.75\baselineskip} % Whitespace above the title

{\Huge \calligra{ Feladatok }\\} % Title

\vspace{0.75\baselineskip} % Whitespace below the title

\rule{\textwidth}{0.4pt}\vspace*{-\baselineskip}\vspace{3.2pt} % Thin horizontal rule
\rule{\textwidth}{1.6pt} % Thick horizontal rule

\vspace{2\baselineskip} % Whitespace after the title block

%------------------------------------------------
%	Subtitle
%------------------------------------------------

\vspace*{2\baselineskip} % Whitespace under the subtitle

{\Huge Módszertani vonatkozások a diszkrét matematika tanításában \par} 

\vspace*{5\baselineskip} % Whitespace under the subtitle



%\vspace{0.5\baselineskip} 

{\scshape   \Large
\begin{tabular}{cc}
	Bartalis Dorottya & Boga Zsombor\tabularnewline
	Csapó Hajnalka & Csurka-Molnár Hanna \tabularnewline
	Czofa Vivien & Domokos Ábel\tabularnewline
	Fábián Nóra & Gál Tamara\tabularnewline
	Gergely Verona & Kis Aranka-Enikő\tabularnewline
	Kis Brigitta & Kiss Andrea-Tímea\tabularnewline
	Kovács Levente & Lőrincz Attila\tabularnewline
	Lukács Andor & Miklós Dóra\tabularnewline
	Péter Róbert & Seres Brigitta\tabularnewline
	Sógor Bence & Szabó Kinga\tabularnewline
	Száfta Antal & Szélyes Klaudia\tabularnewline
\end{tabular}
}
\vspace{5\baselineskip} 

\textit{\Large Babeș-Bolyai Tudományegyetem \\
Kolozsvár, 2025} 

\vfill 

%------------------------------------------------
% Author
%------------------------------------------------

\begin{figure}[!h]
    \centering
    \includegraphics[width = 3cm, height= 3cm]{resource/logo_ubb.png}%% include the university icon here
\end{figure}
\vspace{0.3\baselineskip} 


{\large Szerkesztette\\  Szöllősi István\par}
\vspace{\baselineskip} 
\end{titlepage}

\newpage

\tableofcontents

\chapter*{Előszó}
\addcontentsline{toc}{chapter}{Előszó}
\thispagestyle{plain}

Jelen feladatgyűjtemény a 2024/2025-ös tanév második féléve során
készült a \emph{Korszerű módszerek a matematikatanításban}, illetve
\emph{Didaktikai mesteri -- Matematika} szakos, első- és másodéves
hallgatók közreműködésével, a \emph{Módszertani vonatkozások a diszkrét
matematika tanításában} nevű tantárgy keretén belül (tantárgyfelelős:
dr. Szöllősi István, egyetemi docens, Babeș-Bolyai Tudományegyetem, Matematika és Informatika
Kar, Magyar Matematika és Informatika Intézet, Kolozsvár). 

A félév során minden egyes hallgató választott egy témát egy megadott
listából és annak alapján tartott egy bevezető jellegű tanórát a kollégái
előtt. A megadott témakör fogalmait, tételeit, stb. precízen és érthetően
be kellett vezetni, bizonyítani, megfelelő példákkal érthetővé kellett
tenni és meg kellett fogalmazni a többieknek 5-10, témához kapcsolódó
házi feladatot. Ezután (két hetes határidővel) meg kellett szerkeszteni
az órán leadott anyagot (esetleg kibővített formában) és a házi feladatok
megoldását egy adott \LaTeX{} sablont használva. A kollégák megoldották
a házi feladatokat és kerestek egy-két témához kapcsolódó nehezebb/érdekesebb
feladatot, majd ezeket -- megoldással együtt -- leírták (szintén
a megadott sablont használva). Ebből adódóan a könyv szerkezete a
következő: 
\begin{itemize}
	\item Minden fejezet (vagy alfejezet) egy hosszabb vagy rövidebb bevezető
	szöveggel indít, itt található a témához kapcsolódó elméleti tudnivaló:
	fogalmak meghatározásai, tételek, bizonyítások, példák, magyarázat,
	feladatok. Az adott fejezet elején a szerző neve fel van tüntetve. 
	\item Ezután egy \emph{Házi feladatok} című alfejezet következik, ez tartalmazza
	a szerző 5-10 házi feladatát, a (feltehetően) helyes megoldásokkal
	együtt. 
	\item Minden fejezet egy (általában terjedelmes) \emph{Nehezebb feladatok}
	című résszel zárul, ez tartalmazza a témához kapcsolódó nehezebb/érdekesebb
	feladatokat, szintén megoldásokkal együtt. Minden egyes feladatnál
	szerepel annak a hallgatónak a neve, aki az adott feladatot beküldte
	és a megoldását kidolgozta.
\end{itemize}
Ezzel a módszerrel és a 22 hallgató féléves munkájával készült el
a feladatgyűjtemény, ami -- reményeim szerint -- rajtuk kívül másoknak
(pl. iskolai matematikatanároknak, egyetemi hallgatóknak, matematikai
versenyekre készülőknek, stb.) is hasznára válhat. A gyűjtemény több
mint 500 megoldott feladatot és 22 elméleti gyorstalpalót tartalmaz,
ízelítőt adva a diszkrét matematika különböző fejezeteiből. 

Igyekeztem minden megoldást ellenőrizni és javítottam számos kisebb-nagyobb
hibát, de ennek ellenére figyelembe kell venni a következőket:
\begin{itemize}
	\item Számos elírás, esetleg matematikai hiba is lehet a könyvben, ha bárki
	ilyet észlel, jelezze\footnote{A következő email címre írva lehet jelezni bármilyen hibát, észrevételt: \href{mailto:istvan.szollosi@math.ubbcluj.ro}{istvan.szollosi@math.ubbcluj.ro}}, és amint lehet, javítjuk.
	\item A bevezető szövegeknek (``elméleti tudnivalóknak'') nyilvánvaló okok
	miatt eltérő a stílusa, különböző fokú részletességgel és precizitással
	vannak kidolgozva és tartalmazhatnak pontatlanságokat, elírásokat,
	kisebb hibákat. Ha bárki ilyet észlel, jelezze, és amint lehet, javítjuk.
	\item A feladatok érdekességének vagy nehézségi szintjének szempontjából
	vannak ``hálásabb'' és ``kevésbé hálás'' témák. A gyűjtemény nem feltétlenül
	követ egy ``könnyebbtől a nehezebb felé'' szerkesztési elvet.
	\item A feladatszövegek és/vagy megoldások, illetve felhasznált képek forrásmegjelölése helyenként
	hiányos lehet. A felhasznált irodalom nagy részét a könyv végén feltüntettük, 
	de valószínűleg a lista korántsem teljes vagy pontos. Ha bárki
	segítene e hiányosság pótlásában, vagy szerzőjogi aggályok/problémák
	merülnének fel, jelezze és amint lehet, orvosoljuk. Ez egy ingyenes,
	nyílt forrású, kiadatlan, saját használatra szánt anyag, ahol talán
	vállalható ez a kompromisszum (is). 
\end{itemize}
\newpage
\thispagestyle{plain}
\noindent A gyűjtemény tartalmát a következő hallgatók munkája nyomán állítottuk össze:
\begin{enumerate}
	\item Bartalis Dorottya -- \nameref{chap:dio}: \nameref{sec:lindio}
	\item Boga Zsombor -- \nameref{chap:indukcio}
	\item Csapó Hajnalka -- \nameref{chap:generatorf}: \nameref{sec:mi_a_generatorf}
	\item Csurka-Molnár Hanna -- \nameref{chap:grafok}: \nameref{sec:graf_alapok}
	\item Czofa Vivien -- \nameref{chap:elemi_szamelmelet}: \nameref{sec:szamelmelet_alapok}
	\item Domokos Ábel -- \nameref{chap:fak}: \nameref{sec:cimkezett_fak}
	\item Fábián Nóra -- \nameref{chap:halmazok}
	\item Gál Tamara -- \nameref{chap:ismetleses_perm}
	\item Gergely Verona -- \nameref{chap:grafok}: \nameref{sec:terkepek}
	\item Kis Aranka-Enikő -- \nameref{chap:grafok}: \nameref{sec:sikgrafok}
	\item Kis Brigitta -- \nameref{chap:elemi_szamelmelet}: \nameref{sec:Fermat-Wilson}
	\item Kiss Andrea-Tímea -- \nameref{chap:perm}
	\item Kovács Levente -- \nameref{chap:Fibonacci}
	\item Lőrincz Attila -- \nameref{chap:aszimmetrikus_kripto}
	\item Lukács Andor -- \nameref{chap:generatorf}: \nameref{sec:Catalan}, \nameref{chap:kombgeo}
	\item Miklós Dóra -- \nameref{chap:particiok}
	\item Péter Róbert -- \nameref{chap:skatulya}
	\item Seres Brigitta -- \nameref{chap:Pascal}
	\item Sógor Bence -- \nameref{chap:szimmetrikus-kripto}
	\item Szabó Kinga -- \nameref{chap:szita}
	\item Száfta Antal -- \nameref{chap:dio}: \nameref{sec:vegtelen_leszallas}
	\item Szélyes Klaudia -- \nameref{chap:fak}: \nameref{sec:fak_alapok}
\end{enumerate}
Köszönöm a felsorolt kollégák figyelmét és egész féléves kitartó munkáját.
A tanári hivatáshoz sok sikert, türelmet és megbecsülést kívánok!

\vspace*{1cm}

\begin{tabular*}{1\textwidth}{@{\extracolsep{\fill}}lr}
	dr. Szöllősi István & Kolozsvár,\tabularnewline
    egyetemi docens & 2025. július 7.
\end{tabular*}

\include{\string"content/Fabian_Nora/chapter_halmazok_fuggvenyek\string"}
\include{\string"content/Kiss_Andrea_Timea/chapter_variaciok_permutaciok_kombinaciok\string"}
\include{\string"content/Gal_Tamara/chapter_ismetleses_permutacio_variacio_kombinacio\string"}
\include{\string"content/Boga_Zsombor/chapter_teljes_indukcio\string"}
\include{\string"content/Szabo_Kinga/chapter_szitaformula\string"}
\include{\string"content/Peter_Robert/chapter_skatulyaelv\string"}
\include{\string"content/Seres_Brigitta_Alexandra/chapter_binomialis_pascal\string"}
\include{\string"content/Miklos_Dora/chapter_particiok\string"}
\include{\string"content/Czofa_Vivien-Kis_Brigitta/chapter_szamelmelet\string"}
\include{\string"content/Sogor_Bence/chapter_szimmetrikus_kripto\string"}
\include{\string"content/Lorincz_Attila/chapter_aszimmetrikus_kripto\string"}
\include{\string"content/Csapo_Hajnalka-Lukacs_Andor/chapter_generatorfuggvenyek\string"}
\include{\string"content/Kovacs_Levente/chapter_fibonacci\string"}
\include{\string"content/Bartalis_Dorottya-Szafta_Antal/chapter_diofantikus_egyenletek\string"}
\include{\string"content/Csurka_Molnar_Hanna-Kis_Aranka_Eniko-Gergely_Verona/chapter_grafok\string"}
\include{\string"content/Szelyes_Klaudia-Domokos_Abel/chapter_fak\string"}
\include{\string"content/Lukacs_Andor/chapter_kombinatorikus_geometria\string"}

\nocite{*}
\printbibliography[heading=bibintoc,title={\centering Könyvészet}]

%\bibliographystyle{diszkret} % or alpha, abbrv, etc.

%\bibliography{resource/references} % omit the .bib extension

\end{document}
